Vectorial boolean functions $s: \F\to\F$ have many applications, especially if they come with additional properties. For example, \emph{almost perfect nonlinear} (APN) functions are used in block ciphers to resist differential attacks \cite{Nyberg1993}.

The APN property (among others) is invariant under affine and other transformations, narrowing the search for suitable candidates to representatives from each equivalence class. In fact, if $\alpha, \beta$ are affine bijections, and $\gamma$ is any affine function, then we define $t, s: \F\to\F$ to be \emph{extended affine} (EA) equivalent, if $t = \beta s \alpha + \gamma$.

In \cite{Carlet1998}, a more general equivalence relation was introduced that includes EA equivalence as a special case. Let $\G(s) := \{(x,s(x)) | x \in\F\} \subseteq \F\times\F$ be the graph of the function
$s$. Then a function $t$ is \emph{CCZ} equivalent to $s$ if $\G(t)$ is affine equivalent to $\G(s)$ in $\F\times\F$, that is if there exists an affine bijection $\lambda\in {\mathbb F}_2^{2n}$ such that $\G(t) = \lambda(\G(s))$. It was proven in \cite{Carlet1998} that this equivalence relation stabilises the APN property.

In \cite{Brinkmann2008}, we give a complete classification of APN functions up to dimension~5, applying efficient canonicity filters to backtrack programming.  Using the same methods, we are also able to classify all vectorial boolean functions (i.e., not only those APN) up to dimension~4 modulo extended affine and CCZ equivalence. Although results for higher dimensions are more interesting (e.g., there are 8~bits in a byte), low dimension results can still be useful for efficient construction of functions in higher dimensions, see for example~\cite{Boss2017}. Thus, we give our complete results for dimension~4 here as a reference.
