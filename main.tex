\documentclass[a4paper,twocolumn,10pt]{article}
\usepackage{usenix,epsfig,endnotes}

\usepackage[utf8]{inputenc}%(only for the pdftex engine)
%\RequirePackage[no-math]{fontspec}%(only for the luatex or the xetex engine)
\usepackage[T1]{fontenc}
\usepackage{xspace}
%%% \usepackage{graphicx}
%%% \usepackage[hyphens]{url}
\usepackage{url}
\usepackage{hyperref}
%\usepackage[table]{xcolor}    % loads also »colortbl«
\usepackage{listings}
%%% \usepackage{tabularx}
%%% \usepackage{float}
\usepackage[dvipsnames, table]{xcolor}
\usepackage{colortbl}
%%% \usepackage[flushleft]{threeparttable}
\usepackage{mdframed}
\usepackage{booktabs}
\usepackage{subcaption}
\usepackage{eurosym}
\usepackage{amsmath}
\usepackage{rotating}
\usepackage{multirow}
\usepackage[affil-it]{authblk}
%%% \usepackage{nth}
%%% \usepackage{courier}
\usepackage{wasysym}
\usepackage{pifont}
\usepackage[font={small}]{caption}

\usepackage{amsfonts}
\usepackage{booktabs}
\usepackage{longtable}
\usepackage{float}

%%%%%%%%%%%%%%%%%%%%%%%%%%%%%%%%%%%%%%%%%%%%%%%%%%%%%%%%%%%%%%%%%%
\newcommand{\F}{{\mathbb{F}}_2^n}
\newcommand{\G}{\mathcal{G}}
%%%%%%%%%%%%%%%%%%%%%%%%%%%%%%%%%%%%%%%%%%%%%%%%%%%%%%%%%%%%%%%%%%

\begin{document}

% don't want date printed
\date{}

\title{Extended Affine and CCZ Equivalence up to Dimension 4}

\author{
{\rm Marcus Brinkmann}\thanks{Email: \texttt{marcus.brinkmann@rub.de}}\\
Ruhr University Bochum
}

\maketitle


\subsection*{Abstract}

For all vectorial boolean functions up to dimension 4, we present canonical representatives for all extended affine (EA) and CCZ equivalence classes.  We include the size of each class, as well as its algebraic degree and extended Walsh spectrum.  We also answer the following questions: How large are these classes? Which of these classes contain bijective functions? And how are these classes grouped into CCZ equivalence classes?

\section{Introduction}
Vectorial boolean functions $s: \F\to\F$ have many applications, especially if they come with additional properties. For example, \emph{almost perfect nonlinear} (APN) functions are used in block ciphers to resist differential attacks \cite{Nyberg1993}.

The APN property (among others) is invariant under affine and other transformations, narrowing the search for suitable candidates to representatives from each equivalence class. In fact, if $\alpha, \beta$ are affine bijections, and $\gamma$ is any affine function, then we define $t, s: \F\to\F$ to be \emph{extended affine} (EA) equivalent, if $t = \beta s \alpha + \gamma$.

In \cite{Carlet1998}, a more general equivalence relation was introduced that includes EA equivalence as a special case. Let $\G(s) := \{(x,s(x)) | x \in\F\} \subseteq \F\times\F$ be the graph of the function
$s$. Then a function $t$ is \emph{CCZ} equivalent to $s$ if $\G(t)$ is affine equivalent to $\G(s)$ in $\F\times\F$, that is if there exists an affine bijection $\lambda\in {\mathbb F}_2^{2n}$ such that $\G(t) = \lambda(\G(s))$. It was proven in \cite{Carlet1998} that this equivalence relation stabilises the APN property.

In \cite{Brinkmann2008}, we give a complete classification of APN functions up to dimension~5, applying efficient canonicity filters to backtrack programming.  Using the same methods, we are also able to classify all vectorial boolean functions (i.e., not only those APN) up to dimension~4 modulo extended affine and CCZ equivalence. Although results for higher dimensions are more interesting (e.g., there are 8~bits in a byte), low dimension results can still be useful for efficient construction of functions in higher dimensions, see for example~\cite{Boss2017}. Thus, we give our complete results for dimension~4 here as a reference.

\section{Methodology}
We use depth-first backtrack programming with efficient EA canonicity filters to prune the search tree, as in \cite{Brinkmann2008}.  A function is a \emph{canonical representative} of its class if its function table is the smallest lexicographically in the alphabet $\F$.

\subsection{Table Layout}

The tables below enumerate all EA classes and their canonical representative by giving their function table.

We include the algebraic degree, which is an EA (but not CCZ) invariant. Note that for this to be true, we give the degree of the class containing $s \equiv 0$ as $1$.

We also include the extended Walsh spectrum, which is the multi-set $W_s := \{|w_s(a,b)| \mid a,b\in\F, b \neq 0\}$ where $w_s(a,b)=\sum_{x\in\F}(-1)^{bs(x)+ax}$. The extended Walsh spectrum is known to be a CCZ (and thus also EA) invariant \cite{Carlet1998}.

We further check for CCZ equivalence among the EA canonical representations.  Because CCZ equivalence implies EA equivalence, this is sufficient to find representatives for all CCZ classes.  In the tables below, the column \emph{CCZ} either says "can." if this entry is also a canonical representative for CCZ equivalence, or it gives the index of a CCZ canonical representative to which this entry is CCZ equivalent.

We directly check for EA equivalence to bijective functions. If an entry in the table is EA equivalent to a bijective function, the column "bij.?" contains the letter "Y" (for "yes"), otherwise it contains the letter "N" (for "no").

\subsection{Total Number of Functions}

The total number of functions in $\F\to\F$ is $({2^n})^{2^n}$. This set is partitioned into EA equivalence classes. To determine the size of each class, we consider the group $G$ of all function triplets $(\alpha, \beta, \gamma)$ that acts on $s:\F\to\F$ by means of $(\alpha, \beta, \gamma)\cdot s = \beta s \alpha + \gamma$, where $\alpha$ is an affine bijection, $\beta$ is a linear bijection, and $\gamma$ is any affine function (the affine component of $\beta$ is subsumed by $\gamma$). Then the EA equivalence class of $s$ is the orbit $Gs$ and the size is $|Gs| = |G|/|G_s|$, where $G_s = \{(\alpha, \beta, \gamma) \in G | \beta s \alpha + \gamma = s\}$ is the \emph{stabiliser} of $s$.

The order of $G$ is easily counted, as the number of linear bijections is well known (cf. A002884 in \cite{oeis}):

\begin{equation}
\begin{aligned}
|G| & = |G^\alpha|\cdot|G^\beta|\cdot|G^\gamma| \\
& = \left( 2^n \cdot \prod_{i=0}^{n-1}(2^n-2^i) \right)\cdot
\left( \prod_{i=0}^{n-1}(2^n-2^i)\right)
\cdot{(2^n)}^{n+1}
\end{aligned}
\end{equation}

See \autoref{tab:order} for the order of $|G|$ up to dimension 4.

\begin{table}\centering
\begin{tabular}{>{\bfseries}rrrrr}
\toprule
$n$ & $|G^\alpha|$ & $|G^\beta|$ & $|G^\gamma|$ & $|G|$ \\
\midrule
1 & 2 & 1 & 4 & 8 \\
2 & 24 & 6 & 64 & 9216 \\
3 & 1344 & 168 & 4096 & 924844032 \\
4 & 322560 & 20160 & 1048576 & 6818690079129600 \\
 \bottomrule
\end{tabular}
\caption{The size of the group of affine functions $(\alpha, \beta, \gamma)$ acting on $s$, c.f. A028365, A002884 and A053763 in \cite{oeis}.}
\label{tab:order}
\end{table}

For example, to calculate the size of the equivalence class $s=s_1$ in \autoref{tab:dim4}, we have $|Gs| = |G|/|G_s| = 6818690079129600 / 6502809600 = 1048576$.

\section{Results}
\subsection{Dimension $n=1$}
In dimension $n=1$ we have only a single EA equivalence class, see \autoref{tab:dim1}.  It contains all four functions, including the two bijective functions. All functions are linear, and have the same Walsh spectrum (\autoref{tab:walshdim1}).

\subsection{Dimension $n=2$}
In dimension $n=2$ we have two EA equivalence classes, one containing the bijective functions, see \autoref{tab:dim2}. They have distinct Walsh spectrums (\autoref{tab:walshdim2}) and thus are also different CCZ equivalence classes.

\subsection{Dimension $n=3$}
In dimension $n=3$, we have seven distinct EA equivalence classes, four containing bijective functions (and these are all linear or quadratic), see \autoref{tab:dim3}.  They all have distinct Walsh spectrums (\autoref{tab:walshdim3}), and thus also represent different CCZ equivalence classes.

\subsection{Dimension $n=4$}
In dimension $n=4$, we find 4713 different EA equivalence classes (with 71 different sizes), of which 194 contain bijective functions (while 4519 do not), see \autoref{tab:dim4}. The classes have 481 distinct Walsh spectrums (\autoref{tab:walshdim4}). However, the number of CCZ equivalence classes is 4151 and thus much larger (only 562 EA canonical representatives are not also CCZ canonical representatives).

\appendix

\section{References}
\bibliographystyle{ieeetr}
\bibliography{100-bibliography}

\newpage
\section{Function Tables}
\subsection{Dimension 1}
\begin{table}[H]\centering
\begin{tabular}{>{\bfseries}rccc}
\toprule
& \multicolumn{3}{c}{Multiplicity for $0, ..., 2^n$} \\
\cmidrule{2-4}
\# & 0 & 1 & 2 \\
\midrule
$w_1$ & 1 & 0 & 1 \\
 \bottomrule
\end{tabular}
\caption{Extended Walsh spectrums for dimension $n=1$.}
\label{tab:walshdim1}
\end{table}

\begin{table}[H]\centering
\begin{tabular}{>{\bfseries}rcccrccc}
\toprule
& \multicolumn{2}{c}{$s(x)$ for $x\in\F$} & \\
\cmidrule{2-3}
\# & 0 & 1 & $|G_s|$ & deg & $w$ & bij.? & CCZ \\
\midrule
1 & 0 & 0 & 2 & 1 & $w_1$ & Y & can. \\
 \bottomrule
\end{tabular}
\caption{Extended affine and CCZ equivalence in dimension $n=1$.}
\label{tab:dim1}
\end{table}

\subsection{Dimension 2}
\begin{table}[H]\centering
\begin{tabular}{>{\bfseries}rccccc}
\toprule
& \multicolumn{5}{c}{Multiplicity for $0, ..., 2^n$} \\
\cmidrule{2-6}
\# & 0 & 1 & 2 & 3 & 4 \\
\midrule
$w_1$ & 9 & 0 & 0 & 0 & 3 \\
$w_2$ & 3 & 0 & 8 & 0 & 1 \\
\bottomrule
\end{tabular}
\caption{Walsh spectrums for dimension $n=2$.}
\label{tab:walshdim2}
\end{table}

\begin{table}[H]\centering
\begin{tabular}{>{\bfseries}rccccrcccc}
\toprule
& \multicolumn{4}{c}{$s(x)$ for $x\in\F$} & \\
\cmidrule{2-5}
\# & 0 & 1 & 2 & 3 & $|G_s|$ & deg & $w$ & bij.? & CCZ \\
\midrule
1 & 0 & 0 & 0 & 0 & 144 & 1 & $w_1$ & Y & can. \\
2 & 0 & 0 & 0 & 1 & 48 & 2 & $w_2$ & N & can. \\
 \bottomrule
\end{tabular}
\caption{Extended affine and CCZ equivalence in dimension $n=2$.}
\label{tab:dim2}
\end{table}

\newpage\onecolumn
\subsection{Dimension 3}
\begin{table}[h]\centering
\begin{tabular}{>{\bfseries}rccccccccc}
\toprule
& \multicolumn{9}{c}{Multiplicity for $0, ..., 2^n$} \\
\cmidrule{2-10}
\# & 0 & 1 & 2 & 3 & 4 & 5 & 6 & 7 & 8 \\
\midrule
$w_1$ & 49 & 0 & 0 & 0 & 0 & 0 & 0 & 0 & 7 \\
$w_2$ & 21 & 0 & 28 & 0 & 0 & 0 & 4 & 0 & 3 \\
$w_3$ & 37 & 0 & 0 & 0 & 16 & 0 & 0 & 0 & 3 \\
$w_4$ & 15 & 0 & 28 & 0 & 8 & 0 & 4 & 0 & 1 \\
$w_5$ & 31 & 0 & 0 & 0 & 24 & 0 & 0 & 0 & 1 \\
$w_6$ & 12 & 0 & 28 & 0 & 12 & 0 & 4 & 0 & 0 \\
$w_7$ & 28 & 0 & 0 & 0 & 28 & 0 & 0 & 0 & 0 \\
\bottomrule
\end{tabular}
\caption{Walsh spectrums for dimension $n=3$.}
\label{tab:walshdim3}
\end{table}

\begin{table}[h]\centering
\begin{tabular}{>{\bfseries}rccccccccrcccc}
\toprule
& \multicolumn{8}{c}{$s(x)$ for $x\in\F$} \\
\cmidrule{2-9}
\# & 0 & 1 & 2 & 3 & 4 & 5 & 6 & 7 & $|G_s|$ & deg & $w$ & bij.? & CCZ \\
\midrule
1 & 0 & 0 & 0 & 0 & 0 & 0 & 0 & 0 & 225792 & 1 & $w_1$ & Y & can. \\
2 & 0 & 0 & 0 & 0 & 0 & 0 & 0 & 1 & 4032 & 3 & $w_2$ & N & can. \\
3 & 0 & 0 & 0 & 0 & 0 & 0 & 1 & 1 & 4608 & 2 & $w_3$ & Y & can. \\
4 & 0 & 0 & 0 & 0 & 0 & 0 & 1 & 2 & 192 & 3 & $w_4$ & N & can. \\
5 & 0 & 0 & 0 & 0 & 0 & 1 & 2 & 3 & 768 & 2 & $w_5$ & Y & can. \\
6 & 0 & 0 & 0 & 0 & 0 & 1 & 2 & 4 & 96 & 3 & $w_6$ & N & can. \\
7 & 0 & 0 & 0 & 1 & 0 & 2 & 4 & 7 & 1344 & 2 & $w_7$ & Y & can. \\
 \bottomrule
\end{tabular}
\caption{Extended affine and CCZ equivalence in dimension $n=3$.}
\label{tab:dim3}
\end{table}

\newpage
\subsection{Dimension 4}
\begin{longtable}{>{\bfseries}rccccccccccccccccc}
\caption{Walsh spectrums for dimension $n=4$.} \\
\toprule
& \multicolumn{17}{c}{Multiplicity for $0, 2^i, i \in\F$} \\
\cmidrule{2-18}
\# & 0 & 1 & 2 & 3 & 4 & 5 & 6 & 7 & 8 & 9 & 10 & 11 & 12 & 13 & 14 & 15 & 16 \\
\midrule
\label{tab:walshdim4}
\endfirsthead
\multicolumn{17}{c}%
{\tablename\ \thetable\ -- \textit{Continued from previous page}} \\
\toprule
\# & 0 & 1 & 2 & 3 & 4 & 5 & 6 & 7 & 8 & 9 & 10 & 11 & 12 & 13 & 14 & 15 & 16 \\
\midrule
\endhead
\bottomrule
\multicolumn{17}{r}{\textit{Continued on next page}} \\
\endfoot
\bottomrule
\endlastfoot
$w_{1}$ & 225 & 0 & 0 & 0 & 0 & 0 & 0 & 0 & 0 & 0 & 0 & 0 & 0 & 0 & 0 & 0 & 15 \\
$w_{2}$ & 105 & 0 & 120 & 0 & 0 & 0 & 0 & 0 & 0 & 0 & 0 & 0 & 0 & 0 & 8 & 0 & 7 \\
$w_{3}$ & 169 & 0 & 0 & 0 & 56 & 0 & 0 & 0 & 0 & 0 & 0 & 0 & 8 & 0 & 0 & 0 & 7 \\
$w_{4}$ & 77 & 0 & 120 & 0 & 28 & 0 & 0 & 0 & 0 & 0 & 0 & 0 & 4 & 0 & 8 & 0 & 3 \\
$w_{5}$ & 105 & 0 & 96 & 0 & 0 & 0 & 24 & 0 & 0 & 0 & 8 & 0 & 0 & 0 & 0 & 0 & 7 \\
$w_{6}$ & 77 & 0 & 108 & 0 & 28 & 0 & 12 & 0 & 0 & 0 & 4 & 0 & 4 & 0 & 4 & 0 & 3 \\
$w_{7}$ & 141 & 0 & 0 & 0 & 84 & 0 & 0 & 0 & 0 & 0 & 0 & 0 & 12 & 0 & 0 & 0 & 3 \\
$w_{8}$ & 63 & 0 & 114 & 0 & 42 & 0 & 6 & 0 & 0 & 0 & 2 & 0 & 6 & 0 & 6 & 0 & 1 \\
$w_{9}$ & 201 & 0 & 0 & 0 & 0 & 0 & 0 & 0 & 32 & 0 & 0 & 0 & 0 & 0 & 0 & 0 & 7 \\
$w_{10}$ & 93 & 0 & 108 & 0 & 0 & 0 & 12 & 0 & 16 & 0 & 4 & 0 & 0 & 0 & 4 & 0 & 3 \\
$w_{11}$ & 157 & 0 & 0 & 0 & 56 & 0 & 0 & 0 & 16 & 0 & 0 & 0 & 8 & 0 & 0 & 0 & 3 \\
$w_{12}$ & 77 & 0 & 96 & 0 & 28 & 0 & 24 & 0 & 0 & 0 & 8 & 0 & 4 & 0 & 0 & 0 & 3 \\
$w_{13}$ & 71 & 0 & 108 & 0 & 28 & 0 & 12 & 0 & 8 & 0 & 4 & 0 & 4 & 0 & 4 & 0 & 1 \\
$w_{14}$ & 63 & 0 & 102 & 0 & 42 & 0 & 18 & 0 & 0 & 0 & 6 & 0 & 6 & 0 & 2 & 0 & 1 \\
$w_{15}$ & 135 & 0 & 0 & 0 & 84 & 0 & 0 & 0 & 8 & 0 & 0 & 0 & 12 & 0 & 0 & 0 & 1 \\
$w_{16}$ & 60 & 0 & 108 & 0 & 42 & 0 & 12 & 0 & 4 & 0 & 4 & 0 & 6 & 0 & 4 & 0 & 0 \\
$w_{17}$ & 153 & 0 & 0 & 0 & 64 & 0 & 0 & 0 & 16 & 0 & 0 & 0 & 0 & 0 & 0 & 0 & 7 \\
$w_{18}$ & 69 & 0 & 108 & 0 & 32 & 0 & 12 & 0 & 8 & 0 & 4 & 0 & 0 & 0 & 4 & 0 & 3 \\
$w_{19}$ & 133 & 0 & 0 & 0 & 88 & 0 & 0 & 0 & 8 & 0 & 0 & 0 & 8 & 0 & 0 & 0 & 3 \\
$w_{20}$ & 59 & 0 & 108 & 0 & 44 & 0 & 12 & 0 & 4 & 0 & 4 & 0 & 4 & 0 & 4 & 0 & 1 \\
$w_{21}$ & 123 & 0 & 0 & 0 & 100 & 0 & 0 & 0 & 4 & 0 & 0 & 0 & 12 & 0 & 0 & 0 & 1 \\
$w_{22}$ & 54 & 0 & 108 & 0 & 50 & 0 & 12 & 0 & 2 & 0 & 4 & 0 & 6 & 0 & 4 & 0 & 0 \\
$w_{23}$ & 125 & 0 & 0 & 0 & 92 & 0 & 0 & 0 & 16 & 0 & 0 & 0 & 4 & 0 & 0 & 0 & 3 \\
$w_{24}$ & 55 & 0 & 108 & 0 & 46 & 0 & 12 & 0 & 8 & 0 & 4 & 0 & 2 & 0 & 4 & 0 & 1 \\
$w_{25}$ & 69 & 0 & 96 & 0 & 32 & 0 & 24 & 0 & 8 & 0 & 8 & 0 & 0 & 0 & 0 & 0 & 3 \\
$w_{26}$ & 59 & 0 & 102 & 0 & 44 & 0 & 18 & 0 & 4 & 0 & 6 & 0 & 4 & 0 & 2 & 0 & 1 \\
$w_{27}$ & 55 & 0 & 102 & 0 & 46 & 0 & 18 & 0 & 8 & 0 & 6 & 0 & 2 & 0 & 2 & 0 & 1 \\
$w_{28}$ & 115 & 0 & 0 & 0 & 104 & 0 & 0 & 0 & 12 & 0 & 0 & 0 & 8 & 0 & 0 & 0 & 1 \\
$w_{29}$ & 63 & 0 & 96 & 0 & 42 & 0 & 24 & 0 & 0 & 0 & 8 & 0 & 6 & 0 & 0 & 0 & 1 \\
$w_{30}$ & 50 & 0 & 105 & 0 & 52 & 0 & 15 & 0 & 6 & 0 & 5 & 0 & 4 & 0 & 3 & 0 & 0 \\
$w_{31}$ & 149 & 0 & 0 & 0 & 60 & 0 & 0 & 0 & 24 & 0 & 0 & 0 & 4 & 0 & 0 & 0 & 3 \\
$w_{32}$ & 67 & 0 & 108 & 0 & 30 & 0 & 12 & 0 & 12 & 0 & 4 & 0 & 2 & 0 & 4 & 0 & 1 \\
$w_{33}$ & 93 & 0 & 96 & 0 & 0 & 0 & 24 & 0 & 16 & 0 & 8 & 0 & 0 & 0 & 0 & 0 & 3 \\
$w_{34}$ & 67 & 0 & 102 & 0 & 30 & 0 & 18 & 0 & 12 & 0 & 6 & 0 & 2 & 0 & 2 & 0 & 1 \\
$w_{35}$ & 71 & 0 & 102 & 0 & 28 & 0 & 18 & 0 & 8 & 0 & 6 & 0 & 4 & 0 & 2 & 0 & 1 \\
$w_{36}$ & 127 & 0 & 0 & 0 & 88 & 0 & 0 & 0 & 16 & 0 & 0 & 0 & 8 & 0 & 0 & 0 & 1 \\
$w_{37}$ & 56 & 0 & 105 & 0 & 44 & 0 & 15 & 0 & 8 & 0 & 5 & 0 & 4 & 0 & 3 & 0 & 0 \\
$w_{38}$ & 71 & 0 & 96 & 0 & 28 & 0 & 24 & 0 & 8 & 0 & 8 & 0 & 4 & 0 & 0 & 0 & 1 \\
$w_{39}$ & 59 & 0 & 96 & 0 & 44 & 0 & 24 & 0 & 4 & 0 & 8 & 0 & 4 & 0 & 0 & 0 & 1 \\
$w_{40}$ & 56 & 0 & 102 & 0 & 44 & 0 & 18 & 0 & 8 & 0 & 6 & 0 & 4 & 0 & 2 & 0 & 0 \\
$w_{41}$ & 50 & 0 & 102 & 0 & 52 & 0 & 18 & 0 & 6 & 0 & 6 & 0 & 4 & 0 & 2 & 0 & 0 \\
$w_{42}$ & 54 & 0 & 99 & 0 & 50 & 0 & 21 & 0 & 2 & 0 & 7 & 0 & 6 & 0 & 1 & 0 & 0 \\
$w_{43}$ & 60 & 0 & 99 & 0 & 42 & 0 & 21 & 0 & 4 & 0 & 7 & 0 & 6 & 0 & 1 & 0 & 0 \\
$w_{44}$ & 116 & 0 & 0 & 0 & 102 & 0 & 0 & 0 & 12 & 0 & 0 & 0 & 10 & 0 & 0 & 0 & 0 \\
$w_{45}$ & 189 & 0 & 0 & 0 & 0 & 0 & 0 & 0 & 48 & 0 & 0 & 0 & 0 & 0 & 0 & 0 & 3 \\
$w_{46}$ & 87 & 0 & 102 & 0 & 0 & 0 & 18 & 0 & 24 & 0 & 6 & 0 & 0 & 0 & 2 & 0 & 1 \\
$w_{47}$ & 151 & 0 & 0 & 0 & 56 & 0 & 0 & 0 & 24 & 0 & 0 & 0 & 8 & 0 & 0 & 0 & 1 \\
$w_{48}$ & 68 & 0 & 102 & 0 & 28 & 0 & 18 & 0 & 12 & 0 & 6 & 0 & 4 & 0 & 2 & 0 & 0 \\
$w_{49}$ & 141 & 0 & 0 & 0 & 64 & 0 & 0 & 0 & 32 & 0 & 0 & 0 & 0 & 0 & 0 & 0 & 3 \\
$w_{50}$ & 63 & 0 & 102 & 0 & 32 & 0 & 18 & 0 & 16 & 0 & 6 & 0 & 0 & 0 & 2 & 0 & 1 \\
$w_{51}$ & 119 & 0 & 0 & 0 & 92 & 0 & 0 & 0 & 24 & 0 & 0 & 0 & 4 & 0 & 0 & 0 & 1 \\
$w_{52}$ & 55 & 0 & 96 & 0 & 46 & 0 & 24 & 0 & 8 & 0 & 8 & 0 & 2 & 0 & 0 & 0 & 1 \\
$w_{53}$ & 52 & 0 & 102 & 0 & 46 & 0 & 18 & 0 & 12 & 0 & 6 & 0 & 2 & 0 & 2 & 0 & 0 \\
$w_{54}$ & 143 & 0 & 0 & 0 & 60 & 0 & 0 & 0 & 32 & 0 & 0 & 0 & 4 & 0 & 0 & 0 & 1 \\
$w_{55}$ & 67 & 0 & 96 & 0 & 30 & 0 & 24 & 0 & 12 & 0 & 8 & 0 & 2 & 0 & 0 & 0 & 1 \\
$w_{56}$ & 64 & 0 & 102 & 0 & 30 & 0 & 18 & 0 & 16 & 0 & 6 & 0 & 2 & 0 & 2 & 0 & 0 \\
$w_{57}$ & 56 & 0 & 99 & 0 & 44 & 0 & 21 & 0 & 8 & 0 & 7 & 0 & 4 & 0 & 1 & 0 & 0 \\
$w_{58}$ & 124 & 0 & 0 & 0 & 88 & 0 & 0 & 0 & 20 & 0 & 0 & 0 & 8 & 0 & 0 & 0 & 0 \\
$w_{59}$ & 60 & 0 & 96 & 0 & 42 & 0 & 24 & 0 & 4 & 0 & 8 & 0 & 6 & 0 & 0 & 0 & 0 \\
$w_{60}$ & 117 & 0 & 0 & 0 & 96 & 0 & 0 & 0 & 24 & 0 & 0 & 0 & 0 & 0 & 0 & 0 & 3 \\
$w_{61}$ & 51 & 0 & 102 & 0 & 48 & 0 & 18 & 0 & 12 & 0 & 6 & 0 & 0 & 0 & 2 & 0 & 1 \\
$w_{62}$ & 107 & 0 & 0 & 0 & 108 & 0 & 0 & 0 & 20 & 0 & 0 & 0 & 4 & 0 & 0 & 0 & 1 \\
$w_{63}$ & 46 & 0 & 102 & 0 & 54 & 0 & 18 & 0 & 10 & 0 & 6 & 0 & 2 & 0 & 2 & 0 & 0 \\
$w_{64}$ & 50 & 0 & 99 & 0 & 52 & 0 & 21 & 0 & 6 & 0 & 7 & 0 & 4 & 0 & 1 & 0 & 0 \\
$w_{65}$ & 112 & 0 & 0 & 0 & 104 & 0 & 0 & 0 & 16 & 0 & 0 & 0 & 8 & 0 & 0 & 0 & 0 \\
$w_{66}$ & 54 & 0 & 96 & 0 & 50 & 0 & 24 & 0 & 2 & 0 & 8 & 0 & 6 & 0 & 0 & 0 & 0 \\
$w_{67}$ & 63 & 0 & 96 & 0 & 32 & 0 & 24 & 0 & 16 & 0 & 8 & 0 & 0 & 0 & 0 & 0 & 1 \\
$w_{68}$ & 51 & 0 & 96 & 0 & 48 & 0 & 24 & 0 & 12 & 0 & 8 & 0 & 0 & 0 & 0 & 0 & 1 \\
$w_{69}$ & 52 & 0 & 99 & 0 & 46 & 0 & 21 & 0 & 12 & 0 & 7 & 0 & 2 & 0 & 1 & 0 & 0 \\
$w_{70}$ & 46 & 0 & 99 & 0 & 54 & 0 & 21 & 0 & 10 & 0 & 7 & 0 & 2 & 0 & 1 & 0 & 0 \\
$w_{71}$ & 108 & 0 & 0 & 0 & 106 & 0 & 0 & 0 & 20 & 0 & 0 & 0 & 6 & 0 & 0 & 0 & 0 \\
$w_{72}$ & 50 & 0 & 96 & 0 & 52 & 0 & 24 & 0 & 6 & 0 & 8 & 0 & 4 & 0 & 0 & 0 & 0 \\
$w_{73}$ & 56 & 0 & 96 & 0 & 44 & 0 & 24 & 0 & 8 & 0 & 8 & 0 & 4 & 0 & 0 & 0 & 0 \\
$w_{74}$ & 87 & 0 & 96 & 0 & 0 & 0 & 24 & 0 & 24 & 0 & 8 & 0 & 0 & 0 & 0 & 0 & 1 \\
$w_{75}$ & 64 & 0 & 99 & 0 & 30 & 0 & 21 & 0 & 16 & 0 & 7 & 0 & 2 & 0 & 1 & 0 & 0 \\
$w_{76}$ & 68 & 0 & 96 & 0 & 28 & 0 & 24 & 0 & 12 & 0 & 8 & 0 & 4 & 0 & 0 & 0 & 0 \\
$w_{77}$ & 120 & 0 & 0 & 0 & 90 & 0 & 0 & 0 & 24 & 0 & 0 & 0 & 6 & 0 & 0 & 0 & 0 \\
$w_{78}$ & 183 & 0 & 0 & 0 & 0 & 0 & 0 & 0 & 56 & 0 & 0 & 0 & 0 & 0 & 0 & 0 & 1 \\
$w_{79}$ & 84 & 0 & 99 & 0 & 0 & 0 & 21 & 0 & 28 & 0 & 7 & 0 & 0 & 0 & 1 & 0 & 0 \\
$w_{80}$ & 135 & 0 & 0 & 0 & 64 & 0 & 0 & 0 & 40 & 0 & 0 & 0 & 0 & 0 & 0 & 0 & 1 \\
$w_{81}$ & 60 & 0 & 99 & 0 & 32 & 0 & 21 & 0 & 20 & 0 & 7 & 0 & 0 & 0 & 1 & 0 & 0 \\
$w_{82}$ & 140 & 0 & 0 & 0 & 60 & 0 & 0 & 0 & 36 & 0 & 0 & 0 & 4 & 0 & 0 & 0 & 0 \\
$w_{83}$ & 64 & 0 & 96 & 0 & 30 & 0 & 24 & 0 & 16 & 0 & 8 & 0 & 2 & 0 & 0 & 0 & 0 \\
$w_{84}$ & 111 & 0 & 0 & 0 & 96 & 0 & 0 & 0 & 32 & 0 & 0 & 0 & 0 & 0 & 0 & 0 & 1 \\
$w_{85}$ & 48 & 0 & 99 & 0 & 48 & 0 & 21 & 0 & 16 & 0 & 7 & 0 & 0 & 0 & 1 & 0 & 0 \\
$w_{86}$ & 116 & 0 & 0 & 0 & 92 & 0 & 0 & 0 & 28 & 0 & 0 & 0 & 4 & 0 & 0 & 0 & 0 \\
$w_{87}$ & 52 & 0 & 96 & 0 & 46 & 0 & 24 & 0 & 12 & 0 & 8 & 0 & 2 & 0 & 0 & 0 & 0 \\
$w_{88}$ & 99 & 0 & 0 & 0 & 112 & 0 & 0 & 0 & 28 & 0 & 0 & 0 & 0 & 0 & 0 & 0 & 1 \\
$w_{89}$ & 42 & 0 & 99 & 0 & 56 & 0 & 21 & 0 & 14 & 0 & 7 & 0 & 0 & 0 & 1 & 0 & 0 \\
$w_{90}$ & 104 & 0 & 0 & 0 & 108 & 0 & 0 & 0 & 24 & 0 & 0 & 0 & 4 & 0 & 0 & 0 & 0 \\
$w_{91}$ & 46 & 0 & 96 & 0 & 54 & 0 & 24 & 0 & 10 & 0 & 8 & 0 & 2 & 0 & 0 & 0 & 0 \\
$w_{92}$ & 105 & 0 & 80 & 0 & 0 & 0 & 48 & 0 & 0 & 0 & 0 & 0 & 0 & 0 & 0 & 0 & 7 \\
$w_{93}$ & 69 & 0 & 100 & 0 & 32 & 0 & 24 & 0 & 8 & 0 & 0 & 0 & 0 & 0 & 4 & 0 & 3 \\
$w_{94}$ & 77 & 0 & 88 & 0 & 28 & 0 & 36 & 0 & 0 & 0 & 4 & 0 & 4 & 0 & 0 & 0 & 3 \\
$w_{95}$ & 55 & 0 & 104 & 0 & 46 & 0 & 18 & 0 & 8 & 0 & 2 & 0 & 2 & 0 & 4 & 0 & 1 \\
$w_{96}$ & 59 & 0 & 98 & 0 & 44 & 0 & 24 & 0 & 4 & 0 & 4 & 0 & 4 & 0 & 2 & 0 & 1 \\
$w_{97}$ & 63 & 0 & 92 & 0 & 42 & 0 & 30 & 0 & 0 & 0 & 6 & 0 & 6 & 0 & 0 & 0 & 1 \\
$w_{98}$ & 50 & 0 & 103 & 0 & 52 & 0 & 18 & 0 & 6 & 0 & 4 & 0 & 4 & 0 & 3 & 0 & 0 \\
$w_{99}$ & 77 & 0 & 80 & 0 & 28 & 0 & 48 & 0 & 0 & 0 & 0 & 0 & 4 & 0 & 0 & 0 & 3 \\
$w_{100}$ & 55 & 0 & 100 & 0 & 46 & 0 & 24 & 0 & 8 & 0 & 0 & 0 & 2 & 0 & 4 & 0 & 1 \\
$w_{101}$ & 69 & 0 & 88 & 0 & 32 & 0 & 36 & 0 & 8 & 0 & 4 & 0 & 0 & 0 & 0 & 0 & 3 \\
$w_{102}$ & 55 & 0 & 98 & 0 & 46 & 0 & 24 & 0 & 8 & 0 & 4 & 0 & 2 & 0 & 2 & 0 & 1 \\
$w_{103}$ & 55 & 0 & 94 & 0 & 46 & 0 & 30 & 0 & 8 & 0 & 2 & 0 & 2 & 0 & 2 & 0 & 1 \\
$w_{104}$ & 63 & 0 & 84 & 0 & 42 & 0 & 42 & 0 & 0 & 0 & 2 & 0 & 6 & 0 & 0 & 0 & 1 \\
$w_{105}$ & 111 & 0 & 0 & 0 & 106 & 0 & 0 & 0 & 16 & 0 & 0 & 0 & 6 & 0 & 0 & 0 & 1 \\
$w_{106}$ & 48 & 0 & 99 & 0 & 53 & 0 & 24 & 0 & 8 & 0 & 2 & 0 & 3 & 0 & 3 & 0 & 0 \\
$w_{107}$ & 93 & 0 & 88 & 0 & 0 & 0 & 36 & 0 & 16 & 0 & 4 & 0 & 0 & 0 & 0 & 0 & 3 \\
$w_{108}$ & 67 & 0 & 98 & 0 & 30 & 0 & 24 & 0 & 12 & 0 & 4 & 0 & 2 & 0 & 2 & 0 & 1 \\
$w_{109}$ & 63 & 0 & 98 & 0 & 32 & 0 & 24 & 0 & 16 & 0 & 4 & 0 & 0 & 0 & 2 & 0 & 1 \\
$w_{110}$ & 123 & 0 & 0 & 0 & 90 & 0 & 0 & 0 & 20 & 0 & 0 & 0 & 6 & 0 & 0 & 0 & 1 \\
$w_{111}$ & 71 & 0 & 88 & 0 & 28 & 0 & 36 & 0 & 8 & 0 & 4 & 0 & 4 & 0 & 0 & 0 & 1 \\
$w_{112}$ & 59 & 0 & 88 & 0 & 44 & 0 & 36 & 0 & 4 & 0 & 4 & 0 & 4 & 0 & 0 & 0 & 1 \\
$w_{113}$ & 54 & 0 & 98 & 0 & 45 & 0 & 24 & 0 & 10 & 0 & 4 & 0 & 3 & 0 & 2 & 0 & 0 \\
$w_{114}$ & 55 & 0 & 92 & 0 & 46 & 0 & 30 & 0 & 8 & 0 & 6 & 0 & 2 & 0 & 0 & 0 & 1 \\
$w_{115}$ & 52 & 0 & 98 & 0 & 46 & 0 & 24 & 0 & 12 & 0 & 4 & 0 & 2 & 0 & 2 & 0 & 0 \\
$w_{116}$ & 50 & 0 & 93 & 0 & 52 & 0 & 30 & 0 & 6 & 0 & 4 & 0 & 4 & 0 & 1 & 0 & 0 \\
$w_{117}$ & 48 & 0 & 97 & 0 & 53 & 0 & 24 & 0 & 8 & 0 & 6 & 0 & 3 & 0 & 1 & 0 & 0 \\
$w_{118}$ & 110 & 0 & 0 & 0 & 105 & 0 & 0 & 0 & 18 & 0 & 0 & 0 & 7 & 0 & 0 & 0 & 0 \\
$w_{119}$ & 60 & 0 & 88 & 0 & 42 & 0 & 36 & 0 & 4 & 0 & 4 & 0 & 6 & 0 & 0 & 0 & 0 \\
$w_{120}$ & 54 & 0 & 97 & 0 & 50 & 0 & 24 & 0 & 2 & 0 & 6 & 0 & 6 & 0 & 1 & 0 & 0 \\
$w_{121}$ & 110 & 0 & 0 & 0 & 110 & 0 & 0 & 0 & 10 & 0 & 0 & 0 & 10 & 0 & 0 & 0 & 0 \\
$w_{122}$ & 63 & 0 & 88 & 0 & 42 & 0 & 36 & 0 & 0 & 0 & 4 & 0 & 6 & 0 & 0 & 0 & 1 \\
$w_{123}$ & 48 & 0 & 101 & 0 & 53 & 0 & 21 & 0 & 8 & 0 & 3 & 0 & 3 & 0 & 3 & 0 & 0 \\
$w_{124}$ & 51 & 0 & 98 & 0 & 48 & 0 & 24 & 0 & 12 & 0 & 4 & 0 & 0 & 0 & 2 & 0 & 1 \\
$w_{125}$ & 48 & 0 & 98 & 0 & 53 & 0 & 24 & 0 & 8 & 0 & 4 & 0 & 3 & 0 & 2 & 0 & 0 \\
$w_{126}$ & 59 & 0 & 92 & 0 & 44 & 0 & 30 & 0 & 4 & 0 & 6 & 0 & 4 & 0 & 0 & 0 & 1 \\
$w_{127}$ & 46 & 0 & 100 & 0 & 54 & 0 & 21 & 0 & 10 & 0 & 5 & 0 & 2 & 0 & 2 & 0 & 0 \\
$w_{128}$ & 48 & 0 & 100 & 0 & 53 & 0 & 21 & 0 & 8 & 0 & 5 & 0 & 3 & 0 & 2 & 0 & 0 \\
$w_{129}$ & 46 & 0 & 98 & 0 & 54 & 0 & 24 & 0 & 10 & 0 & 4 & 0 & 2 & 0 & 2 & 0 & 0 \\
$w_{130}$ & 50 & 0 & 95 & 0 & 52 & 0 & 27 & 0 & 6 & 0 & 5 & 0 & 4 & 0 & 1 & 0 & 0 \\
$w_{131}$ & 48 & 0 & 99 & 0 & 53 & 0 & 21 & 0 & 8 & 0 & 7 & 0 & 3 & 0 & 1 & 0 & 0 \\
$w_{132}$ & 104 & 0 & 0 & 0 & 113 & 0 & 0 & 0 & 16 & 0 & 0 & 0 & 7 & 0 & 0 & 0 & 0 \\
$w_{133}$ & 54 & 0 & 90 & 0 & 50 & 0 & 33 & 0 & 2 & 0 & 5 & 0 & 6 & 0 & 0 & 0 & 0 \\
$w_{134}$ & 50 & 0 & 97 & 0 & 52 & 0 & 24 & 0 & 6 & 0 & 6 & 0 & 4 & 0 & 1 & 0 & 0 \\
$w_{135}$ & 46 & 0 & 97 & 0 & 54 & 0 & 24 & 0 & 10 & 0 & 6 & 0 & 2 & 0 & 1 & 0 & 0 \\
$w_{136}$ & 102 & 0 & 0 & 0 & 114 & 0 & 0 & 0 & 18 & 0 & 0 & 0 & 6 & 0 & 0 & 0 & 0 \\
$w_{137}$ & 50 & 0 & 94 & 0 & 52 & 0 & 27 & 0 & 6 & 0 & 7 & 0 & 4 & 0 & 0 & 0 & 0 \\
$w_{138}$ & 54 & 0 & 103 & 0 & 45 & 0 & 18 & 0 & 10 & 0 & 4 & 0 & 3 & 0 & 3 & 0 & 0 \\
$w_{139}$ & 67 & 0 & 92 & 0 & 30 & 0 & 30 & 0 & 12 & 0 & 6 & 0 & 2 & 0 & 0 & 0 & 1 \\
$w_{140}$ & 52 & 0 & 100 & 0 & 46 & 0 & 21 & 0 & 12 & 0 & 5 & 0 & 2 & 0 & 2 & 0 & 0 \\
$w_{141}$ & 54 & 0 & 97 & 0 & 45 & 0 & 24 & 0 & 10 & 0 & 6 & 0 & 3 & 0 & 1 & 0 & 0 \\
$w_{142}$ & 54 & 0 & 102 & 0 & 45 & 0 & 18 & 0 & 10 & 0 & 6 & 0 & 3 & 0 & 2 & 0 & 0 \\
$w_{143}$ & 56 & 0 & 95 & 0 & 44 & 0 & 27 & 0 & 8 & 0 & 5 & 0 & 4 & 0 & 1 & 0 & 0 \\
$w_{144}$ & 54 & 0 & 92 & 0 & 50 & 0 & 30 & 0 & 2 & 0 & 6 & 0 & 6 & 0 & 0 & 0 & 0 \\
$w_{145}$ & 52 & 0 & 97 & 0 & 46 & 0 & 24 & 0 & 12 & 0 & 6 & 0 & 2 & 0 & 1 & 0 & 0 \\
$w_{146}$ & 56 & 0 & 92 & 0 & 44 & 0 & 30 & 0 & 8 & 0 & 6 & 0 & 4 & 0 & 0 & 0 & 0 \\
$w_{147}$ & 54 & 0 & 96 & 0 & 45 & 0 & 24 & 0 & 10 & 0 & 8 & 0 & 3 & 0 & 0 & 0 & 0 \\
$w_{148}$ & 50 & 0 & 92 & 0 & 52 & 0 & 30 & 0 & 6 & 0 & 6 & 0 & 4 & 0 & 0 & 0 & 0 \\
$w_{149}$ & 48 & 0 & 96 & 0 & 53 & 0 & 24 & 0 & 8 & 0 & 8 & 0 & 3 & 0 & 0 & 0 & 0 \\
$w_{150}$ & 93 & 0 & 80 & 0 & 0 & 0 & 48 & 0 & 16 & 0 & 0 & 0 & 0 & 0 & 0 & 0 & 3 \\
$w_{151}$ & 63 & 0 & 94 & 0 & 32 & 0 & 30 & 0 & 16 & 0 & 2 & 0 & 0 & 0 & 2 & 0 & 1 \\
$w_{152}$ & 71 & 0 & 80 & 0 & 28 & 0 & 48 & 0 & 8 & 0 & 0 & 0 & 4 & 0 & 0 & 0 & 1 \\
$w_{153}$ & 55 & 0 & 88 & 0 & 46 & 0 & 36 & 0 & 8 & 0 & 4 & 0 & 2 & 0 & 0 & 0 & 1 \\
$w_{154}$ & 52 & 0 & 94 & 0 & 46 & 0 & 30 & 0 & 12 & 0 & 2 & 0 & 2 & 0 & 2 & 0 & 0 \\
$w_{155}$ & 48 & 0 & 95 & 0 & 53 & 0 & 27 & 0 & 8 & 0 & 5 & 0 & 3 & 0 & 1 & 0 & 0 \\
$w_{156}$ & 69 & 0 & 80 & 0 & 32 & 0 & 48 & 0 & 8 & 0 & 0 & 0 & 0 & 0 & 0 & 0 & 3 \\
$w_{157}$ & 51 & 0 & 94 & 0 & 48 & 0 & 30 & 0 & 12 & 0 & 2 & 0 & 0 & 0 & 2 & 0 & 1 \\
$w_{158}$ & 59 & 0 & 80 & 0 & 44 & 0 & 48 & 0 & 4 & 0 & 0 & 0 & 4 & 0 & 0 & 0 & 1 \\
$w_{159}$ & 46 & 0 & 94 & 0 & 54 & 0 & 30 & 0 & 10 & 0 & 2 & 0 & 2 & 0 & 2 & 0 & 0 \\
$w_{160}$ & 59 & 0 & 84 & 0 & 44 & 0 & 42 & 0 & 4 & 0 & 2 & 0 & 4 & 0 & 0 & 0 & 1 \\
$w_{161}$ & 46 & 0 & 96 & 0 & 54 & 0 & 27 & 0 & 10 & 0 & 3 & 0 & 2 & 0 & 2 & 0 & 0 \\
$w_{162}$ & 48 & 0 & 93 & 0 & 53 & 0 & 30 & 0 & 8 & 0 & 4 & 0 & 3 & 0 & 1 & 0 & 0 \\
$w_{163}$ & 54 & 0 & 86 & 0 & 50 & 0 & 39 & 0 & 2 & 0 & 3 & 0 & 6 & 0 & 0 & 0 & 0 \\
$w_{164}$ & 55 & 0 & 84 & 0 & 46 & 0 & 42 & 0 & 8 & 0 & 2 & 0 & 2 & 0 & 0 & 0 & 1 \\
$w_{165}$ & 103 & 0 & 0 & 0 & 110 & 0 & 0 & 0 & 24 & 0 & 0 & 0 & 2 & 0 & 0 & 0 & 1 \\
$w_{166}$ & 44 & 0 & 96 & 0 & 55 & 0 & 27 & 0 & 12 & 0 & 3 & 0 & 1 & 0 & 2 & 0 & 0 \\
$w_{167}$ & 44 & 0 & 98 & 0 & 55 & 0 & 24 & 0 & 12 & 0 & 4 & 0 & 1 & 0 & 2 & 0 & 0 \\
$w_{168}$ & 51 & 0 & 92 & 0 & 48 & 0 & 30 & 0 & 12 & 0 & 6 & 0 & 0 & 0 & 0 & 0 & 1 \\
$w_{169}$ & 46 & 0 & 95 & 0 & 54 & 0 & 27 & 0 & 10 & 0 & 5 & 0 & 2 & 0 & 1 & 0 & 0 \\
$w_{170}$ & 44 & 0 & 97 & 0 & 55 & 0 & 24 & 0 & 12 & 0 & 6 & 0 & 1 & 0 & 1 & 0 & 0 \\
$w_{171}$ & 46 & 0 & 93 & 0 & 54 & 0 & 30 & 0 & 10 & 0 & 4 & 0 & 2 & 0 & 1 & 0 & 0 \\
$w_{172}$ & 44 & 0 & 99 & 0 & 55 & 0 & 21 & 0 & 12 & 0 & 7 & 0 & 1 & 0 & 1 & 0 & 0 \\
$w_{173}$ & 100 & 0 & 0 & 0 & 115 & 0 & 0 & 0 & 20 & 0 & 0 & 0 & 5 & 0 & 0 & 0 & 0 \\
$w_{174}$ & 50 & 0 & 90 & 0 & 52 & 0 & 33 & 0 & 6 & 0 & 5 & 0 & 4 & 0 & 0 & 0 & 0 \\
$w_{175}$ & 48 & 0 & 94 & 0 & 53 & 0 & 27 & 0 & 8 & 0 & 7 & 0 & 3 & 0 & 0 & 0 & 0 \\
$w_{176}$ & 51 & 0 & 88 & 0 & 48 & 0 & 36 & 0 & 12 & 0 & 4 & 0 & 0 & 0 & 0 & 0 & 1 \\
$w_{177}$ & 44 & 0 & 95 & 0 & 55 & 0 & 27 & 0 & 12 & 0 & 5 & 0 & 1 & 0 & 1 & 0 & 0 \\
$w_{178}$ & 44 & 0 & 93 & 0 & 55 & 0 & 30 & 0 & 12 & 0 & 4 & 0 & 1 & 0 & 1 & 0 & 0 \\
$w_{179}$ & 46 & 0 & 91 & 0 & 54 & 0 & 33 & 0 & 10 & 0 & 3 & 0 & 2 & 0 & 1 & 0 & 0 \\
$w_{180}$ & 50 & 0 & 86 & 0 & 52 & 0 & 39 & 0 & 6 & 0 & 3 & 0 & 4 & 0 & 0 & 0 & 0 \\
$w_{181}$ & 98 & 0 & 0 & 0 & 116 & 0 & 0 & 0 & 22 & 0 & 0 & 0 & 4 & 0 & 0 & 0 & 0 \\
$w_{182}$ & 48 & 0 & 90 & 0 & 53 & 0 & 33 & 0 & 8 & 0 & 5 & 0 & 3 & 0 & 0 & 0 & 0 \\
$w_{183}$ & 48 & 0 & 92 & 0 & 53 & 0 & 30 & 0 & 8 & 0 & 6 & 0 & 3 & 0 & 0 & 0 & 0 \\
$w_{184}$ & 115 & 0 & 0 & 0 & 94 & 0 & 0 & 0 & 28 & 0 & 0 & 0 & 2 & 0 & 0 & 0 & 1 \\
$w_{185}$ & 50 & 0 & 100 & 0 & 47 & 0 & 21 & 0 & 14 & 0 & 5 & 0 & 1 & 0 & 2 & 0 & 0 \\
$w_{186}$ & 63 & 0 & 92 & 0 & 32 & 0 & 30 & 0 & 16 & 0 & 6 & 0 & 0 & 0 & 0 & 0 & 1 \\
$w_{187}$ & 50 & 0 & 97 & 0 & 47 & 0 & 24 & 0 & 14 & 0 & 6 & 0 & 1 & 0 & 1 & 0 & 0 \\
$w_{188}$ & 54 & 0 & 99 & 0 & 45 & 0 & 21 & 0 & 10 & 0 & 7 & 0 & 3 & 0 & 1 & 0 & 0 \\
$w_{189}$ & 52 & 0 & 95 & 0 & 46 & 0 & 27 & 0 & 12 & 0 & 5 & 0 & 2 & 0 & 1 & 0 & 0 \\
$w_{190}$ & 106 & 0 & 0 & 0 & 107 & 0 & 0 & 0 & 22 & 0 & 0 & 0 & 5 & 0 & 0 & 0 & 0 \\
$w_{191}$ & 67 & 0 & 84 & 0 & 30 & 0 & 42 & 0 & 12 & 0 & 2 & 0 & 2 & 0 & 0 & 0 & 1 \\
$w_{192}$ & 50 & 0 & 96 & 0 & 47 & 0 & 27 & 0 & 14 & 0 & 3 & 0 & 1 & 0 & 2 & 0 & 0 \\
$w_{193}$ & 63 & 0 & 88 & 0 & 32 & 0 & 36 & 0 & 16 & 0 & 4 & 0 & 0 & 0 & 0 & 0 & 1 \\
$w_{194}$ & 50 & 0 & 93 & 0 & 47 & 0 & 30 & 0 & 14 & 0 & 4 & 0 & 1 & 0 & 1 & 0 & 0 \\
$w_{195}$ & 52 & 0 & 91 & 0 & 46 & 0 & 33 & 0 & 12 & 0 & 3 & 0 & 2 & 0 & 1 & 0 & 0 \\
$w_{196}$ & 56 & 0 & 84 & 0 & 44 & 0 & 42 & 0 & 8 & 0 & 2 & 0 & 4 & 0 & 0 & 0 & 0 \\
$w_{197}$ & 67 & 0 & 88 & 0 & 30 & 0 & 36 & 0 & 12 & 0 & 4 & 0 & 2 & 0 & 0 & 0 & 1 \\
$w_{198}$ & 52 & 0 & 93 & 0 & 46 & 0 & 30 & 0 & 12 & 0 & 4 & 0 & 2 & 0 & 1 & 0 & 0 \\
$w_{199}$ & 54 & 0 & 88 & 0 & 45 & 0 & 36 & 0 & 10 & 0 & 4 & 0 & 3 & 0 & 0 & 0 & 0 \\
$w_{200}$ & 52 & 0 & 92 & 0 & 46 & 0 & 30 & 0 & 12 & 0 & 6 & 0 & 2 & 0 & 0 & 0 & 0 \\
$w_{201}$ & 48 & 0 & 88 & 0 & 53 & 0 & 36 & 0 & 8 & 0 & 4 & 0 & 3 & 0 & 0 & 0 & 0 \\
$w_{202}$ & 46 & 0 & 92 & 0 & 54 & 0 & 30 & 0 & 10 & 0 & 6 & 0 & 2 & 0 & 0 & 0 & 0 \\
$w_{203}$ & 46 & 0 & 94 & 0 & 54 & 0 & 27 & 0 & 10 & 0 & 7 & 0 & 2 & 0 & 0 & 0 & 0 \\
$w_{204}$ & 54 & 0 & 95 & 0 & 45 & 0 & 27 & 0 & 10 & 0 & 5 & 0 & 3 & 0 & 1 & 0 & 0 \\
$w_{205}$ & 56 & 0 & 88 & 0 & 44 & 0 & 36 & 0 & 8 & 0 & 4 & 0 & 4 & 0 & 0 & 0 & 0 \\
$w_{206}$ & 54 & 0 & 92 & 0 & 45 & 0 & 30 & 0 & 10 & 0 & 6 & 0 & 3 & 0 & 0 & 0 & 0 \\
$w_{207}$ & 139 & 0 & 0 & 0 & 62 & 0 & 0 & 0 & 36 & 0 & 0 & 0 & 2 & 0 & 0 & 0 & 1 \\
$w_{208}$ & 62 & 0 & 102 & 0 & 31 & 0 & 18 & 0 & 18 & 0 & 6 & 0 & 1 & 0 & 2 & 0 & 0 \\
$w_{209}$ & 62 & 0 & 99 & 0 & 31 & 0 & 21 & 0 & 18 & 0 & 7 & 0 & 1 & 0 & 1 & 0 & 0 \\
$w_{210}$ & 68 & 0 & 99 & 0 & 28 & 0 & 21 & 0 & 12 & 0 & 7 & 0 & 4 & 0 & 1 & 0 & 0 \\
$w_{211}$ & 87 & 0 & 88 & 0 & 0 & 0 & 36 & 0 & 24 & 0 & 4 & 0 & 0 & 0 & 0 & 0 & 1 \\
$w_{212}$ & 62 & 0 & 95 & 0 & 31 & 0 & 27 & 0 & 18 & 0 & 5 & 0 & 1 & 0 & 1 & 0 & 0 \\
$w_{213}$ & 68 & 0 & 88 & 0 & 28 & 0 & 36 & 0 & 12 & 0 & 4 & 0 & 4 & 0 & 0 & 0 & 0 \\
$w_{214}$ & 50 & 0 & 98 & 0 & 47 & 0 & 24 & 0 & 14 & 0 & 4 & 0 & 1 & 0 & 2 & 0 & 0 \\
$w_{215}$ & 50 & 0 & 95 & 0 & 47 & 0 & 27 & 0 & 14 & 0 & 5 & 0 & 1 & 0 & 1 & 0 & 0 \\
$w_{216}$ & 50 & 0 & 99 & 0 & 47 & 0 & 21 & 0 & 14 & 0 & 7 & 0 & 1 & 0 & 1 & 0 & 0 \\
$w_{217}$ & 64 & 0 & 95 & 0 & 30 & 0 & 27 & 0 & 16 & 0 & 5 & 0 & 2 & 0 & 1 & 0 & 0 \\
$w_{218}$ & 54 & 0 & 94 & 0 & 45 & 0 & 27 & 0 & 10 & 0 & 7 & 0 & 3 & 0 & 0 & 0 & 0 \\
$w_{219}$ & 64 & 0 & 92 & 0 & 30 & 0 & 30 & 0 & 16 & 0 & 6 & 0 & 2 & 0 & 0 & 0 & 0 \\
$w_{220}$ & 52 & 0 & 94 & 0 & 46 & 0 & 27 & 0 & 12 & 0 & 7 & 0 & 2 & 0 & 0 & 0 & 0 \\
$w_{221}$ & 60 & 0 & 95 & 0 & 32 & 0 & 27 & 0 & 20 & 0 & 5 & 0 & 0 & 0 & 1 & 0 & 0 \\
$w_{222}$ & 48 & 0 & 95 & 0 & 48 & 0 & 27 & 0 & 16 & 0 & 5 & 0 & 0 & 0 & 1 & 0 & 0 \\
$w_{223}$ & 102 & 0 & 0 & 0 & 109 & 0 & 0 & 0 & 26 & 0 & 0 & 0 & 3 & 0 & 0 & 0 & 0 \\
$w_{224}$ & 52 & 0 & 88 & 0 & 46 & 0 & 36 & 0 & 12 & 0 & 4 & 0 & 2 & 0 & 0 & 0 & 0 \\
$w_{225}$ & 46 & 0 & 90 & 0 & 54 & 0 & 33 & 0 & 10 & 0 & 5 & 0 & 2 & 0 & 0 & 0 & 0 \\
$w_{226}$ & 44 & 0 & 94 & 0 & 55 & 0 & 27 & 0 & 12 & 0 & 7 & 0 & 1 & 0 & 0 & 0 & 0 \\
$w_{227}$ & 50 & 0 & 88 & 0 & 52 & 0 & 36 & 0 & 6 & 0 & 4 & 0 & 4 & 0 & 0 & 0 & 0 \\
$w_{228}$ & 114 & 0 & 0 & 0 & 93 & 0 & 0 & 0 & 30 & 0 & 0 & 0 & 3 & 0 & 0 & 0 & 0 \\
$w_{229}$ & 64 & 0 & 88 & 0 & 30 & 0 & 36 & 0 & 16 & 0 & 4 & 0 & 2 & 0 & 0 & 0 & 0 \\
$w_{230}$ & 52 & 0 & 90 & 0 & 46 & 0 & 33 & 0 & 12 & 0 & 5 & 0 & 2 & 0 & 0 & 0 & 0 \\
$w_{231}$ & 50 & 0 & 94 & 0 & 47 & 0 & 27 & 0 & 14 & 0 & 7 & 0 & 1 & 0 & 0 & 0 & 0 \\
$w_{232}$ & 51 & 0 & 90 & 0 & 48 & 0 & 36 & 0 & 12 & 0 & 0 & 0 & 0 & 0 & 2 & 0 & 1 \\
$w_{233}$ & 46 & 0 & 89 & 0 & 54 & 0 & 36 & 0 & 10 & 0 & 2 & 0 & 2 & 0 & 1 & 0 & 0 \\
$w_{234}$ & 54 & 0 & 90 & 0 & 45 & 0 & 33 & 0 & 10 & 0 & 5 & 0 & 3 & 0 & 0 & 0 & 0 \\
$w_{235}$ & 42 & 0 & 95 & 0 & 56 & 0 & 27 & 0 & 14 & 0 & 5 & 0 & 0 & 0 & 1 & 0 & 0 \\
$w_{236}$ & 42 & 0 & 97 & 0 & 56 & 0 & 24 & 0 & 14 & 0 & 6 & 0 & 0 & 0 & 1 & 0 & 0 \\
$w_{237}$ & 96 & 0 & 0 & 0 & 117 & 0 & 0 & 0 & 24 & 0 & 0 & 0 & 3 & 0 & 0 & 0 & 0 \\
$w_{238}$ & 44 & 0 & 96 & 0 & 55 & 0 & 24 & 0 & 12 & 0 & 8 & 0 & 1 & 0 & 0 & 0 & 0 \\
$w_{239}$ & 50 & 0 & 92 & 0 & 47 & 0 & 30 & 0 & 14 & 0 & 6 & 0 & 1 & 0 & 0 & 0 & 0 \\
$w_{240}$ & 42 & 0 & 93 & 0 & 56 & 0 & 30 & 0 & 14 & 0 & 4 & 0 & 0 & 0 & 1 & 0 & 0 \\
$w_{241}$ & 44 & 0 & 92 & 0 & 55 & 0 & 30 & 0 & 12 & 0 & 6 & 0 & 1 & 0 & 0 & 0 & 0 \\
$w_{242}$ & 50 & 0 & 96 & 0 & 47 & 0 & 24 & 0 & 14 & 0 & 8 & 0 & 1 & 0 & 0 & 0 & 0 \\
$w_{243}$ & 48 & 0 & 96 & 0 & 48 & 0 & 24 & 0 & 16 & 0 & 8 & 0 & 0 & 0 & 0 & 0 & 0 \\
$w_{244}$ & 42 & 0 & 96 & 0 & 56 & 0 & 24 & 0 & 14 & 0 & 8 & 0 & 0 & 0 & 0 & 0 & 0 \\
$w_{245}$ & 62 & 0 & 92 & 0 & 31 & 0 & 30 & 0 & 18 & 0 & 6 & 0 & 1 & 0 & 0 & 0 & 0 \\
$w_{246}$ & 60 & 0 & 96 & 0 & 32 & 0 & 24 & 0 & 20 & 0 & 8 & 0 & 0 & 0 & 0 & 0 & 0 \\
$w_{247}$ & 87 & 0 & 80 & 0 & 0 & 0 & 48 & 0 & 24 & 0 & 0 & 0 & 0 & 0 & 0 & 0 & 1 \\
$w_{248}$ & 60 & 0 & 91 & 0 & 32 & 0 & 33 & 0 & 20 & 0 & 3 & 0 & 0 & 0 & 1 & 0 & 0 \\
$w_{249}$ & 63 & 0 & 80 & 0 & 32 & 0 & 48 & 0 & 16 & 0 & 0 & 0 & 0 & 0 & 0 & 0 & 1 \\
$w_{250}$ & 48 & 0 & 91 & 0 & 48 & 0 & 33 & 0 & 16 & 0 & 3 & 0 & 0 & 0 & 1 & 0 & 0 \\
$w_{251}$ & 100 & 0 & 0 & 0 & 110 & 0 & 0 & 0 & 28 & 0 & 0 & 0 & 2 & 0 & 0 & 0 & 0 \\
$w_{252}$ & 112 & 0 & 0 & 0 & 94 & 0 & 0 & 0 & 32 & 0 & 0 & 0 & 2 & 0 & 0 & 0 & 0 \\
$w_{253}$ & 64 & 0 & 84 & 0 & 30 & 0 & 42 & 0 & 16 & 0 & 2 & 0 & 2 & 0 & 0 & 0 & 0 \\
$w_{254}$ & 50 & 0 & 90 & 0 & 47 & 0 & 33 & 0 & 14 & 0 & 5 & 0 & 1 & 0 & 0 & 0 & 0 \\
$w_{255}$ & 51 & 0 & 84 & 0 & 48 & 0 & 42 & 0 & 12 & 0 & 2 & 0 & 0 & 0 & 0 & 0 & 1 \\
$w_{256}$ & 51 & 0 & 80 & 0 & 48 & 0 & 48 & 0 & 12 & 0 & 0 & 0 & 0 & 0 & 0 & 0 & 1 \\
$w_{257}$ & 42 & 0 & 91 & 0 & 56 & 0 & 33 & 0 & 14 & 0 & 3 & 0 & 0 & 0 & 1 & 0 & 0 \\
$w_{258}$ & 46 & 0 & 86 & 0 & 54 & 0 & 39 & 0 & 10 & 0 & 3 & 0 & 2 & 0 & 0 & 0 & 0 \\
$w_{259}$ & 94 & 0 & 0 & 0 & 118 & 0 & 0 & 0 & 26 & 0 & 0 & 0 & 2 & 0 & 0 & 0 & 0 \\
$w_{260}$ & 44 & 0 & 90 & 0 & 55 & 0 & 33 & 0 & 12 & 0 & 5 & 0 & 1 & 0 & 0 & 0 & 0 \\
$w_{261}$ & 50 & 0 & 82 & 0 & 52 & 0 & 45 & 0 & 6 & 0 & 1 & 0 & 4 & 0 & 0 & 0 & 0 \\
$w_{262}$ & 46 & 0 & 88 & 0 & 54 & 0 & 36 & 0 & 10 & 0 & 4 & 0 & 2 & 0 & 0 & 0 & 0 \\
$w_{263}$ & 52 & 0 & 84 & 0 & 46 & 0 & 42 & 0 & 12 & 0 & 2 & 0 & 2 & 0 & 0 & 0 & 0 \\
$w_{264}$ & 50 & 0 & 88 & 0 & 47 & 0 & 36 & 0 & 14 & 0 & 4 & 0 & 1 & 0 & 0 & 0 & 0 \\
$w_{265}$ & 48 & 0 & 92 & 0 & 48 & 0 & 30 & 0 & 16 & 0 & 6 & 0 & 0 & 0 & 0 & 0 & 0 \\
$w_{266}$ & 44 & 0 & 88 & 0 & 55 & 0 & 36 & 0 & 12 & 0 & 4 & 0 & 1 & 0 & 0 & 0 & 0 \\
$w_{267}$ & 42 & 0 & 92 & 0 & 56 & 0 & 30 & 0 & 14 & 0 & 6 & 0 & 0 & 0 & 0 & 0 & 0 \\
$w_{268}$ & 42 & 0 & 94 & 0 & 56 & 0 & 27 & 0 & 14 & 0 & 7 & 0 & 0 & 0 & 0 & 0 & 0 \\
$w_{269}$ & 62 & 0 & 96 & 0 & 31 & 0 & 24 & 0 & 18 & 0 & 8 & 0 & 1 & 0 & 0 & 0 & 0 \\
$w_{270}$ & 60 & 0 & 92 & 0 & 32 & 0 & 30 & 0 & 20 & 0 & 6 & 0 & 0 & 0 & 0 & 0 & 0 \\
$w_{271}$ & 62 & 0 & 88 & 0 & 31 & 0 & 36 & 0 & 18 & 0 & 4 & 0 & 1 & 0 & 0 & 0 & 0 \\
$w_{272}$ & 134 & 0 & 0 & 0 & 63 & 0 & 0 & 0 & 42 & 0 & 0 & 0 & 1 & 0 & 0 & 0 & 0 \\
$w_{273}$ & 84 & 0 & 88 & 0 & 0 & 0 & 36 & 0 & 28 & 0 & 4 & 0 & 0 & 0 & 0 & 0 & 0 \\
$w_{274}$ & 110 & 0 & 0 & 0 & 95 & 0 & 0 & 0 & 34 & 0 & 0 & 0 & 1 & 0 & 0 & 0 & 0 \\
$w_{275}$ & 60 & 0 & 88 & 0 & 32 & 0 & 36 & 0 & 20 & 0 & 4 & 0 & 0 & 0 & 0 & 0 & 0 \\
$w_{276}$ & 48 & 0 & 87 & 0 & 48 & 0 & 39 & 0 & 16 & 0 & 1 & 0 & 0 & 0 & 1 & 0 & 0 \\
$w_{277}$ & 98 & 0 & 0 & 0 & 111 & 0 & 0 & 0 & 30 & 0 & 0 & 0 & 1 & 0 & 0 & 0 & 0 \\
$w_{278}$ & 48 & 0 & 88 & 0 & 48 & 0 & 36 & 0 & 16 & 0 & 4 & 0 & 0 & 0 & 0 & 0 & 0 \\
$w_{279}$ & 42 & 0 & 85 & 0 & 56 & 0 & 42 & 0 & 14 & 0 & 0 & 0 & 0 & 0 & 1 & 0 & 0 \\
$w_{280}$ & 92 & 0 & 0 & 0 & 119 & 0 & 0 & 0 & 28 & 0 & 0 & 0 & 1 & 0 & 0 & 0 & 0 \\
$w_{281}$ & 42 & 0 & 90 & 0 & 56 & 0 & 33 & 0 & 14 & 0 & 5 & 0 & 0 & 0 & 0 & 0 & 0 \\
$w_{282}$ & 148 & 0 & 0 & 0 & 56 & 0 & 0 & 0 & 28 & 0 & 0 & 0 & 8 & 0 & 0 & 0 & 0 \\
$w_{283}$ & 136 & 0 & 0 & 0 & 62 & 0 & 0 & 0 & 40 & 0 & 0 & 0 & 2 & 0 & 0 & 0 & 0 \\
$w_{284}$ & 68 & 0 & 80 & 0 & 28 & 0 & 48 & 0 & 12 & 0 & 0 & 0 & 4 & 0 & 0 & 0 & 0 \\
$w_{285}$ & 48 & 0 & 97 & 0 & 48 & 0 & 24 & 0 & 16 & 0 & 6 & 0 & 0 & 0 & 1 & 0 & 0 \\
$w_{286}$ & 48 & 0 & 94 & 0 & 48 & 0 & 27 & 0 & 16 & 0 & 7 & 0 & 0 & 0 & 0 & 0 & 0 \\
$w_{287}$ & 105 & 0 & 0 & 0 & 128 & 0 & 0 & 0 & 0 & 0 & 0 & 0 & 0 & 0 & 0 & 0 & 7 \\
$w_{288}$ & 45 & 0 & 100 & 0 & 64 & 0 & 24 & 0 & 0 & 0 & 0 & 0 & 0 & 0 & 4 & 0 & 3 \\
$w_{289}$ & 101 & 0 & 0 & 0 & 124 & 0 & 0 & 0 & 8 & 0 & 0 & 0 & 4 & 0 & 0 & 0 & 3 \\
$w_{290}$ & 43 & 0 & 100 & 0 & 62 & 0 & 24 & 0 & 4 & 0 & 0 & 0 & 2 & 0 & 4 & 0 & 1 \\
$w_{291}$ & 45 & 0 & 88 & 0 & 64 & 0 & 36 & 0 & 0 & 0 & 4 & 0 & 0 & 0 & 0 & 0 & 3 \\
$w_{292}$ & 43 & 0 & 94 & 0 & 62 & 0 & 30 & 0 & 4 & 0 & 2 & 0 & 2 & 0 & 2 & 0 & 1 \\
$w_{293}$ & 99 & 0 & 0 & 0 & 122 & 0 & 0 & 0 & 12 & 0 & 0 & 0 & 6 & 0 & 0 & 0 & 1 \\
$w_{294}$ & 63 & 0 & 80 & 0 & 42 & 0 & 48 & 0 & 0 & 0 & 0 & 0 & 6 & 0 & 0 & 0 & 1 \\
$w_{295}$ & 42 & 0 & 97 & 0 & 61 & 0 & 27 & 0 & 6 & 0 & 1 & 0 & 3 & 0 & 3 & 0 & 0 \\
$w_{296}$ & 93 & 0 & 0 & 0 & 128 & 0 & 0 & 0 & 16 & 0 & 0 & 0 & 0 & 0 & 0 & 0 & 3 \\
$w_{297}$ & 39 & 0 & 94 & 0 & 64 & 0 & 30 & 0 & 8 & 0 & 2 & 0 & 0 & 0 & 2 & 0 & 1 \\
$w_{298}$ & 95 & 0 & 0 & 0 & 124 & 0 & 0 & 0 & 16 & 0 & 0 & 0 & 4 & 0 & 0 & 0 & 1 \\
$w_{299}$ & 43 & 0 & 88 & 0 & 62 & 0 & 36 & 0 & 4 & 0 & 4 & 0 & 2 & 0 & 0 & 0 & 1 \\
$w_{300}$ & 40 & 0 & 94 & 0 & 62 & 0 & 30 & 0 & 8 & 0 & 2 & 0 & 2 & 0 & 2 & 0 & 0 \\
$w_{301}$ & 42 & 0 & 91 & 0 & 61 & 0 & 33 & 0 & 6 & 0 & 3 & 0 & 3 & 0 & 1 & 0 & 0 \\
$w_{302}$ & 96 & 0 & 0 & 0 & 122 & 0 & 0 & 0 & 16 & 0 & 0 & 0 & 6 & 0 & 0 & 0 & 0 \\
$w_{303}$ & 60 & 0 & 80 & 0 & 42 & 0 & 48 & 0 & 4 & 0 & 0 & 0 & 6 & 0 & 0 & 0 & 0 \\
$w_{304}$ & 45 & 0 & 96 & 0 & 64 & 0 & 24 & 0 & 0 & 0 & 8 & 0 & 0 & 0 & 0 & 0 & 3 \\
$w_{305}$ & 43 & 0 & 98 & 0 & 62 & 0 & 24 & 0 & 4 & 0 & 4 & 0 & 2 & 0 & 2 & 0 & 1 \\
$w_{306}$ & 39 & 0 & 98 & 0 & 64 & 0 & 24 & 0 & 8 & 0 & 4 & 0 & 0 & 0 & 2 & 0 & 1 \\
$w_{307}$ & 43 & 0 & 96 & 0 & 62 & 0 & 24 & 0 & 4 & 0 & 8 & 0 & 2 & 0 & 0 & 0 & 1 \\
$w_{308}$ & 40 & 0 & 98 & 0 & 62 & 0 & 24 & 0 & 8 & 0 & 4 & 0 & 2 & 0 & 2 & 0 & 0 \\
$w_{309}$ & 42 & 0 & 99 & 0 & 61 & 0 & 24 & 0 & 6 & 0 & 2 & 0 & 3 & 0 & 3 & 0 & 0 \\
$w_{310}$ & 43 & 0 & 92 & 0 & 62 & 0 & 30 & 0 & 4 & 0 & 6 & 0 & 2 & 0 & 0 & 0 & 1 \\
$w_{311}$ & 40 & 0 & 96 & 0 & 62 & 0 & 27 & 0 & 8 & 0 & 3 & 0 & 2 & 0 & 2 & 0 & 0 \\
$w_{312}$ & 42 & 0 & 93 & 0 & 61 & 0 & 30 & 0 & 6 & 0 & 4 & 0 & 3 & 0 & 1 & 0 & 0 \\
$w_{313}$ & 42 & 0 & 95 & 0 & 61 & 0 & 27 & 0 & 6 & 0 & 5 & 0 & 3 & 0 & 1 & 0 & 0 \\
$w_{314}$ & 48 & 0 & 91 & 0 & 53 & 0 & 33 & 0 & 8 & 0 & 3 & 0 & 3 & 0 & 1 & 0 & 0 \\
$w_{315}$ & 54 & 0 & 84 & 0 & 50 & 0 & 42 & 0 & 2 & 0 & 2 & 0 & 6 & 0 & 0 & 0 & 0 \\
$w_{316}$ & 91 & 0 & 0 & 0 & 126 & 0 & 0 & 0 & 20 & 0 & 0 & 0 & 2 & 0 & 0 & 0 & 1 \\
$w_{317}$ & 55 & 0 & 80 & 0 & 46 & 0 & 48 & 0 & 8 & 0 & 0 & 0 & 2 & 0 & 0 & 0 & 1 \\
$w_{318}$ & 38 & 0 & 94 & 0 & 63 & 0 & 30 & 0 & 10 & 0 & 2 & 0 & 1 & 0 & 2 & 0 & 0 \\
$w_{319}$ & 38 & 0 & 96 & 0 & 63 & 0 & 27 & 0 & 10 & 0 & 3 & 0 & 1 & 0 & 2 & 0 & 0 \\
$w_{320}$ & 39 & 0 & 92 & 0 & 64 & 0 & 30 & 0 & 8 & 0 & 6 & 0 & 0 & 0 & 0 & 0 & 1 \\
$w_{321}$ & 40 & 0 & 93 & 0 & 62 & 0 & 30 & 0 & 8 & 0 & 4 & 0 & 2 & 0 & 1 & 0 & 0 \\
$w_{322}$ & 39 & 0 & 88 & 0 & 64 & 0 & 36 & 0 & 8 & 0 & 4 & 0 & 0 & 0 & 0 & 0 & 1 \\
$w_{323}$ & 38 & 0 & 93 & 0 & 63 & 0 & 30 & 0 & 10 & 0 & 4 & 0 & 1 & 0 & 1 & 0 & 0 \\
$w_{324}$ & 40 & 0 & 91 & 0 & 62 & 0 & 33 & 0 & 8 & 0 & 3 & 0 & 2 & 0 & 1 & 0 & 0 \\
$w_{325}$ & 38 & 0 & 95 & 0 & 63 & 0 & 27 & 0 & 10 & 0 & 5 & 0 & 1 & 0 & 1 & 0 & 0 \\
$w_{326}$ & 44 & 0 & 91 & 0 & 55 & 0 & 33 & 0 & 12 & 0 & 3 & 0 & 1 & 0 & 1 & 0 & 0 \\
$w_{327}$ & 50 & 0 & 84 & 0 & 52 & 0 & 42 & 0 & 6 & 0 & 2 & 0 & 4 & 0 & 0 & 0 & 0 \\
$w_{328}$ & 92 & 0 & 0 & 0 & 124 & 0 & 0 & 0 & 20 & 0 & 0 & 0 & 4 & 0 & 0 & 0 & 0 \\
$w_{329}$ & 42 & 0 & 92 & 0 & 61 & 0 & 30 & 0 & 6 & 0 & 6 & 0 & 3 & 0 & 0 & 0 & 0 \\
$w_{330}$ & 67 & 0 & 80 & 0 & 30 & 0 & 48 & 0 & 12 & 0 & 0 & 0 & 2 & 0 & 0 & 0 & 1 \\
$w_{331}$ & 50 & 0 & 94 & 0 & 47 & 0 & 30 & 0 & 14 & 0 & 2 & 0 & 1 & 0 & 2 & 0 & 0 \\
$w_{332}$ & 38 & 0 & 91 & 0 & 63 & 0 & 33 & 0 & 10 & 0 & 3 & 0 & 1 & 0 & 1 & 0 & 0 \\
$w_{333}$ & 50 & 0 & 91 & 0 & 47 & 0 & 33 & 0 & 14 & 0 & 3 & 0 & 1 & 0 & 1 & 0 & 0 \\
$w_{334}$ & 56 & 0 & 80 & 0 & 44 & 0 & 48 & 0 & 8 & 0 & 0 & 0 & 4 & 0 & 0 & 0 & 0 \\
$w_{335}$ & 42 & 0 & 88 & 0 & 61 & 0 & 36 & 0 & 6 & 0 & 4 & 0 & 3 & 0 & 0 & 0 & 0 \\
$w_{336}$ & 39 & 0 & 96 & 0 & 64 & 0 & 24 & 0 & 8 & 0 & 8 & 0 & 0 & 0 & 0 & 0 & 1 \\
$w_{337}$ & 40 & 0 & 96 & 0 & 62 & 0 & 24 & 0 & 8 & 0 & 8 & 0 & 2 & 0 & 0 & 0 & 0 \\
$w_{338}$ & 40 & 0 & 92 & 0 & 62 & 0 & 30 & 0 & 8 & 0 & 6 & 0 & 2 & 0 & 0 & 0 & 0 \\
$w_{339}$ & 40 & 0 & 95 & 0 & 62 & 0 & 27 & 0 & 8 & 0 & 5 & 0 & 2 & 0 & 1 & 0 & 0 \\
$w_{340}$ & 42 & 0 & 90 & 0 & 61 & 0 & 33 & 0 & 6 & 0 & 5 & 0 & 3 & 0 & 0 & 0 & 0 \\
$w_{341}$ & 40 & 0 & 90 & 0 & 62 & 0 & 33 & 0 & 8 & 0 & 5 & 0 & 2 & 0 & 0 & 0 & 0 \\
$w_{342}$ & 87 & 0 & 0 & 0 & 128 & 0 & 0 & 0 & 24 & 0 & 0 & 0 & 0 & 0 & 0 & 0 & 1 \\
$w_{343}$ & 36 & 0 & 91 & 0 & 64 & 0 & 33 & 0 & 12 & 0 & 3 & 0 & 0 & 0 & 1 & 0 & 0 \\
$w_{344}$ & 40 & 0 & 88 & 0 & 62 & 0 & 36 & 0 & 8 & 0 & 4 & 0 & 2 & 0 & 0 & 0 & 0 \\
$w_{345}$ & 88 & 0 & 0 & 0 & 126 & 0 & 0 & 0 & 24 & 0 & 0 & 0 & 2 & 0 & 0 & 0 & 0 \\
$w_{346}$ & 52 & 0 & 80 & 0 & 46 & 0 & 48 & 0 & 12 & 0 & 0 & 0 & 2 & 0 & 0 & 0 & 0 \\
$w_{347}$ & 38 & 0 & 88 & 0 & 63 & 0 & 36 & 0 & 10 & 0 & 4 & 0 & 1 & 0 & 0 & 0 & 0 \\
$w_{348}$ & 64 & 0 & 80 & 0 & 30 & 0 & 48 & 0 & 16 & 0 & 0 & 0 & 2 & 0 & 0 & 0 & 0 \\
$w_{349}$ & 36 & 0 & 95 & 0 & 64 & 0 & 27 & 0 & 12 & 0 & 5 & 0 & 0 & 0 & 1 & 0 & 0 \\
$w_{350}$ & 36 & 0 & 93 & 0 & 64 & 0 & 30 & 0 & 12 & 0 & 4 & 0 & 0 & 0 & 1 & 0 & 0 \\
$w_{351}$ & 46 & 0 & 84 & 0 & 54 & 0 & 42 & 0 & 10 & 0 & 2 & 0 & 2 & 0 & 0 & 0 & 0 \\
$w_{352}$ & 38 & 0 & 92 & 0 & 63 & 0 & 30 & 0 & 10 & 0 & 6 & 0 & 1 & 0 & 0 & 0 & 0 \\
$w_{353}$ & 38 & 0 & 90 & 0 & 63 & 0 & 33 & 0 & 10 & 0 & 5 & 0 & 1 & 0 & 0 & 0 & 0 \\
$w_{354}$ & 36 & 0 & 92 & 0 & 64 & 0 & 30 & 0 & 12 & 0 & 6 & 0 & 0 & 0 & 0 & 0 & 0 \\
$w_{355}$ & 36 & 0 & 90 & 0 & 64 & 0 & 33 & 0 & 12 & 0 & 5 & 0 & 0 & 0 & 0 & 0 & 0 \\
$w_{356}$ & 36 & 0 & 96 & 0 & 64 & 0 & 24 & 0 & 12 & 0 & 8 & 0 & 0 & 0 & 0 & 0 & 0 \\
$w_{357}$ & 42 & 0 & 98 & 0 & 61 & 0 & 24 & 0 & 6 & 0 & 4 & 0 & 3 & 0 & 2 & 0 & 0 \\
$w_{358}$ & 42 & 0 & 96 & 0 & 61 & 0 & 24 & 0 & 6 & 0 & 8 & 0 & 3 & 0 & 0 & 0 & 0 \\
$w_{359}$ & 54 & 0 & 93 & 0 & 45 & 0 & 30 & 0 & 10 & 0 & 4 & 0 & 3 & 0 & 1 & 0 & 0 \\
$w_{360}$ & 42 & 0 & 97 & 0 & 61 & 0 & 24 & 0 & 6 & 0 & 6 & 0 & 3 & 0 & 1 & 0 & 0 \\
$w_{361}$ & 38 & 0 & 97 & 0 & 63 & 0 & 24 & 0 & 10 & 0 & 6 & 0 & 1 & 0 & 1 & 0 & 0 \\
$w_{362}$ & 40 & 0 & 97 & 0 & 62 & 0 & 24 & 0 & 8 & 0 & 6 & 0 & 2 & 0 & 1 & 0 & 0 \\
$w_{363}$ & 94 & 0 & 0 & 0 & 123 & 0 & 0 & 0 & 18 & 0 & 0 & 0 & 5 & 0 & 0 & 0 & 0 \\
$w_{364}$ & 42 & 0 & 94 & 0 & 61 & 0 & 27 & 0 & 6 & 0 & 7 & 0 & 3 & 0 & 0 & 0 & 0 \\
$w_{365}$ & 40 & 0 & 94 & 0 & 62 & 0 & 27 & 0 & 8 & 0 & 7 & 0 & 2 & 0 & 0 & 0 & 0 \\
$w_{366}$ & 98 & 0 & 0 & 0 & 121 & 0 & 0 & 0 & 14 & 0 & 0 & 0 & 7 & 0 & 0 & 0 & 0 \\
$w_{367}$ & 54 & 0 & 88 & 0 & 50 & 0 & 36 & 0 & 2 & 0 & 4 & 0 & 6 & 0 & 0 & 0 & 0 \\
$w_{368}$ & 48 & 0 & 93 & 0 & 48 & 0 & 30 & 0 & 16 & 0 & 4 & 0 & 0 & 0 & 1 & 0 & 0 \\
$w_{369}$ & 36 & 0 & 97 & 0 & 64 & 0 & 24 & 0 & 12 & 0 & 6 & 0 & 0 & 0 & 1 & 0 & 0 \\
$w_{370}$ & 38 & 0 & 96 & 0 & 63 & 0 & 24 & 0 & 10 & 0 & 8 & 0 & 1 & 0 & 0 & 0 & 0 \\
$w_{371}$ & 90 & 0 & 0 & 0 & 125 & 0 & 0 & 0 & 22 & 0 & 0 & 0 & 3 & 0 & 0 & 0 & 0 \\
$w_{372}$ & 38 & 0 & 94 & 0 & 63 & 0 & 27 & 0 & 10 & 0 & 7 & 0 & 1 & 0 & 0 & 0 & 0 \\
$w_{373}$ & 36 & 0 & 94 & 0 & 64 & 0 & 27 & 0 & 12 & 0 & 7 & 0 & 0 & 0 & 0 & 0 & 0 \\
$w_{374}$ & 48 & 0 & 90 & 0 & 48 & 0 & 33 & 0 & 16 & 0 & 5 & 0 & 0 & 0 & 0 & 0 & 0 \\
$w_{375}$ & 132 & 0 & 0 & 0 & 64 & 0 & 0 & 0 & 44 & 0 & 0 & 0 & 0 & 0 & 0 & 0 & 0 \\
$w_{376}$ & 108 & 0 & 0 & 0 & 96 & 0 & 0 & 0 & 36 & 0 & 0 & 0 & 0 & 0 & 0 & 0 & 0 \\
$w_{377}$ & 96 & 0 & 0 & 0 & 112 & 0 & 0 & 0 & 32 & 0 & 0 & 0 & 0 & 0 & 0 & 0 & 0 \\
$w_{378}$ & 86 & 0 & 0 & 0 & 127 & 0 & 0 & 0 & 26 & 0 & 0 & 0 & 1 & 0 & 0 & 0 & 0 \\
$w_{379}$ & 42 & 0 & 88 & 0 & 56 & 0 & 36 & 0 & 14 & 0 & 4 & 0 & 0 & 0 & 0 & 0 & 0 \\
$w_{380}$ & 44 & 0 & 90 & 0 & 55 & 0 & 36 & 0 & 12 & 0 & 0 & 0 & 1 & 0 & 2 & 0 & 0 \\
$w_{381}$ & 44 & 0 & 94 & 0 & 55 & 0 & 30 & 0 & 12 & 0 & 2 & 0 & 1 & 0 & 2 & 0 & 0 \\
$w_{382}$ & 44 & 0 & 89 & 0 & 55 & 0 & 36 & 0 & 12 & 0 & 2 & 0 & 1 & 0 & 1 & 0 & 0 \\
$w_{383}$ & 48 & 0 & 86 & 0 & 53 & 0 & 39 & 0 & 8 & 0 & 3 & 0 & 3 & 0 & 0 & 0 & 0 \\
$w_{384}$ & 63 & 0 & 84 & 0 & 32 & 0 & 42 & 0 & 16 & 0 & 2 & 0 & 0 & 0 & 0 & 0 & 1 \\
$w_{385}$ & 50 & 0 & 89 & 0 & 47 & 0 & 36 & 0 & 14 & 0 & 2 & 0 & 1 & 0 & 1 & 0 & 0 \\
$w_{386}$ & 48 & 0 & 89 & 0 & 48 & 0 & 36 & 0 & 16 & 0 & 2 & 0 & 0 & 0 & 1 & 0 & 0 \\
$w_{387}$ & 54 & 0 & 80 & 0 & 45 & 0 & 48 & 0 & 10 & 0 & 0 & 0 & 3 & 0 & 0 & 0 & 0 \\
$w_{388}$ & 48 & 0 & 84 & 0 & 53 & 0 & 42 & 0 & 8 & 0 & 2 & 0 & 3 & 0 & 0 & 0 & 0 \\
$w_{389}$ & 54 & 0 & 84 & 0 & 45 & 0 & 42 & 0 & 10 & 0 & 2 & 0 & 3 & 0 & 0 & 0 & 0 \\
$w_{390}$ & 54 & 0 & 86 & 0 & 45 & 0 & 39 & 0 & 10 & 0 & 3 & 0 & 3 & 0 & 0 & 0 & 0 \\
$w_{391}$ & 52 & 0 & 86 & 0 & 46 & 0 & 39 & 0 & 12 & 0 & 3 & 0 & 2 & 0 & 0 & 0 & 0 \\
$w_{392}$ & 44 & 0 & 86 & 0 & 55 & 0 & 39 & 0 & 12 & 0 & 3 & 0 & 1 & 0 & 0 & 0 & 0 \\
$w_{393}$ & 50 & 0 & 86 & 0 & 47 & 0 & 39 & 0 & 14 & 0 & 3 & 0 & 1 & 0 & 0 & 0 & 0 \\
$w_{394}$ & 62 & 0 & 84 & 0 & 31 & 0 & 42 & 0 & 18 & 0 & 2 & 0 & 1 & 0 & 0 & 0 & 0 \\
$w_{395}$ & 45 & 0 & 80 & 0 & 64 & 0 & 48 & 0 & 0 & 0 & 0 & 0 & 0 & 0 & 0 & 0 & 3 \\
$w_{396}$ & 39 & 0 & 90 & 0 & 64 & 0 & 36 & 0 & 8 & 0 & 0 & 0 & 0 & 0 & 2 & 0 & 1 \\
$w_{397}$ & 43 & 0 & 84 & 0 & 62 & 0 & 42 & 0 & 4 & 0 & 2 & 0 & 2 & 0 & 0 & 0 & 1 \\
$w_{398}$ & 38 & 0 & 92 & 0 & 63 & 0 & 33 & 0 & 10 & 0 & 1 & 0 & 1 & 0 & 2 & 0 & 0 \\
$w_{399}$ & 44 & 0 & 92 & 0 & 55 & 0 & 33 & 0 & 12 & 0 & 1 & 0 & 1 & 0 & 2 & 0 & 0 \\
$w_{400}$ & 40 & 0 & 89 & 0 & 62 & 0 & 36 & 0 & 8 & 0 & 2 & 0 & 2 & 0 & 1 & 0 & 0 \\
$w_{401}$ & 43 & 0 & 80 & 0 & 62 & 0 & 48 & 0 & 4 & 0 & 0 & 0 & 2 & 0 & 0 & 0 & 1 \\
$w_{402}$ & 39 & 0 & 84 & 0 & 64 & 0 & 42 & 0 & 8 & 0 & 2 & 0 & 0 & 0 & 0 & 0 & 1 \\
$w_{403}$ & 38 & 0 & 87 & 0 & 63 & 0 & 39 & 0 & 10 & 0 & 1 & 0 & 1 & 0 & 1 & 0 & 0 \\
$w_{404}$ & 38 & 0 & 90 & 0 & 63 & 0 & 36 & 0 & 10 & 0 & 0 & 0 & 1 & 0 & 2 & 0 & 0 \\
$w_{405}$ & 38 & 0 & 89 & 0 & 63 & 0 & 36 & 0 & 10 & 0 & 2 & 0 & 1 & 0 & 1 & 0 & 0 \\
$w_{406}$ & 44 & 0 & 87 & 0 & 55 & 0 & 39 & 0 & 12 & 0 & 1 & 0 & 1 & 0 & 1 & 0 & 0 \\
$w_{407}$ & 42 & 0 & 84 & 0 & 61 & 0 & 42 & 0 & 6 & 0 & 2 & 0 & 3 & 0 & 0 & 0 & 0 \\
$w_{408}$ & 48 & 0 & 82 & 0 & 53 & 0 & 45 & 0 & 8 & 0 & 1 & 0 & 3 & 0 & 0 & 0 & 0 \\
$w_{409}$ & 40 & 0 & 86 & 0 & 62 & 0 & 39 & 0 & 8 & 0 & 3 & 0 & 2 & 0 & 0 & 0 & 0 \\
$w_{410}$ & 42 & 0 & 86 & 0 & 61 & 0 & 39 & 0 & 6 & 0 & 3 & 0 & 3 & 0 & 0 & 0 & 0 \\
$w_{411}$ & 42 & 0 & 89 & 0 & 56 & 0 & 36 & 0 & 14 & 0 & 2 & 0 & 0 & 0 & 1 & 0 & 0 \\
$w_{412}$ & 36 & 0 & 89 & 0 & 64 & 0 & 36 & 0 & 12 & 0 & 2 & 0 & 0 & 0 & 1 & 0 & 0 \\
$w_{413}$ & 46 & 0 & 82 & 0 & 54 & 0 & 45 & 0 & 10 & 0 & 1 & 0 & 2 & 0 & 0 & 0 & 0 \\
$w_{414}$ & 54 & 0 & 82 & 0 & 45 & 0 & 45 & 0 & 10 & 0 & 1 & 0 & 3 & 0 & 0 & 0 & 0 \\
$w_{415}$ & 52 & 0 & 82 & 0 & 46 & 0 & 45 & 0 & 12 & 0 & 1 & 0 & 2 & 0 & 0 & 0 & 0 \\
$w_{416}$ & 39 & 0 & 80 & 0 & 64 & 0 & 48 & 0 & 8 & 0 & 0 & 0 & 0 & 0 & 0 & 0 & 1 \\
$w_{417}$ & 36 & 0 & 87 & 0 & 64 & 0 & 39 & 0 & 12 & 0 & 1 & 0 & 0 & 0 & 1 & 0 & 0 \\
$w_{418}$ & 40 & 0 & 84 & 0 & 62 & 0 & 42 & 0 & 8 & 0 & 2 & 0 & 2 & 0 & 0 & 0 & 0 \\
$w_{419}$ & 38 & 0 & 86 & 0 & 63 & 0 & 39 & 0 & 10 & 0 & 3 & 0 & 1 & 0 & 0 & 0 & 0 \\
$w_{420}$ & 48 & 0 & 80 & 0 & 53 & 0 & 48 & 0 & 8 & 0 & 0 & 0 & 3 & 0 & 0 & 0 & 0 \\
$w_{421}$ & 42 & 0 & 87 & 0 & 56 & 0 & 39 & 0 & 14 & 0 & 1 & 0 & 0 & 0 & 1 & 0 & 0 \\
$w_{422}$ & 46 & 0 & 80 & 0 & 54 & 0 & 48 & 0 & 10 & 0 & 0 & 0 & 2 & 0 & 0 & 0 & 0 \\
$w_{423}$ & 44 & 0 & 84 & 0 & 55 & 0 & 42 & 0 & 12 & 0 & 2 & 0 & 1 & 0 & 0 & 0 & 0 \\
$w_{424}$ & 40 & 0 & 82 & 0 & 62 & 0 & 45 & 0 & 8 & 0 & 1 & 0 & 2 & 0 & 0 & 0 & 0 \\
$w_{425}$ & 44 & 0 & 82 & 0 & 55 & 0 & 45 & 0 & 12 & 0 & 1 & 0 & 1 & 0 & 0 & 0 & 0 \\
$w_{426}$ & 38 & 0 & 84 & 0 & 63 & 0 & 42 & 0 & 10 & 0 & 2 & 0 & 1 & 0 & 0 & 0 & 0 \\
$w_{427}$ & 36 & 0 & 88 & 0 & 64 & 0 & 36 & 0 & 12 & 0 & 4 & 0 & 0 & 0 & 0 & 0 & 0 \\
$w_{428}$ & 42 & 0 & 86 & 0 & 56 & 0 & 39 & 0 & 14 & 0 & 3 & 0 & 0 & 0 & 0 & 0 & 0 \\
$w_{429}$ & 36 & 0 & 86 & 0 & 64 & 0 & 39 & 0 & 12 & 0 & 3 & 0 & 0 & 0 & 0 & 0 & 0 \\
$w_{430}$ & 42 & 0 & 84 & 0 & 56 & 0 & 42 & 0 & 14 & 0 & 2 & 0 & 0 & 0 & 0 & 0 & 0 \\
$w_{431}$ & 50 & 0 & 82 & 0 & 47 & 0 & 45 & 0 & 14 & 0 & 1 & 0 & 1 & 0 & 0 & 0 & 0 \\
$w_{432}$ & 48 & 0 & 86 & 0 & 48 & 0 & 39 & 0 & 16 & 0 & 3 & 0 & 0 & 0 & 0 & 0 & 0 \\
$w_{433}$ & 50 & 0 & 84 & 0 & 47 & 0 & 42 & 0 & 14 & 0 & 2 & 0 & 1 & 0 & 0 & 0 & 0 \\
$w_{434}$ & 48 & 0 & 84 & 0 & 48 & 0 & 42 & 0 & 16 & 0 & 2 & 0 & 0 & 0 & 0 & 0 & 0 \\
$w_{435}$ & 62 & 0 & 80 & 0 & 31 & 0 & 48 & 0 & 18 & 0 & 0 & 0 & 1 & 0 & 0 & 0 & 0 \\
$w_{436}$ & 60 & 0 & 84 & 0 & 32 & 0 & 42 & 0 & 20 & 0 & 2 & 0 & 0 & 0 & 0 & 0 & 0 \\
$w_{437}$ & 50 & 0 & 80 & 0 & 47 & 0 & 48 & 0 & 14 & 0 & 0 & 0 & 1 & 0 & 0 & 0 & 0 \\
$w_{438}$ & 84 & 0 & 0 & 0 & 128 & 0 & 0 & 0 & 28 & 0 & 0 & 0 & 0 & 0 & 0 & 0 & 0 \\
$w_{439}$ & 36 & 0 & 85 & 0 & 64 & 0 & 42 & 0 & 12 & 0 & 0 & 0 & 0 & 0 & 1 & 0 & 0 \\
$w_{440}$ & 90 & 0 & 0 & 0 & 120 & 0 & 0 & 0 & 30 & 0 & 0 & 0 & 0 & 0 & 0 & 0 & 0 \\
$w_{441}$ & 42 & 0 & 82 & 0 & 56 & 0 & 45 & 0 & 14 & 0 & 1 & 0 & 0 & 0 & 0 & 0 & 0 \\
$w_{442}$ & 44 & 0 & 80 & 0 & 55 & 0 & 48 & 0 & 12 & 0 & 0 & 0 & 1 & 0 & 0 & 0 & 0 \\
$w_{443}$ & 48 & 0 & 82 & 0 & 48 & 0 & 45 & 0 & 16 & 0 & 1 & 0 & 0 & 0 & 0 & 0 & 0 \\
$w_{444}$ & 42 & 0 & 80 & 0 & 61 & 0 & 48 & 0 & 6 & 0 & 0 & 0 & 3 & 0 & 0 & 0 & 0 \\
$w_{445}$ & 40 & 0 & 80 & 0 & 62 & 0 & 48 & 0 & 8 & 0 & 0 & 0 & 2 & 0 & 0 & 0 & 0 \\
$w_{446}$ & 38 & 0 & 80 & 0 & 63 & 0 & 48 & 0 & 10 & 0 & 0 & 0 & 1 & 0 & 0 & 0 & 0 \\
$w_{447}$ & 45 & 0 & 0 & 0 & 192 & 0 & 0 & 0 & 0 & 0 & 0 & 0 & 0 & 0 & 0 & 0 & 3 \\
$w_{448}$ & 15 & 0 & 90 & 0 & 96 & 0 & 36 & 0 & 0 & 0 & 0 & 0 & 0 & 0 & 2 & 0 & 1 \\
$w_{449}$ & 67 & 0 & 0 & 0 & 158 & 0 & 0 & 0 & 12 & 0 & 0 & 0 & 2 & 0 & 0 & 0 & 1 \\
$w_{450}$ & 26 & 0 & 90 & 0 & 79 & 0 & 36 & 0 & 6 & 0 & 0 & 0 & 1 & 0 & 2 & 0 & 0 \\
$w_{451}$ & 15 & 0 & 84 & 0 & 96 & 0 & 42 & 0 & 0 & 0 & 2 & 0 & 0 & 0 & 0 & 0 & 1 \\
$w_{452}$ & 26 & 0 & 87 & 0 & 79 & 0 & 39 & 0 & 6 & 0 & 1 & 0 & 1 & 0 & 1 & 0 & 0 \\
$w_{453}$ & 78 & 0 & 0 & 0 & 141 & 0 & 0 & 0 & 18 & 0 & 0 & 0 & 3 & 0 & 0 & 0 & 0 \\
$w_{454}$ & 39 & 0 & 0 & 0 & 192 & 0 & 0 & 0 & 8 & 0 & 0 & 0 & 0 & 0 & 0 & 0 & 1 \\
$w_{455}$ & 12 & 0 & 87 & 0 & 96 & 0 & 39 & 0 & 4 & 0 & 1 & 0 & 0 & 0 & 1 & 0 & 0 \\
$w_{456}$ & 64 & 0 & 0 & 0 & 158 & 0 & 0 & 0 & 16 & 0 & 0 & 0 & 2 & 0 & 0 & 0 & 0 \\
$w_{457}$ & 26 & 0 & 84 & 0 & 79 & 0 & 42 & 0 & 6 & 0 & 2 & 0 & 1 & 0 & 0 & 0 & 0 \\
$w_{458}$ & 15 & 0 & 92 & 0 & 96 & 0 & 30 & 0 & 0 & 0 & 6 & 0 & 0 & 0 & 0 & 0 & 1 \\
$w_{459}$ & 26 & 0 & 91 & 0 & 79 & 0 & 33 & 0 & 6 & 0 & 3 & 0 & 1 & 0 & 1 & 0 & 0 \\
$w_{460}$ & 42 & 0 & 82 & 0 & 61 & 0 & 45 & 0 & 6 & 0 & 1 & 0 & 3 & 0 & 0 & 0 & 0 \\
$w_{461}$ & 63 & 0 & 0 & 0 & 160 & 0 & 0 & 0 & 16 & 0 & 0 & 0 & 0 & 0 & 0 & 0 & 1 \\
$w_{462}$ & 24 & 0 & 91 & 0 & 80 & 0 & 33 & 0 & 8 & 0 & 3 & 0 & 0 & 0 & 1 & 0 & 0 \\
$w_{463}$ & 76 & 0 & 0 & 0 & 142 & 0 & 0 & 0 & 20 & 0 & 0 & 0 & 2 & 0 & 0 & 0 & 0 \\
$w_{464}$ & 26 & 0 & 92 & 0 & 79 & 0 & 30 & 0 & 6 & 0 & 6 & 0 & 1 & 0 & 0 & 0 & 0 \\
$w_{465}$ & 24 & 0 & 87 & 0 & 80 & 0 & 39 & 0 & 8 & 0 & 1 & 0 & 0 & 0 & 1 & 0 & 0 \\
$w_{466}$ & 26 & 0 & 88 & 0 & 79 & 0 & 36 & 0 & 6 & 0 & 4 & 0 & 1 & 0 & 0 & 0 & 0 \\
$w_{467}$ & 38 & 0 & 82 & 0 & 63 & 0 & 45 & 0 & 10 & 0 & 1 & 0 & 1 & 0 & 0 & 0 & 0 \\
$w_{468}$ & 74 & 0 & 0 & 0 & 143 & 0 & 0 & 0 & 22 & 0 & 0 & 0 & 1 & 0 & 0 & 0 & 0 \\
$w_{469}$ & 24 & 0 & 88 & 0 & 80 & 0 & 36 & 0 & 8 & 0 & 4 & 0 & 0 & 0 & 0 & 0 & 0 \\
$w_{470}$ & 36 & 0 & 84 & 0 & 64 & 0 & 42 & 0 & 12 & 0 & 2 & 0 & 0 & 0 & 0 & 0 & 0 \\
$w_{471}$ & 62 & 0 & 0 & 0 & 159 & 0 & 0 & 0 & 18 & 0 & 0 & 0 & 1 & 0 & 0 & 0 & 0 \\
$w_{472}$ & 12 & 0 & 92 & 0 & 96 & 0 & 30 & 0 & 4 & 0 & 6 & 0 & 0 & 0 & 0 & 0 & 0 \\
$w_{473}$ & 48 & 0 & 80 & 0 & 48 & 0 & 48 & 0 & 16 & 0 & 0 & 0 & 0 & 0 & 0 & 0 & 0 \\
$w_{474}$ & 24 & 0 & 92 & 0 & 80 & 0 & 30 & 0 & 8 & 0 & 6 & 0 & 0 & 0 & 0 & 0 & 0 \\
$w_{475}$ & 36 & 0 & 82 & 0 & 64 & 0 & 45 & 0 & 12 & 0 & 1 & 0 & 0 & 0 & 0 & 0 & 0 \\
$w_{476}$ & 42 & 0 & 80 & 0 & 56 & 0 & 48 & 0 & 14 & 0 & 0 & 0 & 0 & 0 & 0 & 0 & 0 \\
$w_{477}$ & 24 & 0 & 84 & 0 & 80 & 0 & 42 & 0 & 8 & 0 & 2 & 0 & 0 & 0 & 0 & 0 & 0 \\
$w_{478}$ & 12 & 0 & 88 & 0 & 96 & 0 & 36 & 0 & 4 & 0 & 4 & 0 & 0 & 0 & 0 & 0 & 0 \\
$w_{479}$ & 72 & 0 & 0 & 0 & 144 & 0 & 0 & 0 & 24 & 0 & 0 & 0 & 0 & 0 & 0 & 0 & 0 \\
$w_{480}$ & 36 & 0 & 80 & 0 & 64 & 0 & 48 & 0 & 12 & 0 & 0 & 0 & 0 & 0 & 0 & 0 & 0 \\
$w_{481}$ & 60 & 0 & 0 & 0 & 160 & 0 & 0 & 0 & 20 & 0 & 0 & 0 & 0 & 0 & 0 & 0 & 0 \\

\end{longtable}

\newpage
\begin{footnotesize}
\begin{longtable}{>{\bfseries}rccccccccccccccccrcccc}
\caption{All extended affine and CCZ equivalence classes in dimension $n=4$.} \\
\toprule
& \multicolumn{16}{c}{$s(x)$ for $x\in\F$} \\
\cmidrule{2-17}
\# & 0 & 1 & 2 & 3 & 4 & 5 & 6 & 7 & 8 & 9 & 10 & 11 & 12 & 13 & 14 & 15 & $|G_s|$ & deg & $w$ & bij.? & CCZ \\
\midrule
\label{tab:dim4}
\endfirsthead
\multicolumn{20}{c}%
{\tablename\ \thetable\ -- \textit{Continued from previous page}} \\
\toprule
\# & 0 & 1 & 2 & 3 & 4 & 5 & 6 & 7 & 8 & 9 & 10 & 11 & 12 & 13 & 14 & 15 & $|G_s|$ & deg & $w$ & bij. & CCZ \\
\midrule
\endhead
\bottomrule
\multicolumn{20}{r}{\textit{Continued on next page}} \\
\endfoot
\bottomrule
\endlastfoot
1 & 0 & 0 & 0 & 0 & 0 & 0 & 0 & 0 & 0 & 0 & 0 & 0 & 0 & 0 & 0 & 0 & 6502809600 & 1 & $w_{1}$ & Y & can. \\
2 & 0 & 0 & 0 & 0 & 0 & 0 & 0 & 0 & 0 & 0 & 0 & 0 & 0 & 0 & 0 & 1 & 27095040 & 4 & $w_{2}$ & N & can. \\
3 & 0 & 0 & 0 & 0 & 0 & 0 & 0 & 0 & 0 & 0 & 0 & 0 & 0 & 0 & 1 & 1 & 3612672 & 3 & $w_{3}$ & Y & can. \\
4 & 0 & 0 & 0 & 0 & 0 & 0 & 0 & 0 & 0 & 0 & 0 & 0 & 0 & 0 & 1 & 2 & 258048 & 4 & $w_{4}$ & N & can. \\
5 & 0 & 0 & 0 & 0 & 0 & 0 & 0 & 0 & 0 & 0 & 0 & 0 & 0 & 1 & 1 & 1 & 774144 & 4 & $w_{5}$ & N & can. \\
6 & 0 & 0 & 0 & 0 & 0 & 0 & 0 & 0 & 0 & 0 & 0 & 0 & 0 & 1 & 1 & 2 & 18432 & 4 & $w_{6}$ & N & can. \\
7 & 0 & 0 & 0 & 0 & 0 & 0 & 0 & 0 & 0 & 0 & 0 & 0 & 0 & 1 & 2 & 3 & 55296 & 3 & $w_{7}$ & Y & can. \\
8 & 0 & 0 & 0 & 0 & 0 & 0 & 0 & 0 & 0 & 0 & 0 & 0 & 0 & 1 & 2 & 4 & 4608 & 4 & $w_{8}$ & N & can. \\
9 & 0 & 0 & 0 & 0 & 0 & 0 & 0 & 0 & 0 & 0 & 0 & 0 & 1 & 1 & 1 & 1 & 12386304 & 2 & $w_{9}$ & Y & can. \\
10 & 0 & 0 & 0 & 0 & 0 & 0 & 0 & 0 & 0 & 0 & 0 & 0 & 1 & 1 & 1 & 2 & 55296 & 4 & $w_{10}$ & N & can. \\
11 & 0 & 0 & 0 & 0 & 0 & 0 & 0 & 0 & 0 & 0 & 0 & 0 & 1 & 1 & 2 & 2 & 73728 & 3 & $w_{11}$ & Y & can. \\
12 & 0 & 0 & 0 & 0 & 0 & 0 & 0 & 0 & 0 & 0 & 0 & 0 & 1 & 1 & 2 & 3 & 36864 & 4 & $w_{12}$ & N & can. \\
13 & 0 & 0 & 0 & 0 & 0 & 0 & 0 & 0 & 0 & 0 & 0 & 0 & 1 & 1 & 2 & 4 & 3072 & 4 & $w_{13}$ & N & can. \\
14 & 0 & 0 & 0 & 0 & 0 & 0 & 0 & 0 & 0 & 0 & 0 & 0 & 1 & 2 & 3 & 4 & 4608 & 4 & $w_{14}$ & N & can. \\
15 & 0 & 0 & 0 & 0 & 0 & 0 & 0 & 0 & 0 & 0 & 0 & 0 & 1 & 2 & 4 & 7 & 18432 & 3 & $w_{15}$ & N & can. \\
16 & 0 & 0 & 0 & 0 & 0 & 0 & 0 & 0 & 0 & 0 & 0 & 0 & 1 & 2 & 4 & 8 & 2304 & 4 & $w_{16}$ & N & can. \\
17 & 0 & 0 & 0 & 0 & 0 & 0 & 0 & 0 & 0 & 0 & 0 & 1 & 0 & 1 & 1 & 1 & 516096 & 3 & $w_{17}$ & N & can. \\
18 & 0 & 0 & 0 & 0 & 0 & 0 & 0 & 0 & 0 & 0 & 0 & 1 & 0 & 1 & 1 & 2 & 4608 & 4 & $w_{18}$ & N & can. \\
19 & 0 & 0 & 0 & 0 & 0 & 0 & 0 & 0 & 0 & 0 & 0 & 1 & 0 & 1 & 2 & 2 & 6144 & 3 & $w_{19}$ & Y & can. \\
20 & 0 & 0 & 0 & 0 & 0 & 0 & 0 & 0 & 0 & 0 & 0 & 1 & 0 & 1 & 2 & 3 & 3072 & 4 & $w_{12}$ & N & can. \\
21 & 0 & 0 & 0 & 0 & 0 & 0 & 0 & 0 & 0 & 0 & 0 & 1 & 0 & 1 & 2 & 4 & 256 & 4 & $w_{20}$ & N & can. \\
22 & 0 & 0 & 0 & 0 & 0 & 0 & 0 & 0 & 0 & 0 & 0 & 1 & 0 & 2 & 3 & 4 & 384 & 4 & $w_{14}$ & N & can. \\
23 & 0 & 0 & 0 & 0 & 0 & 0 & 0 & 0 & 0 & 0 & 0 & 1 & 0 & 2 & 4 & 7 & 1536 & 3 & $w_{21}$ & Y & can. \\
24 & 0 & 0 & 0 & 0 & 0 & 0 & 0 & 0 & 0 & 0 & 0 & 1 & 0 & 2 & 4 & 8 & 192 & 4 & $w_{22}$ & N & can. \\
25 & 0 & 0 & 0 & 0 & 0 & 0 & 0 & 0 & 0 & 0 & 0 & 1 & 1 & 1 & 2 & 2 & 6144 & 4 & $w_{12}$ & N & can. \\
26 & 0 & 0 & 0 & 0 & 0 & 0 & 0 & 0 & 0 & 0 & 0 & 1 & 1 & 1 & 2 & 3 & 6144 & 3 & $w_{23}$ & N & can. \\
27 & 0 & 0 & 0 & 0 & 0 & 0 & 0 & 0 & 0 & 0 & 0 & 1 & 1 & 1 & 2 & 4 & 512 & 4 & $w_{24}$ & N & can. \\
28 & 0 & 0 & 0 & 0 & 0 & 0 & 0 & 0 & 0 & 0 & 0 & 1 & 1 & 2 & 2 & 2 & 9216 & 4 & $w_{12}$ & N & can. \\
29 & 0 & 0 & 0 & 0 & 0 & 0 & 0 & 0 & 0 & 0 & 0 & 1 & 1 & 2 & 2 & 3 & 1536 & 4 & $w_{25}$ & N & can. \\
30 & 0 & 0 & 0 & 0 & 0 & 0 & 0 & 0 & 0 & 0 & 0 & 1 & 1 & 2 & 2 & 4 & 128 & 4 & $w_{26}$ & N & can. \\
31 & 0 & 0 & 0 & 0 & 0 & 0 & 0 & 0 & 0 & 0 & 0 & 1 & 1 & 2 & 3 & 4 & 128 & 4 & $w_{27}$ & N & can. \\
32 & 0 & 0 & 0 & 0 & 0 & 0 & 0 & 0 & 0 & 0 & 0 & 1 & 1 & 2 & 4 & 6 & 384 & 3 & $w_{28}$ & Y & can. \\
33 & 0 & 0 & 0 & 0 & 0 & 0 & 0 & 0 & 0 & 0 & 0 & 1 & 1 & 2 & 4 & 7 & 384 & 4 & $w_{29}$ & N & can. \\
34 & 0 & 0 & 0 & 0 & 0 & 0 & 0 & 0 & 0 & 0 & 0 & 1 & 1 & 2 & 4 & 8 & 48 & 4 & $w_{30}$ & N & can. \\
35 & 0 & 0 & 0 & 0 & 0 & 0 & 0 & 0 & 0 & 0 & 0 & 1 & 2 & 2 & 2 & 3 & 9216 & 3 & $w_{31}$ & Y & can. \\
36 & 0 & 0 & 0 & 0 & 0 & 0 & 0 & 0 & 0 & 0 & 0 & 1 & 2 & 2 & 2 & 4 & 768 & 4 & $w_{32}$ & N & can. \\
37 & 0 & 0 & 0 & 0 & 0 & 0 & 0 & 0 & 0 & 0 & 0 & 1 & 2 & 2 & 3 & 3 & 6144 & 4 & $w_{33}$ & N & can. \\
38 & 0 & 0 & 0 & 0 & 0 & 0 & 0 & 0 & 0 & 0 & 0 & 1 & 2 & 2 & 3 & 4 & 128 & 4 & $w_{34}$ & N & can. \\
39 & 0 & 0 & 0 & 0 & 0 & 0 & 0 & 0 & 0 & 0 & 0 & 1 & 2 & 2 & 4 & 4 & 512 & 4 & $w_{35}$ & N & can. \\
40 & 0 & 0 & 0 & 0 & 0 & 0 & 0 & 0 & 0 & 0 & 0 & 1 & 2 & 2 & 4 & 5 & 256 & 3 & $w_{36}$ & Y & can. \\
41 & 0 & 0 & 0 & 0 & 0 & 0 & 0 & 0 & 0 & 0 & 0 & 1 & 2 & 2 & 4 & 6 & 256 & 4 & $w_{27}$ & N & can. \\
42 & 0 & 0 & 0 & 0 & 0 & 0 & 0 & 0 & 0 & 0 & 0 & 1 & 2 & 2 & 4 & 7 & 256 & 4 & $w_{29}$ & N & can. \\
43 & 0 & 0 & 0 & 0 & 0 & 0 & 0 & 0 & 0 & 0 & 0 & 1 & 2 & 2 & 4 & 8 & 32 & 4 & $w_{37}$ & N & can. \\
44 & 0 & 0 & 0 & 0 & 0 & 0 & 0 & 0 & 0 & 0 & 0 & 1 & 2 & 3 & 4 & 5 & 512 & 4 & $w_{38}$ & N & can. \\
45 & 0 & 0 & 0 & 0 & 0 & 0 & 0 & 0 & 0 & 0 & 0 & 1 & 2 & 3 & 4 & 6 & 128 & 4 & $w_{39}$ & N & can. \\
46 & 0 & 0 & 0 & 0 & 0 & 0 & 0 & 0 & 0 & 0 & 0 & 1 & 2 & 3 & 4 & 8 & 32 & 4 & $w_{40}$ & N & can. \\
47 & 0 & 0 & 0 & 0 & 0 & 0 & 0 & 0 & 0 & 0 & 0 & 1 & 2 & 4 & 6 & 8 & 48 & 4 & $w_{41}$ & N & can. \\
48 & 0 & 0 & 0 & 0 & 0 & 0 & 0 & 0 & 0 & 0 & 0 & 1 & 2 & 4 & 7 & 8 & 48 & 4 & $w_{42}$ & N & can. \\
49 & 0 & 0 & 0 & 0 & 0 & 0 & 0 & 0 & 0 & 0 & 0 & 1 & 2 & 4 & 8 & 14 & 192 & 4 & $w_{43}$ & N & can. \\
50 & 0 & 0 & 0 & 0 & 0 & 0 & 0 & 0 & 0 & 0 & 0 & 1 & 2 & 4 & 8 & 15 & 192 & 3 & $w_{44}$ & N & can. \\
51 & 0 & 0 & 0 & 0 & 0 & 0 & 0 & 0 & 0 & 0 & 1 & 1 & 2 & 2 & 3 & 3 & 294912 & 2 & $w_{45}$ & Y & can. \\
52 & 0 & 0 & 0 & 0 & 0 & 0 & 0 & 0 & 0 & 0 & 1 & 1 & 2 & 2 & 3 & 4 & 1536 & 4 & $w_{46}$ & N & can. \\
53 & 0 & 0 & 0 & 0 & 0 & 0 & 0 & 0 & 0 & 0 & 1 & 1 & 2 & 2 & 4 & 4 & 12288 & 3 & $w_{47}$ & Y & can. \\
54 & 0 & 0 & 0 & 0 & 0 & 0 & 0 & 0 & 0 & 0 & 1 & 1 & 2 & 2 & 4 & 5 & 1024 & 4 & $w_{38}$ & N & can. \\
55 & 0 & 0 & 0 & 0 & 0 & 0 & 0 & 0 & 0 & 0 & 1 & 1 & 2 & 2 & 4 & 8 & 384 & 4 & $w_{48}$ & N & can. \\
56 & 0 & 0 & 0 & 0 & 0 & 0 & 0 & 0 & 0 & 0 & 1 & 1 & 2 & 3 & 2 & 3 & 24576 & 3 & $w_{49}$ & N & can. \\
57 & 0 & 0 & 0 & 0 & 0 & 0 & 0 & 0 & 0 & 0 & 1 & 1 & 2 & 3 & 2 & 4 & 256 & 4 & $w_{50}$ & N & can. \\
58 & 0 & 0 & 0 & 0 & 0 & 0 & 0 & 0 & 0 & 0 & 1 & 1 & 2 & 3 & 4 & 5 & 2048 & 3 & $w_{51}$ & N & can. \\
59 & 0 & 0 & 0 & 0 & 0 & 0 & 0 & 0 & 0 & 0 & 1 & 1 & 2 & 3 & 4 & 6 & 512 & 4 & $w_{52}$ & N & can. \\
60 & 0 & 0 & 0 & 0 & 0 & 0 & 0 & 0 & 0 & 0 & 1 & 1 & 2 & 3 & 4 & 8 & 128 & 4 & $w_{53}$ & N & can. \\
61 & 0 & 0 & 0 & 0 & 0 & 0 & 0 & 0 & 0 & 0 & 1 & 1 & 2 & 4 & 2 & 4 & 2048 & 3 & $w_{36}$ & N & can. \\
62 & 0 & 0 & 0 & 0 & 0 & 0 & 0 & 0 & 0 & 0 & 1 & 1 & 2 & 4 & 2 & 5 & 512 & 4 & $w_{38}$ & N & can. \\
63 & 0 & 0 & 0 & 0 & 0 & 0 & 0 & 0 & 0 & 0 & 1 & 1 & 2 & 4 & 2 & 6 & 256 & 4 & $w_{39}$ & N & can. \\
64 & 0 & 0 & 0 & 0 & 0 & 0 & 0 & 0 & 0 & 0 & 1 & 1 & 2 & 4 & 2 & 8 & 64 & 4 & $w_{40}$ & N & can. \\
65 & 0 & 0 & 0 & 0 & 0 & 0 & 0 & 0 & 0 & 0 & 1 & 1 & 2 & 4 & 3 & 5 & 1024 & 3 & $w_{54}$ & Y & can. \\
66 & 0 & 0 & 0 & 0 & 0 & 0 & 0 & 0 & 0 & 0 & 1 & 1 & 2 & 4 & 3 & 6 & 256 & 4 & $w_{55}$ & N & can. \\
67 & 0 & 0 & 0 & 0 & 0 & 0 & 0 & 0 & 0 & 0 & 1 & 1 & 2 & 4 & 3 & 8 & 64 & 4 & $w_{56}$ & N & can. \\
68 & 0 & 0 & 0 & 0 & 0 & 0 & 0 & 0 & 0 & 0 & 1 & 1 & 2 & 4 & 6 & 8 & 32 & 4 & $w_{57}$ & N & can. \\
69 & 0 & 0 & 0 & 0 & 0 & 0 & 0 & 0 & 0 & 0 & 1 & 1 & 2 & 4 & 8 & 14 & 256 & 3 & $w_{58}$ & N & can. \\
70 & 0 & 0 & 0 & 0 & 0 & 0 & 0 & 0 & 0 & 0 & 1 & 1 & 2 & 4 & 8 & 15 & 256 & 4 & $w_{59}$ & N & can. \\
71 & 0 & 0 & 0 & 0 & 0 & 0 & 0 & 0 & 0 & 0 & 1 & 2 & 1 & 3 & 2 & 3 & 18432 & 3 & $w_{60}$ & Y & can. \\
72 & 0 & 0 & 0 & 0 & 0 & 0 & 0 & 0 & 0 & 0 & 1 & 2 & 1 & 3 & 2 & 4 & 192 & 4 & $w_{61}$ & N & can. \\
73 & 0 & 0 & 0 & 0 & 0 & 0 & 0 & 0 & 0 & 0 & 1 & 2 & 1 & 3 & 4 & 5 & 512 & 3 & $w_{62}$ & Y & can. \\
74 & 0 & 0 & 0 & 0 & 0 & 0 & 0 & 0 & 0 & 0 & 1 & 2 & 1 & 3 & 4 & 6 & 128 & 4 & $w_{52}$ & N & can. \\
75 & 0 & 0 & 0 & 0 & 0 & 0 & 0 & 0 & 0 & 0 & 1 & 2 & 1 & 3 & 4 & 8 & 32 & 4 & $w_{63}$ & N & can. \\
76 & 0 & 0 & 0 & 0 & 0 & 0 & 0 & 0 & 0 & 0 & 1 & 2 & 1 & 4 & 2 & 4 & 1536 & 3 & $w_{28}$ & Y & can. \\
77 & 0 & 0 & 0 & 0 & 0 & 0 & 0 & 0 & 0 & 0 & 1 & 2 & 1 & 4 & 2 & 5 & 128 & 4 & $w_{39}$ & N & can. \\
78 & 0 & 0 & 0 & 0 & 0 & 0 & 0 & 0 & 0 & 0 & 1 & 2 & 1 & 4 & 2 & 8 & 48 & 4 & $w_{41}$ & N & can. \\
79 & 0 & 0 & 0 & 0 & 0 & 0 & 0 & 0 & 0 & 0 & 1 & 2 & 1 & 4 & 3 & 5 & 256 & 3 & $w_{51}$ & Y & can. \\
80 & 0 & 0 & 0 & 0 & 0 & 0 & 0 & 0 & 0 & 0 & 1 & 2 & 1 & 4 & 3 & 6 & 128 & 4 & $w_{55}$ & N & can. \\
81 & 0 & 0 & 0 & 0 & 0 & 0 & 0 & 0 & 0 & 0 & 1 & 2 & 1 & 4 & 3 & 7 & 128 & 4 & $w_{52}$ & N & can. \\
82 & 0 & 0 & 0 & 0 & 0 & 0 & 0 & 0 & 0 & 0 & 1 & 2 & 1 & 4 & 3 & 8 & 16 & 4 & $w_{53}$ & N & can. \\
83 & 0 & 0 & 0 & 0 & 0 & 0 & 0 & 0 & 0 & 0 & 1 & 2 & 1 & 4 & 6 & 8 & 16 & 4 & $w_{57}$ & N & can. \\
84 & 0 & 0 & 0 & 0 & 0 & 0 & 0 & 0 & 0 & 0 & 1 & 2 & 1 & 4 & 7 & 8 & 16 & 4 & $w_{64}$ & N & can. \\
85 & 0 & 0 & 0 & 0 & 0 & 0 & 0 & 0 & 0 & 0 & 1 & 2 & 1 & 4 & 8 & 14 & 64 & 3 & $w_{65}$ & N & can. \\
86 & 0 & 0 & 0 & 0 & 0 & 0 & 0 & 0 & 0 & 0 & 1 & 2 & 1 & 4 & 8 & 15 & 64 & 4 & $w_{66}$ & N & can. \\
87 & 0 & 0 & 0 & 0 & 0 & 0 & 0 & 0 & 0 & 0 & 1 & 2 & 3 & 4 & 5 & 6 & 256 & 4 & $w_{67}$ & N & can. \\
88 & 0 & 0 & 0 & 0 & 0 & 0 & 0 & 0 & 0 & 0 & 1 & 2 & 3 & 4 & 5 & 7 & 192 & 4 & $w_{68}$ & N & can. \\
89 & 0 & 0 & 0 & 0 & 0 & 0 & 0 & 0 & 0 & 0 & 1 & 2 & 3 & 4 & 5 & 8 & 8 & 4 & $w_{69}$ & N & can. \\
90 & 0 & 0 & 0 & 0 & 0 & 0 & 0 & 0 & 0 & 0 & 1 & 2 & 3 & 4 & 7 & 8 & 16 & 4 & $w_{70}$ & N & can. \\
91 & 0 & 0 & 0 & 0 & 0 & 0 & 0 & 0 & 0 & 0 & 1 & 2 & 3 & 4 & 8 & 11 & 64 & 4 & $w_{69}$ & N & can. \\
92 & 0 & 0 & 0 & 0 & 0 & 0 & 0 & 0 & 0 & 0 & 1 & 2 & 3 & 4 & 8 & 12 & 64 & 3 & $w_{71}$ & N & can. \\
93 & 0 & 0 & 0 & 0 & 0 & 0 & 0 & 0 & 0 & 0 & 1 & 2 & 3 & 4 & 8 & 13 & 16 & 4 & $w_{72}$ & N & can. \\
94 & 0 & 0 & 0 & 0 & 0 & 0 & 0 & 0 & 0 & 0 & 1 & 2 & 3 & 4 & 8 & 15 & 32 & 4 & $w_{73}$ & N & can. \\
95 & 0 & 0 & 0 & 0 & 0 & 0 & 0 & 0 & 0 & 0 & 1 & 2 & 4 & 7 & 5 & 6 & 1536 & 4 & $w_{74}$ & N & can. \\
96 & 0 & 0 & 0 & 0 & 0 & 0 & 0 & 0 & 0 & 0 & 1 & 2 & 4 & 7 & 5 & 8 & 32 & 4 & $w_{75}$ & N & can. \\
97 & 0 & 0 & 0 & 0 & 0 & 0 & 0 & 0 & 0 & 0 & 1 & 2 & 4 & 7 & 8 & 11 & 384 & 4 & $w_{76}$ & N & can. \\
98 & 0 & 0 & 0 & 0 & 0 & 0 & 0 & 0 & 0 & 0 & 1 & 2 & 4 & 7 & 8 & 13 & 64 & 4 & $w_{73}$ & N & can. \\
99 & 0 & 0 & 0 & 0 & 0 & 0 & 0 & 0 & 0 & 0 & 1 & 2 & 4 & 8 & 5 & 10 & 96 & 3 & $w_{77}$ & N & can. \\
100 & 0 & 0 & 0 & 0 & 0 & 0 & 0 & 0 & 0 & 0 & 1 & 2 & 4 & 8 & 5 & 11 & 16 & 4 & $w_{73}$ & N & can. \\
101 & 0 & 0 & 0 & 0 & 0 & 0 & 0 & 0 & 0 & 0 & 1 & 2 & 4 & 8 & 7 & 13 & 48 & 4 & $w_{72}$ & N & can. \\
102 & 0 & 0 & 0 & 0 & 0 & 0 & 0 & 0 & 0 & 1 & 2 & 3 & 4 & 5 & 6 & 7 & 172032 & 2 & $w_{78}$ & Y & can. \\
103 & 0 & 0 & 0 & 0 & 0 & 0 & 0 & 0 & 0 & 1 & 2 & 3 & 4 & 5 & 6 & 8 & 1344 & 4 & $w_{79}$ & N & can. \\
104 & 0 & 0 & 0 & 0 & 0 & 0 & 0 & 0 & 0 & 1 & 2 & 3 & 4 & 5 & 7 & 6 & 12288 & 3 & $w_{80}$ & N & can. \\
105 & 0 & 0 & 0 & 0 & 0 & 0 & 0 & 0 & 0 & 1 & 2 & 3 & 4 & 5 & 7 & 8 & 192 & 4 & $w_{81}$ & N & can. \\
106 & 0 & 0 & 0 & 0 & 0 & 0 & 0 & 0 & 0 & 1 & 2 & 3 & 4 & 5 & 8 & 9 & 1536 & 3 & $w_{82}$ & N & can. \\
107 & 0 & 0 & 0 & 0 & 0 & 0 & 0 & 0 & 0 & 1 & 2 & 3 & 4 & 5 & 8 & 10 & 64 & 4 & $w_{83}$ & N & can. \\
108 & 0 & 0 & 0 & 0 & 0 & 0 & 0 & 0 & 0 & 1 & 2 & 3 & 4 & 6 & 7 & 5 & 6144 & 3 & $w_{84}$ & N & can. \\
109 & 0 & 0 & 0 & 0 & 0 & 0 & 0 & 0 & 0 & 1 & 2 & 3 & 4 & 6 & 7 & 8 & 96 & 4 & $w_{85}$ & N & can. \\
110 & 0 & 0 & 0 & 0 & 0 & 0 & 0 & 0 & 0 & 1 & 2 & 3 & 4 & 6 & 8 & 10 & 256 & 3 & $w_{86}$ & N & can. \\
111 & 0 & 0 & 0 & 0 & 0 & 0 & 0 & 0 & 0 & 1 & 2 & 3 & 4 & 6 & 8 & 11 & 128 & 4 & $w_{87}$ & N & can. \\
112 & 0 & 0 & 0 & 0 & 0 & 0 & 0 & 0 & 0 & 1 & 2 & 3 & 4 & 6 & 8 & 12 & 16 & 4 & $w_{87}$ & N & can. \\
113 & 0 & 0 & 0 & 0 & 0 & 0 & 0 & 0 & 0 & 1 & 2 & 4 & 3 & 6 & 7 & 5 & 10752 & 3 & $w_{88}$ & N & can. \\
114 & 0 & 0 & 0 & 0 & 0 & 0 & 0 & 0 & 0 & 1 & 2 & 4 & 3 & 6 & 7 & 8 & 168 & 4 & $w_{89}$ & N & can. \\
115 & 0 & 0 & 0 & 0 & 0 & 0 & 0 & 0 & 0 & 1 & 2 & 4 & 3 & 6 & 8 & 10 & 192 & 3 & $w_{90}$ & N & can. \\
116 & 0 & 0 & 0 & 0 & 0 & 0 & 0 & 0 & 0 & 1 & 2 & 4 & 3 & 6 & 8 & 11 & 32 & 4 & $w_{91}$ & N & can. \\
117 & 0 & 0 & 0 & 0 & 0 & 0 & 0 & 1 & 0 & 0 & 0 & 1 & 0 & 1 & 1 & 1 & 967680 & 4 & $w_{92}$ & N & can. \\
118 & 0 & 0 & 0 & 0 & 0 & 0 & 0 & 1 & 0 & 0 & 0 & 1 & 0 & 1 & 1 & 2 & 4608 & 4 & $w_{93}$ & N & can. \\
119 & 0 & 0 & 0 & 0 & 0 & 0 & 0 & 1 & 0 & 0 & 0 & 1 & 0 & 1 & 2 & 2 & 1152 & 4 & $w_{94}$ & N & can. \\
120 & 0 & 0 & 0 & 0 & 0 & 0 & 0 & 1 & 0 & 0 & 0 & 1 & 0 & 1 & 2 & 3 & 1152 & 3 & $w_{23}$ & N & can. \\
121 & 0 & 0 & 0 & 0 & 0 & 0 & 0 & 1 & 0 & 0 & 0 & 1 & 0 & 1 & 2 & 4 & 96 & 4 & $w_{95}$ & N & can. \\
122 & 0 & 0 & 0 & 0 & 0 & 0 & 0 & 1 & 0 & 0 & 0 & 1 & 0 & 2 & 2 & 3 & 768 & 4 & $w_{25}$ & N & can. \\
123 & 0 & 0 & 0 & 0 & 0 & 0 & 0 & 1 & 0 & 0 & 0 & 1 & 0 & 2 & 2 & 4 & 64 & 4 & $w_{96}$ & N & can. \\
124 & 0 & 0 & 0 & 0 & 0 & 0 & 0 & 1 & 0 & 0 & 0 & 1 & 0 & 2 & 3 & 4 & 32 & 4 & $w_{27}$ & N & can. \\
125 & 0 & 0 & 0 & 0 & 0 & 0 & 0 & 1 & 0 & 0 & 0 & 1 & 0 & 2 & 4 & 6 & 96 & 3 & $w_{28}$ & Y & can. \\
126 & 0 & 0 & 0 & 0 & 0 & 0 & 0 & 1 & 0 & 0 & 0 & 1 & 0 & 2 & 4 & 7 & 96 & 4 & $w_{97}$ & N & can. \\
127 & 0 & 0 & 0 & 0 & 0 & 0 & 0 & 1 & 0 & 0 & 0 & 1 & 0 & 2 & 4 & 8 & 12 & 4 & $w_{98}$ & N & can. \\
128 & 0 & 0 & 0 & 0 & 0 & 0 & 0 & 1 & 0 & 0 & 0 & 1 & 1 & 1 & 2 & 2 & 3072 & 3 & $w_{23}$ & N & can. \\
129 & 0 & 0 & 0 & 0 & 0 & 0 & 0 & 1 & 0 & 0 & 0 & 1 & 1 & 1 & 2 & 3 & 3072 & 4 & $w_{99}$ & N & can. \\
130 & 0 & 0 & 0 & 0 & 0 & 0 & 0 & 1 & 0 & 0 & 0 & 1 & 1 & 1 & 2 & 4 & 256 & 4 & $w_{100}$ & N & can. \\
131 & 0 & 0 & 0 & 0 & 0 & 0 & 0 & 1 & 0 & 0 & 0 & 1 & 1 & 2 & 2 & 2 & 1152 & 4 & $w_{25}$ & N & can. \\
132 & 0 & 0 & 0 & 0 & 0 & 0 & 0 & 1 & 0 & 0 & 0 & 1 & 1 & 2 & 2 & 3 & 384 & 4 & $w_{101}$ & N & can. \\
133 & 0 & 0 & 0 & 0 & 0 & 0 & 0 & 1 & 0 & 0 & 0 & 1 & 1 & 2 & 2 & 4 & 32 & 4 & $w_{102}$ & N & can. \\
134 & 0 & 0 & 0 & 0 & 0 & 0 & 0 & 1 & 0 & 0 & 0 & 1 & 1 & 2 & 3 & 4 & 32 & 4 & $w_{103}$ & N & can. \\
135 & 0 & 0 & 0 & 0 & 0 & 0 & 0 & 1 & 0 & 0 & 0 & 1 & 1 & 2 & 4 & 6 & 96 & 4 & $w_{104}$ & N & can. \\
136 & 0 & 0 & 0 & 0 & 0 & 0 & 0 & 1 & 0 & 0 & 0 & 1 & 1 & 2 & 4 & 7 & 96 & 3 & $w_{105}$ & N & can. \\
137 & 0 & 0 & 0 & 0 & 0 & 0 & 0 & 1 & 0 & 0 & 0 & 1 & 1 & 2 & 4 & 8 & 12 & 4 & $w_{106}$ & N & can. \\
138 & 0 & 0 & 0 & 0 & 0 & 0 & 0 & 1 & 0 & 0 & 0 & 1 & 2 & 2 & 2 & 3 & 3456 & 4 & $w_{107}$ & N & can. \\
139 & 0 & 0 & 0 & 0 & 0 & 0 & 0 & 1 & 0 & 0 & 0 & 1 & 2 & 2 & 2 & 4 & 96 & 4 & $w_{108}$ & N & can. \\
140 & 0 & 0 & 0 & 0 & 0 & 0 & 0 & 1 & 0 & 0 & 0 & 1 & 2 & 2 & 3 & 3 & 3072 & 3 & $w_{49}$ & N & can. \\
141 & 0 & 0 & 0 & 0 & 0 & 0 & 0 & 1 & 0 & 0 & 0 & 1 & 2 & 2 & 3 & 4 & 32 & 4 & $w_{109}$ & N & can. \\
142 & 0 & 0 & 0 & 0 & 0 & 0 & 0 & 1 & 0 & 0 & 0 & 1 & 2 & 2 & 4 & 4 & 128 & 3 & $w_{110}$ & Y & can. \\
143 & 0 & 0 & 0 & 0 & 0 & 0 & 0 & 1 & 0 & 0 & 0 & 1 & 2 & 2 & 4 & 5 & 64 & 4 & $w_{111}$ & N & can. \\
144 & 0 & 0 & 0 & 0 & 0 & 0 & 0 & 1 & 0 & 0 & 0 & 1 & 2 & 2 & 4 & 6 & 64 & 4 & $w_{112}$ & N & can. \\
145 & 0 & 0 & 0 & 0 & 0 & 0 & 0 & 1 & 0 & 0 & 0 & 1 & 2 & 2 & 4 & 7 & 64 & 4 & $w_{52}$ & N & can. \\
146 & 0 & 0 & 0 & 0 & 0 & 0 & 0 & 1 & 0 & 0 & 0 & 1 & 2 & 2 & 4 & 8 & 8 & 4 & $w_{113}$ & N & can. \\
147 & 0 & 0 & 0 & 0 & 0 & 0 & 0 & 1 & 0 & 0 & 0 & 1 & 2 & 3 & 4 & 5 & 128 & 3 & $w_{51}$ & N & can. \\
148 & 0 & 0 & 0 & 0 & 0 & 0 & 0 & 1 & 0 & 0 & 0 & 1 & 2 & 3 & 4 & 6 & 32 & 4 & $w_{114}$ & N & can. \\
149 & 0 & 0 & 0 & 0 & 0 & 0 & 0 & 1 & 0 & 0 & 0 & 1 & 2 & 3 & 4 & 8 & 8 & 4 & $w_{115}$ & N & can. \\
150 & 0 & 0 & 0 & 0 & 0 & 0 & 0 & 1 & 0 & 0 & 0 & 1 & 2 & 4 & 6 & 8 & 12 & 4 & $w_{116}$ & N & can. \\
151 & 0 & 0 & 0 & 0 & 0 & 0 & 0 & 1 & 0 & 0 & 0 & 1 & 2 & 4 & 7 & 8 & 12 & 4 & $w_{117}$ & N & can. \\
152 & 0 & 0 & 0 & 0 & 0 & 0 & 0 & 1 & 0 & 0 & 0 & 1 & 2 & 4 & 8 & 14 & 48 & 3 & $w_{118}$ & N & can. \\
153 & 0 & 0 & 0 & 0 & 0 & 0 & 0 & 1 & 0 & 0 & 0 & 1 & 2 & 4 & 8 & 15 & 48 & 4 & $w_{119}$ & N & can. \\
154 & 0 & 0 & 0 & 0 & 0 & 0 & 0 & 1 & 0 & 0 & 0 & 2 & 0 & 3 & 4 & 5 & 64 & 4 & $w_{39}$ & N & can. \\
155 & 0 & 0 & 0 & 0 & 0 & 0 & 0 & 1 & 0 & 0 & 0 & 2 & 0 & 3 & 4 & 8 & 12 & 4 & $w_{41}$ & N & can. \\
156 & 0 & 0 & 0 & 0 & 0 & 0 & 0 & 1 & 0 & 0 & 0 & 2 & 0 & 4 & 7 & 8 & 24 & 4 & $w_{120}$ & N & can. \\
157 & 0 & 0 & 0 & 0 & 0 & 0 & 0 & 1 & 0 & 0 & 0 & 2 & 0 & 4 & 8 & 15 & 120 & 3 & $w_{121}$ & Y & can. \\
158 & 0 & 0 & 0 & 0 & 0 & 0 & 0 & 1 & 0 & 0 & 0 & 2 & 1 & 1 & 2 & 2 & 768 & 4 & $w_{25}$ & N & can. \\
159 & 0 & 0 & 0 & 0 & 0 & 0 & 0 & 1 & 0 & 0 & 0 & 2 & 1 & 1 & 2 & 3 & 384 & 4 & $w_{101}$ & N & can. \\
160 & 0 & 0 & 0 & 0 & 0 & 0 & 0 & 1 & 0 & 0 & 0 & 2 & 1 & 1 & 2 & 4 & 16 & 4 & $w_{102}$ & N & can. \\
161 & 0 & 0 & 0 & 0 & 0 & 0 & 0 & 1 & 0 & 0 & 0 & 2 & 1 & 1 & 4 & 4 & 64 & 4 & $w_{27}$ & N & can. \\
162 & 0 & 0 & 0 & 0 & 0 & 0 & 0 & 1 & 0 & 0 & 0 & 2 & 1 & 1 & 4 & 5 & 64 & 4 & $w_{103}$ & N & can. \\
163 & 0 & 0 & 0 & 0 & 0 & 0 & 0 & 1 & 0 & 0 & 0 & 2 & 1 & 1 & 4 & 6 & 64 & 4 & $w_{122}$ & N & can. \\
164 & 0 & 0 & 0 & 0 & 0 & 0 & 0 & 1 & 0 & 0 & 0 & 2 & 1 & 1 & 4 & 7 & 64 & 3 & $w_{105}$ & N & can. \\
165 & 0 & 0 & 0 & 0 & 0 & 0 & 0 & 1 & 0 & 0 & 0 & 2 & 1 & 1 & 4 & 8 & 8 & 4 & $w_{123}$ & N & can. \\
166 & 0 & 0 & 0 & 0 & 0 & 0 & 0 & 1 & 0 & 0 & 0 & 2 & 1 & 2 & 3 & 3 & 384 & 3 & $w_{60}$ & N & can. \\
167 & 0 & 0 & 0 & 0 & 0 & 0 & 0 & 1 & 0 & 0 & 0 & 2 & 1 & 2 & 3 & 4 & 16 & 4 & $w_{124}$ & N & can. \\
168 & 0 & 0 & 0 & 0 & 0 & 0 & 0 & 1 & 0 & 0 & 0 & 2 & 1 & 2 & 4 & 4 & 32 & 3 & $w_{105}$ & Y & can. \\
169 & 0 & 0 & 0 & 0 & 0 & 0 & 0 & 1 & 0 & 0 & 0 & 2 & 1 & 2 & 4 & 5 & 16 & 4 & $w_{112}$ & N & can. \\
170 & 0 & 0 & 0 & 0 & 0 & 0 & 0 & 1 & 0 & 0 & 0 & 2 & 1 & 2 & 4 & 7 & 32 & 4 & $w_{52}$ & N & can. \\
171 & 0 & 0 & 0 & 0 & 0 & 0 & 0 & 1 & 0 & 0 & 0 & 2 & 1 & 2 & 4 & 8 & 4 & 4 & $w_{125}$ & N & can. \\
172 & 0 & 0 & 0 & 0 & 0 & 0 & 0 & 1 & 0 & 0 & 0 & 2 & 1 & 3 & 3 & 4 & 16 & 4 & $w_{61}$ & N & can. \\
173 & 0 & 0 & 0 & 0 & 0 & 0 & 0 & 1 & 0 & 0 & 0 & 2 & 1 & 3 & 4 & 4 & 16 & 4 & $w_{126}$ & N & can. \\
174 & 0 & 0 & 0 & 0 & 0 & 0 & 0 & 1 & 0 & 0 & 0 & 2 & 1 & 3 & 4 & 5 & 16 & 3 & $w_{62}$ & N & can. \\
175 & 0 & 0 & 0 & 0 & 0 & 0 & 0 & 1 & 0 & 0 & 0 & 2 & 1 & 3 & 4 & 6 & 16 & 4 & $w_{52}$ & N & can. \\
176 & 0 & 0 & 0 & 0 & 0 & 0 & 0 & 1 & 0 & 0 & 0 & 2 & 1 & 3 & 4 & 7 & 16 & 4 & $w_{114}$ & N & can. \\
177 & 0 & 0 & 0 & 0 & 0 & 0 & 0 & 1 & 0 & 0 & 0 & 2 & 1 & 3 & 4 & 8 & 2 & 4 & $w_{127}$ & N & can. \\
178 & 0 & 0 & 0 & 0 & 0 & 0 & 0 & 1 & 0 & 0 & 0 & 2 & 1 & 4 & 4 & 4 & 96 & 4 & $w_{27}$ & N & can. \\
179 & 0 & 0 & 0 & 0 & 0 & 0 & 0 & 1 & 0 & 0 & 0 & 2 & 1 & 4 & 4 & 5 & 16 & 4 & $w_{124}$ & N & can. \\
180 & 0 & 0 & 0 & 0 & 0 & 0 & 0 & 1 & 0 & 0 & 0 & 2 & 1 & 4 & 4 & 6 & 16 & 4 & $w_{126}$ & N & can. \\
181 & 0 & 0 & 0 & 0 & 0 & 0 & 0 & 1 & 0 & 0 & 0 & 2 & 1 & 4 & 4 & 7 & 16 & 4 & $w_{52}$ & N & can. \\
182 & 0 & 0 & 0 & 0 & 0 & 0 & 0 & 1 & 0 & 0 & 0 & 2 & 1 & 4 & 4 & 8 & 2 & 4 & $w_{128}$ & N & can. \\
183 & 0 & 0 & 0 & 0 & 0 & 0 & 0 & 1 & 0 & 0 & 0 & 2 & 1 & 4 & 5 & 6 & 8 & 4 & $w_{114}$ & N & can. \\
184 & 0 & 0 & 0 & 0 & 0 & 0 & 0 & 1 & 0 & 0 & 0 & 2 & 1 & 4 & 5 & 8 & 2 & 4 & $w_{129}$ & N & can. \\
185 & 0 & 0 & 0 & 0 & 0 & 0 & 0 & 1 & 0 & 0 & 0 & 2 & 1 & 4 & 6 & 8 & 2 & 4 & $w_{130}$ & N & can. \\
186 & 0 & 0 & 0 & 0 & 0 & 0 & 0 & 1 & 0 & 0 & 0 & 2 & 1 & 4 & 7 & 8 & 2 & 4 & $w_{117}$ & N & can. \\
187 & 0 & 0 & 0 & 0 & 0 & 0 & 0 & 1 & 0 & 0 & 0 & 2 & 1 & 4 & 8 & 12 & 6 & 4 & $w_{116}$ & N & can. \\
188 & 0 & 0 & 0 & 0 & 0 & 0 & 0 & 1 & 0 & 0 & 0 & 2 & 1 & 4 & 8 & 13 & 6 & 4 & $w_{131}$ & N & can. \\
189 & 0 & 0 & 0 & 0 & 0 & 0 & 0 & 1 & 0 & 0 & 0 & 2 & 1 & 4 & 8 & 14 & 6 & 3 & $w_{132}$ & Y & can. \\
190 & 0 & 0 & 0 & 0 & 0 & 0 & 0 & 1 & 0 & 0 & 0 & 2 & 1 & 4 & 8 & 15 & 6 & 4 & $w_{133}$ & N & can. \\
191 & 0 & 0 & 0 & 0 & 0 & 0 & 0 & 1 & 0 & 0 & 0 & 2 & 3 & 3 & 4 & 4 & 64 & 4 & $w_{39}$ & N & can. \\
192 & 0 & 0 & 0 & 0 & 0 & 0 & 0 & 1 & 0 & 0 & 0 & 2 & 3 & 3 & 4 & 5 & 32 & 4 & $w_{52}$ & N & can. \\
193 & 0 & 0 & 0 & 0 & 0 & 0 & 0 & 1 & 0 & 0 & 0 & 2 & 3 & 3 & 4 & 7 & 64 & 3 & $w_{62}$ & N & can. \\
194 & 0 & 0 & 0 & 0 & 0 & 0 & 0 & 1 & 0 & 0 & 0 & 2 & 3 & 3 & 4 & 8 & 8 & 4 & $w_{63}$ & N & can. \\
195 & 0 & 0 & 0 & 0 & 0 & 0 & 0 & 1 & 0 & 0 & 0 & 2 & 3 & 4 & 4 & 4 & 288 & 4 & $w_{97}$ & N & can. \\
196 & 0 & 0 & 0 & 0 & 0 & 0 & 0 & 1 & 0 & 0 & 0 & 2 & 3 & 4 & 4 & 5 & 16 & 4 & $w_{52}$ & N & can. \\
197 & 0 & 0 & 0 & 0 & 0 & 0 & 0 & 1 & 0 & 0 & 0 & 2 & 3 & 4 & 4 & 7 & 32 & 4 & $w_{114}$ & N & can. \\
198 & 0 & 0 & 0 & 0 & 0 & 0 & 0 & 1 & 0 & 0 & 0 & 2 & 3 & 4 & 4 & 8 & 4 & 4 & $w_{134}$ & N & can. \\
199 & 0 & 0 & 0 & 0 & 0 & 0 & 0 & 1 & 0 & 0 & 0 & 2 & 3 & 4 & 5 & 6 & 16 & 4 & $w_{68}$ & N & can. \\
200 & 0 & 0 & 0 & 0 & 0 & 0 & 0 & 1 & 0 & 0 & 0 & 2 & 3 & 4 & 5 & 8 & 2 & 4 & $w_{70}$ & N & can. \\
201 & 0 & 0 & 0 & 0 & 0 & 0 & 0 & 1 & 0 & 0 & 0 & 2 & 3 & 4 & 7 & 8 & 4 & 4 & $w_{135}$ & N & can. \\
202 & 0 & 0 & 0 & 0 & 0 & 0 & 0 & 1 & 0 & 0 & 0 & 2 & 3 & 4 & 8 & 12 & 12 & 3 & $w_{136}$ & Y & can. \\
203 & 0 & 0 & 0 & 0 & 0 & 0 & 0 & 1 & 0 & 0 & 0 & 2 & 3 & 4 & 8 & 13 & 6 & 4 & $w_{137}$ & N & can. \\
204 & 0 & 0 & 0 & 0 & 0 & 0 & 0 & 1 & 0 & 0 & 0 & 2 & 3 & 4 & 8 & 15 & 12 & 4 & $w_{72}$ & N & can. \\
205 & 0 & 0 & 0 & 0 & 0 & 0 & 0 & 1 & 0 & 0 & 0 & 2 & 4 & 4 & 4 & 7 & 288 & 3 & $w_{110}$ & Y & can. \\
206 & 0 & 0 & 0 & 0 & 0 & 0 & 0 & 1 & 0 & 0 & 0 & 2 & 4 & 4 & 4 & 8 & 36 & 4 & $w_{138}$ & N & can. \\
207 & 0 & 0 & 0 & 0 & 0 & 0 & 0 & 1 & 0 & 0 & 0 & 2 & 4 & 4 & 5 & 5 & 64 & 4 & $w_{50}$ & N & can. \\
208 & 0 & 0 & 0 & 0 & 0 & 0 & 0 & 1 & 0 & 0 & 0 & 2 & 4 & 4 & 5 & 6 & 32 & 3 & $w_{51}$ & Y & can. \\
209 & 0 & 0 & 0 & 0 & 0 & 0 & 0 & 1 & 0 & 0 & 0 & 2 & 4 & 4 & 5 & 7 & 16 & 4 & $w_{139}$ & N & can. \\
210 & 0 & 0 & 0 & 0 & 0 & 0 & 0 & 1 & 0 & 0 & 0 & 2 & 4 & 4 & 5 & 8 & 2 & 4 & $w_{140}$ & N & can. \\
211 & 0 & 0 & 0 & 0 & 0 & 0 & 0 & 1 & 0 & 0 & 0 & 2 & 4 & 4 & 7 & 7 & 128 & 4 & $w_{55}$ & N & can. \\
212 & 0 & 0 & 0 & 0 & 0 & 0 & 0 & 1 & 0 & 0 & 0 & 2 & 4 & 4 & 7 & 8 & 4 & 4 & $w_{141}$ & N & can. \\
213 & 0 & 0 & 0 & 0 & 0 & 0 & 0 & 1 & 0 & 0 & 0 & 2 & 4 & 4 & 8 & 8 & 16 & 4 & $w_{142}$ & N & can. \\
214 & 0 & 0 & 0 & 0 & 0 & 0 & 0 & 1 & 0 & 0 & 0 & 2 & 4 & 4 & 8 & 9 & 4 & 4 & $w_{143}$ & N & can. \\
215 & 0 & 0 & 0 & 0 & 0 & 0 & 0 & 1 & 0 & 0 & 0 & 2 & 4 & 4 & 8 & 11 & 8 & 3 & $w_{118}$ & Y & can. \\
216 & 0 & 0 & 0 & 0 & 0 & 0 & 0 & 1 & 0 & 0 & 0 & 2 & 4 & 4 & 8 & 12 & 8 & 4 & $w_{129}$ & N & can. \\
217 & 0 & 0 & 0 & 0 & 0 & 0 & 0 & 1 & 0 & 0 & 0 & 2 & 4 & 4 & 8 & 13 & 4 & 4 & $w_{131}$ & N & can. \\
218 & 0 & 0 & 0 & 0 & 0 & 0 & 0 & 1 & 0 & 0 & 0 & 2 & 4 & 4 & 8 & 15 & 8 & 4 & $w_{144}$ & N & can. \\
219 & 0 & 0 & 0 & 0 & 0 & 0 & 0 & 1 & 0 & 0 & 0 & 2 & 4 & 5 & 6 & 7 & 64 & 4 & $w_{67}$ & N & can. \\
220 & 0 & 0 & 0 & 0 & 0 & 0 & 0 & 1 & 0 & 0 & 0 & 2 & 4 & 5 & 6 & 8 & 2 & 4 & $w_{145}$ & N & can. \\
221 & 0 & 0 & 0 & 0 & 0 & 0 & 0 & 1 & 0 & 0 & 0 & 2 & 4 & 5 & 8 & 9 & 8 & 4 & $w_{69}$ & N & can. \\
222 & 0 & 0 & 0 & 0 & 0 & 0 & 0 & 1 & 0 & 0 & 0 & 2 & 4 & 5 & 8 & 10 & 8 & 3 & $w_{71}$ & Y & can. \\
223 & 0 & 0 & 0 & 0 & 0 & 0 & 0 & 1 & 0 & 0 & 0 & 2 & 4 & 5 & 8 & 11 & 4 & 4 & $w_{146}$ & N & can. \\
224 & 0 & 0 & 0 & 0 & 0 & 0 & 0 & 1 & 0 & 0 & 0 & 2 & 4 & 5 & 8 & 12 & 2 & 4 & $w_{135}$ & N & can. \\
225 & 0 & 0 & 0 & 0 & 0 & 0 & 0 & 1 & 0 & 0 & 0 & 2 & 4 & 5 & 8 & 14 & 2 & 4 & $w_{137}$ & N & can. \\
226 & 0 & 0 & 0 & 0 & 0 & 0 & 0 & 1 & 0 & 0 & 0 & 2 & 4 & 7 & 8 & 11 & 16 & 4 & $w_{147}$ & N & can. \\
227 & 0 & 0 & 0 & 0 & 0 & 0 & 0 & 1 & 0 & 0 & 0 & 2 & 4 & 7 & 8 & 12 & 4 & 4 & $w_{148}$ & N & can. \\
228 & 0 & 0 & 0 & 0 & 0 & 0 & 0 & 1 & 0 & 0 & 0 & 2 & 4 & 7 & 8 & 13 & 4 & 4 & $w_{149}$ & N & can. \\
229 & 0 & 0 & 0 & 0 & 0 & 0 & 0 & 1 & 0 & 0 & 1 & 1 & 2 & 2 & 4 & 4 & 512 & 4 & $w_{38}$ & N & can. \\
230 & 0 & 0 & 0 & 0 & 0 & 0 & 0 & 1 & 0 & 0 & 1 & 1 & 2 & 2 & 4 & 5 & 256 & 3 & $w_{51}$ & N & can. \\
231 & 0 & 0 & 0 & 0 & 0 & 0 & 0 & 1 & 0 & 0 & 1 & 1 & 2 & 2 & 4 & 6 & 128 & 4 & $w_{52}$ & N & can. \\
232 & 0 & 0 & 0 & 0 & 0 & 0 & 0 & 1 & 0 & 0 & 1 & 1 & 2 & 2 & 4 & 8 & 32 & 4 & $w_{53}$ & N & can. \\
233 & 0 & 0 & 0 & 0 & 0 & 0 & 0 & 1 & 0 & 0 & 1 & 1 & 2 & 3 & 2 & 3 & 9216 & 4 & $w_{150}$ & N & can. \\
234 & 0 & 0 & 0 & 0 & 0 & 0 & 0 & 1 & 0 & 0 & 1 & 1 & 2 & 3 & 2 & 4 & 64 & 4 & $w_{151}$ & N & can. \\
235 & 0 & 0 & 0 & 0 & 0 & 0 & 0 & 1 & 0 & 0 & 1 & 1 & 2 & 3 & 4 & 5 & 512 & 4 & $w_{152}$ & N & can. \\
236 & 0 & 0 & 0 & 0 & 0 & 0 & 0 & 1 & 0 & 0 & 1 & 1 & 2 & 3 & 4 & 6 & 128 & 4 & $w_{153}$ & N & can. \\
237 & 0 & 0 & 0 & 0 & 0 & 0 & 0 & 1 & 0 & 0 & 1 & 1 & 2 & 3 & 4 & 8 & 32 & 4 & $w_{154}$ & N & can. \\
238 & 0 & 0 & 0 & 0 & 0 & 0 & 0 & 1 & 0 & 0 & 1 & 1 & 2 & 4 & 2 & 4 & 128 & 4 & $w_{111}$ & N & can. \\
239 & 0 & 0 & 0 & 0 & 0 & 0 & 0 & 1 & 0 & 0 & 1 & 1 & 2 & 4 & 2 & 5 & 128 & 3 & $w_{51}$ & N & can. \\
240 & 0 & 0 & 0 & 0 & 0 & 0 & 0 & 1 & 0 & 0 & 1 & 1 & 2 & 4 & 2 & 6 & 32 & 4 & $w_{114}$ & N & can. \\
241 & 0 & 0 & 0 & 0 & 0 & 0 & 0 & 1 & 0 & 0 & 1 & 1 & 2 & 4 & 2 & 8 & 8 & 4 & $w_{115}$ & N & can. \\
242 & 0 & 0 & 0 & 0 & 0 & 0 & 0 & 1 & 0 & 0 & 1 & 1 & 2 & 4 & 6 & 8 & 8 & 4 & $w_{155}$ & N & can. \\
243 & 0 & 0 & 0 & 0 & 0 & 0 & 0 & 1 & 0 & 0 & 1 & 1 & 2 & 4 & 8 & 14 & 64 & 4 & $w_{119}$ & N & can. \\
244 & 0 & 0 & 0 & 0 & 0 & 0 & 0 & 1 & 0 & 0 & 1 & 1 & 2 & 4 & 8 & 15 & 64 & 3 & $w_{71}$ & N & can. \\
245 & 0 & 0 & 0 & 0 & 0 & 0 & 0 & 1 & 0 & 0 & 1 & 2 & 1 & 2 & 3 & 3 & 1536 & 4 & $w_{101}$ & N & can. \\
246 & 0 & 0 & 0 & 0 & 0 & 0 & 0 & 1 & 0 & 0 & 1 & 2 & 1 & 2 & 3 & 4 & 16 & 4 & $w_{124}$ & N & can. \\
247 & 0 & 0 & 0 & 0 & 0 & 0 & 0 & 1 & 0 & 0 & 1 & 2 & 1 & 2 & 4 & 4 & 64 & 4 & $w_{126}$ & N & can. \\
248 & 0 & 0 & 0 & 0 & 0 & 0 & 0 & 1 & 0 & 0 & 1 & 2 & 1 & 2 & 4 & 5 & 32 & 3 & $w_{62}$ & N & can. \\
249 & 0 & 0 & 0 & 0 & 0 & 0 & 0 & 1 & 0 & 0 & 1 & 2 & 1 & 2 & 4 & 6 & 32 & 4 & $w_{114}$ & N & can. \\
250 & 0 & 0 & 0 & 0 & 0 & 0 & 0 & 1 & 0 & 0 & 1 & 2 & 1 & 2 & 4 & 7 & 32 & 4 & $w_{52}$ & N & can. \\
251 & 0 & 0 & 0 & 0 & 0 & 0 & 0 & 1 & 0 & 0 & 1 & 2 & 1 & 2 & 4 & 8 & 4 & 4 & $w_{127}$ & N & can. \\
252 & 0 & 0 & 0 & 0 & 0 & 0 & 0 & 1 & 0 & 0 & 1 & 2 & 1 & 3 & 2 & 3 & 768 & 4 & $w_{156}$ & N & can. \\
253 & 0 & 0 & 0 & 0 & 0 & 0 & 0 & 1 & 0 & 0 & 1 & 2 & 1 & 3 & 2 & 4 & 8 & 4 & $w_{157}$ & N & can. \\
254 & 0 & 0 & 0 & 0 & 0 & 0 & 0 & 1 & 0 & 0 & 1 & 2 & 1 & 3 & 4 & 5 & 64 & 4 & $w_{158}$ & N & can. \\
255 & 0 & 0 & 0 & 0 & 0 & 0 & 0 & 1 & 0 & 0 & 1 & 2 & 1 & 3 & 4 & 6 & 16 & 4 & $w_{153}$ & N & can. \\
256 & 0 & 0 & 0 & 0 & 0 & 0 & 0 & 1 & 0 & 0 & 1 & 2 & 1 & 3 & 4 & 8 & 4 & 4 & $w_{159}$ & N & can. \\
257 & 0 & 0 & 0 & 0 & 0 & 0 & 0 & 1 & 0 & 0 & 1 & 2 & 1 & 4 & 2 & 4 & 32 & 4 & $w_{112}$ & N & can. \\
258 & 0 & 0 & 0 & 0 & 0 & 0 & 0 & 1 & 0 & 0 & 1 & 2 & 1 & 4 & 2 & 5 & 16 & 3 & $w_{62}$ & N & can. \\
259 & 0 & 0 & 0 & 0 & 0 & 0 & 0 & 1 & 0 & 0 & 1 & 2 & 1 & 4 & 2 & 6 & 8 & 4 & $w_{114}$ & N & can. \\
260 & 0 & 0 & 0 & 0 & 0 & 0 & 0 & 1 & 0 & 0 & 1 & 2 & 1 & 4 & 2 & 8 & 2 & 4 & $w_{129}$ & N & can. \\
261 & 0 & 0 & 0 & 0 & 0 & 0 & 0 & 1 & 0 & 0 & 1 & 2 & 1 & 4 & 3 & 5 & 32 & 4 & $w_{160}$ & N & can. \\
262 & 0 & 0 & 0 & 0 & 0 & 0 & 0 & 1 & 0 & 0 & 1 & 2 & 1 & 4 & 3 & 6 & 16 & 4 & $w_{114}$ & N & can. \\
263 & 0 & 0 & 0 & 0 & 0 & 0 & 0 & 1 & 0 & 0 & 1 & 2 & 1 & 4 & 3 & 7 & 16 & 4 & $w_{153}$ & N & can. \\
264 & 0 & 0 & 0 & 0 & 0 & 0 & 0 & 1 & 0 & 0 & 1 & 2 & 1 & 4 & 3 & 8 & 2 & 4 & $w_{161}$ & N & can. \\
265 & 0 & 0 & 0 & 0 & 0 & 0 & 0 & 1 & 0 & 0 & 1 & 2 & 1 & 4 & 6 & 8 & 2 & 4 & $w_{155}$ & N & can. \\
266 & 0 & 0 & 0 & 0 & 0 & 0 & 0 & 1 & 0 & 0 & 1 & 2 & 1 & 4 & 7 & 8 & 2 & 4 & $w_{162}$ & N & can. \\
267 & 0 & 0 & 0 & 0 & 0 & 0 & 0 & 1 & 0 & 0 & 1 & 2 & 1 & 4 & 8 & 14 & 8 & 4 & $w_{163}$ & N & can. \\
268 & 0 & 0 & 0 & 0 & 0 & 0 & 0 & 1 & 0 & 0 & 1 & 2 & 1 & 4 & 8 & 15 & 8 & 3 & $w_{136}$ & N & can. \\
269 & 0 & 0 & 0 & 0 & 0 & 0 & 0 & 1 & 0 & 0 & 1 & 2 & 2 & 3 & 4 & 4 & 16 & 4 & $w_{114}$ & N & can. \\
270 & 0 & 0 & 0 & 0 & 0 & 0 & 0 & 1 & 0 & 0 & 1 & 2 & 2 & 3 & 4 & 5 & 16 & 4 & $w_{153}$ & N & can. \\
271 & 0 & 0 & 0 & 0 & 0 & 0 & 0 & 1 & 0 & 0 & 1 & 2 & 2 & 3 & 4 & 6 & 16 & 4 & $w_{164}$ & N & can. \\
272 & 0 & 0 & 0 & 0 & 0 & 0 & 0 & 1 & 0 & 0 & 1 & 2 & 2 & 3 & 4 & 7 & 16 & 3 & $w_{165}$ & N & can. \\
273 & 0 & 0 & 0 & 0 & 0 & 0 & 0 & 1 & 0 & 0 & 1 & 2 & 2 & 3 & 4 & 8 & 2 & 4 & $w_{166}$ & N & can. \\
274 & 0 & 0 & 0 & 0 & 0 & 0 & 0 & 1 & 0 & 0 & 1 & 2 & 2 & 4 & 3 & 4 & 16 & 4 & $w_{114}$ & N & can. \\
275 & 0 & 0 & 0 & 0 & 0 & 0 & 0 & 1 & 0 & 0 & 1 & 2 & 2 & 4 & 3 & 5 & 16 & 4 & $w_{114}$ & N & can. \\
276 & 0 & 0 & 0 & 0 & 0 & 0 & 0 & 1 & 0 & 0 & 1 & 2 & 2 & 4 & 3 & 6 & 16 & 4 & $w_{153}$ & N & can. \\
277 & 0 & 0 & 0 & 0 & 0 & 0 & 0 & 1 & 0 & 0 & 1 & 2 & 2 & 4 & 3 & 7 & 16 & 3 & $w_{165}$ & N & can. \\
278 & 0 & 0 & 0 & 0 & 0 & 0 & 0 & 1 & 0 & 0 & 1 & 2 & 2 & 4 & 3 & 8 & 2 & 4 & $w_{167}$ & N & can. \\
279 & 0 & 0 & 0 & 0 & 0 & 0 & 0 & 1 & 0 & 0 & 1 & 2 & 2 & 4 & 4 & 5 & 16 & 4 & $w_{114}$ & N & can. \\
280 & 0 & 0 & 0 & 0 & 0 & 0 & 0 & 1 & 0 & 0 & 1 & 2 & 2 & 4 & 4 & 6 & 8 & 4 & $w_{153}$ & N & can. \\
281 & 0 & 0 & 0 & 0 & 0 & 0 & 0 & 1 & 0 & 0 & 1 & 2 & 2 & 4 & 4 & 7 & 8 & 4 & $w_{68}$ & N & can. \\
282 & 0 & 0 & 0 & 0 & 0 & 0 & 0 & 1 & 0 & 0 & 1 & 2 & 2 & 4 & 4 & 8 & 1 & 4 & $w_{155}$ & N & can. \\
283 & 0 & 0 & 0 & 0 & 0 & 0 & 0 & 1 & 0 & 0 & 1 & 2 & 2 & 4 & 5 & 5 & 32 & 4 & $w_{52}$ & N & can. \\
284 & 0 & 0 & 0 & 0 & 0 & 0 & 0 & 1 & 0 & 0 & 1 & 2 & 2 & 4 & 5 & 6 & 8 & 4 & $w_{68}$ & N & can. \\
285 & 0 & 0 & 0 & 0 & 0 & 0 & 0 & 1 & 0 & 0 & 1 & 2 & 2 & 4 & 5 & 7 & 8 & 4 & $w_{168}$ & N & can. \\
286 & 0 & 0 & 0 & 0 & 0 & 0 & 0 & 1 & 0 & 0 & 1 & 2 & 2 & 4 & 5 & 8 & 1 & 4 & $w_{135}$ & N & can. \\
287 & 0 & 0 & 0 & 0 & 0 & 0 & 0 & 1 & 0 & 0 & 1 & 2 & 2 & 4 & 6 & 6 & 16 & 4 & $w_{114}$ & N & can. \\
288 & 0 & 0 & 0 & 0 & 0 & 0 & 0 & 1 & 0 & 0 & 1 & 2 & 2 & 4 & 6 & 7 & 8 & 4 & $w_{168}$ & N & can. \\
289 & 0 & 0 & 0 & 0 & 0 & 0 & 0 & 1 & 0 & 0 & 1 & 2 & 2 & 4 & 6 & 8 & 1 & 4 & $w_{169}$ & N & can. \\
290 & 0 & 0 & 0 & 0 & 0 & 0 & 0 & 1 & 0 & 0 & 1 & 2 & 2 & 4 & 7 & 7 & 16 & 4 & $w_{68}$ & N & can. \\
291 & 0 & 0 & 0 & 0 & 0 & 0 & 0 & 1 & 0 & 0 & 1 & 2 & 2 & 4 & 7 & 8 & 1 & 4 & $w_{170}$ & N & can. \\
292 & 0 & 0 & 0 & 0 & 0 & 0 & 0 & 1 & 0 & 0 & 1 & 2 & 2 & 4 & 8 & 8 & 2 & 4 & $w_{117}$ & N & can. \\
293 & 0 & 0 & 0 & 0 & 0 & 0 & 0 & 1 & 0 & 0 & 1 & 2 & 2 & 4 & 8 & 9 & 2 & 4 & $w_{169}$ & N & can. \\
294 & 0 & 0 & 0 & 0 & 0 & 0 & 0 & 1 & 0 & 0 & 1 & 2 & 2 & 4 & 8 & 10 & 2 & 4 & $w_{171}$ & N & can. \\
295 & 0 & 0 & 0 & 0 & 0 & 0 & 0 & 1 & 0 & 0 & 1 & 2 & 2 & 4 & 8 & 11 & 2 & 4 & $w_{172}$ & N & can. \\
296 & 0 & 0 & 0 & 0 & 0 & 0 & 0 & 1 & 0 & 0 & 1 & 2 & 2 & 4 & 8 & 12 & 2 & 3 & $w_{173}$ & Y & can. \\
297 & 0 & 0 & 0 & 0 & 0 & 0 & 0 & 1 & 0 & 0 & 1 & 2 & 2 & 4 & 8 & 13 & 2 & 4 & $w_{174}$ & N & can. \\
298 & 0 & 0 & 0 & 0 & 0 & 0 & 0 & 1 & 0 & 0 & 1 & 2 & 2 & 4 & 8 & 14 & 2 & 4 & $w_{148}$ & N & can. \\
299 & 0 & 0 & 0 & 0 & 0 & 0 & 0 & 1 & 0 & 0 & 1 & 2 & 2 & 4 & 8 & 15 & 2 & 4 & $w_{175}$ & N & can. \\
300 & 0 & 0 & 0 & 0 & 0 & 0 & 0 & 1 & 0 & 0 & 1 & 2 & 3 & 3 & 4 & 4 & 64 & 4 & $w_{52}$ & N & can. \\
301 & 0 & 0 & 0 & 0 & 0 & 0 & 0 & 1 & 0 & 0 & 1 & 2 & 3 & 3 & 4 & 5 & 64 & 4 & $w_{153}$ & N & can. \\
302 & 0 & 0 & 0 & 0 & 0 & 0 & 0 & 1 & 0 & 0 & 1 & 2 & 3 & 3 & 4 & 6 & 64 & 3 & $w_{165}$ & N & can. \\
303 & 0 & 0 & 0 & 0 & 0 & 0 & 0 & 1 & 0 & 0 & 1 & 2 & 3 & 3 & 4 & 7 & 64 & 4 & $w_{153}$ & N & can. \\
304 & 0 & 0 & 0 & 0 & 0 & 0 & 0 & 1 & 0 & 0 & 1 & 2 & 3 & 3 & 4 & 8 & 8 & 4 & $w_{167}$ & N & can. \\
305 & 0 & 0 & 0 & 0 & 0 & 0 & 0 & 1 & 0 & 0 & 1 & 2 & 3 & 4 & 4 & 5 & 16 & 4 & $w_{153}$ & N & can. \\
306 & 0 & 0 & 0 & 0 & 0 & 0 & 0 & 1 & 0 & 0 & 1 & 2 & 3 & 4 & 4 & 6 & 8 & 4 & $w_{168}$ & N & can. \\
307 & 0 & 0 & 0 & 0 & 0 & 0 & 0 & 1 & 0 & 0 & 1 & 2 & 3 & 4 & 4 & 7 & 8 & 4 & $w_{168}$ & N & can. \\
308 & 0 & 0 & 0 & 0 & 0 & 0 & 0 & 1 & 0 & 0 & 1 & 2 & 3 & 4 & 4 & 8 & 1 & 4 & $w_{169}$ & N & can. \\
309 & 0 & 0 & 0 & 0 & 0 & 0 & 0 & 1 & 0 & 0 & 1 & 2 & 3 & 4 & 5 & 5 & 32 & 4 & $w_{114}$ & N & can. \\
310 & 0 & 0 & 0 & 0 & 0 & 0 & 0 & 1 & 0 & 0 & 1 & 2 & 3 & 4 & 5 & 6 & 8 & 4 & $w_{168}$ & N & can. \\
311 & 0 & 0 & 0 & 0 & 0 & 0 & 0 & 1 & 0 & 0 & 1 & 2 & 3 & 4 & 5 & 7 & 8 & 4 & $w_{176}$ & N & can. \\
312 & 0 & 0 & 0 & 0 & 0 & 0 & 0 & 1 & 0 & 0 & 1 & 2 & 3 & 4 & 5 & 8 & 1 & 4 & $w_{171}$ & N & can. \\
313 & 0 & 0 & 0 & 0 & 0 & 0 & 0 & 1 & 0 & 0 & 1 & 2 & 3 & 4 & 6 & 6 & 16 & 4 & $w_{68}$ & N & can. \\
314 & 0 & 0 & 0 & 0 & 0 & 0 & 0 & 1 & 0 & 0 & 1 & 2 & 3 & 4 & 6 & 7 & 8 & 4 & $w_{176}$ & N & can. \\
315 & 0 & 0 & 0 & 0 & 0 & 0 & 0 & 1 & 0 & 0 & 1 & 2 & 3 & 4 & 6 & 8 & 1 & 4 & $w_{177}$ & N & can. \\
316 & 0 & 0 & 0 & 0 & 0 & 0 & 0 & 1 & 0 & 0 & 1 & 2 & 3 & 4 & 7 & 7 & 16 & 4 & $w_{168}$ & N & can. \\
317 & 0 & 0 & 0 & 0 & 0 & 0 & 0 & 1 & 0 & 0 & 1 & 2 & 3 & 4 & 7 & 8 & 1 & 4 & $w_{178}$ & N & can. \\
318 & 0 & 0 & 0 & 0 & 0 & 0 & 0 & 1 & 0 & 0 & 1 & 2 & 3 & 4 & 8 & 8 & 2 & 4 & $w_{135}$ & N & can. \\
319 & 0 & 0 & 0 & 0 & 0 & 0 & 0 & 1 & 0 & 0 & 1 & 2 & 3 & 4 & 8 & 9 & 2 & 4 & $w_{179}$ & N & can. \\
320 & 0 & 0 & 0 & 0 & 0 & 0 & 0 & 1 & 0 & 0 & 1 & 2 & 3 & 4 & 8 & 10 & 2 & 4 & $w_{178}$ & N & can. \\
321 & 0 & 0 & 0 & 0 & 0 & 0 & 0 & 1 & 0 & 0 & 1 & 2 & 3 & 4 & 8 & 11 & 2 & 4 & $w_{177}$ & N & can. \\
322 & 0 & 0 & 0 & 0 & 0 & 0 & 0 & 1 & 0 & 0 & 1 & 2 & 3 & 4 & 8 & 12 & 2 & 4 & $w_{180}$ & N & can. \\
323 & 0 & 0 & 0 & 0 & 0 & 0 & 0 & 1 & 0 & 0 & 1 & 2 & 3 & 4 & 8 & 13 & 2 & 3 & $w_{181}$ & N & can. \\
324 & 0 & 0 & 0 & 0 & 0 & 0 & 0 & 1 & 0 & 0 & 1 & 2 & 3 & 4 & 8 & 14 & 2 & 4 & $w_{182}$ & N & can. \\
325 & 0 & 0 & 0 & 0 & 0 & 0 & 0 & 1 & 0 & 0 & 1 & 2 & 3 & 4 & 8 & 15 & 2 & 4 & $w_{183}$ & N & can. \\
326 & 0 & 0 & 0 & 0 & 0 & 0 & 0 & 1 & 0 & 0 & 1 & 2 & 4 & 4 & 5 & 6 & 32 & 4 & $w_{139}$ & N & can. \\
327 & 0 & 0 & 0 & 0 & 0 & 0 & 0 & 1 & 0 & 0 & 1 & 2 & 4 & 4 & 5 & 7 & 32 & 3 & $w_{184}$ & N & can. \\
328 & 0 & 0 & 0 & 0 & 0 & 0 & 0 & 1 & 0 & 0 & 1 & 2 & 4 & 4 & 5 & 8 & 4 & 4 & $w_{185}$ & N & can. \\
329 & 0 & 0 & 0 & 0 & 0 & 0 & 0 & 1 & 0 & 0 & 1 & 2 & 4 & 4 & 6 & 6 & 128 & 4 & $w_{55}$ & N & can. \\
330 & 0 & 0 & 0 & 0 & 0 & 0 & 0 & 1 & 0 & 0 & 1 & 2 & 4 & 4 & 6 & 7 & 16 & 4 & $w_{186}$ & N & can. \\
331 & 0 & 0 & 0 & 0 & 0 & 0 & 0 & 1 & 0 & 0 & 1 & 2 & 4 & 4 & 6 & 8 & 2 & 4 & $w_{145}$ & N & can. \\
332 & 0 & 0 & 0 & 0 & 0 & 0 & 0 & 1 & 0 & 0 & 1 & 2 & 4 & 4 & 7 & 7 & 64 & 4 & $w_{67}$ & N & can. \\
333 & 0 & 0 & 0 & 0 & 0 & 0 & 0 & 1 & 0 & 0 & 1 & 2 & 4 & 4 & 7 & 8 & 2 & 4 & $w_{187}$ & N & can. \\
334 & 0 & 0 & 0 & 0 & 0 & 0 & 0 & 1 & 0 & 0 & 1 & 2 & 4 & 4 & 8 & 8 & 8 & 4 & $w_{188}$ & N & can. \\
335 & 0 & 0 & 0 & 0 & 0 & 0 & 0 & 1 & 0 & 0 & 1 & 2 & 4 & 4 & 8 & 9 & 4 & 4 & $w_{189}$ & N & can. \\
336 & 0 & 0 & 0 & 0 & 0 & 0 & 0 & 1 & 0 & 0 & 1 & 2 & 4 & 4 & 8 & 10 & 4 & 3 & $w_{190}$ & Y & can. \\
337 & 0 & 0 & 0 & 0 & 0 & 0 & 0 & 1 & 0 & 0 & 1 & 2 & 4 & 4 & 8 & 11 & 4 & 4 & $w_{146}$ & N & can. \\
338 & 0 & 0 & 0 & 0 & 0 & 0 & 0 & 1 & 0 & 0 & 1 & 2 & 4 & 4 & 8 & 12 & 4 & 4 & $w_{169}$ & N & can. \\
339 & 0 & 0 & 0 & 0 & 0 & 0 & 0 & 1 & 0 & 0 & 1 & 2 & 4 & 4 & 8 & 13 & 4 & 4 & $w_{172}$ & N & can. \\
340 & 0 & 0 & 0 & 0 & 0 & 0 & 0 & 1 & 0 & 0 & 1 & 2 & 4 & 4 & 8 & 14 & 4 & 4 & $w_{148}$ & N & can. \\
341 & 0 & 0 & 0 & 0 & 0 & 0 & 0 & 1 & 0 & 0 & 1 & 2 & 4 & 4 & 8 & 15 & 4 & 4 & $w_{149}$ & N & can. \\
342 & 0 & 0 & 0 & 0 & 0 & 0 & 0 & 1 & 0 & 0 & 1 & 2 & 4 & 5 & 4 & 6 & 16 & 4 & $w_{191}$ & N & can. \\
343 & 0 & 0 & 0 & 0 & 0 & 0 & 0 & 1 & 0 & 0 & 1 & 2 & 4 & 5 & 4 & 7 & 16 & 3 & $w_{184}$ & N & can. \\
344 & 0 & 0 & 0 & 0 & 0 & 0 & 0 & 1 & 0 & 0 & 1 & 2 & 4 & 5 & 4 & 8 & 2 & 4 & $w_{192}$ & N & can. \\
345 & 0 & 0 & 0 & 0 & 0 & 0 & 0 & 1 & 0 & 0 & 1 & 2 & 4 & 5 & 6 & 7 & 32 & 4 & $w_{193}$ & N & can. \\
346 & 0 & 0 & 0 & 0 & 0 & 0 & 0 & 1 & 0 & 0 & 1 & 2 & 4 & 5 & 6 & 8 & 1 & 4 & $w_{194}$ & N & can. \\
347 & 0 & 0 & 0 & 0 & 0 & 0 & 0 & 1 & 0 & 0 & 1 & 2 & 4 & 5 & 8 & 9 & 8 & 4 & $w_{195}$ & N & can. \\
348 & 0 & 0 & 0 & 0 & 0 & 0 & 0 & 1 & 0 & 0 & 1 & 2 & 4 & 5 & 8 & 10 & 4 & 4 & $w_{196}$ & N & can. \\
349 & 0 & 0 & 0 & 0 & 0 & 0 & 0 & 1 & 0 & 0 & 1 & 2 & 4 & 5 & 8 & 11 & 4 & 3 & $w_{90}$ & N & can. \\
350 & 0 & 0 & 0 & 0 & 0 & 0 & 0 & 1 & 0 & 0 & 1 & 2 & 4 & 5 & 8 & 12 & 2 & 4 & $w_{178}$ & N & can. \\
351 & 0 & 0 & 0 & 0 & 0 & 0 & 0 & 1 & 0 & 0 & 1 & 2 & 4 & 5 & 8 & 14 & 2 & 4 & $w_{182}$ & N & can. \\
352 & 0 & 0 & 0 & 0 & 0 & 0 & 0 & 1 & 0 & 0 & 1 & 2 & 4 & 6 & 4 & 6 & 32 & 4 & $w_{197}$ & N & can. \\
353 & 0 & 0 & 0 & 0 & 0 & 0 & 0 & 1 & 0 & 0 & 1 & 2 & 4 & 6 & 4 & 7 & 8 & 4 & $w_{186}$ & N & can. \\
354 & 0 & 0 & 0 & 0 & 0 & 0 & 0 & 1 & 0 & 0 & 1 & 2 & 4 & 6 & 4 & 8 & 1 & 4 & $w_{198}$ & N & can. \\
355 & 0 & 0 & 0 & 0 & 0 & 0 & 0 & 1 & 0 & 0 & 1 & 2 & 4 & 6 & 5 & 7 & 32 & 4 & $w_{193}$ & N & can. \\
356 & 0 & 0 & 0 & 0 & 0 & 0 & 0 & 1 & 0 & 0 & 1 & 2 & 4 & 6 & 5 & 8 & 1 & 4 & $w_{194}$ & N & can. \\
357 & 0 & 0 & 0 & 0 & 0 & 0 & 0 & 1 & 0 & 0 & 1 & 2 & 4 & 6 & 8 & 10 & 8 & 4 & $w_{199}$ & N & can. \\
358 & 0 & 0 & 0 & 0 & 0 & 0 & 0 & 1 & 0 & 0 & 1 & 2 & 4 & 6 & 8 & 11 & 4 & 4 & $w_{200}$ & N & can. \\
359 & 0 & 0 & 0 & 0 & 0 & 0 & 0 & 1 & 0 & 0 & 1 & 2 & 4 & 6 & 8 & 12 & 2 & 4 & $w_{201}$ & N & can. \\
360 & 0 & 0 & 0 & 0 & 0 & 0 & 0 & 1 & 0 & 0 & 1 & 2 & 4 & 6 & 8 & 13 & 2 & 4 & $w_{202}$ & N & can. \\
361 & 0 & 0 & 0 & 0 & 0 & 0 & 0 & 1 & 0 & 0 & 1 & 2 & 4 & 7 & 4 & 7 & 32 & 4 & $w_{67}$ & N & can. \\
362 & 0 & 0 & 0 & 0 & 0 & 0 & 0 & 1 & 0 & 0 & 1 & 2 & 4 & 7 & 4 & 8 & 1 & 4 & $w_{187}$ & N & can. \\
363 & 0 & 0 & 0 & 0 & 0 & 0 & 0 & 1 & 0 & 0 & 1 & 2 & 4 & 7 & 5 & 6 & 64 & 4 & $w_{67}$ & N & can. \\
364 & 0 & 0 & 0 & 0 & 0 & 0 & 0 & 1 & 0 & 0 & 1 & 2 & 4 & 7 & 5 & 8 & 1 & 4 & $w_{187}$ & N & can. \\
365 & 0 & 0 & 0 & 0 & 0 & 0 & 0 & 1 & 0 & 0 & 1 & 2 & 4 & 7 & 8 & 11 & 8 & 4 & $w_{87}$ & N & can. \\
366 & 0 & 0 & 0 & 0 & 0 & 0 & 0 & 1 & 0 & 0 & 1 & 2 & 4 & 7 & 8 & 12 & 2 & 4 & $w_{203}$ & N & can. \\
367 & 0 & 0 & 0 & 0 & 0 & 0 & 0 & 1 & 0 & 0 & 1 & 2 & 4 & 7 & 8 & 13 & 2 & 4 & $w_{203}$ & N & can. \\
368 & 0 & 0 & 0 & 0 & 0 & 0 & 0 & 1 & 0 & 0 & 1 & 2 & 4 & 8 & 4 & 8 & 4 & 4 & $w_{204}$ & N & can. \\
369 & 0 & 0 & 0 & 0 & 0 & 0 & 0 & 1 & 0 & 0 & 1 & 2 & 4 & 8 & 4 & 9 & 2 & 4 & $w_{189}$ & N & can. \\
370 & 0 & 0 & 0 & 0 & 0 & 0 & 0 & 1 & 0 & 0 & 1 & 2 & 4 & 8 & 4 & 10 & 2 & 3 & $w_{190}$ & Y & can. \\
371 & 0 & 0 & 0 & 0 & 0 & 0 & 0 & 1 & 0 & 0 & 1 & 2 & 4 & 8 & 4 & 11 & 2 & 4 & $w_{205}$ & N & can. \\
372 & 0 & 0 & 0 & 0 & 0 & 0 & 0 & 1 & 0 & 0 & 1 & 2 & 4 & 8 & 4 & 12 & 2 & 4 & $w_{171}$ & N & can. \\
373 & 0 & 0 & 0 & 0 & 0 & 0 & 0 & 1 & 0 & 0 & 1 & 2 & 4 & 8 & 4 & 13 & 2 & 4 & $w_{170}$ & N & can. \\
374 & 0 & 0 & 0 & 0 & 0 & 0 & 0 & 1 & 0 & 0 & 1 & 2 & 4 & 8 & 4 & 14 & 2 & 4 & $w_{174}$ & N & can. \\
375 & 0 & 0 & 0 & 0 & 0 & 0 & 0 & 1 & 0 & 0 & 1 & 2 & 4 & 8 & 4 & 15 & 2 & 4 & $w_{175}$ & N & can. \\
376 & 0 & 0 & 0 & 0 & 0 & 0 & 0 & 1 & 0 & 0 & 1 & 2 & 4 & 8 & 5 & 9 & 4 & 4 & $w_{189}$ & N & can. \\
377 & 0 & 0 & 0 & 0 & 0 & 0 & 0 & 1 & 0 & 0 & 1 & 2 & 4 & 8 & 5 & 10 & 2 & 4 & $w_{205}$ & N & can. \\
378 & 0 & 0 & 0 & 0 & 0 & 0 & 0 & 1 & 0 & 0 & 1 & 2 & 4 & 8 & 5 & 11 & 2 & 3 & $w_{90}$ & N & can. \\
379 & 0 & 0 & 0 & 0 & 0 & 0 & 0 & 1 & 0 & 0 & 1 & 2 & 4 & 8 & 5 & 12 & 1 & 4 & $w_{177}$ & N & can. \\
380 & 0 & 0 & 0 & 0 & 0 & 0 & 0 & 1 & 0 & 0 & 1 & 2 & 4 & 8 & 5 & 14 & 1 & 4 & $w_{183}$ & N & can. \\
381 & 0 & 0 & 0 & 0 & 0 & 0 & 0 & 1 & 0 & 0 & 1 & 2 & 4 & 8 & 6 & 10 & 4 & 4 & $w_{206}$ & N & can. \\
382 & 0 & 0 & 0 & 0 & 0 & 0 & 0 & 1 & 0 & 0 & 1 & 2 & 4 & 8 & 6 & 11 & 2 & 4 & $w_{200}$ & N & can. \\
383 & 0 & 0 & 0 & 0 & 0 & 0 & 0 & 1 & 0 & 0 & 1 & 2 & 4 & 8 & 6 & 12 & 1 & 4 & $w_{182}$ & N & can. \\
384 & 0 & 0 & 0 & 0 & 0 & 0 & 0 & 1 & 0 & 0 & 1 & 2 & 4 & 8 & 6 & 13 & 1 & 4 & $w_{203}$ & N & can. \\
385 & 0 & 0 & 0 & 0 & 0 & 0 & 0 & 1 & 0 & 0 & 1 & 2 & 4 & 8 & 7 & 11 & 4 & 4 & $w_{200}$ & N & can. \\
386 & 0 & 0 & 0 & 0 & 0 & 0 & 0 & 1 & 0 & 0 & 1 & 2 & 4 & 8 & 7 & 12 & 1 & 4 & $w_{202}$ & N & can. \\
387 & 0 & 0 & 0 & 0 & 0 & 0 & 0 & 1 & 0 & 0 & 1 & 2 & 4 & 8 & 7 & 13 & 1 & 4 & $w_{202}$ & N & can. \\
388 & 0 & 0 & 0 & 0 & 0 & 0 & 0 & 1 & 0 & 0 & 2 & 2 & 4 & 4 & 6 & 7 & 384 & 3 & $w_{207}$ & Y & can. \\
389 & 0 & 0 & 0 & 0 & 0 & 0 & 0 & 1 & 0 & 0 & 2 & 2 & 4 & 4 & 6 & 8 & 48 & 4 & $w_{208}$ & N & can. \\
390 & 0 & 0 & 0 & 0 & 0 & 0 & 0 & 1 & 0 & 0 & 2 & 2 & 4 & 4 & 7 & 7 & 1536 & 4 & $w_{74}$ & N & can. \\
391 & 0 & 0 & 0 & 0 & 0 & 0 & 0 & 1 & 0 & 0 & 2 & 2 & 4 & 4 & 7 & 8 & 24 & 4 & $w_{209}$ & N & can. \\
392 & 0 & 0 & 0 & 0 & 0 & 0 & 0 & 1 & 0 & 0 & 2 & 2 & 4 & 4 & 8 & 8 & 192 & 4 & $w_{210}$ & N & can. \\
393 & 0 & 0 & 0 & 0 & 0 & 0 & 0 & 1 & 0 & 0 & 2 & 2 & 4 & 4 & 8 & 9 & 48 & 3 & $w_{77}$ & Y & can. \\
394 & 0 & 0 & 0 & 0 & 0 & 0 & 0 & 1 & 0 & 0 & 2 & 2 & 4 & 4 & 8 & 10 & 16 & 4 & $w_{69}$ & N & can. \\
395 & 0 & 0 & 0 & 0 & 0 & 0 & 0 & 1 & 0 & 0 & 2 & 2 & 4 & 4 & 8 & 11 & 16 & 4 & $w_{73}$ & N & can. \\
396 & 0 & 0 & 0 & 0 & 0 & 0 & 0 & 1 & 0 & 0 & 2 & 2 & 4 & 5 & 6 & 7 & 128 & 4 & $w_{211}$ & N & can. \\
397 & 0 & 0 & 0 & 0 & 0 & 0 & 0 & 1 & 0 & 0 & 2 & 2 & 4 & 5 & 6 & 8 & 4 & 4 & $w_{212}$ & N & can. \\
398 & 0 & 0 & 0 & 0 & 0 & 0 & 0 & 1 & 0 & 0 & 2 & 2 & 4 & 5 & 8 & 9 & 32 & 4 & $w_{213}$ & N & can. \\
399 & 0 & 0 & 0 & 0 & 0 & 0 & 0 & 1 & 0 & 0 & 2 & 2 & 4 & 5 & 8 & 10 & 16 & 4 & $w_{205}$ & N & can. \\
400 & 0 & 0 & 0 & 0 & 0 & 0 & 0 & 1 & 0 & 0 & 2 & 2 & 4 & 5 & 8 & 11 & 16 & 4 & $w_{87}$ & N & can. \\
401 & 0 & 0 & 0 & 0 & 0 & 0 & 0 & 1 & 0 & 0 & 2 & 2 & 4 & 5 & 8 & 12 & 4 & 4 & $w_{206}$ & N & can. \\
402 & 0 & 0 & 0 & 0 & 0 & 0 & 0 & 1 & 0 & 0 & 2 & 2 & 4 & 6 & 4 & 6 & 256 & 4 & $w_{151}$ & N & can. \\
403 & 0 & 0 & 0 & 0 & 0 & 0 & 0 & 1 & 0 & 0 & 2 & 2 & 4 & 6 & 4 & 7 & 32 & 3 & $w_{184}$ & N & can. \\
404 & 0 & 0 & 0 & 0 & 0 & 0 & 0 & 1 & 0 & 0 & 2 & 2 & 4 & 6 & 4 & 8 & 4 & 4 & $w_{214}$ & N & can. \\
405 & 0 & 0 & 0 & 0 & 0 & 0 & 0 & 1 & 0 & 0 & 2 & 2 & 4 & 6 & 5 & 7 & 128 & 4 & $w_{193}$ & N & can. \\
406 & 0 & 0 & 0 & 0 & 0 & 0 & 0 & 1 & 0 & 0 & 2 & 2 & 4 & 6 & 5 & 8 & 4 & 4 & $w_{215}$ & N & can. \\
407 & 0 & 0 & 0 & 0 & 0 & 0 & 0 & 1 & 0 & 0 & 2 & 2 & 4 & 6 & 8 & 10 & 32 & 4 & $w_{195}$ & N & can. \\
408 & 0 & 0 & 0 & 0 & 0 & 0 & 0 & 1 & 0 & 0 & 2 & 2 & 4 & 6 & 8 & 11 & 16 & 3 & $w_{90}$ & N & can. \\
409 & 0 & 0 & 0 & 0 & 0 & 0 & 0 & 1 & 0 & 0 & 2 & 2 & 4 & 6 & 8 & 12 & 8 & 4 & $w_{177}$ & N & can. \\
410 & 0 & 0 & 0 & 0 & 0 & 0 & 0 & 1 & 0 & 0 & 2 & 2 & 4 & 6 & 8 & 13 & 8 & 4 & $w_{183}$ & N & can. \\
411 & 0 & 0 & 0 & 0 & 0 & 0 & 0 & 1 & 0 & 0 & 2 & 2 & 4 & 7 & 4 & 7 & 256 & 4 & $w_{67}$ & N & can. \\
412 & 0 & 0 & 0 & 0 & 0 & 0 & 0 & 1 & 0 & 0 & 2 & 2 & 4 & 7 & 4 & 8 & 4 & 4 & $w_{216}$ & N & can. \\
413 & 0 & 0 & 0 & 0 & 0 & 0 & 0 & 1 & 0 & 0 & 2 & 2 & 4 & 7 & 5 & 6 & 128 & 4 & $w_{67}$ & N & can. \\
414 & 0 & 0 & 0 & 0 & 0 & 0 & 0 & 1 & 0 & 0 & 2 & 2 & 4 & 7 & 5 & 8 & 4 & 4 & $w_{216}$ & N & can. \\
415 & 0 & 0 & 0 & 0 & 0 & 0 & 0 & 1 & 0 & 0 & 2 & 2 & 4 & 7 & 8 & 11 & 32 & 4 & $w_{87}$ & N & can. \\
416 & 0 & 0 & 0 & 0 & 0 & 0 & 0 & 1 & 0 & 0 & 2 & 2 & 4 & 7 & 8 & 12 & 4 & 4 & $w_{91}$ & N & can. \\
417 & 0 & 0 & 0 & 0 & 0 & 0 & 0 & 1 & 0 & 0 & 2 & 2 & 4 & 8 & 4 & 8 & 32 & 4 & $w_{143}$ & N & can. \\
418 & 0 & 0 & 0 & 0 & 0 & 0 & 0 & 1 & 0 & 0 & 2 & 2 & 4 & 8 & 4 & 9 & 8 & 3 & $w_{71}$ & N & can. \\
419 & 0 & 0 & 0 & 0 & 0 & 0 & 0 & 1 & 0 & 0 & 2 & 2 & 4 & 8 & 4 & 10 & 8 & 4 & $w_{69}$ & N & can. \\
420 & 0 & 0 & 0 & 0 & 0 & 0 & 0 & 1 & 0 & 0 & 2 & 2 & 4 & 8 & 4 & 11 & 8 & 4 & $w_{146}$ & N & can. \\
421 & 0 & 0 & 0 & 0 & 0 & 0 & 0 & 1 & 0 & 0 & 2 & 2 & 4 & 8 & 4 & 12 & 4 & 4 & $w_{135}$ & N & can. \\
422 & 0 & 0 & 0 & 0 & 0 & 0 & 0 & 1 & 0 & 0 & 2 & 2 & 4 & 8 & 4 & 13 & 4 & 4 & $w_{137}$ & N & can. \\
423 & 0 & 0 & 0 & 0 & 0 & 0 & 0 & 1 & 0 & 0 & 2 & 2 & 4 & 8 & 5 & 9 & 16 & 4 & $w_{146}$ & N & can. \\
424 & 0 & 0 & 0 & 0 & 0 & 0 & 0 & 1 & 0 & 0 & 2 & 2 & 4 & 8 & 5 & 10 & 8 & 4 & $w_{146}$ & N & can. \\
425 & 0 & 0 & 0 & 0 & 0 & 0 & 0 & 1 & 0 & 0 & 2 & 2 & 4 & 8 & 5 & 11 & 8 & 4 & $w_{87}$ & N & can. \\
426 & 0 & 0 & 0 & 0 & 0 & 0 & 0 & 1 & 0 & 0 & 2 & 2 & 4 & 8 & 5 & 12 & 2 & 4 & $w_{175}$ & N & can. \\
427 & 0 & 0 & 0 & 0 & 0 & 0 & 0 & 1 & 0 & 0 & 2 & 2 & 4 & 8 & 6 & 10 & 16 & 4 & $w_{217}$ & N & can. \\
428 & 0 & 0 & 0 & 0 & 0 & 0 & 0 & 1 & 0 & 0 & 2 & 2 & 4 & 8 & 6 & 11 & 8 & 3 & $w_{86}$ & Y & can. \\
429 & 0 & 0 & 0 & 0 & 0 & 0 & 0 & 1 & 0 & 0 & 2 & 2 & 4 & 8 & 6 & 12 & 4 & 4 & $w_{187}$ & N & can. \\
430 & 0 & 0 & 0 & 0 & 0 & 0 & 0 & 1 & 0 & 0 & 2 & 2 & 4 & 8 & 6 & 13 & 4 & 4 & $w_{218}$ & N & can. \\
431 & 0 & 0 & 0 & 0 & 0 & 0 & 0 & 1 & 0 & 0 & 2 & 2 & 4 & 8 & 7 & 11 & 16 & 4 & $w_{219}$ & N & can. \\
432 & 0 & 0 & 0 & 0 & 0 & 0 & 0 & 1 & 0 & 0 & 2 & 2 & 4 & 8 & 7 & 12 & 2 & 4 & $w_{220}$ & N & can. \\
433 & 0 & 0 & 0 & 0 & 0 & 0 & 0 & 1 & 0 & 0 & 2 & 3 & 4 & 5 & 6 & 7 & 384 & 3 & $w_{80}$ & N & can. \\
434 & 0 & 0 & 0 & 0 & 0 & 0 & 0 & 1 & 0 & 0 & 2 & 3 & 4 & 5 & 6 & 8 & 4 & 4 & $w_{221}$ & N & can. \\
435 & 0 & 0 & 0 & 0 & 0 & 0 & 0 & 1 & 0 & 0 & 2 & 3 & 4 & 5 & 8 & 9 & 48 & 3 & $w_{86}$ & N & can. \\
436 & 0 & 0 & 0 & 0 & 0 & 0 & 0 & 1 & 0 & 0 & 2 & 3 & 4 & 5 & 8 & 10 & 4 & 4 & $w_{200}$ & N & can. \\
437 & 0 & 0 & 0 & 0 & 0 & 0 & 0 & 1 & 0 & 0 & 2 & 3 & 4 & 5 & 8 & 14 & 8 & 4 & $w_{200}$ & N & can. \\
438 & 0 & 0 & 0 & 0 & 0 & 0 & 0 & 1 & 0 & 0 & 2 & 3 & 4 & 6 & 4 & 7 & 32 & 4 & $w_{193}$ & N & can. \\
439 & 0 & 0 & 0 & 0 & 0 & 0 & 0 & 1 & 0 & 0 & 2 & 3 & 4 & 6 & 4 & 8 & 1 & 4 & $w_{215}$ & N & can. \\
440 & 0 & 0 & 0 & 0 & 0 & 0 & 0 & 1 & 0 & 0 & 2 & 3 & 4 & 6 & 5 & 7 & 32 & 3 & $w_{84}$ & N & can. \\
441 & 0 & 0 & 0 & 0 & 0 & 0 & 0 & 1 & 0 & 0 & 2 & 3 & 4 & 6 & 5 & 8 & 1 & 4 & $w_{222}$ & N & can. \\
442 & 0 & 0 & 0 & 0 & 0 & 0 & 0 & 1 & 0 & 0 & 2 & 3 & 4 & 6 & 8 & 10 & 8 & 3 & $w_{223}$ & N & can. \\
443 & 0 & 0 & 0 & 0 & 0 & 0 & 0 & 1 & 0 & 0 & 2 & 3 & 4 & 6 & 8 & 11 & 8 & 4 & $w_{224}$ & N & can. \\
444 & 0 & 0 & 0 & 0 & 0 & 0 & 0 & 1 & 0 & 0 & 2 & 3 & 4 & 6 & 8 & 12 & 2 & 4 & $w_{225}$ & N & can. \\
445 & 0 & 0 & 0 & 0 & 0 & 0 & 0 & 1 & 0 & 0 & 2 & 3 & 4 & 6 & 8 & 13 & 2 & 4 & $w_{226}$ & N & can. \\
446 & 0 & 0 & 0 & 0 & 0 & 0 & 0 & 1 & 0 & 0 & 2 & 3 & 4 & 8 & 4 & 9 & 8 & 4 & $w_{205}$ & N & can. \\
447 & 0 & 0 & 0 & 0 & 0 & 0 & 0 & 1 & 0 & 0 & 2 & 3 & 4 & 8 & 4 & 10 & 2 & 4 & $w_{206}$ & N & can. \\
448 & 0 & 0 & 0 & 0 & 0 & 0 & 0 & 1 & 0 & 0 & 2 & 3 & 4 & 8 & 4 & 12 & 4 & 4 & $w_{227}$ & N & can. \\
449 & 0 & 0 & 0 & 0 & 0 & 0 & 0 & 1 & 0 & 0 & 2 & 3 & 4 & 8 & 4 & 13 & 4 & 4 & $w_{91}$ & N & can. \\
450 & 0 & 0 & 0 & 0 & 0 & 0 & 0 & 1 & 0 & 0 & 2 & 3 & 4 & 8 & 4 & 14 & 2 & 4 & $w_{183}$ & N & can. \\
451 & 0 & 0 & 0 & 0 & 0 & 0 & 0 & 1 & 0 & 0 & 2 & 3 & 4 & 8 & 5 & 9 & 8 & 3 & $w_{90}$ & N & can. \\
452 & 0 & 0 & 0 & 0 & 0 & 0 & 0 & 1 & 0 & 0 & 2 & 3 & 4 & 8 & 5 & 10 & 2 & 4 & $w_{200}$ & N & can. \\
453 & 0 & 0 & 0 & 0 & 0 & 0 & 0 & 1 & 0 & 0 & 2 & 3 & 4 & 8 & 5 & 12 & 2 & 4 & $w_{202}$ & N & can. \\
454 & 0 & 0 & 0 & 0 & 0 & 0 & 0 & 1 & 0 & 0 & 2 & 3 & 4 & 8 & 5 & 14 & 2 & 4 & $w_{202}$ & N & can. \\
455 & 0 & 0 & 0 & 0 & 0 & 0 & 0 & 1 & 0 & 0 & 2 & 3 & 4 & 8 & 6 & 10 & 4 & 3 & $w_{228}$ & Y & can. \\
456 & 0 & 0 & 0 & 0 & 0 & 0 & 0 & 1 & 0 & 0 & 2 & 3 & 4 & 8 & 6 & 11 & 4 & 4 & $w_{229}$ & N & can. \\
457 & 0 & 0 & 0 & 0 & 0 & 0 & 0 & 1 & 0 & 0 & 2 & 3 & 4 & 8 & 6 & 12 & 1 & 4 & $w_{230}$ & N & can. \\
458 & 0 & 0 & 0 & 0 & 0 & 0 & 0 & 1 & 0 & 0 & 2 & 3 & 4 & 8 & 6 & 13 & 1 & 4 & $w_{231}$ & N & can. \\
459 & 0 & 0 & 0 & 0 & 0 & 0 & 0 & 1 & 0 & 0 & 2 & 4 & 2 & 6 & 4 & 6 & 192 & 4 & $w_{232}$ & N & can. \\
460 & 0 & 0 & 0 & 0 & 0 & 0 & 0 & 1 & 0 & 0 & 2 & 4 & 2 & 6 & 4 & 7 & 24 & 3 & $w_{165}$ & Y & can. \\
461 & 0 & 0 & 0 & 0 & 0 & 0 & 0 & 1 & 0 & 0 & 2 & 4 & 2 & 6 & 4 & 8 & 3 & 4 & $w_{166}$ & N & can. \\
462 & 0 & 0 & 0 & 0 & 0 & 0 & 0 & 1 & 0 & 0 & 2 & 4 & 2 & 6 & 5 & 7 & 32 & 4 & $w_{176}$ & N & can. \\
463 & 0 & 0 & 0 & 0 & 0 & 0 & 0 & 1 & 0 & 0 & 2 & 4 & 2 & 6 & 5 & 8 & 1 & 4 & $w_{177}$ & N & can. \\
464 & 0 & 0 & 0 & 0 & 0 & 0 & 0 & 1 & 0 & 0 & 2 & 4 & 2 & 6 & 8 & 10 & 8 & 4 & $w_{233}$ & N & can. \\
465 & 0 & 0 & 0 & 0 & 0 & 0 & 0 & 1 & 0 & 0 & 2 & 4 & 2 & 6 & 8 & 11 & 4 & 3 & $w_{181}$ & Y & can. \\
466 & 0 & 0 & 0 & 0 & 0 & 0 & 0 & 1 & 0 & 0 & 2 & 4 & 2 & 6 & 8 & 12 & 2 & 4 & $w_{177}$ & N & can. \\
467 & 0 & 0 & 0 & 0 & 0 & 0 & 0 & 1 & 0 & 0 & 2 & 4 & 2 & 6 & 8 & 13 & 2 & 4 & $w_{182}$ & N & can. \\
468 & 0 & 0 & 0 & 0 & 0 & 0 & 0 & 1 & 0 & 0 & 2 & 4 & 2 & 7 & 4 & 7 & 192 & 4 & $w_{68}$ & N & can. \\
469 & 0 & 0 & 0 & 0 & 0 & 0 & 0 & 1 & 0 & 0 & 2 & 4 & 2 & 7 & 4 & 8 & 3 & 4 & $w_{172}$ & N & can. \\
470 & 0 & 0 & 0 & 0 & 0 & 0 & 0 & 1 & 0 & 0 & 2 & 4 & 2 & 7 & 5 & 6 & 32 & 4 & $w_{168}$ & N & can. \\
471 & 0 & 0 & 0 & 0 & 0 & 0 & 0 & 1 & 0 & 0 & 2 & 4 & 2 & 7 & 5 & 8 & 1 & 4 & $w_{170}$ & N & can. \\
472 & 0 & 0 & 0 & 0 & 0 & 0 & 0 & 1 & 0 & 0 & 2 & 4 & 2 & 7 & 8 & 11 & 8 & 4 & $w_{203}$ & N & can. \\
473 & 0 & 0 & 0 & 0 & 0 & 0 & 0 & 1 & 0 & 0 & 2 & 4 & 2 & 7 & 8 & 12 & 2 & 4 & $w_{203}$ & N & can. \\
474 & 0 & 0 & 0 & 0 & 0 & 0 & 0 & 1 & 0 & 0 & 2 & 4 & 2 & 7 & 8 & 13 & 2 & 4 & $w_{91}$ & N & can. \\
475 & 0 & 0 & 0 & 0 & 0 & 0 & 0 & 1 & 0 & 0 & 2 & 4 & 2 & 8 & 4 & 8 & 24 & 4 & $w_{116}$ & N & can. \\
476 & 0 & 0 & 0 & 0 & 0 & 0 & 0 & 1 & 0 & 0 & 2 & 4 & 2 & 8 & 4 & 9 & 6 & 3 & $w_{136}$ & Y & can. \\
477 & 0 & 0 & 0 & 0 & 0 & 0 & 0 & 1 & 0 & 0 & 2 & 4 & 2 & 8 & 4 & 10 & 2 & 4 & $w_{135}$ & N & can. \\
478 & 0 & 0 & 0 & 0 & 0 & 0 & 0 & 1 & 0 & 0 & 2 & 4 & 2 & 8 & 4 & 11 & 2 & 4 & $w_{148}$ & N & can. \\
479 & 0 & 0 & 0 & 0 & 0 & 0 & 0 & 1 & 0 & 0 & 2 & 4 & 2 & 8 & 5 & 9 & 4 & 4 & $w_{174}$ & N & can. \\
480 & 0 & 0 & 0 & 0 & 0 & 0 & 0 & 1 & 0 & 0 & 2 & 4 & 2 & 8 & 5 & 10 & 2 & 4 & $w_{174}$ & N & can. \\
481 & 0 & 0 & 0 & 0 & 0 & 0 & 0 & 1 & 0 & 0 & 2 & 4 & 2 & 8 & 5 & 11 & 2 & 4 & $w_{91}$ & N & can. \\
482 & 0 & 0 & 0 & 0 & 0 & 0 & 0 & 1 & 0 & 0 & 2 & 4 & 2 & 8 & 5 & 12 & 1 & 4 & $w_{175}$ & N & can. \\
483 & 0 & 0 & 0 & 0 & 0 & 0 & 0 & 1 & 0 & 0 & 2 & 4 & 2 & 8 & 5 & 13 & 1 & 4 & $w_{183}$ & N & can. \\
484 & 0 & 0 & 0 & 0 & 0 & 0 & 0 & 1 & 0 & 0 & 2 & 4 & 2 & 8 & 6 & 10 & 4 & 4 & $w_{195}$ & N & can. \\
485 & 0 & 0 & 0 & 0 & 0 & 0 & 0 & 1 & 0 & 0 & 2 & 4 & 2 & 8 & 6 & 11 & 2 & 3 & $w_{90}$ & Y & can. \\
486 & 0 & 0 & 0 & 0 & 0 & 0 & 0 & 1 & 0 & 0 & 2 & 4 & 2 & 8 & 6 & 12 & 2 & 4 & $w_{187}$ & N & can. \\
487 & 0 & 0 & 0 & 0 & 0 & 0 & 0 & 1 & 0 & 0 & 2 & 4 & 2 & 8 & 6 & 13 & 2 & 4 & $w_{234}$ & N & can. \\
488 & 0 & 0 & 0 & 0 & 0 & 0 & 0 & 1 & 0 & 0 & 2 & 4 & 2 & 8 & 6 & 14 & 2 & 4 & $w_{177}$ & N & can. \\
489 & 0 & 0 & 0 & 0 & 0 & 0 & 0 & 1 & 0 & 0 & 2 & 4 & 2 & 8 & 6 & 15 & 2 & 4 & $w_{183}$ & N & can. \\
490 & 0 & 0 & 0 & 0 & 0 & 0 & 0 & 1 & 0 & 0 & 2 & 4 & 2 & 8 & 7 & 11 & 4 & 4 & $w_{200}$ & N & can. \\
491 & 0 & 0 & 0 & 0 & 0 & 0 & 0 & 1 & 0 & 0 & 2 & 4 & 2 & 8 & 7 & 12 & 1 & 4 & $w_{220}$ & N & can. \\
492 & 0 & 0 & 0 & 0 & 0 & 0 & 0 & 1 & 0 & 0 & 2 & 4 & 2 & 8 & 7 & 14 & 1 & 4 & $w_{203}$ & N & can. \\
493 & 0 & 0 & 0 & 0 & 0 & 0 & 0 & 1 & 0 & 0 & 2 & 4 & 3 & 5 & 6 & 8 & 4 & 4 & $w_{222}$ & N & can. \\
494 & 0 & 0 & 0 & 0 & 0 & 0 & 0 & 1 & 0 & 0 & 2 & 4 & 3 & 5 & 7 & 8 & 4 & 4 & $w_{85}$ & N & can. \\
495 & 0 & 0 & 0 & 0 & 0 & 0 & 0 & 1 & 0 & 0 & 2 & 4 & 3 & 5 & 8 & 10 & 2 & 4 & $w_{203}$ & N & can. \\
496 & 0 & 0 & 0 & 0 & 0 & 0 & 0 & 1 & 0 & 0 & 2 & 4 & 3 & 5 & 8 & 14 & 8 & 4 & $w_{87}$ & N & can. \\
497 & 0 & 0 & 0 & 0 & 0 & 0 & 0 & 1 & 0 & 0 & 2 & 4 & 3 & 5 & 8 & 15 & 8 & 4 & $w_{200}$ & N & can. \\
498 & 0 & 0 & 0 & 0 & 0 & 0 & 0 & 1 & 0 & 0 & 2 & 4 & 3 & 6 & 5 & 7 & 24 & 3 & $w_{88}$ & N & can. \\
499 & 0 & 0 & 0 & 0 & 0 & 0 & 0 & 1 & 0 & 0 & 2 & 4 & 3 & 6 & 5 & 8 & 1 & 4 & $w_{235}$ & N & can. \\
500 & 0 & 0 & 0 & 0 & 0 & 0 & 0 & 1 & 0 & 0 & 2 & 4 & 3 & 6 & 7 & 8 & 1 & 4 & $w_{236}$ & N & can. \\
501 & 0 & 0 & 0 & 0 & 0 & 0 & 0 & 1 & 0 & 0 & 2 & 4 & 3 & 6 & 8 & 10 & 2 & 3 & $w_{237}$ & Y & can. \\
502 & 0 & 0 & 0 & 0 & 0 & 0 & 0 & 1 & 0 & 0 & 2 & 4 & 3 & 6 & 8 & 11 & 2 & 4 & $w_{225}$ & N & can. \\
503 & 0 & 0 & 0 & 0 & 0 & 0 & 0 & 1 & 0 & 0 & 2 & 4 & 3 & 6 & 8 & 12 & 2 & 4 & $w_{225}$ & N & can. \\
504 & 0 & 0 & 0 & 0 & 0 & 0 & 0 & 1 & 0 & 0 & 2 & 4 & 3 & 6 & 8 & 13 & 2 & 4 & $w_{238}$ & N & can. \\
505 & 0 & 0 & 0 & 0 & 0 & 0 & 0 & 1 & 0 & 0 & 2 & 4 & 3 & 6 & 8 & 14 & 2 & 4 & $w_{202}$ & N & can. \\
506 & 0 & 0 & 0 & 0 & 0 & 0 & 0 & 1 & 0 & 0 & 2 & 4 & 3 & 6 & 8 & 15 & 2 & 4 & $w_{226}$ & N & can. \\
507 & 0 & 0 & 0 & 0 & 0 & 0 & 0 & 1 & 0 & 0 & 2 & 4 & 3 & 8 & 5 & 9 & 6 & 3 & $w_{181}$ & N & can. \\
508 & 0 & 0 & 0 & 0 & 0 & 0 & 0 & 1 & 0 & 0 & 2 & 4 & 3 & 8 & 5 & 10 & 1 & 4 & $w_{203}$ & N & can. \\
509 & 0 & 0 & 0 & 0 & 0 & 0 & 0 & 1 & 0 & 0 & 2 & 4 & 3 & 8 & 5 & 11 & 1 & 4 & $w_{202}$ & N & can. \\
510 & 0 & 0 & 0 & 0 & 0 & 0 & 0 & 1 & 0 & 0 & 2 & 4 & 3 & 8 & 5 & 14 & 2 & 4 & $w_{203}$ & N & can. \\
511 & 0 & 0 & 0 & 0 & 0 & 0 & 0 & 1 & 0 & 0 & 2 & 4 & 3 & 8 & 5 & 15 & 2 & 4 & $w_{202}$ & N & can. \\
512 & 0 & 0 & 0 & 0 & 0 & 0 & 0 & 1 & 0 & 0 & 2 & 4 & 3 & 8 & 6 & 10 & 1 & 3 & $w_{223}$ & Y & can. \\
513 & 0 & 0 & 0 & 0 & 0 & 0 & 0 & 1 & 0 & 0 & 2 & 4 & 3 & 8 & 6 & 11 & 2 & 4 & $w_{224}$ & N & can. \\
514 & 0 & 0 & 0 & 0 & 0 & 0 & 0 & 1 & 0 & 0 & 2 & 4 & 3 & 8 & 6 & 12 & 1 & 4 & $w_{230}$ & N & can. \\
515 & 0 & 0 & 0 & 0 & 0 & 0 & 0 & 1 & 0 & 0 & 2 & 4 & 3 & 8 & 6 & 13 & 1 & 4 & $w_{231}$ & N & can. \\
516 & 0 & 0 & 0 & 0 & 0 & 0 & 0 & 1 & 0 & 0 & 2 & 4 & 3 & 8 & 6 & 14 & 1 & 4 & $w_{225}$ & N & can. \\
517 & 0 & 0 & 0 & 0 & 0 & 0 & 0 & 1 & 0 & 0 & 2 & 4 & 3 & 8 & 6 & 15 & 1 & 4 & $w_{226}$ & N & can. \\
518 & 0 & 0 & 0 & 0 & 0 & 0 & 0 & 1 & 0 & 0 & 2 & 4 & 3 & 8 & 7 & 10 & 2 & 4 & $w_{200}$ & N & can. \\
519 & 0 & 0 & 0 & 0 & 0 & 0 & 0 & 1 & 0 & 0 & 2 & 4 & 3 & 8 & 7 & 12 & 1 & 4 & $w_{220}$ & N & can. \\
520 & 0 & 0 & 0 & 0 & 0 & 0 & 0 & 1 & 0 & 0 & 2 & 4 & 3 & 8 & 7 & 13 & 1 & 4 & $w_{231}$ & N & can. \\
521 & 0 & 0 & 0 & 0 & 0 & 0 & 0 & 1 & 0 & 0 & 2 & 4 & 3 & 8 & 7 & 14 & 1 & 4 & $w_{238}$ & N & can. \\
522 & 0 & 0 & 0 & 0 & 0 & 0 & 0 & 1 & 0 & 0 & 2 & 4 & 3 & 8 & 7 & 15 & 1 & 4 & $w_{202}$ & N & can. \\
523 & 0 & 0 & 0 & 0 & 0 & 0 & 0 & 1 & 0 & 0 & 2 & 4 & 3 & 8 & 12 & 14 & 1 & 4 & $w_{202}$ & N & can. \\
524 & 0 & 0 & 0 & 0 & 0 & 0 & 0 & 1 & 0 & 0 & 2 & 4 & 3 & 8 & 13 & 14 & 1 & 4 & $w_{226}$ & N & can. \\
525 & 0 & 0 & 0 & 0 & 0 & 0 & 0 & 1 & 0 & 0 & 2 & 4 & 6 & 8 & 10 & 12 & 4 & 4 & $w_{222}$ & N & can. \\
526 & 0 & 0 & 0 & 0 & 0 & 0 & 0 & 1 & 0 & 0 & 2 & 4 & 6 & 8 & 10 & 13 & 1 & 4 & $w_{239}$ & N & can. \\
527 & 0 & 0 & 0 & 0 & 0 & 0 & 0 & 1 & 0 & 0 & 2 & 4 & 6 & 8 & 10 & 14 & 3 & 4 & $w_{240}$ & N & can. \\
528 & 0 & 0 & 0 & 0 & 0 & 0 & 0 & 1 & 0 & 0 & 2 & 4 & 6 & 8 & 10 & 15 & 1 & 4 & $w_{226}$ & N & can. \\
529 & 0 & 0 & 0 & 0 & 0 & 0 & 0 & 1 & 0 & 0 & 2 & 4 & 6 & 8 & 11 & 13 & 4 & 4 & $w_{239}$ & N & can. \\
530 & 0 & 0 & 0 & 0 & 0 & 0 & 0 & 1 & 0 & 0 & 2 & 4 & 6 & 8 & 11 & 14 & 1 & 4 & $w_{241}$ & N & can. \\
531 & 0 & 0 & 0 & 0 & 0 & 0 & 0 & 1 & 0 & 0 & 2 & 4 & 7 & 8 & 10 & 12 & 4 & 4 & $w_{242}$ & N & can. \\
532 & 0 & 0 & 0 & 0 & 0 & 0 & 0 & 1 & 0 & 0 & 2 & 4 & 7 & 8 & 11 & 13 & 4 & 4 & $w_{243}$ & N & can. \\
533 & 0 & 0 & 0 & 0 & 0 & 0 & 0 & 1 & 0 & 0 & 2 & 4 & 7 & 8 & 11 & 14 & 3 & 4 & $w_{244}$ & N & can. \\
534 & 0 & 0 & 0 & 0 & 0 & 0 & 0 & 1 & 0 & 0 & 2 & 4 & 8 & 14 & 10 & 12 & 24 & 4 & $w_{81}$ & N & can. \\
535 & 0 & 0 & 0 & 0 & 0 & 0 & 0 & 1 & 0 & 0 & 2 & 4 & 8 & 14 & 10 & 13 & 4 & 4 & $w_{245}$ & N & can. \\
536 & 0 & 0 & 0 & 0 & 0 & 0 & 0 & 1 & 0 & 0 & 2 & 4 & 8 & 14 & 11 & 13 & 24 & 4 & $w_{246}$ & N & can. \\
537 & 0 & 0 & 0 & 0 & 0 & 0 & 0 & 1 & 0 & 1 & 2 & 3 & 4 & 5 & 7 & 6 & 1536 & 4 & $w_{247}$ & N & can. \\
538 & 0 & 0 & 0 & 0 & 0 & 0 & 0 & 1 & 0 & 1 & 2 & 3 & 4 & 5 & 7 & 8 & 24 & 4 & $w_{248}$ & N & can. \\
539 & 0 & 0 & 0 & 0 & 0 & 0 & 0 & 1 & 0 & 1 & 2 & 3 & 4 & 5 & 8 & 9 & 192 & 4 & $w_{213}$ & N & can. \\
540 & 0 & 0 & 0 & 0 & 0 & 0 & 0 & 1 & 0 & 1 & 2 & 3 & 4 & 5 & 8 & 10 & 8 & 4 & $w_{200}$ & N & can. \\
541 & 0 & 0 & 0 & 0 & 0 & 0 & 0 & 1 & 0 & 1 & 2 & 3 & 4 & 6 & 5 & 7 & 256 & 4 & $w_{193}$ & N & can. \\
542 & 0 & 0 & 0 & 0 & 0 & 0 & 0 & 1 & 0 & 1 & 2 & 3 & 4 & 6 & 5 & 8 & 4 & 4 & $w_{222}$ & N & can. \\
543 & 0 & 0 & 0 & 0 & 0 & 0 & 0 & 1 & 0 & 1 & 2 & 3 & 4 & 6 & 7 & 5 & 256 & 4 & $w_{249}$ & N & can. \\
544 & 0 & 0 & 0 & 0 & 0 & 0 & 0 & 1 & 0 & 1 & 2 & 3 & 4 & 6 & 7 & 8 & 4 & 4 & $w_{250}$ & N & can. \\
545 & 0 & 0 & 0 & 0 & 0 & 0 & 0 & 1 & 0 & 1 & 2 & 3 & 4 & 6 & 8 & 10 & 16 & 4 & $w_{224}$ & N & can. \\
546 & 0 & 0 & 0 & 0 & 0 & 0 & 0 & 1 & 0 & 1 & 2 & 3 & 4 & 6 & 8 & 11 & 16 & 3 & $w_{251}$ & N & can. \\
547 & 0 & 0 & 0 & 0 & 0 & 0 & 0 & 1 & 0 & 1 & 2 & 3 & 4 & 6 & 8 & 12 & 2 & 4 & $w_{241}$ & N & can. \\
548 & 0 & 0 & 0 & 0 & 0 & 0 & 0 & 1 & 0 & 1 & 2 & 3 & 4 & 8 & 5 & 9 & 32 & 4 & $w_{196}$ & N & can. \\
549 & 0 & 0 & 0 & 0 & 0 & 0 & 0 & 1 & 0 & 1 & 2 & 3 & 4 & 8 & 5 & 10 & 8 & 4 & $w_{200}$ & N & can. \\
550 & 0 & 0 & 0 & 0 & 0 & 0 & 0 & 1 & 0 & 1 & 2 & 3 & 4 & 8 & 5 & 11 & 8 & 4 & $w_{224}$ & N & can. \\
551 & 0 & 0 & 0 & 0 & 0 & 0 & 0 & 1 & 0 & 1 & 2 & 3 & 4 & 8 & 5 & 12 & 2 & 4 & $w_{225}$ & N & can. \\
552 & 0 & 0 & 0 & 0 & 0 & 0 & 0 & 1 & 0 & 1 & 2 & 3 & 4 & 8 & 6 & 10 & 16 & 4 & $w_{219}$ & N & can. \\
553 & 0 & 0 & 0 & 0 & 0 & 0 & 0 & 1 & 0 & 1 & 2 & 3 & 4 & 8 & 6 & 11 & 8 & 3 & $w_{252}$ & N & can. \\
554 & 0 & 0 & 0 & 0 & 0 & 0 & 0 & 1 & 0 & 1 & 2 & 3 & 4 & 8 & 6 & 12 & 2 & 4 & $w_{231}$ & N & can. \\
555 & 0 & 0 & 0 & 0 & 0 & 0 & 0 & 1 & 0 & 1 & 2 & 3 & 4 & 8 & 7 & 11 & 16 & 4 & $w_{253}$ & N & can. \\
556 & 0 & 0 & 0 & 0 & 0 & 0 & 0 & 1 & 0 & 1 & 2 & 3 & 4 & 8 & 7 & 12 & 2 & 4 & $w_{254}$ & N & can. \\
557 & 0 & 0 & 0 & 0 & 0 & 0 & 0 & 1 & 0 & 1 & 2 & 4 & 3 & 6 & 5 & 7 & 192 & 4 & $w_{255}$ & N & can. \\
558 & 0 & 0 & 0 & 0 & 0 & 0 & 0 & 1 & 0 & 1 & 2 & 4 & 3 & 6 & 5 & 8 & 3 & 4 & $w_{240}$ & N & can. \\
559 & 0 & 0 & 0 & 0 & 0 & 0 & 0 & 1 & 0 & 1 & 2 & 4 & 3 & 6 & 7 & 5 & 192 & 4 & $w_{256}$ & N & can. \\
560 & 0 & 0 & 0 & 0 & 0 & 0 & 0 & 1 & 0 & 1 & 2 & 4 & 3 & 6 & 7 & 8 & 3 & 4 & $w_{257}$ & N & can. \\
561 & 0 & 0 & 0 & 0 & 0 & 0 & 0 & 1 & 0 & 1 & 2 & 4 & 3 & 6 & 8 & 10 & 4 & 4 & $w_{258}$ & N & can. \\
562 & 0 & 0 & 0 & 0 & 0 & 0 & 0 & 1 & 0 & 1 & 2 & 4 & 3 & 6 & 8 & 11 & 4 & 3 & $w_{259}$ & N & can. \\
563 & 0 & 0 & 0 & 0 & 0 & 0 & 0 & 1 & 0 & 1 & 2 & 4 & 3 & 6 & 8 & 12 & 1 & 4 & $w_{260}$ & N & can. \\
564 & 0 & 0 & 0 & 0 & 0 & 0 & 0 & 1 & 0 & 1 & 2 & 4 & 3 & 6 & 8 & 13 & 1 & 4 & $w_{241}$ & N & can. \\
565 & 0 & 0 & 0 & 0 & 0 & 0 & 0 & 1 & 0 & 1 & 2 & 4 & 3 & 8 & 5 & 9 & 24 & 4 & $w_{261}$ & N & can. \\
566 & 0 & 0 & 0 & 0 & 0 & 0 & 0 & 1 & 0 & 1 & 2 & 4 & 3 & 8 & 5 & 10 & 2 & 4 & $w_{225}$ & N & can. \\
567 & 0 & 0 & 0 & 0 & 0 & 0 & 0 & 1 & 0 & 1 & 2 & 4 & 3 & 8 & 5 & 11 & 2 & 4 & $w_{262}$ & N & can. \\
568 & 0 & 0 & 0 & 0 & 0 & 0 & 0 & 1 & 0 & 1 & 2 & 4 & 3 & 8 & 6 & 10 & 4 & 4 & $w_{224}$ & N & can. \\
569 & 0 & 0 & 0 & 0 & 0 & 0 & 0 & 1 & 0 & 1 & 2 & 4 & 3 & 8 & 6 & 11 & 2 & 3 & $w_{251}$ & N & can. \\
570 & 0 & 0 & 0 & 0 & 0 & 0 & 0 & 1 & 0 & 1 & 2 & 4 & 3 & 8 & 6 & 12 & 2 & 4 & $w_{254}$ & N & can. \\
571 & 0 & 0 & 0 & 0 & 0 & 0 & 0 & 1 & 0 & 1 & 2 & 4 & 3 & 8 & 6 & 13 & 2 & 4 & $w_{231}$ & N & can. \\
572 & 0 & 0 & 0 & 0 & 0 & 0 & 0 & 1 & 0 & 1 & 2 & 4 & 3 & 8 & 6 & 14 & 2 & 4 & $w_{241}$ & N & can. \\
573 & 0 & 0 & 0 & 0 & 0 & 0 & 0 & 1 & 0 & 1 & 2 & 4 & 3 & 8 & 6 & 15 & 2 & 4 & $w_{241}$ & N & can. \\
574 & 0 & 0 & 0 & 0 & 0 & 0 & 0 & 1 & 0 & 1 & 2 & 4 & 3 & 8 & 7 & 11 & 4 & 4 & $w_{263}$ & N & can. \\
575 & 0 & 0 & 0 & 0 & 0 & 0 & 0 & 1 & 0 & 1 & 2 & 4 & 3 & 8 & 7 & 12 & 1 & 4 & $w_{254}$ & N & can. \\
576 & 0 & 0 & 0 & 0 & 0 & 0 & 0 & 1 & 0 & 1 & 2 & 4 & 3 & 8 & 7 & 14 & 1 & 4 & $w_{260}$ & N & can. \\
577 & 0 & 0 & 0 & 0 & 0 & 0 & 0 & 1 & 0 & 1 & 2 & 4 & 6 & 8 & 10 & 12 & 4 & 4 & $w_{264}$ & N & can. \\
578 & 0 & 0 & 0 & 0 & 0 & 0 & 0 & 1 & 0 & 1 & 2 & 4 & 6 & 8 & 10 & 13 & 1 & 4 & $w_{265}$ & N & can. \\
579 & 0 & 0 & 0 & 0 & 0 & 0 & 0 & 1 & 0 & 1 & 2 & 4 & 6 & 8 & 10 & 14 & 3 & 4 & $w_{266}$ & N & can. \\
580 & 0 & 0 & 0 & 0 & 0 & 0 & 0 & 1 & 0 & 1 & 2 & 4 & 6 & 8 & 10 & 15 & 1 & 4 & $w_{267}$ & N & can. \\
581 & 0 & 0 & 0 & 0 & 0 & 0 & 0 & 1 & 0 & 1 & 2 & 4 & 6 & 8 & 11 & 13 & 4 & 4 & $w_{243}$ & N & can. \\
582 & 0 & 0 & 0 & 0 & 0 & 0 & 0 & 1 & 0 & 1 & 2 & 4 & 6 & 8 & 11 & 14 & 1 & 4 & $w_{268}$ & N & can. \\
583 & 0 & 0 & 0 & 0 & 0 & 0 & 0 & 1 & 0 & 1 & 2 & 4 & 6 & 8 & 12 & 10 & 4 & 4 & $w_{239}$ & N & can. \\
584 & 0 & 0 & 0 & 0 & 0 & 0 & 0 & 1 & 0 & 1 & 2 & 4 & 6 & 8 & 12 & 14 & 3 & 4 & $w_{260}$ & N & can. \\
585 & 0 & 0 & 0 & 0 & 0 & 0 & 0 & 1 & 0 & 1 & 2 & 4 & 6 & 8 & 13 & 11 & 4 & 4 & $w_{265}$ & N & can. \\
586 & 0 & 0 & 0 & 0 & 0 & 0 & 0 & 1 & 0 & 1 & 2 & 4 & 8 & 14 & 10 & 12 & 48 & 4 & $w_{269}$ & N & can. \\
587 & 0 & 0 & 0 & 0 & 0 & 0 & 0 & 1 & 0 & 1 & 2 & 4 & 8 & 14 & 10 & 13 & 4 & 4 & $w_{270}$ & N & can. \\
588 & 0 & 0 & 0 & 0 & 0 & 0 & 0 & 1 & 0 & 1 & 2 & 4 & 8 & 14 & 11 & 13 & 24 & 4 & $w_{271}$ & N & can. \\
589 & 0 & 0 & 0 & 0 & 0 & 0 & 0 & 1 & 0 & 2 & 4 & 6 & 8 & 10 & 12 & 15 & 336 & 3 & $w_{272}$ & Y & can. \\
590 & 0 & 0 & 0 & 0 & 0 & 0 & 0 & 1 & 0 & 2 & 4 & 6 & 8 & 10 & 13 & 15 & 192 & 4 & $w_{273}$ & N & can. \\
591 & 0 & 0 & 0 & 0 & 0 & 0 & 0 & 1 & 0 & 2 & 4 & 6 & 8 & 10 & 14 & 12 & 192 & 4 & $w_{248}$ & N & can. \\
592 & 0 & 0 & 0 & 0 & 0 & 0 & 0 & 1 & 0 & 2 & 4 & 6 & 8 & 10 & 14 & 13 & 24 & 3 & $w_{274}$ & N & can. \\
593 & 0 & 0 & 0 & 0 & 0 & 0 & 0 & 1 & 0 & 2 & 4 & 6 & 8 & 10 & 15 & 13 & 192 & 4 & $w_{246}$ & N & can. \\
594 & 0 & 0 & 0 & 0 & 0 & 0 & 0 & 1 & 0 & 2 & 4 & 6 & 8 & 11 & 14 & 13 & 32 & 4 & $w_{275}$ & N & can. \\
595 & 0 & 0 & 0 & 0 & 0 & 0 & 0 & 1 & 0 & 2 & 4 & 6 & 8 & 12 & 14 & 10 & 96 & 4 & $w_{276}$ & N & can. \\
596 & 0 & 0 & 0 & 0 & 0 & 0 & 0 & 1 & 0 & 2 & 4 & 6 & 8 & 12 & 14 & 11 & 12 & 3 & $w_{277}$ & N & can. \\
597 & 0 & 0 & 0 & 0 & 0 & 0 & 0 & 1 & 0 & 2 & 4 & 6 & 8 & 12 & 15 & 11 & 32 & 4 & $w_{265}$ & N & can. \\
598 & 0 & 0 & 0 & 0 & 0 & 0 & 0 & 1 & 0 & 2 & 4 & 7 & 8 & 12 & 14 & 11 & 24 & 4 & $w_{278}$ & N & can. \\
599 & 0 & 0 & 0 & 0 & 0 & 0 & 0 & 1 & 0 & 2 & 4 & 8 & 6 & 12 & 14 & 10 & 168 & 4 & $w_{279}$ & N & can. \\
600 & 0 & 0 & 0 & 0 & 0 & 0 & 0 & 1 & 0 & 2 & 4 & 8 & 6 & 12 & 14 & 11 & 21 & 3 & $w_{280}$ & N & can. \\
601 & 0 & 0 & 0 & 0 & 0 & 0 & 0 & 1 & 0 & 2 & 4 & 8 & 6 & 12 & 15 & 11 & 24 & 4 & $w_{281}$ & N & can. \\
602 & 0 & 0 & 0 & 0 & 0 & 0 & 1 & 1 & 0 & 0 & 2 & 2 & 4 & 4 & 7 & 7 & 172032 & 2 & $w_{78}$ & Y & can. \\
603 & 0 & 0 & 0 & 0 & 0 & 0 & 1 & 1 & 0 & 0 & 2 & 2 & 4 & 4 & 7 & 8 & 1344 & 4 & $w_{79}$ & N & can. \\
604 & 0 & 0 & 0 & 0 & 0 & 0 & 1 & 1 & 0 & 0 & 2 & 2 & 4 & 4 & 8 & 8 & 21504 & 3 & $w_{282}$ & Y & can. \\
605 & 0 & 0 & 0 & 0 & 0 & 0 & 1 & 1 & 0 & 0 & 2 & 2 & 4 & 4 & 8 & 9 & 384 & 4 & $w_{76}$ & N & can. \\
606 & 0 & 0 & 0 & 0 & 0 & 0 & 1 & 1 & 0 & 0 & 2 & 2 & 4 & 5 & 4 & 5 & 1024 & 3 & $w_{51}$ & N & can. \\
607 & 0 & 0 & 0 & 0 & 0 & 0 & 1 & 1 & 0 & 0 & 2 & 2 & 4 & 5 & 4 & 6 & 256 & 4 & $w_{67}$ & N & can. \\
608 & 0 & 0 & 0 & 0 & 0 & 0 & 1 & 1 & 0 & 0 & 2 & 2 & 4 & 5 & 4 & 8 & 32 & 4 & $w_{69}$ & N & can. \\
609 & 0 & 0 & 0 & 0 & 0 & 0 & 1 & 1 & 0 & 0 & 2 & 2 & 4 & 5 & 6 & 7 & 1024 & 3 & $w_{80}$ & N & can. \\
610 & 0 & 0 & 0 & 0 & 0 & 0 & 1 & 1 & 0 & 0 & 2 & 2 & 4 & 5 & 6 & 8 & 32 & 4 & $w_{81}$ & N & can. \\
611 & 0 & 0 & 0 & 0 & 0 & 0 & 1 & 1 & 0 & 0 & 2 & 2 & 4 & 5 & 8 & 9 & 256 & 3 & $w_{86}$ & N & can. \\
612 & 0 & 0 & 0 & 0 & 0 & 0 & 1 & 1 & 0 & 0 & 2 & 2 & 4 & 5 & 8 & 10 & 128 & 4 & $w_{87}$ & N & can. \\
613 & 0 & 0 & 0 & 0 & 0 & 0 & 1 & 1 & 0 & 0 & 2 & 2 & 4 & 5 & 8 & 12 & 32 & 4 & $w_{87}$ & N & can. \\
614 & 0 & 0 & 0 & 0 & 0 & 0 & 1 & 1 & 0 & 0 & 2 & 2 & 4 & 8 & 4 & 8 & 128 & 3 & $w_{77}$ & N & can. \\
615 & 0 & 0 & 0 & 0 & 0 & 0 & 1 & 1 & 0 & 0 & 2 & 2 & 4 & 8 & 4 & 9 & 64 & 4 & $w_{73}$ & N & can. \\
616 & 0 & 0 & 0 & 0 & 0 & 0 & 1 & 1 & 0 & 0 & 2 & 2 & 4 & 8 & 4 & 11 & 64 & 4 & $w_{83}$ & N & can. \\
617 & 0 & 0 & 0 & 0 & 0 & 0 & 1 & 1 & 0 & 0 & 2 & 2 & 4 & 8 & 4 & 12 & 16 & 4 & $w_{147}$ & N & can. \\
618 & 0 & 0 & 0 & 0 & 0 & 0 & 1 & 1 & 0 & 0 & 2 & 2 & 4 & 8 & 7 & 11 & 384 & 3 & $w_{283}$ & Y & can. \\
619 & 0 & 0 & 0 & 0 & 0 & 0 & 1 & 1 & 0 & 0 & 2 & 2 & 4 & 8 & 7 & 12 & 48 & 4 & $w_{269}$ & N & can. \\
620 & 0 & 0 & 0 & 0 & 0 & 0 & 1 & 1 & 0 & 0 & 2 & 3 & 2 & 3 & 2 & 3 & 9216 & 4 & $w_{99}$ & N & can. \\
621 & 0 & 0 & 0 & 0 & 0 & 0 & 1 & 1 & 0 & 0 & 2 & 3 & 2 & 3 & 2 & 4 & 128 & 4 & $w_{103}$ & N & can. \\
622 & 0 & 0 & 0 & 0 & 0 & 0 & 1 & 1 & 0 & 0 & 2 & 3 & 2 & 3 & 4 & 5 & 512 & 4 & $w_{152}$ & N & can. \\
623 & 0 & 0 & 0 & 0 & 0 & 0 & 1 & 1 & 0 & 0 & 2 & 3 & 2 & 3 & 4 & 6 & 256 & 4 & $w_{153}$ & N & can. \\
624 & 0 & 0 & 0 & 0 & 0 & 0 & 1 & 1 & 0 & 0 & 2 & 3 & 2 & 3 & 4 & 8 & 64 & 4 & $w_{154}$ & N & can. \\
625 & 0 & 0 & 0 & 0 & 0 & 0 & 1 & 1 & 0 & 0 & 2 & 3 & 2 & 4 & 2 & 4 & 64 & 4 & $w_{112}$ & N & can. \\
626 & 0 & 0 & 0 & 0 & 0 & 0 & 1 & 1 & 0 & 0 & 2 & 3 & 2 & 4 & 2 & 5 & 64 & 3 & $w_{62}$ & N & can. \\
627 & 0 & 0 & 0 & 0 & 0 & 0 & 1 & 1 & 0 & 0 & 2 & 3 & 2 & 4 & 2 & 6 & 32 & 4 & $w_{114}$ & N & can. \\
628 & 0 & 0 & 0 & 0 & 0 & 0 & 1 & 1 & 0 & 0 & 2 & 3 & 2 & 4 & 2 & 8 & 8 & 4 & $w_{129}$ & N & can. \\
629 & 0 & 0 & 0 & 0 & 0 & 0 & 1 & 1 & 0 & 0 & 2 & 3 & 2 & 4 & 4 & 5 & 128 & 4 & $w_{153}$ & N & can. \\
630 & 0 & 0 & 0 & 0 & 0 & 0 & 1 & 1 & 0 & 0 & 2 & 3 & 2 & 4 & 4 & 6 & 16 & 4 & $w_{168}$ & N & can. \\
631 & 0 & 0 & 0 & 0 & 0 & 0 & 1 & 1 & 0 & 0 & 2 & 3 & 2 & 4 & 4 & 8 & 4 & 4 & $w_{169}$ & N & can. \\
632 & 0 & 0 & 0 & 0 & 0 & 0 & 1 & 1 & 0 & 0 & 2 & 3 & 2 & 4 & 6 & 7 & 64 & 4 & $w_{193}$ & N & can. \\
633 & 0 & 0 & 0 & 0 & 0 & 0 & 1 & 1 & 0 & 0 & 2 & 3 & 2 & 4 & 6 & 8 & 4 & 4 & $w_{215}$ & N & can. \\
634 & 0 & 0 & 0 & 0 & 0 & 0 & 1 & 1 & 0 & 0 & 2 & 3 & 2 & 4 & 8 & 9 & 16 & 4 & $w_{195}$ & N & can. \\
635 & 0 & 0 & 0 & 0 & 0 & 0 & 1 & 1 & 0 & 0 & 2 & 3 & 2 & 4 & 8 & 10 & 8 & 4 & $w_{177}$ & N & can. \\
636 & 0 & 0 & 0 & 0 & 0 & 0 & 1 & 1 & 0 & 0 & 2 & 3 & 2 & 4 & 8 & 12 & 8 & 4 & $w_{183}$ & N & can. \\
637 & 0 & 0 & 0 & 0 & 0 & 0 & 1 & 1 & 0 & 0 & 2 & 3 & 2 & 4 & 8 & 14 & 16 & 4 & $w_{205}$ & N & can. \\
638 & 0 & 0 & 0 & 0 & 0 & 0 & 1 & 1 & 0 & 0 & 2 & 3 & 2 & 4 & 8 & 15 & 16 & 3 & $w_{90}$ & N & can. \\
639 & 0 & 0 & 0 & 0 & 0 & 0 & 1 & 1 & 0 & 0 & 2 & 3 & 4 & 5 & 6 & 8 & 32 & 4 & $w_{248}$ & N & can. \\
640 & 0 & 0 & 0 & 0 & 0 & 0 & 1 & 1 & 0 & 0 & 2 & 3 & 4 & 5 & 8 & 9 & 384 & 4 & $w_{284}$ & N & can. \\
641 & 0 & 0 & 0 & 0 & 0 & 0 & 1 & 1 & 0 & 0 & 2 & 3 & 4 & 5 & 8 & 10 & 32 & 4 & $w_{224}$ & N & can. \\
642 & 0 & 0 & 0 & 0 & 0 & 0 & 1 & 1 & 0 & 0 & 2 & 3 & 4 & 5 & 8 & 14 & 64 & 4 & $w_{224}$ & N & can. \\
643 & 0 & 0 & 0 & 0 & 0 & 0 & 1 & 1 & 0 & 0 & 2 & 3 & 4 & 6 & 4 & 6 & 128 & 4 & $w_{193}$ & N & can. \\
644 & 0 & 0 & 0 & 0 & 0 & 0 & 1 & 1 & 0 & 0 & 2 & 3 & 4 & 6 & 4 & 7 & 128 & 3 & $w_{84}$ & N & can. \\
645 & 0 & 0 & 0 & 0 & 0 & 0 & 1 & 1 & 0 & 0 & 2 & 3 & 4 & 6 & 4 & 8 & 4 & 4 & $w_{222}$ & N & can. \\
646 & 0 & 0 & 0 & 0 & 0 & 0 & 1 & 1 & 0 & 0 & 2 & 3 & 4 & 6 & 8 & 10 & 64 & 4 & $w_{224}$ & N & can. \\
647 & 0 & 0 & 0 & 0 & 0 & 0 & 1 & 1 & 0 & 0 & 2 & 3 & 4 & 6 & 8 & 11 & 64 & 3 & $w_{251}$ & N & can. \\
648 & 0 & 0 & 0 & 0 & 0 & 0 & 1 & 1 & 0 & 0 & 2 & 3 & 4 & 6 & 8 & 12 & 8 & 4 & $w_{241}$ & N & can. \\
649 & 0 & 0 & 0 & 0 & 0 & 0 & 1 & 1 & 0 & 0 & 2 & 3 & 4 & 8 & 4 & 8 & 32 & 4 & $w_{205}$ & N & can. \\
650 & 0 & 0 & 0 & 0 & 0 & 0 & 1 & 1 & 0 & 0 & 2 & 3 & 4 & 8 & 4 & 9 & 32 & 3 & $w_{90}$ & N & can. \\
651 & 0 & 0 & 0 & 0 & 0 & 0 & 1 & 1 & 0 & 0 & 2 & 3 & 4 & 8 & 4 & 10 & 8 & 4 & $w_{200}$ & N & can. \\
652 & 0 & 0 & 0 & 0 & 0 & 0 & 1 & 1 & 0 & 0 & 2 & 3 & 4 & 8 & 4 & 12 & 8 & 4 & $w_{202}$ & N & can. \\
653 & 0 & 0 & 0 & 0 & 0 & 0 & 1 & 1 & 0 & 0 & 2 & 3 & 4 & 8 & 4 & 14 & 8 & 4 & $w_{202}$ & N & can. \\
654 & 0 & 0 & 0 & 0 & 0 & 0 & 1 & 1 & 0 & 0 & 2 & 3 & 4 & 8 & 6 & 10 & 32 & 4 & $w_{229}$ & N & can. \\
655 & 0 & 0 & 0 & 0 & 0 & 0 & 1 & 1 & 0 & 0 & 2 & 3 & 4 & 8 & 6 & 11 & 32 & 3 & $w_{252}$ & N & can. \\
656 & 0 & 0 & 0 & 0 & 0 & 0 & 1 & 1 & 0 & 0 & 2 & 3 & 4 & 8 & 6 & 12 & 4 & 4 & $w_{239}$ & N & can. \\
657 & 0 & 0 & 0 & 0 & 0 & 0 & 1 & 1 & 0 & 0 & 2 & 4 & 2 & 4 & 2 & 4 & 192 & 4 & $w_{111}$ & N & can. \\
658 & 0 & 0 & 0 & 0 & 0 & 0 & 1 & 1 & 0 & 0 & 2 & 4 & 2 & 4 & 2 & 5 & 64 & 4 & $w_{52}$ & N & can. \\
659 & 0 & 0 & 0 & 0 & 0 & 0 & 1 & 1 & 0 & 0 & 2 & 4 & 2 & 4 & 2 & 6 & 32 & 4 & $w_{114}$ & N & can. \\
660 & 0 & 0 & 0 & 0 & 0 & 0 & 1 & 1 & 0 & 0 & 2 & 4 & 2 & 4 & 2 & 7 & 32 & 4 & $w_{186}$ & N & can. \\
661 & 0 & 0 & 0 & 0 & 0 & 0 & 1 & 1 & 0 & 0 & 2 & 4 & 2 & 4 & 2 & 8 & 4 & 4 & $w_{204}$ & N & can. \\
662 & 0 & 0 & 0 & 0 & 0 & 0 & 1 & 1 & 0 & 0 & 2 & 4 & 2 & 4 & 6 & 8 & 16 & 4 & $w_{189}$ & N & can. \\
663 & 0 & 0 & 0 & 0 & 0 & 0 & 1 & 1 & 0 & 0 & 2 & 4 & 2 & 4 & 7 & 8 & 16 & 4 & $w_{221}$ & N & can. \\
664 & 0 & 0 & 0 & 0 & 0 & 0 & 1 & 1 & 0 & 0 & 2 & 4 & 2 & 4 & 8 & 10 & 8 & 4 & $w_{206}$ & N & can. \\
665 & 0 & 0 & 0 & 0 & 0 & 0 & 1 & 1 & 0 & 0 & 2 & 4 & 2 & 4 & 8 & 14 & 32 & 4 & $w_{213}$ & N & can. \\
666 & 0 & 0 & 0 & 0 & 0 & 0 & 1 & 1 & 0 & 0 & 2 & 4 & 2 & 4 & 8 & 15 & 32 & 4 & $w_{87}$ & N & can. \\
667 & 0 & 0 & 0 & 0 & 0 & 0 & 1 & 1 & 0 & 0 & 2 & 4 & 2 & 5 & 2 & 6 & 16 & 4 & $w_{68}$ & N & can. \\
668 & 0 & 0 & 0 & 0 & 0 & 0 & 1 & 1 & 0 & 0 & 2 & 4 & 2 & 5 & 2 & 8 & 4 & 4 & $w_{70}$ & N & can. \\
669 & 0 & 0 & 0 & 0 & 0 & 0 & 1 & 1 & 0 & 0 & 2 & 4 & 2 & 5 & 6 & 8 & 8 & 4 & $w_{85}$ & N & can. \\
670 & 0 & 0 & 0 & 0 & 0 & 0 & 1 & 1 & 0 & 0 & 2 & 4 & 2 & 5 & 8 & 10 & 8 & 4 & $w_{91}$ & N & can. \\
671 & 0 & 0 & 0 & 0 & 0 & 0 & 1 & 1 & 0 & 0 & 2 & 4 & 2 & 5 & 8 & 14 & 16 & 4 & $w_{87}$ & N & can. \\
672 & 0 & 0 & 0 & 0 & 0 & 0 & 1 & 1 & 0 & 0 & 2 & 4 & 2 & 6 & 2 & 8 & 4 & 4 & $w_{135}$ & N & can. \\
673 & 0 & 0 & 0 & 0 & 0 & 0 & 1 & 1 & 0 & 0 & 2 & 4 & 2 & 6 & 3 & 8 & 4 & 4 & $w_{187}$ & N & can. \\
674 & 0 & 0 & 0 & 0 & 0 & 0 & 1 & 1 & 0 & 0 & 2 & 4 & 2 & 6 & 4 & 6 & 96 & 3 & $w_{165}$ & Y & can. \\
675 & 0 & 0 & 0 & 0 & 0 & 0 & 1 & 1 & 0 & 0 & 2 & 4 & 2 & 6 & 4 & 7 & 48 & 4 & $w_{168}$ & N & can. \\
676 & 0 & 0 & 0 & 0 & 0 & 0 & 1 & 1 & 0 & 0 & 2 & 4 & 2 & 6 & 4 & 8 & 2 & 4 & $w_{170}$ & N & can. \\
677 & 0 & 0 & 0 & 0 & 0 & 0 & 1 & 1 & 0 & 0 & 2 & 4 & 2 & 6 & 5 & 7 & 96 & 3 & $w_{84}$ & Y & can. \\
678 & 0 & 0 & 0 & 0 & 0 & 0 & 1 & 1 & 0 & 0 & 2 & 4 & 2 & 6 & 5 & 8 & 2 & 4 & $w_{285}$ & N & can. \\
679 & 0 & 0 & 0 & 0 & 0 & 0 & 1 & 1 & 0 & 0 & 2 & 4 & 2 & 6 & 8 & 10 & 8 & 3 & $w_{223}$ & Y & can. \\
680 & 0 & 0 & 0 & 0 & 0 & 0 & 1 & 1 & 0 & 0 & 2 & 4 & 2 & 6 & 8 & 11 & 8 & 4 & $w_{202}$ & N & can. \\
681 & 0 & 0 & 0 & 0 & 0 & 0 & 1 & 1 & 0 & 0 & 2 & 4 & 2 & 6 & 8 & 12 & 4 & 4 & $w_{200}$ & N & can. \\
682 & 0 & 0 & 0 & 0 & 0 & 0 & 1 & 1 & 0 & 0 & 2 & 4 & 2 & 6 & 8 & 13 & 4 & 4 & $w_{238}$ & N & can. \\
683 & 0 & 0 & 0 & 0 & 0 & 0 & 1 & 1 & 0 & 0 & 2 & 4 & 2 & 8 & 2 & 12 & 12 & 4 & $w_{137}$ & N & can. \\
684 & 0 & 0 & 0 & 0 & 0 & 0 & 1 & 1 & 0 & 0 & 2 & 4 & 2 & 8 & 2 & 13 & 12 & 4 & $w_{218}$ & N & can. \\
685 & 0 & 0 & 0 & 0 & 0 & 0 & 1 & 1 & 0 & 0 & 2 & 4 & 2 & 8 & 2 & 14 & 12 & 3 & $w_{190}$ & Y & can. \\
686 & 0 & 0 & 0 & 0 & 0 & 0 & 1 & 1 & 0 & 0 & 2 & 4 & 2 & 8 & 2 & 15 & 12 & 4 & $w_{148}$ & N & can. \\
687 & 0 & 0 & 0 & 0 & 0 & 0 & 1 & 1 & 0 & 0 & 2 & 4 & 2 & 8 & 4 & 8 & 12 & 3 & $w_{190}$ & Y & can. \\
688 & 0 & 0 & 0 & 0 & 0 & 0 & 1 & 1 & 0 & 0 & 2 & 4 & 2 & 8 & 4 & 9 & 12 & 4 & $w_{148}$ & N & can. \\
689 & 0 & 0 & 0 & 0 & 0 & 0 & 1 & 1 & 0 & 0 & 2 & 4 & 2 & 8 & 4 & 10 & 2 & 4 & $w_{175}$ & N & can. \\
690 & 0 & 0 & 0 & 0 & 0 & 0 & 1 & 1 & 0 & 0 & 2 & 4 & 2 & 8 & 4 & 11 & 2 & 4 & $w_{220}$ & N & can. \\
691 & 0 & 0 & 0 & 0 & 0 & 0 & 1 & 1 & 0 & 0 & 2 & 4 & 2 & 8 & 4 & 14 & 4 & 4 & $w_{206}$ & N & can. \\
692 & 0 & 0 & 0 & 0 & 0 & 0 & 1 & 1 & 0 & 0 & 2 & 4 & 2 & 8 & 4 & 15 & 4 & 4 & $w_{91}$ & N & can. \\
693 & 0 & 0 & 0 & 0 & 0 & 0 & 1 & 1 & 0 & 0 & 2 & 4 & 2 & 8 & 6 & 10 & 8 & 3 & $w_{90}$ & N & can. \\
694 & 0 & 0 & 0 & 0 & 0 & 0 & 1 & 1 & 0 & 0 & 2 & 4 & 2 & 8 & 6 & 11 & 4 & 4 & $w_{200}$ & N & can. \\
695 & 0 & 0 & 0 & 0 & 0 & 0 & 1 & 1 & 0 & 0 & 2 & 4 & 2 & 8 & 6 & 12 & 4 & 4 & $w_{206}$ & N & can. \\
696 & 0 & 0 & 0 & 0 & 0 & 0 & 1 & 1 & 0 & 0 & 2 & 4 & 2 & 8 & 6 & 13 & 4 & 4 & $w_{242}$ & N & can. \\
697 & 0 & 0 & 0 & 0 & 0 & 0 & 1 & 1 & 0 & 0 & 2 & 4 & 2 & 8 & 6 & 14 & 2 & 4 & $w_{203}$ & N & can. \\
698 & 0 & 0 & 0 & 0 & 0 & 0 & 1 & 1 & 0 & 0 & 2 & 4 & 2 & 8 & 7 & 11 & 8 & 3 & $w_{252}$ & Y & can. \\
699 & 0 & 0 & 0 & 0 & 0 & 0 & 1 & 1 & 0 & 0 & 2 & 4 & 2 & 8 & 7 & 12 & 4 & 4 & $w_{242}$ & N & can. \\
700 & 0 & 0 & 0 & 0 & 0 & 0 & 1 & 1 & 0 & 0 & 2 & 4 & 2 & 8 & 7 & 13 & 4 & 4 & $w_{245}$ & N & can. \\
701 & 0 & 0 & 0 & 0 & 0 & 0 & 1 & 1 & 0 & 0 & 2 & 4 & 2 & 8 & 7 & 14 & 2 & 4 & $w_{231}$ & N & can. \\
702 & 0 & 0 & 0 & 0 & 0 & 0 & 1 & 1 & 0 & 0 & 2 & 4 & 2 & 8 & 12 & 14 & 4 & 4 & $w_{203}$ & N & can. \\
703 & 0 & 0 & 0 & 0 & 0 & 0 & 1 & 1 & 0 & 0 & 2 & 4 & 2 & 8 & 13 & 14 & 4 & 4 & $w_{231}$ & N & can. \\
704 & 0 & 0 & 0 & 0 & 0 & 0 & 1 & 1 & 0 & 0 & 2 & 4 & 6 & 8 & 10 & 12 & 16 & 4 & $w_{200}$ & N & can. \\
705 & 0 & 0 & 0 & 0 & 0 & 0 & 1 & 1 & 0 & 0 & 2 & 4 & 6 & 8 & 10 & 13 & 8 & 4 & $w_{243}$ & N & can. \\
706 & 0 & 0 & 0 & 0 & 0 & 0 & 1 & 1 & 0 & 0 & 2 & 4 & 6 & 8 & 10 & 14 & 6 & 4 & $w_{226}$ & N & can. \\
707 & 0 & 0 & 0 & 0 & 0 & 0 & 1 & 1 & 0 & 0 & 2 & 4 & 6 & 8 & 11 & 13 & 16 & 4 & $w_{270}$ & N & can. \\
708 & 0 & 0 & 0 & 0 & 0 & 0 & 1 & 1 & 0 & 0 & 2 & 4 & 6 & 8 & 11 & 14 & 6 & 4 & $w_{286}$ & N & can. \\
709 & 0 & 0 & 0 & 0 & 0 & 0 & 1 & 1 & 0 & 0 & 2 & 4 & 8 & 14 & 10 & 13 & 32 & 4 & $w_{246}$ & N & can. \\
710 & 0 & 0 & 0 & 0 & 0 & 0 & 1 & 1 & 0 & 0 & 2 & 4 & 8 & 14 & 11 & 13 & 192 & 4 & $w_{273}$ & N & can. \\
711 & 0 & 0 & 0 & 0 & 0 & 0 & 1 & 1 & 0 & 1 & 0 & 1 & 0 & 1 & 1 & 0 & 15482880 & 2 & $w_{287}$ & N & can. \\
712 & 0 & 0 & 0 & 0 & 0 & 0 & 1 & 1 & 0 & 1 & 0 & 1 & 0 & 1 & 1 & 2 & 69120 & 4 & $w_{288}$ & N & can. \\
713 & 0 & 0 & 0 & 0 & 0 & 0 & 1 & 1 & 0 & 1 & 0 & 1 & 0 & 1 & 2 & 3 & 9216 & 3 & $w_{289}$ & N & can. \\
714 & 0 & 0 & 0 & 0 & 0 & 0 & 1 & 1 & 0 & 1 & 0 & 1 & 0 & 1 & 2 & 4 & 768 & 4 & $w_{290}$ & N & can. \\
715 & 0 & 0 & 0 & 0 & 0 & 0 & 1 & 1 & 0 & 1 & 0 & 1 & 0 & 2 & 2 & 3 & 4608 & 4 & $w_{291}$ & N & can. \\
716 & 0 & 0 & 0 & 0 & 0 & 0 & 1 & 1 & 0 & 1 & 0 & 1 & 0 & 2 & 2 & 4 & 128 & 4 & $w_{292}$ & N & can. \\
717 & 0 & 0 & 0 & 0 & 0 & 0 & 1 & 1 & 0 & 1 & 0 & 1 & 0 & 2 & 4 & 6 & 384 & 3 & $w_{293}$ & N & can. \\
718 & 0 & 0 & 0 & 0 & 0 & 0 & 1 & 1 & 0 & 1 & 0 & 1 & 0 & 2 & 4 & 7 & 384 & 4 & $w_{294}$ & N & can. \\
719 & 0 & 0 & 0 & 0 & 0 & 0 & 1 & 1 & 0 & 1 & 0 & 1 & 0 & 2 & 4 & 8 & 48 & 4 & $w_{295}$ & N & can. \\
720 & 0 & 0 & 0 & 0 & 0 & 0 & 1 & 1 & 0 & 1 & 0 & 1 & 2 & 3 & 3 & 2 & 147456 & 2 & $w_{296}$ & N & can. \\
721 & 0 & 0 & 0 & 0 & 0 & 0 & 1 & 1 & 0 & 1 & 0 & 1 & 2 & 3 & 3 & 4 & 768 & 4 & $w_{297}$ & N & can. \\
722 & 0 & 0 & 0 & 0 & 0 & 0 & 1 & 1 & 0 & 1 & 0 & 1 & 2 & 3 & 4 & 5 & 1024 & 3 & $w_{298}$ & N & can. \\
723 & 0 & 0 & 0 & 0 & 0 & 0 & 1 & 1 & 0 & 1 & 0 & 1 & 2 & 3 & 4 & 6 & 256 & 4 & $w_{299}$ & N & can. \\
724 & 0 & 0 & 0 & 0 & 0 & 0 & 1 & 1 & 0 & 1 & 0 & 1 & 2 & 3 & 4 & 8 & 64 & 4 & $w_{300}$ & N & can. \\
725 & 0 & 0 & 0 & 0 & 0 & 0 & 1 & 1 & 0 & 1 & 0 & 1 & 2 & 4 & 6 & 8 & 48 & 4 & $w_{301}$ & N & can. \\
726 & 0 & 0 & 0 & 0 & 0 & 0 & 1 & 1 & 0 & 1 & 0 & 1 & 2 & 4 & 8 & 14 & 384 & 3 & $w_{302}$ & N & can. \\
727 & 0 & 0 & 0 & 0 & 0 & 0 & 1 & 1 & 0 & 1 & 0 & 1 & 2 & 4 & 8 & 15 & 384 & 4 & $w_{303}$ & N & can. \\
728 & 0 & 0 & 0 & 0 & 0 & 0 & 1 & 1 & 0 & 1 & 0 & 2 & 0 & 2 & 1 & 2 & 1152 & 4 & $w_{101}$ & N & can. \\
729 & 0 & 0 & 0 & 0 & 0 & 0 & 1 & 1 & 0 & 1 & 0 & 2 & 0 & 2 & 1 & 3 & 6912 & 4 & $w_{304}$ & N & can. \\
730 & 0 & 0 & 0 & 0 & 0 & 0 & 1 & 1 & 0 & 1 & 0 & 2 & 0 & 2 & 1 & 4 & 96 & 4 & $w_{305}$ & N & can. \\
731 & 0 & 0 & 0 & 0 & 0 & 0 & 1 & 1 & 0 & 1 & 0 & 2 & 0 & 2 & 2 & 3 & 3072 & 3 & $w_{296}$ & N & can. \\
732 & 0 & 0 & 0 & 0 & 0 & 0 & 1 & 1 & 0 & 1 & 0 & 2 & 0 & 2 & 2 & 4 & 64 & 4 & $w_{306}$ & N & can. \\
733 & 0 & 0 & 0 & 0 & 0 & 0 & 1 & 1 & 0 & 1 & 0 & 2 & 0 & 2 & 3 & 2 & 3072 & 3 & $w_{60}$ & N & can. \\
734 & 0 & 0 & 0 & 0 & 0 & 0 & 1 & 1 & 0 & 1 & 0 & 2 & 0 & 2 & 3 & 4 & 32 & 4 & $w_{124}$ & N & can. \\
735 & 0 & 0 & 0 & 0 & 0 & 0 & 1 & 1 & 0 & 1 & 0 & 2 & 0 & 2 & 4 & 5 & 64 & 3 & $w_{298}$ & N & can. \\
736 & 0 & 0 & 0 & 0 & 0 & 0 & 1 & 1 & 0 & 1 & 0 & 2 & 0 & 2 & 4 & 6 & 64 & 4 & $w_{307}$ & N & can. \\
737 & 0 & 0 & 0 & 0 & 0 & 0 & 1 & 1 & 0 & 1 & 0 & 2 & 0 & 2 & 4 & 7 & 64 & 4 & $w_{153}$ & N & can. \\
738 & 0 & 0 & 0 & 0 & 0 & 0 & 1 & 1 & 0 & 1 & 0 & 2 & 0 & 2 & 4 & 8 & 8 & 4 & $w_{308}$ & N & can. \\
739 & 0 & 0 & 0 & 0 & 0 & 0 & 1 & 1 & 0 & 1 & 0 & 2 & 0 & 3 & 1 & 4 & 96 & 4 & $w_{103}$ & N & can. \\
740 & 0 & 0 & 0 & 0 & 0 & 0 & 1 & 1 & 0 & 1 & 0 & 2 & 0 & 3 & 2 & 3 & 768 & 4 & $w_{156}$ & N & can. \\
741 & 0 & 0 & 0 & 0 & 0 & 0 & 1 & 1 & 0 & 1 & 0 & 2 & 0 & 3 & 2 & 4 & 16 & 4 & $w_{157}$ & N & can. \\
742 & 0 & 0 & 0 & 0 & 0 & 0 & 1 & 1 & 0 & 1 & 0 & 2 & 0 & 3 & 4 & 5 & 64 & 4 & $w_{158}$ & N & can. \\
743 & 0 & 0 & 0 & 0 & 0 & 0 & 1 & 1 & 0 & 1 & 0 & 2 & 0 & 3 & 4 & 6 & 32 & 4 & $w_{153}$ & N & can. \\
744 & 0 & 0 & 0 & 0 & 0 & 0 & 1 & 1 & 0 & 1 & 0 & 2 & 0 & 3 & 4 & 8 & 8 & 4 & $w_{159}$ & N & can. \\
745 & 0 & 0 & 0 & 0 & 0 & 0 & 1 & 1 & 0 & 1 & 0 & 2 & 0 & 4 & 1 & 6 & 288 & 3 & $w_{293}$ & N & can. \\
746 & 0 & 0 & 0 & 0 & 0 & 0 & 1 & 1 & 0 & 1 & 0 & 2 & 0 & 4 & 1 & 7 & 288 & 4 & $w_{104}$ & N & can. \\
747 & 0 & 0 & 0 & 0 & 0 & 0 & 1 & 1 & 0 & 1 & 0 & 2 & 0 & 4 & 1 & 8 & 36 & 4 & $w_{309}$ & N & can. \\
748 & 0 & 0 & 0 & 0 & 0 & 0 & 1 & 1 & 0 & 1 & 0 & 2 & 0 & 4 & 2 & 3 & 32 & 4 & $w_{297}$ & N & can. \\
749 & 0 & 0 & 0 & 0 & 0 & 0 & 1 & 1 & 0 & 1 & 0 & 2 & 0 & 4 & 2 & 4 & 16 & 4 & $w_{160}$ & N & can. \\
750 & 0 & 0 & 0 & 0 & 0 & 0 & 1 & 1 & 0 & 1 & 0 & 2 & 0 & 4 & 2 & 5 & 32 & 3 & $w_{298}$ & N & can. \\
751 & 0 & 0 & 0 & 0 & 0 & 0 & 1 & 1 & 0 & 1 & 0 & 2 & 0 & 4 & 2 & 6 & 16 & 4 & $w_{310}$ & N & can. \\
752 & 0 & 0 & 0 & 0 & 0 & 0 & 1 & 1 & 0 & 1 & 0 & 2 & 0 & 4 & 2 & 7 & 16 & 4 & $w_{153}$ & N & can. \\
753 & 0 & 0 & 0 & 0 & 0 & 0 & 1 & 1 & 0 & 1 & 0 & 2 & 0 & 4 & 2 & 8 & 2 & 4 & $w_{311}$ & N & can. \\
754 & 0 & 0 & 0 & 0 & 0 & 0 & 1 & 1 & 0 & 1 & 0 & 2 & 0 & 4 & 3 & 2 & 32 & 4 & $w_{157}$ & N & can. \\
755 & 0 & 0 & 0 & 0 & 0 & 0 & 1 & 1 & 0 & 1 & 0 & 2 & 0 & 4 & 3 & 4 & 32 & 3 & $w_{62}$ & N & can. \\
756 & 0 & 0 & 0 & 0 & 0 & 0 & 1 & 1 & 0 & 1 & 0 & 2 & 0 & 4 & 3 & 6 & 16 & 4 & $w_{153}$ & N & can. \\
757 & 0 & 0 & 0 & 0 & 0 & 0 & 1 & 1 & 0 & 1 & 0 & 2 & 0 & 4 & 3 & 7 & 16 & 4 & $w_{114}$ & N & can. \\
758 & 0 & 0 & 0 & 0 & 0 & 0 & 1 & 1 & 0 & 1 & 0 & 2 & 0 & 4 & 3 & 8 & 2 & 4 & $w_{161}$ & N & can. \\
759 & 0 & 0 & 0 & 0 & 0 & 0 & 1 & 1 & 0 & 1 & 0 & 2 & 0 & 4 & 6 & 7 & 64 & 4 & $w_{153}$ & N & can. \\
760 & 0 & 0 & 0 & 0 & 0 & 0 & 1 & 1 & 0 & 1 & 0 & 2 & 0 & 4 & 6 & 8 & 4 & 4 & $w_{162}$ & N & can. \\
761 & 0 & 0 & 0 & 0 & 0 & 0 & 1 & 1 & 0 & 1 & 0 & 2 & 0 & 4 & 7 & 6 & 64 & 4 & $w_{299}$ & N & can. \\
762 & 0 & 0 & 0 & 0 & 0 & 0 & 1 & 1 & 0 & 1 & 0 & 2 & 0 & 4 & 7 & 8 & 4 & 4 & $w_{312}$ & N & can. \\
763 & 0 & 0 & 0 & 0 & 0 & 0 & 1 & 1 & 0 & 1 & 0 & 2 & 0 & 4 & 8 & 9 & 8 & 4 & $w_{300}$ & N & can. \\
764 & 0 & 0 & 0 & 0 & 0 & 0 & 1 & 1 & 0 & 1 & 0 & 2 & 0 & 4 & 8 & 10 & 4 & 4 & $w_{313}$ & N & can. \\
765 & 0 & 0 & 0 & 0 & 0 & 0 & 1 & 1 & 0 & 1 & 0 & 2 & 0 & 4 & 8 & 11 & 4 & 4 & $w_{314}$ & N & can. \\
766 & 0 & 0 & 0 & 0 & 0 & 0 & 1 & 1 & 0 & 1 & 0 & 2 & 0 & 4 & 8 & 14 & 8 & 4 & $w_{315}$ & N & can. \\
767 & 0 & 0 & 0 & 0 & 0 & 0 & 1 & 1 & 0 & 1 & 0 & 2 & 0 & 4 & 8 & 15 & 8 & 3 & $w_{302}$ & N & can. \\
768 & 0 & 0 & 0 & 0 & 0 & 0 & 1 & 1 & 0 & 1 & 0 & 2 & 2 & 3 & 4 & 5 & 128 & 4 & $w_{299}$ & N & can. \\
769 & 0 & 0 & 0 & 0 & 0 & 0 & 1 & 1 & 0 & 1 & 0 & 2 & 2 & 3 & 4 & 6 & 128 & 3 & $w_{316}$ & N & can. \\
770 & 0 & 0 & 0 & 0 & 0 & 0 & 1 & 1 & 0 & 1 & 0 & 2 & 2 & 3 & 4 & 7 & 128 & 4 & $w_{317}$ & N & can. \\
771 & 0 & 0 & 0 & 0 & 0 & 0 & 1 & 1 & 0 & 1 & 0 & 2 & 2 & 3 & 4 & 8 & 16 & 4 & $w_{318}$ & N & can. \\
772 & 0 & 0 & 0 & 0 & 0 & 0 & 1 & 1 & 0 & 1 & 0 & 2 & 2 & 4 & 2 & 5 & 32 & 4 & $w_{153}$ & N & can. \\
773 & 0 & 0 & 0 & 0 & 0 & 0 & 1 & 1 & 0 & 1 & 0 & 2 & 2 & 4 & 2 & 6 & 32 & 4 & $w_{164}$ & N & can. \\
774 & 0 & 0 & 0 & 0 & 0 & 0 & 1 & 1 & 0 & 1 & 0 & 2 & 2 & 4 & 2 & 7 & 32 & 3 & $w_{165}$ & N & can. \\
775 & 0 & 0 & 0 & 0 & 0 & 0 & 1 & 1 & 0 & 1 & 0 & 2 & 2 & 4 & 2 & 8 & 4 & 4 & $w_{166}$ & N & can. \\
776 & 0 & 0 & 0 & 0 & 0 & 0 & 1 & 1 & 0 & 1 & 0 & 2 & 2 & 4 & 3 & 4 & 32 & 4 & $w_{153}$ & N & can. \\
777 & 0 & 0 & 0 & 0 & 0 & 0 & 1 & 1 & 0 & 1 & 0 & 2 & 2 & 4 & 3 & 5 & 32 & 4 & $w_{310}$ & N & can. \\
778 & 0 & 0 & 0 & 0 & 0 & 0 & 1 & 1 & 0 & 1 & 0 & 2 & 2 & 4 & 3 & 6 & 32 & 3 & $w_{316}$ & N & can. \\
779 & 0 & 0 & 0 & 0 & 0 & 0 & 1 & 1 & 0 & 1 & 0 & 2 & 2 & 4 & 3 & 7 & 32 & 4 & $w_{164}$ & N & can. \\
780 & 0 & 0 & 0 & 0 & 0 & 0 & 1 & 1 & 0 & 1 & 0 & 2 & 2 & 4 & 3 & 8 & 4 & 4 & $w_{319}$ & N & can. \\
781 & 0 & 0 & 0 & 0 & 0 & 0 & 1 & 1 & 0 & 1 & 0 & 2 & 2 & 4 & 4 & 5 & 64 & 4 & $w_{299}$ & N & can. \\
782 & 0 & 0 & 0 & 0 & 0 & 0 & 1 & 1 & 0 & 1 & 0 & 2 & 2 & 4 & 4 & 6 & 16 & 4 & $w_{320}$ & N & can. \\
783 & 0 & 0 & 0 & 0 & 0 & 0 & 1 & 1 & 0 & 1 & 0 & 2 & 2 & 4 & 4 & 7 & 16 & 4 & $w_{176}$ & N & can. \\
784 & 0 & 0 & 0 & 0 & 0 & 0 & 1 & 1 & 0 & 1 & 0 & 2 & 2 & 4 & 4 & 8 & 2 & 4 & $w_{321}$ & N & can. \\
785 & 0 & 0 & 0 & 0 & 0 & 0 & 1 & 1 & 0 & 1 & 0 & 2 & 2 & 4 & 5 & 4 & 64 & 4 & $w_{153}$ & N & can. \\
786 & 0 & 0 & 0 & 0 & 0 & 0 & 1 & 1 & 0 & 1 & 0 & 2 & 2 & 4 & 5 & 6 & 16 & 4 & $w_{176}$ & N & can. \\
787 & 0 & 0 & 0 & 0 & 0 & 0 & 1 & 1 & 0 & 1 & 0 & 2 & 2 & 4 & 5 & 7 & 16 & 4 & $w_{168}$ & N & can. \\
788 & 0 & 0 & 0 & 0 & 0 & 0 & 1 & 1 & 0 & 1 & 0 & 2 & 2 & 4 & 5 & 8 & 2 & 4 & $w_{171}$ & N & can. \\
789 & 0 & 0 & 0 & 0 & 0 & 0 & 1 & 1 & 0 & 1 & 0 & 2 & 2 & 4 & 6 & 7 & 32 & 4 & $w_{176}$ & N & can. \\
790 & 0 & 0 & 0 & 0 & 0 & 0 & 1 & 1 & 0 & 1 & 0 & 2 & 2 & 4 & 6 & 8 & 2 & 4 & $w_{178}$ & N & can. \\
791 & 0 & 0 & 0 & 0 & 0 & 0 & 1 & 1 & 0 & 1 & 0 & 2 & 2 & 4 & 7 & 6 & 32 & 4 & $w_{322}$ & N & can. \\
792 & 0 & 0 & 0 & 0 & 0 & 0 & 1 & 1 & 0 & 1 & 0 & 2 & 2 & 4 & 7 & 8 & 2 & 4 & $w_{323}$ & N & can. \\
793 & 0 & 0 & 0 & 0 & 0 & 0 & 1 & 1 & 0 & 1 & 0 & 2 & 2 & 4 & 8 & 9 & 4 & 4 & $w_{324}$ & N & can. \\
794 & 0 & 0 & 0 & 0 & 0 & 0 & 1 & 1 & 0 & 1 & 0 & 2 & 2 & 4 & 8 & 10 & 4 & 4 & $w_{325}$ & N & can. \\
795 & 0 & 0 & 0 & 0 & 0 & 0 & 1 & 1 & 0 & 1 & 0 & 2 & 2 & 4 & 8 & 11 & 4 & 4 & $w_{326}$ & N & can. \\
796 & 0 & 0 & 0 & 0 & 0 & 0 & 1 & 1 & 0 & 1 & 0 & 2 & 2 & 4 & 8 & 12 & 4 & 4 & $w_{327}$ & N & can. \\
797 & 0 & 0 & 0 & 0 & 0 & 0 & 1 & 1 & 0 & 1 & 0 & 2 & 2 & 4 & 8 & 13 & 4 & 3 & $w_{328}$ & N & can. \\
798 & 0 & 0 & 0 & 0 & 0 & 0 & 1 & 1 & 0 & 1 & 0 & 2 & 2 & 4 & 8 & 14 & 4 & 4 & $w_{329}$ & N & can. \\
799 & 0 & 0 & 0 & 0 & 0 & 0 & 1 & 1 & 0 & 1 & 0 & 2 & 2 & 4 & 8 & 15 & 4 & 4 & $w_{201}$ & N & can. \\
800 & 0 & 0 & 0 & 0 & 0 & 0 & 1 & 1 & 0 & 1 & 0 & 2 & 4 & 5 & 4 & 6 & 128 & 3 & $w_{184}$ & N & can. \\
801 & 0 & 0 & 0 & 0 & 0 & 0 & 1 & 1 & 0 & 1 & 0 & 2 & 4 & 5 & 4 & 7 & 128 & 4 & $w_{330}$ & N & can. \\
802 & 0 & 0 & 0 & 0 & 0 & 0 & 1 & 1 & 0 & 1 & 0 & 2 & 4 & 5 & 4 & 8 & 16 & 4 & $w_{331}$ & N & can. \\
803 & 0 & 0 & 0 & 0 & 0 & 0 & 1 & 1 & 0 & 1 & 0 & 2 & 4 & 5 & 5 & 6 & 128 & 4 & $w_{330}$ & N & can. \\
804 & 0 & 0 & 0 & 0 & 0 & 0 & 1 & 1 & 0 & 1 & 0 & 2 & 4 & 5 & 5 & 7 & 128 & 3 & $w_{316}$ & N & can. \\
805 & 0 & 0 & 0 & 0 & 0 & 0 & 1 & 1 & 0 & 1 & 0 & 2 & 4 & 5 & 5 & 8 & 16 & 4 & $w_{318}$ & N & can. \\
806 & 0 & 0 & 0 & 0 & 0 & 0 & 1 & 1 & 0 & 1 & 0 & 2 & 4 & 5 & 6 & 7 & 256 & 4 & $w_{322}$ & N & can. \\
807 & 0 & 0 & 0 & 0 & 0 & 0 & 1 & 1 & 0 & 1 & 0 & 2 & 4 & 5 & 6 & 8 & 8 & 4 & $w_{332}$ & N & can. \\
808 & 0 & 0 & 0 & 0 & 0 & 0 & 1 & 1 & 0 & 1 & 0 & 2 & 4 & 5 & 7 & 8 & 8 & 4 & $w_{333}$ & N & can. \\
809 & 0 & 0 & 0 & 0 & 0 & 0 & 1 & 1 & 0 & 1 & 0 & 2 & 4 & 5 & 8 & 9 & 32 & 4 & $w_{324}$ & N & can. \\
810 & 0 & 0 & 0 & 0 & 0 & 0 & 1 & 1 & 0 & 1 & 0 & 2 & 4 & 5 & 8 & 10 & 16 & 3 & $w_{328}$ & N & can. \\
811 & 0 & 0 & 0 & 0 & 0 & 0 & 1 & 1 & 0 & 1 & 0 & 2 & 4 & 5 & 8 & 11 & 16 & 4 & $w_{334}$ & N & can. \\
812 & 0 & 0 & 0 & 0 & 0 & 0 & 1 & 1 & 0 & 1 & 0 & 2 & 4 & 5 & 8 & 12 & 8 & 4 & $w_{332}$ & N & can. \\
813 & 0 & 0 & 0 & 0 & 0 & 0 & 1 & 1 & 0 & 1 & 0 & 2 & 4 & 5 & 8 & 14 & 8 & 4 & $w_{335}$ & N & can. \\
814 & 0 & 0 & 0 & 0 & 0 & 0 & 1 & 1 & 0 & 1 & 0 & 2 & 4 & 6 & 4 & 7 & 32 & 4 & $w_{193}$ & N & can. \\
815 & 0 & 0 & 0 & 0 & 0 & 0 & 1 & 1 & 0 & 1 & 0 & 2 & 4 & 6 & 4 & 8 & 4 & 4 & $w_{215}$ & N & can. \\
816 & 0 & 0 & 0 & 0 & 0 & 0 & 1 & 1 & 0 & 1 & 0 & 2 & 4 & 6 & 5 & 7 & 256 & 4 & $w_{336}$ & N & can. \\
817 & 0 & 0 & 0 & 0 & 0 & 0 & 1 & 1 & 0 & 1 & 0 & 2 & 4 & 6 & 5 & 8 & 4 & 4 & $w_{325}$ & N & can. \\
818 & 0 & 0 & 0 & 0 & 0 & 0 & 1 & 1 & 0 & 1 & 0 & 2 & 4 & 6 & 8 & 10 & 32 & 4 & $w_{337}$ & N & can. \\
819 & 0 & 0 & 0 & 0 & 0 & 0 & 1 & 1 & 0 & 1 & 0 & 2 & 4 & 6 & 8 & 11 & 16 & 4 & $w_{224}$ & N & can. \\
820 & 0 & 0 & 0 & 0 & 0 & 0 & 1 & 1 & 0 & 1 & 0 & 2 & 4 & 6 & 8 & 12 & 4 & 4 & $w_{338}$ & N & can. \\
821 & 0 & 0 & 0 & 0 & 0 & 0 & 1 & 1 & 0 & 1 & 0 & 2 & 4 & 7 & 4 & 8 & 4 & 4 & $w_{333}$ & N & can. \\
822 & 0 & 0 & 0 & 0 & 0 & 0 & 1 & 1 & 0 & 1 & 0 & 2 & 4 & 7 & 5 & 6 & 128 & 4 & $w_{193}$ & N & can. \\
823 & 0 & 0 & 0 & 0 & 0 & 0 & 1 & 1 & 0 & 1 & 0 & 2 & 4 & 7 & 5 & 8 & 4 & 4 & $w_{333}$ & N & can. \\
824 & 0 & 0 & 0 & 0 & 0 & 0 & 1 & 1 & 0 & 1 & 0 & 2 & 4 & 7 & 8 & 11 & 32 & 4 & $w_{224}$ & N & can. \\
825 & 0 & 0 & 0 & 0 & 0 & 0 & 1 & 1 & 0 & 1 & 0 & 2 & 4 & 7 & 8 & 12 & 4 & 4 & $w_{262}$ & N & can. \\
826 & 0 & 0 & 0 & 0 & 0 & 0 & 1 & 1 & 0 & 1 & 0 & 2 & 4 & 8 & 4 & 9 & 8 & 4 & $w_{195}$ & N & can. \\
827 & 0 & 0 & 0 & 0 & 0 & 0 & 1 & 1 & 0 & 1 & 0 & 2 & 4 & 8 & 4 & 10 & 8 & 4 & $w_{196}$ & N & can. \\
828 & 0 & 0 & 0 & 0 & 0 & 0 & 1 & 1 & 0 & 1 & 0 & 2 & 4 & 8 & 4 & 11 & 8 & 3 & $w_{90}$ & N & can. \\
829 & 0 & 0 & 0 & 0 & 0 & 0 & 1 & 1 & 0 & 1 & 0 & 2 & 4 & 8 & 4 & 12 & 4 & 4 & $w_{178}$ & N & can. \\
830 & 0 & 0 & 0 & 0 & 0 & 0 & 1 & 1 & 0 & 1 & 0 & 2 & 4 & 8 & 4 & 14 & 4 & 4 & $w_{182}$ & N & can. \\
831 & 0 & 0 & 0 & 0 & 0 & 0 & 1 & 1 & 0 & 1 & 0 & 2 & 4 & 8 & 5 & 9 & 16 & 4 & $w_{339}$ & N & can. \\
832 & 0 & 0 & 0 & 0 & 0 & 0 & 1 & 1 & 0 & 1 & 0 & 2 & 4 & 8 & 5 & 10 & 8 & 3 & $w_{328}$ & N & can. \\
833 & 0 & 0 & 0 & 0 & 0 & 0 & 1 & 1 & 0 & 1 & 0 & 2 & 4 & 8 & 5 & 11 & 8 & 4 & $w_{196}$ & N & can. \\
834 & 0 & 0 & 0 & 0 & 0 & 0 & 1 & 1 & 0 & 1 & 0 & 2 & 4 & 8 & 5 & 12 & 4 & 4 & $w_{323}$ & N & can. \\
835 & 0 & 0 & 0 & 0 & 0 & 0 & 1 & 1 & 0 & 1 & 0 & 2 & 4 & 8 & 5 & 14 & 4 & 4 & $w_{340}$ & N & can. \\
836 & 0 & 0 & 0 & 0 & 0 & 0 & 1 & 1 & 0 & 1 & 0 & 2 & 4 & 8 & 6 & 10 & 16 & 4 & $w_{338}$ & N & can. \\
837 & 0 & 0 & 0 & 0 & 0 & 0 & 1 & 1 & 0 & 1 & 0 & 2 & 4 & 8 & 6 & 11 & 8 & 4 & $w_{224}$ & N & can. \\
838 & 0 & 0 & 0 & 0 & 0 & 0 & 1 & 1 & 0 & 1 & 0 & 2 & 4 & 8 & 6 & 12 & 2 & 4 & $w_{341}$ & N & can. \\
839 & 0 & 0 & 0 & 0 & 0 & 0 & 1 & 1 & 0 & 1 & 0 & 2 & 4 & 8 & 7 & 11 & 16 & 4 & $w_{200}$ & N & can. \\
840 & 0 & 0 & 0 & 0 & 0 & 0 & 1 & 1 & 0 & 1 & 0 & 2 & 4 & 8 & 7 & 12 & 2 & 4 & $w_{225}$ & N & can. \\
841 & 0 & 0 & 0 & 0 & 0 & 0 & 1 & 1 & 0 & 1 & 2 & 3 & 4 & 5 & 7 & 6 & 24576 & 2 & $w_{342}$ & N & can. \\
842 & 0 & 0 & 0 & 0 & 0 & 0 & 1 & 1 & 0 & 1 & 2 & 3 & 4 & 5 & 7 & 8 & 192 & 4 & $w_{343}$ & N & can. \\
843 & 0 & 0 & 0 & 0 & 0 & 0 & 1 & 1 & 0 & 1 & 2 & 3 & 4 & 5 & 8 & 9 & 1536 & 3 & $w_{328}$ & N & can. \\
844 & 0 & 0 & 0 & 0 & 0 & 0 & 1 & 1 & 0 & 1 & 2 & 3 & 4 & 5 & 8 & 10 & 64 & 4 & $w_{344}$ & N & can. \\
845 & 0 & 0 & 0 & 0 & 0 & 0 & 1 & 1 & 0 & 1 & 2 & 3 & 4 & 6 & 4 & 6 & 1024 & 3 & $w_{342}$ & N & can. \\
846 & 0 & 0 & 0 & 0 & 0 & 0 & 1 & 1 & 0 & 1 & 2 & 3 & 4 & 6 & 4 & 7 & 256 & 4 & $w_{249}$ & N & can. \\
847 & 0 & 0 & 0 & 0 & 0 & 0 & 1 & 1 & 0 & 1 & 2 & 3 & 4 & 6 & 4 & 8 & 16 & 4 & $w_{343}$ & N & can. \\
848 & 0 & 0 & 0 & 0 & 0 & 0 & 1 & 1 & 0 & 1 & 2 & 3 & 4 & 6 & 5 & 7 & 1024 & 3 & $w_{84}$ & N & can. \\
849 & 0 & 0 & 0 & 0 & 0 & 0 & 1 & 1 & 0 & 1 & 2 & 3 & 4 & 6 & 5 & 8 & 16 & 4 & $w_{250}$ & N & can. \\
850 & 0 & 0 & 0 & 0 & 0 & 0 & 1 & 1 & 0 & 1 & 2 & 3 & 4 & 6 & 8 & 10 & 128 & 3 & $w_{345}$ & N & can. \\
851 & 0 & 0 & 0 & 0 & 0 & 0 & 1 & 1 & 0 & 1 & 2 & 3 & 4 & 6 & 8 & 11 & 128 & 4 & $w_{346}$ & N & can. \\
852 & 0 & 0 & 0 & 0 & 0 & 0 & 1 & 1 & 0 & 1 & 2 & 3 & 4 & 6 & 8 & 12 & 16 & 4 & $w_{347}$ & N & can. \\
853 & 0 & 0 & 0 & 0 & 0 & 0 & 1 & 1 & 0 & 1 & 2 & 3 & 4 & 8 & 4 & 8 & 64 & 3 & $w_{328}$ & N & can. \\
854 & 0 & 0 & 0 & 0 & 0 & 0 & 1 & 1 & 0 & 1 & 2 & 3 & 4 & 8 & 4 & 9 & 64 & 4 & $w_{334}$ & N & can. \\
855 & 0 & 0 & 0 & 0 & 0 & 0 & 1 & 1 & 0 & 1 & 2 & 3 & 4 & 8 & 4 & 10 & 32 & 4 & $w_{224}$ & N & can. \\
856 & 0 & 0 & 0 & 0 & 0 & 0 & 1 & 1 & 0 & 1 & 2 & 3 & 4 & 8 & 4 & 11 & 32 & 4 & $w_{344}$ & N & can. \\
857 & 0 & 0 & 0 & 0 & 0 & 0 & 1 & 1 & 0 & 1 & 2 & 3 & 4 & 8 & 4 & 12 & 8 & 4 & $w_{344}$ & N & can. \\
858 & 0 & 0 & 0 & 0 & 0 & 0 & 1 & 1 & 0 & 1 & 2 & 3 & 4 & 8 & 6 & 10 & 128 & 3 & $w_{252}$ & N & can. \\
859 & 0 & 0 & 0 & 0 & 0 & 0 & 1 & 1 & 0 & 1 & 2 & 3 & 4 & 8 & 6 & 11 & 64 & 4 & $w_{348}$ & N & can. \\
860 & 0 & 0 & 0 & 0 & 0 & 0 & 1 & 1 & 0 & 1 & 2 & 3 & 4 & 8 & 6 & 12 & 16 & 4 & $w_{264}$ & N & can. \\
861 & 0 & 0 & 0 & 0 & 0 & 0 & 1 & 1 & 0 & 1 & 2 & 3 & 4 & 8 & 7 & 11 & 128 & 3 & $w_{345}$ & N & can. \\
862 & 0 & 0 & 0 & 0 & 0 & 0 & 1 & 1 & 0 & 1 & 2 & 3 & 4 & 8 & 7 & 12 & 16 & 4 & $w_{347}$ & N & can. \\
863 & 0 & 0 & 0 & 0 & 0 & 0 & 1 & 1 & 0 & 1 & 2 & 4 & 2 & 4 & 2 & 4 & 192 & 4 & $w_{307}$ & N & can. \\
864 & 0 & 0 & 0 & 0 & 0 & 0 & 1 & 1 & 0 & 1 & 2 & 4 & 2 & 4 & 2 & 5 & 64 & 4 & $w_{153}$ & N & can. \\
865 & 0 & 0 & 0 & 0 & 0 & 0 & 1 & 1 & 0 & 1 & 2 & 4 & 2 & 4 & 2 & 6 & 32 & 4 & $w_{168}$ & N & can. \\
866 & 0 & 0 & 0 & 0 & 0 & 0 & 1 & 1 & 0 & 1 & 2 & 4 & 2 & 4 & 2 & 7 & 64 & 4 & $w_{320}$ & N & can. \\
867 & 0 & 0 & 0 & 0 & 0 & 0 & 1 & 1 & 0 & 1 & 2 & 4 & 2 & 4 & 2 & 8 & 4 & 4 & $w_{339}$ & N & can. \\
868 & 0 & 0 & 0 & 0 & 0 & 0 & 1 & 1 & 0 & 1 & 2 & 4 & 2 & 4 & 6 & 8 & 32 & 4 & $w_{349}$ & N & can. \\
869 & 0 & 0 & 0 & 0 & 0 & 0 & 1 & 1 & 0 & 1 & 2 & 4 & 2 & 4 & 7 & 8 & 16 & 4 & $w_{222}$ & N & can. \\
870 & 0 & 0 & 0 & 0 & 0 & 0 & 1 & 1 & 0 & 1 & 2 & 4 & 2 & 4 & 8 & 10 & 8 & 4 & $w_{338}$ & N & can. \\
871 & 0 & 0 & 0 & 0 & 0 & 0 & 1 & 1 & 0 & 1 & 2 & 4 & 2 & 4 & 8 & 14 & 32 & 4 & $w_{337}$ & N & can. \\
872 & 0 & 0 & 0 & 0 & 0 & 0 & 1 & 1 & 0 & 1 & 2 & 4 & 2 & 4 & 8 & 15 & 32 & 4 & $w_{224}$ & N & can. \\
873 & 0 & 0 & 0 & 0 & 0 & 0 & 1 & 1 & 0 & 1 & 2 & 4 & 2 & 5 & 2 & 6 & 16 & 4 & $w_{176}$ & N & can. \\
874 & 0 & 0 & 0 & 0 & 0 & 0 & 1 & 1 & 0 & 1 & 2 & 4 & 2 & 5 & 2 & 8 & 4 & 4 & $w_{179}$ & N & can. \\
875 & 0 & 0 & 0 & 0 & 0 & 0 & 1 & 1 & 0 & 1 & 2 & 4 & 2 & 5 & 6 & 8 & 8 & 4 & $w_{250}$ & N & can. \\
876 & 0 & 0 & 0 & 0 & 0 & 0 & 1 & 1 & 0 & 1 & 2 & 4 & 2 & 5 & 8 & 10 & 8 & 4 & $w_{262}$ & N & can. \\
877 & 0 & 0 & 0 & 0 & 0 & 0 & 1 & 1 & 0 & 1 & 2 & 4 & 2 & 5 & 8 & 14 & 16 & 4 & $w_{224}$ & N & can. \\
878 & 0 & 0 & 0 & 0 & 0 & 0 & 1 & 1 & 0 & 1 & 2 & 4 & 2 & 6 & 2 & 8 & 4 & 4 & $w_{178}$ & N & can. \\
879 & 0 & 0 & 0 & 0 & 0 & 0 & 1 & 1 & 0 & 1 & 2 & 4 & 2 & 6 & 3 & 8 & 4 & 4 & $w_{323}$ & N & can. \\
880 & 0 & 0 & 0 & 0 & 0 & 0 & 1 & 1 & 0 & 1 & 2 & 4 & 2 & 6 & 4 & 6 & 48 & 4 & $w_{255}$ & N & can. \\
881 & 0 & 0 & 0 & 0 & 0 & 0 & 1 & 1 & 0 & 1 & 2 & 4 & 2 & 6 & 4 & 7 & 96 & 3 & $w_{342}$ & N & can. \\
882 & 0 & 0 & 0 & 0 & 0 & 0 & 1 & 1 & 0 & 1 & 2 & 4 & 2 & 6 & 4 & 8 & 2 & 4 & $w_{350}$ & N & can. \\
883 & 0 & 0 & 0 & 0 & 0 & 0 & 1 & 1 & 0 & 1 & 2 & 4 & 2 & 6 & 5 & 6 & 96 & 3 & $w_{88}$ & N & can. \\
884 & 0 & 0 & 0 & 0 & 0 & 0 & 1 & 1 & 0 & 1 & 2 & 4 & 2 & 6 & 5 & 8 & 2 & 4 & $w_{240}$ & N & can. \\
885 & 0 & 0 & 0 & 0 & 0 & 0 & 1 & 1 & 0 & 1 & 2 & 4 & 2 & 6 & 8 & 10 & 8 & 4 & $w_{351}$ & N & can. \\
886 & 0 & 0 & 0 & 0 & 0 & 0 & 1 & 1 & 0 & 1 & 2 & 4 & 2 & 6 & 8 & 11 & 8 & 3 & $w_{345}$ & N & can. \\
887 & 0 & 0 & 0 & 0 & 0 & 0 & 1 & 1 & 0 & 1 & 2 & 4 & 2 & 6 & 8 & 12 & 4 & 4 & $w_{352}$ & N & can. \\
888 & 0 & 0 & 0 & 0 & 0 & 0 & 1 & 1 & 0 & 1 & 2 & 4 & 2 & 6 & 8 & 13 & 4 & 4 & $w_{266}$ & N & can. \\
889 & 0 & 0 & 0 & 0 & 0 & 0 & 1 & 1 & 0 & 1 & 2 & 4 & 2 & 8 & 2 & 12 & 12 & 4 & $w_{340}$ & N & can. \\
890 & 0 & 0 & 0 & 0 & 0 & 0 & 1 & 1 & 0 & 1 & 2 & 4 & 2 & 8 & 2 & 13 & 12 & 4 & $w_{182}$ & N & can. \\
891 & 0 & 0 & 0 & 0 & 0 & 0 & 1 & 1 & 0 & 1 & 2 & 4 & 2 & 8 & 2 & 14 & 12 & 4 & $w_{327}$ & N & can. \\
892 & 0 & 0 & 0 & 0 & 0 & 0 & 1 & 1 & 0 & 1 & 2 & 4 & 2 & 8 & 2 & 15 & 12 & 3 & $w_{328}$ & N & can. \\
893 & 0 & 0 & 0 & 0 & 0 & 0 & 1 & 1 & 0 & 1 & 2 & 4 & 2 & 8 & 4 & 8 & 12 & 4 & $w_{327}$ & N & can. \\
894 & 0 & 0 & 0 & 0 & 0 & 0 & 1 & 1 & 0 & 1 & 2 & 4 & 2 & 8 & 4 & 9 & 12 & 3 & $w_{328}$ & N & can. \\
895 & 0 & 0 & 0 & 0 & 0 & 0 & 1 & 1 & 0 & 1 & 2 & 4 & 2 & 8 & 4 & 10 & 2 & 4 & $w_{225}$ & N & can. \\
896 & 0 & 0 & 0 & 0 & 0 & 0 & 1 & 1 & 0 & 1 & 2 & 4 & 2 & 8 & 4 & 11 & 2 & 4 & $w_{341}$ & N & can. \\
897 & 0 & 0 & 0 & 0 & 0 & 0 & 1 & 1 & 0 & 1 & 2 & 4 & 2 & 8 & 4 & 14 & 4 & 4 & $w_{338}$ & N & can. \\
898 & 0 & 0 & 0 & 0 & 0 & 0 & 1 & 1 & 0 & 1 & 2 & 4 & 2 & 8 & 4 & 15 & 4 & 4 & $w_{262}$ & N & can. \\
899 & 0 & 0 & 0 & 0 & 0 & 0 & 1 & 1 & 0 & 1 & 2 & 4 & 2 & 8 & 6 & 10 & 4 & 4 & $w_{263}$ & N & can. \\
900 & 0 & 0 & 0 & 0 & 0 & 0 & 1 & 1 & 0 & 1 & 2 & 4 & 2 & 8 & 6 & 11 & 8 & 3 & $w_{345}$ & N & can. \\
901 & 0 & 0 & 0 & 0 & 0 & 0 & 1 & 1 & 0 & 1 & 2 & 4 & 2 & 8 & 6 & 12 & 4 & 4 & $w_{352}$ & N & can. \\
902 & 0 & 0 & 0 & 0 & 0 & 0 & 1 & 1 & 0 & 1 & 2 & 4 & 2 & 8 & 6 & 13 & 4 & 4 & $w_{264}$ & N & can. \\
903 & 0 & 0 & 0 & 0 & 0 & 0 & 1 & 1 & 0 & 1 & 2 & 4 & 2 & 8 & 6 & 14 & 2 & 4 & $w_{353}$ & N & can. \\
904 & 0 & 0 & 0 & 0 & 0 & 0 & 1 & 1 & 0 & 1 & 2 & 4 & 2 & 8 & 7 & 10 & 8 & 3 & $w_{251}$ & N & can. \\
905 & 0 & 0 & 0 & 0 & 0 & 0 & 1 & 1 & 0 & 1 & 2 & 4 & 2 & 8 & 7 & 12 & 4 & 4 & $w_{264}$ & N & can. \\
906 & 0 & 0 & 0 & 0 & 0 & 0 & 1 & 1 & 0 & 1 & 2 & 4 & 2 & 8 & 7 & 13 & 4 & 4 & $w_{239}$ & N & can. \\
907 & 0 & 0 & 0 & 0 & 0 & 0 & 1 & 1 & 0 & 1 & 2 & 4 & 2 & 8 & 7 & 14 & 2 & 4 & $w_{260}$ & N & can. \\
908 & 0 & 0 & 0 & 0 & 0 & 0 & 1 & 1 & 0 & 1 & 2 & 4 & 2 & 8 & 12 & 14 & 4 & 4 & $w_{260}$ & N & can. \\
909 & 0 & 0 & 0 & 0 & 0 & 0 & 1 & 1 & 0 & 1 & 2 & 4 & 2 & 8 & 13 & 14 & 4 & 4 & $w_{353}$ & N & can. \\
910 & 0 & 0 & 0 & 0 & 0 & 0 & 1 & 1 & 0 & 1 & 2 & 4 & 6 & 8 & 10 & 12 & 16 & 4 & $w_{354}$ & N & can. \\
911 & 0 & 0 & 0 & 0 & 0 & 0 & 1 & 1 & 0 & 1 & 2 & 4 & 6 & 8 & 10 & 13 & 8 & 4 & $w_{278}$ & N & can. \\
912 & 0 & 0 & 0 & 0 & 0 & 0 & 1 & 1 & 0 & 1 & 2 & 4 & 6 & 8 & 10 & 14 & 6 & 4 & $w_{355}$ & N & can. \\
913 & 0 & 0 & 0 & 0 & 0 & 0 & 1 & 1 & 0 & 1 & 2 & 4 & 6 & 8 & 11 & 13 & 16 & 4 & $w_{265}$ & N & can. \\
914 & 0 & 0 & 0 & 0 & 0 & 0 & 1 & 1 & 0 & 1 & 2 & 4 & 6 & 8 & 11 & 14 & 6 & 4 & $w_{281}$ & N & can. \\
915 & 0 & 0 & 0 & 0 & 0 & 0 & 1 & 1 & 0 & 1 & 2 & 4 & 8 & 14 & 10 & 13 & 32 & 4 & $w_{275}$ & N & can. \\
916 & 0 & 0 & 0 & 0 & 0 & 0 & 1 & 1 & 0 & 1 & 2 & 4 & 8 & 14 & 11 & 13 & 192 & 4 & $w_{356}$ & N & can. \\
917 & 0 & 0 & 0 & 0 & 0 & 0 & 1 & 1 & 0 & 2 & 0 & 2 & 0 & 2 & 1 & 3 & 110592 & 2 & $w_{49}$ & N & can. \\
918 & 0 & 0 & 0 & 0 & 0 & 0 & 1 & 1 & 0 & 2 & 0 & 2 & 0 & 2 & 1 & 4 & 576 & 4 & $w_{109}$ & N & can. \\
919 & 0 & 0 & 0 & 0 & 0 & 0 & 1 & 1 & 0 & 2 & 0 & 2 & 0 & 2 & 3 & 1 & 18432 & 3 & $w_{49}$ & Y & \#917 \\
920 & 0 & 0 & 0 & 0 & 0 & 0 & 1 & 1 & 0 & 2 & 0 & 2 & 0 & 2 & 3 & 4 & 96 & 4 & $w_{109}$ & N & \#918 \\
921 & 0 & 0 & 0 & 0 & 0 & 0 & 1 & 1 & 0 & 2 & 0 & 2 & 0 & 2 & 4 & 6 & 384 & 3 & $w_{51}$ & N & can. \\
922 & 0 & 0 & 0 & 0 & 0 & 0 & 1 & 1 & 0 & 2 & 0 & 2 & 0 & 2 & 4 & 7 & 192 & 4 & $w_{197}$ & N & can. \\
923 & 0 & 0 & 0 & 0 & 0 & 0 & 1 & 1 & 0 & 2 & 0 & 2 & 0 & 2 & 4 & 8 & 24 & 4 & $w_{115}$ & N & can. \\
924 & 0 & 0 & 0 & 0 & 0 & 0 & 1 & 1 & 0 & 2 & 0 & 2 & 0 & 3 & 3 & 0 & 9216 & 3 & $w_{60}$ & Y & can. \\
925 & 0 & 0 & 0 & 0 & 0 & 0 & 1 & 1 & 0 & 2 & 0 & 2 & 0 & 3 & 3 & 4 & 64 & 4 & $w_{124}$ & N & can. \\
926 & 0 & 0 & 0 & 0 & 0 & 0 & 1 & 1 & 0 & 2 & 0 & 2 & 0 & 3 & 4 & 6 & 64 & 4 & $w_{197}$ & N & can. \\
927 & 0 & 0 & 0 & 0 & 0 & 0 & 1 & 1 & 0 & 2 & 0 & 2 & 0 & 3 & 4 & 7 & 64 & 3 & $w_{184}$ & Y & can. \\
928 & 0 & 0 & 0 & 0 & 0 & 0 & 1 & 1 & 0 & 2 & 0 & 2 & 0 & 3 & 4 & 8 & 8 & 4 & $w_{214}$ & N & can. \\
929 & 0 & 0 & 0 & 0 & 0 & 0 & 1 & 1 & 0 & 2 & 0 & 2 & 0 & 4 & 4 & 0 & 384 & 3 & $w_{293}$ & N & can. \\
930 & 0 & 0 & 0 & 0 & 0 & 0 & 1 & 1 & 0 & 2 & 0 & 2 & 0 & 4 & 4 & 1 & 64 & 4 & $w_{112}$ & N & can. \\
931 & 0 & 0 & 0 & 0 & 0 & 0 & 1 & 1 & 0 & 2 & 0 & 2 & 0 & 4 & 4 & 3 & 128 & 4 & $w_{307}$ & N & can. \\
932 & 0 & 0 & 0 & 0 & 0 & 0 & 1 & 1 & 0 & 2 & 0 & 2 & 0 & 4 & 4 & 7 & 64 & 4 & $w_{168}$ & N & can. \\
933 & 0 & 0 & 0 & 0 & 0 & 0 & 1 & 1 & 0 & 2 & 0 & 2 & 0 & 4 & 4 & 8 & 8 & 4 & $w_{313}$ & N & can. \\
934 & 0 & 0 & 0 & 0 & 0 & 0 & 1 & 1 & 0 & 2 & 0 & 2 & 0 & 4 & 5 & 0 & 64 & 4 & $w_{112}$ & N & can. \\
935 & 0 & 0 & 0 & 0 & 0 & 0 & 1 & 1 & 0 & 2 & 0 & 2 & 0 & 4 & 5 & 1 & 128 & 3 & $w_{51}$ & Y & \#921 \\
936 & 0 & 0 & 0 & 0 & 0 & 0 & 1 & 1 & 0 & 2 & 0 & 2 & 0 & 4 & 5 & 2 & 64 & 4 & $w_{52}$ & N & can. \\
937 & 0 & 0 & 0 & 0 & 0 & 0 & 1 & 1 & 0 & 2 & 0 & 2 & 0 & 4 & 5 & 3 & 64 & 4 & $w_{197}$ & N & \#926 \\
938 & 0 & 0 & 0 & 0 & 0 & 0 & 1 & 1 & 0 & 2 & 0 & 2 & 0 & 4 & 5 & 7 & 32 & 4 & $w_{186}$ & N & can. \\
939 & 0 & 0 & 0 & 0 & 0 & 0 & 1 & 1 & 0 & 2 & 0 & 2 & 0 & 4 & 5 & 8 & 4 & 4 & $w_{189}$ & N & can. \\
940 & 0 & 0 & 0 & 0 & 0 & 0 & 1 & 1 & 0 & 2 & 0 & 2 & 0 & 4 & 7 & 0 & 128 & 4 & $w_{307}$ & N & can. \\
941 & 0 & 0 & 0 & 0 & 0 & 0 & 1 & 1 & 0 & 2 & 0 & 2 & 0 & 4 & 7 & 1 & 64 & 4 & $w_{197}$ & N & \#922 \\
942 & 0 & 0 & 0 & 0 & 0 & 0 & 1 & 1 & 0 & 2 & 0 & 2 & 0 & 4 & 7 & 3 & 128 & 3 & $w_{184}$ & N & \#927 \\
943 & 0 & 0 & 0 & 0 & 0 & 0 & 1 & 1 & 0 & 2 & 0 & 2 & 0 & 4 & 7 & 5 & 64 & 4 & $w_{186}$ & N & \#938 \\
944 & 0 & 0 & 0 & 0 & 0 & 0 & 1 & 1 & 0 & 2 & 0 & 2 & 0 & 4 & 7 & 8 & 8 & 4 & $w_{215}$ & N & can. \\
945 & 0 & 0 & 0 & 0 & 0 & 0 & 1 & 1 & 0 & 2 & 0 & 2 & 0 & 4 & 8 & 0 & 16 & 4 & $w_{357}$ & N & can. \\
946 & 0 & 0 & 0 & 0 & 0 & 0 & 1 & 1 & 0 & 2 & 0 & 2 & 0 & 4 & 8 & 1 & 8 & 4 & $w_{115}$ & N & \#923 \\
947 & 0 & 0 & 0 & 0 & 0 & 0 & 1 & 1 & 0 & 2 & 0 & 2 & 0 & 4 & 8 & 3 & 16 & 4 & $w_{214}$ & N & \#928 \\
948 & 0 & 0 & 0 & 0 & 0 & 0 & 1 & 1 & 0 & 2 & 0 & 2 & 0 & 4 & 8 & 5 & 4 & 4 & $w_{189}$ & N & \#939 \\
949 & 0 & 0 & 0 & 0 & 0 & 0 & 1 & 1 & 0 & 2 & 0 & 2 & 0 & 4 & 8 & 7 & 4 & 4 & $w_{215}$ & N & \#944 \\
950 & 0 & 0 & 0 & 0 & 0 & 0 & 1 & 1 & 0 & 2 & 0 & 2 & 0 & 4 & 8 & 12 & 8 & 3 & $w_{190}$ & N & can. \\
951 & 0 & 0 & 0 & 0 & 0 & 0 & 1 & 1 & 0 & 2 & 0 & 2 & 0 & 4 & 8 & 13 & 4 & 4 & $w_{205}$ & N & can. \\
952 & 0 & 0 & 0 & 0 & 0 & 0 & 1 & 1 & 0 & 2 & 0 & 2 & 0 & 4 & 8 & 15 & 8 & 4 & $w_{358}$ & N & can. \\
953 & 0 & 0 & 0 & 0 & 0 & 0 & 1 & 1 & 0 & 2 & 0 & 2 & 4 & 6 & 5 & 7 & 4096 & 2 & $w_{80}$ & N & can. \\
954 & 0 & 0 & 0 & 0 & 0 & 0 & 1 & 1 & 0 & 2 & 0 & 2 & 4 & 6 & 5 & 8 & 32 & 4 & $w_{221}$ & N & can. \\
955 & 0 & 0 & 0 & 0 & 0 & 0 & 1 & 1 & 0 & 2 & 0 & 2 & 4 & 6 & 7 & 5 & 2048 & 3 & $w_{80}$ & Y & \#953 \\
956 & 0 & 0 & 0 & 0 & 0 & 0 & 1 & 1 & 0 & 2 & 0 & 2 & 4 & 6 & 7 & 8 & 16 & 4 & $w_{221}$ & N & \#954 \\
957 & 0 & 0 & 0 & 0 & 0 & 0 & 1 & 1 & 0 & 2 & 0 & 2 & 4 & 6 & 8 & 10 & 128 & 3 & $w_{86}$ & N & can. \\
958 & 0 & 0 & 0 & 0 & 0 & 0 & 1 & 1 & 0 & 2 & 0 & 2 & 4 & 6 & 8 & 11 & 32 & 4 & $w_{229}$ & N & can. \\
959 & 0 & 0 & 0 & 0 & 0 & 0 & 1 & 1 & 0 & 2 & 0 & 2 & 4 & 6 & 8 & 12 & 8 & 4 & $w_{200}$ & N & can. \\
960 & 0 & 0 & 0 & 0 & 0 & 0 & 1 & 1 & 0 & 2 & 0 & 2 & 4 & 7 & 7 & 4 & 3072 & 3 & $w_{84}$ & N & can. \\
961 & 0 & 0 & 0 & 0 & 0 & 0 & 1 & 1 & 0 & 2 & 0 & 2 & 4 & 7 & 7 & 8 & 32 & 4 & $w_{222}$ & N & can. \\
962 & 0 & 0 & 0 & 0 & 0 & 0 & 1 & 1 & 0 & 2 & 0 & 2 & 4 & 7 & 8 & 11 & 64 & 3 & $w_{252}$ & N & can. \\
963 & 0 & 0 & 0 & 0 & 0 & 0 & 1 & 1 & 0 & 2 & 0 & 2 & 4 & 7 & 8 & 12 & 8 & 4 & $w_{239}$ & N & can. \\
964 & 0 & 0 & 0 & 0 & 0 & 0 & 1 & 1 & 0 & 2 & 0 & 2 & 4 & 8 & 8 & 4 & 384 & 3 & $w_{302}$ & N & can. \\
965 & 0 & 0 & 0 & 0 & 0 & 0 & 1 & 1 & 0 & 2 & 0 & 2 & 4 & 8 & 8 & 5 & 32 & 4 & $w_{205}$ & N & can. \\
966 & 0 & 0 & 0 & 0 & 0 & 0 & 1 & 1 & 0 & 2 & 0 & 2 & 4 & 8 & 8 & 7 & 64 & 4 & $w_{337}$ & N & can. \\
967 & 0 & 0 & 0 & 0 & 0 & 0 & 1 & 1 & 0 & 2 & 0 & 2 & 4 & 8 & 8 & 12 & 16 & 4 & $w_{329}$ & N & can. \\
968 & 0 & 0 & 0 & 0 & 0 & 0 & 1 & 1 & 0 & 2 & 0 & 2 & 4 & 8 & 9 & 5 & 128 & 3 & $w_{86}$ & N & \#957 \\
969 & 0 & 0 & 0 & 0 & 0 & 0 & 1 & 1 & 0 & 2 & 0 & 2 & 4 & 8 & 9 & 6 & 64 & 4 & $w_{87}$ & N & can. \\
970 & 0 & 0 & 0 & 0 & 0 & 0 & 1 & 1 & 0 & 2 & 0 & 2 & 4 & 8 & 9 & 7 & 32 & 4 & $w_{229}$ & N & \#958 \\
971 & 0 & 0 & 0 & 0 & 0 & 0 & 1 & 1 & 0 & 2 & 0 & 2 & 4 & 8 & 9 & 12 & 8 & 4 & $w_{200}$ & N & \#959 \\
972 & 0 & 0 & 0 & 0 & 0 & 0 & 1 & 1 & 0 & 2 & 0 & 2 & 4 & 8 & 11 & 7 & 128 & 3 & $w_{252}$ & N & \#962 \\
973 & 0 & 0 & 0 & 0 & 0 & 0 & 1 & 1 & 0 & 2 & 0 & 2 & 4 & 8 & 11 & 12 & 16 & 4 & $w_{239}$ & N & \#963 \\
974 & 0 & 0 & 0 & 0 & 0 & 0 & 1 & 1 & 0 & 2 & 0 & 3 & 0 & 4 & 1 & 4 & 384 & 3 & $w_{51}$ & Y & \#921 \\
975 & 0 & 0 & 0 & 0 & 0 & 0 & 1 & 1 & 0 & 2 & 0 & 3 & 0 & 4 & 1 & 6 & 96 & 4 & $w_{114}$ & N & can. \\
976 & 0 & 0 & 0 & 0 & 0 & 0 & 1 & 1 & 0 & 2 & 0 & 3 & 0 & 4 & 1 & 8 & 24 & 4 & $w_{115}$ & N & \#923 \\
977 & 0 & 0 & 0 & 0 & 0 & 0 & 1 & 1 & 0 & 2 & 0 & 3 & 0 & 4 & 2 & 4 & 16 & 4 & $w_{114}$ & N & can. \\
978 & 0 & 0 & 0 & 0 & 0 & 0 & 1 & 1 & 0 & 2 & 0 & 3 & 0 & 4 & 2 & 6 & 32 & 4 & $w_{153}$ & N & can. \\
979 & 0 & 0 & 0 & 0 & 0 & 0 & 1 & 1 & 0 & 2 & 0 & 3 & 0 & 4 & 2 & 7 & 32 & 3 & $w_{165}$ & Y & can. \\
980 & 0 & 0 & 0 & 0 & 0 & 0 & 1 & 1 & 0 & 2 & 0 & 3 & 0 & 4 & 2 & 8 & 4 & 4 & $w_{167}$ & N & can. \\
981 & 0 & 0 & 0 & 0 & 0 & 0 & 1 & 1 & 0 & 2 & 0 & 3 & 0 & 4 & 4 & 1 & 128 & 3 & $w_{62}$ & N & can. \\
982 & 0 & 0 & 0 & 0 & 0 & 0 & 1 & 1 & 0 & 2 & 0 & 3 & 0 & 4 & 4 & 2 & 16 & 4 & $w_{114}$ & N & can. \\
983 & 0 & 0 & 0 & 0 & 0 & 0 & 1 & 1 & 0 & 2 & 0 & 3 & 0 & 4 & 4 & 6 & 16 & 4 & $w_{168}$ & N & can. \\
984 & 0 & 0 & 0 & 0 & 0 & 0 & 1 & 1 & 0 & 2 & 0 & 3 & 0 & 4 & 4 & 8 & 4 & 4 & $w_{169}$ & N & can. \\
985 & 0 & 0 & 0 & 0 & 0 & 0 & 1 & 1 & 0 & 2 & 0 & 3 & 0 & 4 & 5 & 0 & 128 & 3 & $w_{62}$ & Y & can. \\
986 & 0 & 0 & 0 & 0 & 0 & 0 & 1 & 1 & 0 & 2 & 0 & 3 & 0 & 4 & 5 & 6 & 16 & 4 & $w_{168}$ & N & can. \\
987 & 0 & 0 & 0 & 0 & 0 & 0 & 1 & 1 & 0 & 2 & 0 & 3 & 0 & 4 & 5 & 8 & 4 & 4 & $w_{169}$ & N & can. \\
988 & 0 & 0 & 0 & 0 & 0 & 0 & 1 & 1 & 0 & 2 & 0 & 3 & 0 & 4 & 6 & 0 & 32 & 4 & $w_{114}$ & N & can. \\
989 & 0 & 0 & 0 & 0 & 0 & 0 & 1 & 1 & 0 & 2 & 0 & 3 & 0 & 4 & 6 & 1 & 32 & 4 & $w_{114}$ & N & can. \\
990 & 0 & 0 & 0 & 0 & 0 & 0 & 1 & 1 & 0 & 2 & 0 & 3 & 0 & 4 & 6 & 2 & 32 & 4 & $w_{197}$ & N & \#926 \\
991 & 0 & 0 & 0 & 0 & 0 & 0 & 1 & 1 & 0 & 2 & 0 & 3 & 0 & 4 & 6 & 3 & 32 & 3 & $w_{184}$ & Y & \#927 \\
992 & 0 & 0 & 0 & 0 & 0 & 0 & 1 & 1 & 0 & 2 & 0 & 3 & 0 & 4 & 6 & 4 & 16 & 4 & $w_{186}$ & N & \#938 \\
993 & 0 & 0 & 0 & 0 & 0 & 0 & 1 & 1 & 0 & 2 & 0 & 3 & 0 & 4 & 6 & 5 & 16 & 4 & $w_{186}$ & N & can. \\
994 & 0 & 0 & 0 & 0 & 0 & 0 & 1 & 1 & 0 & 2 & 0 & 3 & 0 & 4 & 6 & 8 & 2 & 4 & $w_{215}$ & N & can. \\
995 & 0 & 0 & 0 & 0 & 0 & 0 & 1 & 1 & 0 & 2 & 0 & 3 & 0 & 4 & 8 & 0 & 8 & 4 & $w_{129}$ & N & can. \\
996 & 0 & 0 & 0 & 0 & 0 & 0 & 1 & 1 & 0 & 2 & 0 & 3 & 0 & 4 & 8 & 1 & 8 & 4 & $w_{129}$ & N & can. \\
997 & 0 & 0 & 0 & 0 & 0 & 0 & 1 & 1 & 0 & 2 & 0 & 3 & 0 & 4 & 8 & 2 & 4 & 4 & $w_{214}$ & N & \#928 \\
998 & 0 & 0 & 0 & 0 & 0 & 0 & 1 & 1 & 0 & 2 & 0 & 3 & 0 & 4 & 8 & 4 & 4 & 4 & $w_{189}$ & N & \#939 \\
999 & 0 & 0 & 0 & 0 & 0 & 0 & 1 & 1 & 0 & 2 & 0 & 3 & 0 & 4 & 8 & 5 & 4 & 4 & $w_{189}$ & N & can. \\
1000 & 0 & 0 & 0 & 0 & 0 & 0 & 1 & 1 & 0 & 2 & 0 & 3 & 0 & 4 & 8 & 6 & 2 & 4 & $w_{177}$ & N & can. \\
1001 & 0 & 0 & 0 & 0 & 0 & 0 & 1 & 1 & 0 & 2 & 0 & 3 & 0 & 4 & 8 & 10 & 2 & 4 & $w_{215}$ & N & \#994 \\
1002 & 0 & 0 & 0 & 0 & 0 & 0 & 1 & 1 & 0 & 2 & 0 & 3 & 0 & 4 & 8 & 12 & 4 & 4 & $w_{205}$ & N & can. \\
1003 & 0 & 0 & 0 & 0 & 0 & 0 & 1 & 1 & 0 & 2 & 0 & 3 & 0 & 4 & 8 & 13 & 4 & 3 & $w_{90}$ & Y & can. \\
1004 & 0 & 0 & 0 & 0 & 0 & 0 & 1 & 1 & 0 & 2 & 0 & 3 & 0 & 4 & 8 & 14 & 2 & 4 & $w_{183}$ & N & can. \\
1005 & 0 & 0 & 0 & 0 & 0 & 0 & 1 & 1 & 0 & 2 & 0 & 3 & 4 & 6 & 4 & 7 & 512 & 3 & $w_{80}$ & Y & \#953 \\
1006 & 0 & 0 & 0 & 0 & 0 & 0 & 1 & 1 & 0 & 2 & 0 & 3 & 4 & 6 & 4 & 8 & 4 & 4 & $w_{221}$ & N & \#954 \\
1007 & 0 & 0 & 0 & 0 & 0 & 0 & 1 & 1 & 0 & 2 & 0 & 3 & 4 & 6 & 6 & 5 & 256 & 3 & $w_{84}$ & Y & can. \\
1008 & 0 & 0 & 0 & 0 & 0 & 0 & 1 & 1 & 0 & 2 & 0 & 3 & 4 & 6 & 6 & 8 & 4 & 4 & $w_{222}$ & N & can. \\
1009 & 0 & 0 & 0 & 0 & 0 & 0 & 1 & 1 & 0 & 2 & 0 & 3 & 4 & 6 & 8 & 10 & 32 & 4 & $w_{229}$ & N & can. \\
1010 & 0 & 0 & 0 & 0 & 0 & 0 & 1 & 1 & 0 & 2 & 0 & 3 & 4 & 6 & 8 & 11 & 32 & 3 & $w_{252}$ & Y & can. \\
1011 & 0 & 0 & 0 & 0 & 0 & 0 & 1 & 1 & 0 & 2 & 0 & 3 & 4 & 6 & 8 & 12 & 4 & 4 & $w_{239}$ & N & can. \\
1012 & 0 & 0 & 0 & 0 & 0 & 0 & 1 & 1 & 0 & 2 & 0 & 3 & 4 & 8 & 4 & 9 & 64 & 3 & $w_{86}$ & Y & \#957 \\
1013 & 0 & 0 & 0 & 0 & 0 & 0 & 1 & 1 & 0 & 2 & 0 & 3 & 4 & 8 & 4 & 10 & 8 & 4 & $w_{200}$ & N & can. \\
1014 & 0 & 0 & 0 & 0 & 0 & 0 & 1 & 1 & 0 & 2 & 0 & 3 & 4 & 8 & 4 & 12 & 8 & 4 & $w_{200}$ & N & can. \\
1015 & 0 & 0 & 0 & 0 & 0 & 0 & 1 & 1 & 0 & 2 & 0 & 3 & 4 & 8 & 4 & 14 & 8 & 4 & $w_{200}$ & N & \#959 \\
1016 & 0 & 0 & 0 & 0 & 0 & 0 & 1 & 1 & 0 & 2 & 0 & 3 & 4 & 8 & 6 & 10 & 32 & 4 & $w_{224}$ & N & can. \\
1017 & 0 & 0 & 0 & 0 & 0 & 0 & 1 & 1 & 0 & 2 & 0 & 3 & 4 & 8 & 6 & 11 & 32 & 3 & $w_{251}$ & N & can. \\
1018 & 0 & 0 & 0 & 0 & 0 & 0 & 1 & 1 & 0 & 2 & 0 & 3 & 4 & 8 & 6 & 12 & 4 & 4 & $w_{241}$ & N & can. \\
1019 & 0 & 0 & 0 & 0 & 0 & 0 & 1 & 1 & 0 & 2 & 0 & 3 & 4 & 8 & 8 & 5 & 64 & 3 & $w_{90}$ & N & can. \\
1020 & 0 & 0 & 0 & 0 & 0 & 0 & 1 & 1 & 0 & 2 & 0 & 3 & 4 & 8 & 8 & 6 & 8 & 4 & $w_{200}$ & N & can. \\
1021 & 0 & 0 & 0 & 0 & 0 & 0 & 1 & 1 & 0 & 2 & 0 & 3 & 4 & 8 & 8 & 12 & 8 & 4 & $w_{202}$ & N & can. \\
1022 & 0 & 0 & 0 & 0 & 0 & 0 & 1 & 1 & 0 & 2 & 0 & 3 & 4 & 8 & 8 & 14 & 8 & 4 & $w_{202}$ & N & can. \\
1023 & 0 & 0 & 0 & 0 & 0 & 0 & 1 & 1 & 0 & 2 & 0 & 3 & 4 & 8 & 10 & 6 & 32 & 4 & $w_{229}$ & N & \#1009 \\
1024 & 0 & 0 & 0 & 0 & 0 & 0 & 1 & 1 & 0 & 2 & 0 & 3 & 4 & 8 & 10 & 7 & 32 & 3 & $w_{252}$ & Y & \#1010 \\
1025 & 0 & 0 & 0 & 0 & 0 & 0 & 1 & 1 & 0 & 2 & 0 & 3 & 4 & 8 & 10 & 12 & 4 & 4 & $w_{239}$ & N & \#1011 \\
1026 & 0 & 0 & 0 & 0 & 0 & 0 & 1 & 1 & 0 & 2 & 0 & 4 & 0 & 6 & 1 & 8 & 36 & 4 & $w_{359}$ & N & can. \\
1027 & 0 & 0 & 0 & 0 & 0 & 0 & 1 & 1 & 0 & 2 & 0 & 4 & 0 & 6 & 2 & 4 & 16 & 4 & $w_{153}$ & N & can. \\
1028 & 0 & 0 & 0 & 0 & 0 & 0 & 1 & 1 & 0 & 2 & 0 & 4 & 0 & 6 & 2 & 5 & 16 & 4 & $w_{168}$ & N & can. \\
1029 & 0 & 0 & 0 & 0 & 0 & 0 & 1 & 1 & 0 & 2 & 0 & 4 & 0 & 6 & 2 & 8 & 4 & 4 & $w_{171}$ & N & can. \\
1030 & 0 & 0 & 0 & 0 & 0 & 0 & 1 & 1 & 0 & 2 & 0 & 4 & 0 & 6 & 3 & 4 & 16 & 4 & $w_{168}$ & N & can. \\
1031 & 0 & 0 & 0 & 0 & 0 & 0 & 1 & 1 & 0 & 2 & 0 & 4 & 0 & 6 & 3 & 5 & 16 & 4 & $w_{193}$ & N & can. \\
1032 & 0 & 0 & 0 & 0 & 0 & 0 & 1 & 1 & 0 & 2 & 0 & 4 & 0 & 6 & 3 & 8 & 4 & 4 & $w_{194}$ & N & can. \\
1033 & 0 & 0 & 0 & 0 & 0 & 0 & 1 & 1 & 0 & 2 & 0 & 4 & 0 & 6 & 8 & 0 & 12 & 4 & $w_{116}$ & N & can. \\
1034 & 0 & 0 & 0 & 0 & 0 & 0 & 1 & 1 & 0 & 2 & 0 & 4 & 0 & 6 & 8 & 1 & 12 & 4 & $w_{359}$ & N & \#1026 \\
1035 & 0 & 0 & 0 & 0 & 0 & 0 & 1 & 1 & 0 & 2 & 0 & 4 & 0 & 6 & 8 & 2 & 4 & 4 & $w_{171}$ & N & can. \\
1036 & 0 & 0 & 0 & 0 & 0 & 0 & 1 & 1 & 0 & 2 & 0 & 4 & 0 & 6 & 8 & 3 & 4 & 4 & $w_{194}$ & N & \#1032 \\
1037 & 0 & 0 & 0 & 0 & 0 & 0 & 1 & 1 & 0 & 2 & 0 & 4 & 0 & 6 & 8 & 10 & 4 & 4 & $w_{199}$ & N & can. \\
1038 & 0 & 0 & 0 & 0 & 0 & 0 & 1 & 1 & 0 & 2 & 0 & 4 & 0 & 6 & 8 & 11 & 4 & 4 & $w_{202}$ & N & can. \\
1039 & 0 & 0 & 0 & 0 & 0 & 0 & 1 & 1 & 0 & 2 & 0 & 4 & 0 & 7 & 1 & 8 & 36 & 4 & $w_{360}$ & N & can. \\
1040 & 0 & 0 & 0 & 0 & 0 & 0 & 1 & 1 & 0 & 2 & 0 & 4 & 0 & 7 & 2 & 4 & 16 & 4 & $w_{168}$ & N & can. \\
1041 & 0 & 0 & 0 & 0 & 0 & 0 & 1 & 1 & 0 & 2 & 0 & 4 & 0 & 7 & 2 & 5 & 16 & 4 & $w_{336}$ & N & can. \\
1042 & 0 & 0 & 0 & 0 & 0 & 0 & 1 & 1 & 0 & 2 & 0 & 4 & 0 & 7 & 2 & 8 & 4 & 4 & $w_{361}$ & N & can. \\
1043 & 0 & 0 & 0 & 0 & 0 & 0 & 1 & 1 & 0 & 2 & 0 & 4 & 0 & 7 & 3 & 4 & 16 & 4 & $w_{68}$ & N & can. \\
1044 & 0 & 0 & 0 & 0 & 0 & 0 & 1 & 1 & 0 & 2 & 0 & 4 & 0 & 7 & 3 & 5 & 16 & 4 & $w_{168}$ & N & can. \\
1045 & 0 & 0 & 0 & 0 & 0 & 0 & 1 & 1 & 0 & 2 & 0 & 4 & 0 & 7 & 3 & 8 & 4 & 4 & $w_{170}$ & N & can. \\
1046 & 0 & 0 & 0 & 0 & 0 & 0 & 1 & 1 & 0 & 2 & 0 & 4 & 0 & 7 & 8 & 0 & 12 & 4 & $w_{360}$ & N & can. \\
1047 & 0 & 0 & 0 & 0 & 0 & 0 & 1 & 1 & 0 & 2 & 0 & 4 & 0 & 7 & 8 & 1 & 12 & 4 & $w_{117}$ & N & can. \\
1048 & 0 & 0 & 0 & 0 & 0 & 0 & 1 & 1 & 0 & 2 & 0 & 4 & 0 & 7 & 8 & 2 & 4 & 4 & $w_{170}$ & N & can. \\
1049 & 0 & 0 & 0 & 0 & 0 & 0 & 1 & 1 & 0 & 2 & 0 & 4 & 0 & 7 & 8 & 3 & 4 & 4 & $w_{361}$ & N & can. \\
1050 & 0 & 0 & 0 & 0 & 0 & 0 & 1 & 1 & 0 & 2 & 0 & 4 & 0 & 7 & 8 & 10 & 4 & 4 & $w_{337}$ & N & can. \\
1051 & 0 & 0 & 0 & 0 & 0 & 0 & 1 & 1 & 0 & 2 & 0 & 4 & 0 & 7 & 8 & 11 & 4 & 4 & $w_{202}$ & N & can. \\
1052 & 0 & 0 & 0 & 0 & 0 & 0 & 1 & 1 & 0 & 2 & 0 & 4 & 0 & 8 & 1 & 14 & 144 & 4 & $w_{119}$ & N & can. \\
1053 & 0 & 0 & 0 & 0 & 0 & 0 & 1 & 1 & 0 & 2 & 0 & 4 & 0 & 8 & 1 & 15 & 144 & 3 & $w_{71}$ & N & can. \\
1054 & 0 & 0 & 0 & 0 & 0 & 0 & 1 & 1 & 0 & 2 & 0 & 4 & 0 & 8 & 2 & 4 & 2 & 4 & $w_{162}$ & N & can. \\
1055 & 0 & 0 & 0 & 0 & 0 & 0 & 1 & 1 & 0 & 2 & 0 & 4 & 0 & 8 & 2 & 5 & 2 & 4 & $w_{362}$ & N & can. \\
1056 & 0 & 0 & 0 & 0 & 0 & 0 & 1 & 1 & 0 & 2 & 0 & 4 & 0 & 8 & 2 & 6 & 2 & 4 & $w_{339}$ & N & can. \\
1057 & 0 & 0 & 0 & 0 & 0 & 0 & 1 & 1 & 0 & 2 & 0 & 4 & 0 & 8 & 2 & 7 & 2 & 4 & $w_{177}$ & N & can. \\
1058 & 0 & 0 & 0 & 0 & 0 & 0 & 1 & 1 & 0 & 2 & 0 & 4 & 0 & 8 & 2 & 12 & 4 & 3 & $w_{363}$ & N & can. \\
1059 & 0 & 0 & 0 & 0 & 0 & 0 & 1 & 1 & 0 & 2 & 0 & 4 & 0 & 8 & 2 & 13 & 4 & 4 & $w_{227}$ & N & can. \\
1060 & 0 & 0 & 0 & 0 & 0 & 0 & 1 & 1 & 0 & 2 & 0 & 4 & 0 & 8 & 2 & 14 & 4 & 4 & $w_{174}$ & N & can. \\
1061 & 0 & 0 & 0 & 0 & 0 & 0 & 1 & 1 & 0 & 2 & 0 & 4 & 0 & 8 & 2 & 15 & 4 & 4 & $w_{364}$ & N & can. \\
1062 & 0 & 0 & 0 & 0 & 0 & 0 & 1 & 1 & 0 & 2 & 0 & 4 & 0 & 8 & 3 & 4 & 2 & 4 & $w_{135}$ & N & can. \\
1063 & 0 & 0 & 0 & 0 & 0 & 0 & 1 & 1 & 0 & 2 & 0 & 4 & 0 & 8 & 3 & 5 & 2 & 4 & $w_{198}$ & N & can. \\
1064 & 0 & 0 & 0 & 0 & 0 & 0 & 1 & 1 & 0 & 2 & 0 & 4 & 0 & 8 & 3 & 6 & 2 & 4 & $w_{177}$ & N & \#1000 \\
1065 & 0 & 0 & 0 & 0 & 0 & 0 & 1 & 1 & 0 & 2 & 0 & 4 & 0 & 8 & 3 & 7 & 2 & 4 & $w_{215}$ & N & \#994 \\
1066 & 0 & 0 & 0 & 0 & 0 & 0 & 1 & 1 & 0 & 2 & 0 & 4 & 0 & 8 & 3 & 12 & 4 & 4 & $w_{227}$ & N & can. \\
1067 & 0 & 0 & 0 & 0 & 0 & 0 & 1 & 1 & 0 & 2 & 0 & 4 & 0 & 8 & 3 & 13 & 4 & 3 & $w_{90}$ & Y & \#1003 \\
1068 & 0 & 0 & 0 & 0 & 0 & 0 & 1 & 1 & 0 & 2 & 0 & 4 & 0 & 8 & 3 & 14 & 4 & 4 & $w_{175}$ & N & can. \\
1069 & 0 & 0 & 0 & 0 & 0 & 0 & 1 & 1 & 0 & 2 & 0 & 4 & 0 & 8 & 3 & 15 & 4 & 4 & $w_{234}$ & N & can. \\
1070 & 0 & 0 & 0 & 0 & 0 & 0 & 1 & 1 & 0 & 2 & 0 & 4 & 0 & 8 & 6 & 2 & 2 & 4 & $w_{189}$ & N & \#939 \\
1071 & 0 & 0 & 0 & 0 & 0 & 0 & 1 & 1 & 0 & 2 & 0 & 4 & 0 & 8 & 6 & 3 & 2 & 4 & $w_{215}$ & N & \#994 \\
1072 & 0 & 0 & 0 & 0 & 0 & 0 & 1 & 1 & 0 & 2 & 0 & 4 & 0 & 8 & 6 & 8 & 4 & 3 & $w_{190}$ & Y & \#950 \\
1073 & 0 & 0 & 0 & 0 & 0 & 0 & 1 & 1 & 0 & 2 & 0 & 4 & 0 & 8 & 6 & 9 & 4 & 4 & $w_{205}$ & N & \#1002 \\
1074 & 0 & 0 & 0 & 0 & 0 & 0 & 1 & 1 & 0 & 2 & 0 & 4 & 0 & 8 & 6 & 10 & 2 & 4 & $w_{182}$ & N & can. \\
1075 & 0 & 0 & 0 & 0 & 0 & 0 & 1 & 1 & 0 & 2 & 0 & 4 & 0 & 8 & 6 & 11 & 2 & 4 & $w_{203}$ & N & can. \\
1076 & 0 & 0 & 0 & 0 & 0 & 0 & 1 & 1 & 0 & 2 & 0 & 4 & 0 & 8 & 6 & 14 & 4 & 4 & $w_{206}$ & N & can. \\
1077 & 0 & 0 & 0 & 0 & 0 & 0 & 1 & 1 & 0 & 2 & 0 & 4 & 0 & 8 & 6 & 15 & 4 & 4 & $w_{200}$ & N & can. \\
1078 & 0 & 0 & 0 & 0 & 0 & 0 & 1 & 1 & 0 & 2 & 0 & 4 & 0 & 8 & 7 & 2 & 2 & 4 & $w_{215}$ & N & \#944 \\
1079 & 0 & 0 & 0 & 0 & 0 & 0 & 1 & 1 & 0 & 2 & 0 & 4 & 0 & 8 & 7 & 3 & 2 & 4 & $w_{215}$ & N & \#994 \\
1080 & 0 & 0 & 0 & 0 & 0 & 0 & 1 & 1 & 0 & 2 & 0 & 4 & 0 & 8 & 7 & 8 & 4 & 4 & $w_{205}$ & N & \#951 \\
1081 & 0 & 0 & 0 & 0 & 0 & 0 & 1 & 1 & 0 & 2 & 0 & 4 & 0 & 8 & 7 & 9 & 4 & 3 & $w_{90}$ & N & \#1003 \\
1082 & 0 & 0 & 0 & 0 & 0 & 0 & 1 & 1 & 0 & 2 & 0 & 4 & 0 & 8 & 7 & 10 & 2 & 4 & $w_{365}$ & N & can. \\
1083 & 0 & 0 & 0 & 0 & 0 & 0 & 1 & 1 & 0 & 2 & 0 & 4 & 0 & 8 & 7 & 11 & 2 & 4 & $w_{230}$ & N & can. \\
1084 & 0 & 0 & 0 & 0 & 0 & 0 & 1 & 1 & 0 & 2 & 0 & 4 & 0 & 8 & 7 & 14 & 4 & 4 & $w_{200}$ & N & can. \\
1085 & 0 & 0 & 0 & 0 & 0 & 0 & 1 & 1 & 0 & 2 & 0 & 4 & 0 & 8 & 7 & 15 & 4 & 4 & $w_{200}$ & N & \#1077 \\
1086 & 0 & 0 & 0 & 0 & 0 & 0 & 1 & 1 & 0 & 2 & 0 & 4 & 0 & 8 & 14 & 0 & 48 & 3 & $w_{366}$ & N & can. \\
1087 & 0 & 0 & 0 & 0 & 0 & 0 & 1 & 1 & 0 & 2 & 0 & 4 & 0 & 8 & 14 & 1 & 48 & 4 & $w_{367}$ & N & can. \\
1088 & 0 & 0 & 0 & 0 & 0 & 0 & 1 & 1 & 0 & 2 & 0 & 4 & 0 & 8 & 14 & 2 & 4 & 4 & $w_{174}$ & N & can. \\
1089 & 0 & 0 & 0 & 0 & 0 & 0 & 1 & 1 & 0 & 2 & 0 & 4 & 0 & 8 & 14 & 3 & 4 & 4 & $w_{364}$ & N & can. \\
1090 & 0 & 0 & 0 & 0 & 0 & 0 & 1 & 1 & 0 & 2 & 0 & 4 & 0 & 8 & 14 & 6 & 4 & 4 & $w_{329}$ & N & can. \\
1091 & 0 & 0 & 0 & 0 & 0 & 0 & 1 & 1 & 0 & 2 & 0 & 4 & 0 & 8 & 14 & 7 & 4 & 4 & $w_{202}$ & N & can. \\
1092 & 0 & 0 & 0 & 0 & 0 & 0 & 1 & 1 & 0 & 2 & 0 & 4 & 0 & 8 & 15 & 0 & 48 & 4 & $w_{367}$ & N & can. \\
1093 & 0 & 0 & 0 & 0 & 0 & 0 & 1 & 1 & 0 & 2 & 0 & 4 & 0 & 8 & 15 & 1 & 48 & 3 & $w_{71}$ & Y & \#1053 \\
1094 & 0 & 0 & 0 & 0 & 0 & 0 & 1 & 1 & 0 & 2 & 0 & 4 & 0 & 8 & 15 & 2 & 4 & 4 & $w_{175}$ & N & can. \\
1095 & 0 & 0 & 0 & 0 & 0 & 0 & 1 & 1 & 0 & 2 & 0 & 4 & 0 & 8 & 15 & 3 & 4 & 4 & $w_{234}$ & N & \#1069 \\
1096 & 0 & 0 & 0 & 0 & 0 & 0 & 1 & 1 & 0 & 2 & 0 & 4 & 0 & 8 & 15 & 6 & 4 & 4 & $w_{202}$ & N & can. \\
1097 & 0 & 0 & 0 & 0 & 0 & 0 & 1 & 1 & 0 & 2 & 0 & 4 & 0 & 8 & 15 & 7 & 4 & 4 & $w_{200}$ & N & \#1077 \\
1098 & 0 & 0 & 0 & 0 & 0 & 0 & 1 & 1 & 0 & 2 & 0 & 4 & 2 & 4 & 6 & 8 & 6 & 4 & $w_{178}$ & N & can. \\
1099 & 0 & 0 & 0 & 0 & 0 & 0 & 1 & 1 & 0 & 2 & 0 & 4 & 2 & 4 & 7 & 8 & 6 & 4 & $w_{368}$ & N & can. \\
1100 & 0 & 0 & 0 & 0 & 0 & 0 & 1 & 1 & 0 & 2 & 0 & 4 & 2 & 4 & 8 & 10 & 4 & 4 & $w_{199}$ & N & can. \\
1101 & 0 & 0 & 0 & 0 & 0 & 0 & 1 & 1 & 0 & 2 & 0 & 4 & 2 & 4 & 8 & 11 & 4 & 4 & $w_{202}$ & N & can. \\
1102 & 0 & 0 & 0 & 0 & 0 & 0 & 1 & 1 & 0 & 2 & 0 & 4 & 2 & 5 & 6 & 8 & 6 & 4 & $w_{369}$ & N & can. \\
1103 & 0 & 0 & 0 & 0 & 0 & 0 & 1 & 1 & 0 & 2 & 0 & 4 & 2 & 5 & 7 & 8 & 6 & 4 & $w_{236}$ & N & can. \\
1104 & 0 & 0 & 0 & 0 & 0 & 0 & 1 & 1 & 0 & 2 & 0 & 4 & 2 & 5 & 8 & 10 & 4 & 4 & $w_{337}$ & N & can. \\
1105 & 0 & 0 & 0 & 0 & 0 & 0 & 1 & 1 & 0 & 2 & 0 & 4 & 2 & 5 & 8 & 11 & 4 & 4 & $w_{202}$ & N & can. \\
1106 & 0 & 0 & 0 & 0 & 0 & 0 & 1 & 1 & 0 & 2 & 0 & 4 & 2 & 6 & 4 & 7 & 32 & 4 & $w_{193}$ & N & can. \\
1107 & 0 & 0 & 0 & 0 & 0 & 0 & 1 & 1 & 0 & 2 & 0 & 4 & 2 & 6 & 4 & 8 & 2 & 4 & $w_{215}$ & N & can. \\
1108 & 0 & 0 & 0 & 0 & 0 & 0 & 1 & 1 & 0 & 2 & 0 & 4 & 2 & 6 & 5 & 7 & 128 & 3 & $w_{84}$ & N & can. \\
1109 & 0 & 0 & 0 & 0 & 0 & 0 & 1 & 1 & 0 & 2 & 0 & 4 & 2 & 6 & 5 & 8 & 2 & 4 & $w_{222}$ & N & can. \\
1110 & 0 & 0 & 0 & 0 & 0 & 0 & 1 & 1 & 0 & 2 & 0 & 4 & 2 & 6 & 6 & 4 & 128 & 3 & $w_{316}$ & N & can. \\
1111 & 0 & 0 & 0 & 0 & 0 & 0 & 1 & 1 & 0 & 2 & 0 & 4 & 2 & 6 & 6 & 5 & 32 & 4 & $w_{176}$ & N & can. \\
1112 & 0 & 0 & 0 & 0 & 0 & 0 & 1 & 1 & 0 & 2 & 0 & 4 & 2 & 6 & 6 & 8 & 2 & 4 & $w_{325}$ & N & can. \\
1113 & 0 & 0 & 0 & 0 & 0 & 0 & 1 & 1 & 0 & 2 & 0 & 4 & 2 & 6 & 7 & 5 & 128 & 3 & $w_{84}$ & Y & \#1108 \\
1114 & 0 & 0 & 0 & 0 & 0 & 0 & 1 & 1 & 0 & 2 & 0 & 4 & 2 & 6 & 7 & 8 & 2 & 4 & $w_{222}$ & N & \#1109 \\
1115 & 0 & 0 & 0 & 0 & 0 & 0 & 1 & 1 & 0 & 2 & 0 & 4 & 2 & 6 & 8 & 10 & 4 & 3 & $w_{223}$ & N & can. \\
1116 & 0 & 0 & 0 & 0 & 0 & 0 & 1 & 1 & 0 & 2 & 0 & 4 & 2 & 6 & 8 & 11 & 4 & 4 & $w_{224}$ & N & can. \\
1117 & 0 & 0 & 0 & 0 & 0 & 0 & 1 & 1 & 0 & 2 & 0 & 4 & 2 & 6 & 8 & 12 & 4 & 4 & $w_{200}$ & N & can. \\
1118 & 0 & 0 & 0 & 0 & 0 & 0 & 1 & 1 & 0 & 2 & 0 & 4 & 2 & 6 & 8 & 13 & 4 & 4 & $w_{239}$ & N & can. \\
1119 & 0 & 0 & 0 & 0 & 0 & 0 & 1 & 1 & 0 & 2 & 0 & 4 & 2 & 6 & 8 & 14 & 4 & 4 & $w_{224}$ & N & can. \\
1120 & 0 & 0 & 0 & 0 & 0 & 0 & 1 & 1 & 0 & 2 & 0 & 4 & 2 & 6 & 8 & 15 & 4 & 4 & $w_{370}$ & N & can. \\
1121 & 0 & 0 & 0 & 0 & 0 & 0 & 1 & 1 & 0 & 2 & 0 & 4 & 2 & 7 & 4 & 7 & 128 & 3 & $w_{84}$ & N & can. \\
1122 & 0 & 0 & 0 & 0 & 0 & 0 & 1 & 1 & 0 & 2 & 0 & 4 & 2 & 7 & 4 & 8 & 2 & 4 & $w_{222}$ & N & can. \\
1123 & 0 & 0 & 0 & 0 & 0 & 0 & 1 & 1 & 0 & 2 & 0 & 4 & 2 & 7 & 5 & 6 & 128 & 3 & $w_{84}$ & N & \#1108 \\
1124 & 0 & 0 & 0 & 0 & 0 & 0 & 1 & 1 & 0 & 2 & 0 & 4 & 2 & 7 & 5 & 8 & 2 & 4 & $w_{222}$ & N & \#1109 \\
1125 & 0 & 0 & 0 & 0 & 0 & 0 & 1 & 1 & 0 & 2 & 0 & 4 & 2 & 7 & 6 & 5 & 128 & 3 & $w_{88}$ & N & can. \\
1126 & 0 & 0 & 0 & 0 & 0 & 0 & 1 & 1 & 0 & 2 & 0 & 4 & 2 & 7 & 6 & 8 & 2 & 4 & $w_{235}$ & N & can. \\
1127 & 0 & 0 & 0 & 0 & 0 & 0 & 1 & 1 & 0 & 2 & 0 & 4 & 2 & 7 & 7 & 4 & 128 & 3 & $w_{88}$ & N & can. \\
1128 & 0 & 0 & 0 & 0 & 0 & 0 & 1 & 1 & 0 & 2 & 0 & 4 & 2 & 7 & 7 & 8 & 2 & 4 & $w_{235}$ & N & can. \\
1129 & 0 & 0 & 0 & 0 & 0 & 0 & 1 & 1 & 0 & 2 & 0 & 4 & 2 & 7 & 8 & 10 & 4 & 4 & $w_{224}$ & N & \#1116 \\
1130 & 0 & 0 & 0 & 0 & 0 & 0 & 1 & 1 & 0 & 2 & 0 & 4 & 2 & 7 & 8 & 11 & 4 & 3 & $w_{251}$ & Y & can. \\
1131 & 0 & 0 & 0 & 0 & 0 & 0 & 1 & 1 & 0 & 2 & 0 & 4 & 2 & 7 & 8 & 12 & 2 & 4 & $w_{239}$ & N & can. \\
1132 & 0 & 0 & 0 & 0 & 0 & 0 & 1 & 1 & 0 & 2 & 0 & 4 & 2 & 7 & 8 & 14 & 2 & 4 & $w_{241}$ & N & can. \\
1133 & 0 & 0 & 0 & 0 & 0 & 0 & 1 & 1 & 0 & 2 & 0 & 4 & 2 & 8 & 4 & 8 & 16 & 3 & $w_{190}$ & N & can. \\
1134 & 0 & 0 & 0 & 0 & 0 & 0 & 1 & 1 & 0 & 2 & 0 & 4 & 2 & 8 & 4 & 9 & 8 & 4 & $w_{205}$ & N & can. \\
1135 & 0 & 0 & 0 & 0 & 0 & 0 & 1 & 1 & 0 & 2 & 0 & 4 & 2 & 8 & 4 & 10 & 2 & 4 & $w_{182}$ & N & can. \\
1136 & 0 & 0 & 0 & 0 & 0 & 0 & 1 & 1 & 0 & 2 & 0 & 4 & 2 & 8 & 4 & 11 & 2 & 4 & $w_{203}$ & N & can. \\
1137 & 0 & 0 & 0 & 0 & 0 & 0 & 1 & 1 & 0 & 2 & 0 & 4 & 2 & 8 & 4 & 14 & 4 & 4 & $w_{206}$ & N & can. \\
1138 & 0 & 0 & 0 & 0 & 0 & 0 & 1 & 1 & 0 & 2 & 0 & 4 & 2 & 8 & 4 & 15 & 4 & 4 & $w_{200}$ & N & can. \\
1139 & 0 & 0 & 0 & 0 & 0 & 0 & 1 & 1 & 0 & 2 & 0 & 4 & 2 & 8 & 5 & 9 & 16 & 3 & $w_{90}$ & N & can. \\
1140 & 0 & 0 & 0 & 0 & 0 & 0 & 1 & 1 & 0 & 2 & 0 & 4 & 2 & 8 & 5 & 10 & 2 & 4 & $w_{365}$ & N & can. \\
1141 & 0 & 0 & 0 & 0 & 0 & 0 & 1 & 1 & 0 & 2 & 0 & 4 & 2 & 8 & 5 & 11 & 2 & 4 & $w_{230}$ & N & can. \\
1142 & 0 & 0 & 0 & 0 & 0 & 0 & 1 & 1 & 0 & 2 & 0 & 4 & 2 & 8 & 5 & 14 & 4 & 4 & $w_{200}$ & N & can. \\
1143 & 0 & 0 & 0 & 0 & 0 & 0 & 1 & 1 & 0 & 2 & 0 & 4 & 2 & 8 & 5 & 15 & 4 & 4 & $w_{200}$ & N & \#1142 \\
1144 & 0 & 0 & 0 & 0 & 0 & 0 & 1 & 1 & 0 & 2 & 0 & 4 & 2 & 8 & 6 & 10 & 4 & 3 & $w_{371}$ & N & can. \\
1145 & 0 & 0 & 0 & 0 & 0 & 0 & 1 & 1 & 0 & 2 & 0 & 4 & 2 & 8 & 6 & 11 & 2 & 4 & $w_{262}$ & N & can. \\
1146 & 0 & 0 & 0 & 0 & 0 & 0 & 1 & 1 & 0 & 2 & 0 & 4 & 2 & 8 & 6 & 12 & 2 & 4 & $w_{338}$ & N & can. \\
1147 & 0 & 0 & 0 & 0 & 0 & 0 & 1 & 1 & 0 & 2 & 0 & 4 & 2 & 8 & 6 & 13 & 2 & 4 & $w_{241}$ & N & can. \\
1148 & 0 & 0 & 0 & 0 & 0 & 0 & 1 & 1 & 0 & 2 & 0 & 4 & 2 & 8 & 6 & 14 & 2 & 4 & $w_{225}$ & N & can. \\
1149 & 0 & 0 & 0 & 0 & 0 & 0 & 1 & 1 & 0 & 2 & 0 & 4 & 2 & 8 & 6 & 15 & 2 & 4 & $w_{372}$ & N & can. \\
1150 & 0 & 0 & 0 & 0 & 0 & 0 & 1 & 1 & 0 & 2 & 0 & 4 & 2 & 8 & 7 & 11 & 4 & 3 & $w_{251}$ & Y & \#1130 \\
1151 & 0 & 0 & 0 & 0 & 0 & 0 & 1 & 1 & 0 & 2 & 0 & 4 & 2 & 8 & 7 & 12 & 2 & 4 & $w_{241}$ & N & can. \\
1152 & 0 & 0 & 0 & 0 & 0 & 0 & 1 & 1 & 0 & 2 & 0 & 4 & 2 & 8 & 7 & 13 & 2 & 4 & $w_{239}$ & N & \#1131 \\
1153 & 0 & 0 & 0 & 0 & 0 & 0 & 1 & 1 & 0 & 2 & 0 & 4 & 2 & 8 & 7 & 14 & 2 & 4 & $w_{226}$ & N & can. \\
1154 & 0 & 0 & 0 & 0 & 0 & 0 & 1 & 1 & 0 & 2 & 0 & 4 & 2 & 8 & 7 & 15 & 2 & 4 & $w_{254}$ & N & can. \\
1155 & 0 & 0 & 0 & 0 & 0 & 0 & 1 & 1 & 0 & 2 & 0 & 4 & 2 & 8 & 8 & 4 & 16 & 3 & $w_{363}$ & N & can. \\
1156 & 0 & 0 & 0 & 0 & 0 & 0 & 1 & 1 & 0 & 2 & 0 & 4 & 2 & 8 & 8 & 5 & 8 & 4 & $w_{227}$ & N & can. \\
1157 & 0 & 0 & 0 & 0 & 0 & 0 & 1 & 1 & 0 & 2 & 0 & 4 & 2 & 8 & 8 & 6 & 2 & 4 & $w_{182}$ & N & can. \\
1158 & 0 & 0 & 0 & 0 & 0 & 0 & 1 & 1 & 0 & 2 & 0 & 4 & 2 & 8 & 8 & 7 & 2 & 4 & $w_{365}$ & N & can. \\
1159 & 0 & 0 & 0 & 0 & 0 & 0 & 1 & 1 & 0 & 2 & 0 & 4 & 2 & 8 & 8 & 12 & 4 & 4 & $w_{329}$ & N & can. \\
1160 & 0 & 0 & 0 & 0 & 0 & 0 & 1 & 1 & 0 & 2 & 0 & 4 & 2 & 8 & 8 & 13 & 4 & 4 & $w_{202}$ & N & can. \\
1161 & 0 & 0 & 0 & 0 & 0 & 0 & 1 & 1 & 0 & 2 & 0 & 4 & 2 & 8 & 9 & 5 & 16 & 3 & $w_{90}$ & Y & \#1139 \\
1162 & 0 & 0 & 0 & 0 & 0 & 0 & 1 & 1 & 0 & 2 & 0 & 4 & 2 & 8 & 9 & 6 & 2 & 4 & $w_{203}$ & N & can. \\
1163 & 0 & 0 & 0 & 0 & 0 & 0 & 1 & 1 & 0 & 2 & 0 & 4 & 2 & 8 & 9 & 7 & 2 & 4 & $w_{230}$ & N & \#1141 \\
1164 & 0 & 0 & 0 & 0 & 0 & 0 & 1 & 1 & 0 & 2 & 0 & 4 & 2 & 8 & 9 & 12 & 4 & 4 & $w_{202}$ & N & can. \\
1165 & 0 & 0 & 0 & 0 & 0 & 0 & 1 & 1 & 0 & 2 & 0 & 4 & 2 & 8 & 9 & 13 & 4 & 4 & $w_{200}$ & N & \#1142 \\
1166 & 0 & 0 & 0 & 0 & 0 & 0 & 1 & 1 & 0 & 2 & 0 & 4 & 2 & 8 & 10 & 6 & 4 & 3 & $w_{223}$ & Y & \#1115 \\
1167 & 0 & 0 & 0 & 0 & 0 & 0 & 1 & 1 & 0 & 2 & 0 & 4 & 2 & 8 & 10 & 7 & 2 & 4 & $w_{224}$ & N & \#1116 \\
1168 & 0 & 0 & 0 & 0 & 0 & 0 & 1 & 1 & 0 & 2 & 0 & 4 & 2 & 8 & 10 & 12 & 2 & 4 & $w_{225}$ & N & can. \\
1169 & 0 & 0 & 0 & 0 & 0 & 0 & 1 & 1 & 0 & 2 & 0 & 4 & 2 & 8 & 10 & 13 & 2 & 4 & $w_{226}$ & N & can. \\
1170 & 0 & 0 & 0 & 0 & 0 & 0 & 1 & 1 & 0 & 2 & 0 & 4 & 2 & 8 & 10 & 14 & 2 & 4 & $w_{200}$ & N & \#1117 \\
1171 & 0 & 0 & 0 & 0 & 0 & 0 & 1 & 1 & 0 & 2 & 0 & 4 & 2 & 8 & 10 & 15 & 2 & 4 & $w_{239}$ & N & \#1118 \\
1172 & 0 & 0 & 0 & 0 & 0 & 0 & 1 & 1 & 0 & 2 & 0 & 4 & 2 & 8 & 11 & 7 & 4 & 3 & $w_{251}$ & N & \#1130 \\
1173 & 0 & 0 & 0 & 0 & 0 & 0 & 1 & 1 & 0 & 2 & 0 & 4 & 2 & 8 & 11 & 12 & 2 & 4 & $w_{372}$ & N & can. \\
1174 & 0 & 0 & 0 & 0 & 0 & 0 & 1 & 1 & 0 & 2 & 0 & 4 & 2 & 8 & 11 & 13 & 2 & 4 & $w_{254}$ & N & \#1154 \\
1175 & 0 & 0 & 0 & 0 & 0 & 0 & 1 & 1 & 0 & 2 & 0 & 4 & 2 & 8 & 11 & 14 & 2 & 4 & $w_{239}$ & N & \#1131 \\
1176 & 0 & 0 & 0 & 0 & 0 & 0 & 1 & 1 & 0 & 2 & 0 & 4 & 2 & 8 & 11 & 15 & 2 & 4 & $w_{239}$ & N & \#1131 \\
1177 & 0 & 0 & 0 & 0 & 0 & 0 & 1 & 1 & 0 & 2 & 0 & 4 & 2 & 8 & 12 & 14 & 4 & 4 & $w_{200}$ & N & \#1117 \\
1178 & 0 & 0 & 0 & 0 & 0 & 0 & 1 & 1 & 0 & 2 & 0 & 4 & 2 & 8 & 12 & 15 & 2 & 4 & $w_{239}$ & N & \#1131 \\
1179 & 0 & 0 & 0 & 0 & 0 & 0 & 1 & 1 & 0 & 2 & 0 & 4 & 2 & 8 & 13 & 15 & 4 & 4 & $w_{239}$ & N & \#1118 \\
1180 & 0 & 0 & 0 & 0 & 0 & 0 & 1 & 1 & 0 & 2 & 0 & 4 & 2 & 8 & 14 & 12 & 4 & 4 & $w_{338}$ & N & can. \\
1181 & 0 & 0 & 0 & 0 & 0 & 0 & 1 & 1 & 0 & 2 & 0 & 4 & 2 & 8 & 14 & 13 & 2 & 4 & $w_{241}$ & N & \#1151 \\
1182 & 0 & 0 & 0 & 0 & 0 & 0 & 1 & 1 & 0 & 2 & 0 & 4 & 2 & 8 & 15 & 13 & 4 & 4 & $w_{239}$ & N & \#1118 \\
1183 & 0 & 0 & 0 & 0 & 0 & 0 & 1 & 1 & 0 & 2 & 0 & 4 & 6 & 8 & 6 & 10 & 4 & 4 & $w_{225}$ & N & can. \\
1184 & 0 & 0 & 0 & 0 & 0 & 0 & 1 & 1 & 0 & 2 & 0 & 4 & 6 & 8 & 6 & 11 & 4 & 4 & $w_{226}$ & N & can. \\
1185 & 0 & 0 & 0 & 0 & 0 & 0 & 1 & 1 & 0 & 2 & 0 & 4 & 6 & 8 & 6 & 14 & 16 & 3 & $w_{223}$ & N & can. \\
1186 & 0 & 0 & 0 & 0 & 0 & 0 & 1 & 1 & 0 & 2 & 0 & 4 & 6 & 8 & 6 & 15 & 16 & 4 & $w_{224}$ & N & can. \\
1187 & 0 & 0 & 0 & 0 & 0 & 0 & 1 & 1 & 0 & 2 & 0 & 4 & 6 & 8 & 7 & 10 & 4 & 4 & $w_{372}$ & N & can. \\
1188 & 0 & 0 & 0 & 0 & 0 & 0 & 1 & 1 & 0 & 2 & 0 & 4 & 6 & 8 & 7 & 11 & 4 & 4 & $w_{254}$ & N & can. \\
1189 & 0 & 0 & 0 & 0 & 0 & 0 & 1 & 1 & 0 & 2 & 0 & 4 & 6 & 8 & 7 & 14 & 16 & 4 & $w_{224}$ & N & can. \\
1190 & 0 & 0 & 0 & 0 & 0 & 0 & 1 & 1 & 0 & 2 & 0 & 4 & 6 & 8 & 7 & 15 & 16 & 3 & $w_{251}$ & N & can. \\
1191 & 0 & 0 & 0 & 0 & 0 & 0 & 1 & 1 & 0 & 2 & 0 & 4 & 6 & 8 & 8 & 10 & 4 & 4 & $w_{338}$ & N & can. \\
1192 & 0 & 0 & 0 & 0 & 0 & 0 & 1 & 1 & 0 & 2 & 0 & 4 & 6 & 8 & 8 & 11 & 4 & 4 & $w_{241}$ & N & can. \\
1193 & 0 & 0 & 0 & 0 & 0 & 0 & 1 & 1 & 0 & 2 & 0 & 4 & 6 & 8 & 8 & 14 & 4 & 4 & $w_{225}$ & N & can. \\
1194 & 0 & 0 & 0 & 0 & 0 & 0 & 1 & 1 & 0 & 2 & 0 & 4 & 6 & 8 & 8 & 15 & 4 & 4 & $w_{372}$ & N & can. \\
1195 & 0 & 0 & 0 & 0 & 0 & 0 & 1 & 1 & 0 & 2 & 0 & 4 & 6 & 8 & 9 & 10 & 4 & 4 & $w_{241}$ & N & \#1018 \\
1196 & 0 & 0 & 0 & 0 & 0 & 0 & 1 & 1 & 0 & 2 & 0 & 4 & 6 & 8 & 9 & 11 & 4 & 4 & $w_{239}$ & N & \#1011 \\
1197 & 0 & 0 & 0 & 0 & 0 & 0 & 1 & 1 & 0 & 2 & 0 & 4 & 6 & 8 & 9 & 14 & 4 & 4 & $w_{226}$ & N & can. \\
1198 & 0 & 0 & 0 & 0 & 0 & 0 & 1 & 1 & 0 & 2 & 0 & 4 & 6 & 8 & 9 & 15 & 4 & 4 & $w_{254}$ & N & \#1188 \\
1199 & 0 & 0 & 0 & 0 & 0 & 0 & 1 & 1 & 0 & 2 & 0 & 4 & 6 & 8 & 10 & 8 & 4 & 4 & $w_{200}$ & N & \#959 \\
1200 & 0 & 0 & 0 & 0 & 0 & 0 & 1 & 1 & 0 & 2 & 0 & 4 & 6 & 8 & 10 & 9 & 4 & 4 & $w_{239}$ & N & \#1011 \\
1201 & 0 & 0 & 0 & 0 & 0 & 0 & 1 & 1 & 0 & 2 & 0 & 4 & 6 & 8 & 10 & 12 & 2 & 4 & $w_{260}$ & N & can. \\
1202 & 0 & 0 & 0 & 0 & 0 & 0 & 1 & 1 & 0 & 2 & 0 & 4 & 6 & 8 & 10 & 13 & 2 & 4 & $w_{268}$ & N & can. \\
1203 & 0 & 0 & 0 & 0 & 0 & 0 & 1 & 1 & 0 & 2 & 0 & 4 & 6 & 8 & 10 & 14 & 2 & 4 & $w_{239}$ & N & can. \\
1204 & 0 & 0 & 0 & 0 & 0 & 0 & 1 & 1 & 0 & 2 & 0 & 4 & 6 & 8 & 10 & 15 & 2 & 4 & $w_{265}$ & N & can. \\
1205 & 0 & 0 & 0 & 0 & 0 & 0 & 1 & 1 & 0 & 2 & 0 & 4 & 6 & 8 & 11 & 8 & 4 & 4 & $w_{239}$ & N & \#963 \\
1206 & 0 & 0 & 0 & 0 & 0 & 0 & 1 & 1 & 0 & 2 & 0 & 4 & 6 & 8 & 11 & 9 & 4 & 4 & $w_{239}$ & N & \#1011 \\
1207 & 0 & 0 & 0 & 0 & 0 & 0 & 1 & 1 & 0 & 2 & 0 & 4 & 6 & 8 & 11 & 12 & 2 & 4 & $w_{373}$ & N & can. \\
1208 & 0 & 0 & 0 & 0 & 0 & 0 & 1 & 1 & 0 & 2 & 0 & 4 & 6 & 8 & 11 & 13 & 2 & 4 & $w_{374}$ & N & can. \\
1209 & 0 & 0 & 0 & 0 & 0 & 0 & 1 & 1 & 0 & 2 & 0 & 4 & 6 & 8 & 11 & 14 & 2 & 4 & $w_{265}$ & N & can. \\
1210 & 0 & 0 & 0 & 0 & 0 & 0 & 1 & 1 & 0 & 2 & 0 & 4 & 6 & 8 & 11 & 15 & 2 & 4 & $w_{265}$ & N & \#1204 \\
1211 & 0 & 0 & 0 & 0 & 0 & 0 & 1 & 1 & 0 & 2 & 0 & 4 & 6 & 8 & 14 & 6 & 16 & 3 & $w_{371}$ & N & can. \\
1212 & 0 & 0 & 0 & 0 & 0 & 0 & 1 & 1 & 0 & 2 & 0 & 4 & 6 & 8 & 14 & 7 & 16 & 4 & $w_{262}$ & N & can. \\
1213 & 0 & 0 & 0 & 0 & 0 & 0 & 1 & 1 & 0 & 2 & 0 & 4 & 6 & 8 & 14 & 10 & 2 & 4 & $w_{352}$ & N & can. \\
1214 & 0 & 0 & 0 & 0 & 0 & 0 & 1 & 1 & 0 & 2 & 0 & 4 & 6 & 8 & 14 & 11 & 2 & 4 & $w_{267}$ & N & can. \\
1215 & 0 & 0 & 0 & 0 & 0 & 0 & 1 & 1 & 0 & 2 & 0 & 4 & 6 & 8 & 15 & 6 & 16 & 4 & $w_{262}$ & N & can. \\
1216 & 0 & 0 & 0 & 0 & 0 & 0 & 1 & 1 & 0 & 2 & 0 & 4 & 6 & 8 & 15 & 7 & 16 & 3 & $w_{251}$ & Y & \#1190 \\
1217 & 0 & 0 & 0 & 0 & 0 & 0 & 1 & 1 & 0 & 2 & 0 & 4 & 6 & 8 & 15 & 10 & 2 & 4 & $w_{267}$ & N & can. \\
1218 & 0 & 0 & 0 & 0 & 0 & 0 & 1 & 1 & 0 & 2 & 0 & 4 & 6 & 8 & 15 & 11 & 2 & 4 & $w_{265}$ & N & \#1204 \\
1219 & 0 & 0 & 0 & 0 & 0 & 0 & 1 & 1 & 0 & 2 & 0 & 4 & 8 & 10 & 8 & 12 & 32 & 3 & $w_{86}$ & Y & \#957 \\
1220 & 0 & 0 & 0 & 0 & 0 & 0 & 1 & 1 & 0 & 2 & 0 & 4 & 8 & 10 & 8 & 13 & 8 & 4 & $w_{229}$ & N & \#958 \\
1221 & 0 & 0 & 0 & 0 & 0 & 0 & 1 & 1 & 0 & 2 & 0 & 4 & 8 & 10 & 8 & 14 & 8 & 4 & $w_{199}$ & N & can. \\
1222 & 0 & 0 & 0 & 0 & 0 & 0 & 1 & 1 & 0 & 2 & 0 & 4 & 8 & 10 & 8 & 15 & 8 & 4 & $w_{242}$ & N & can. \\
1223 & 0 & 0 & 0 & 0 & 0 & 0 & 1 & 1 & 0 & 2 & 0 & 4 & 8 & 10 & 9 & 12 & 8 & 4 & $w_{229}$ & N & \#1009 \\
1224 & 0 & 0 & 0 & 0 & 0 & 0 & 1 & 1 & 0 & 2 & 0 & 4 & 8 & 10 & 9 & 13 & 16 & 3 & $w_{252}$ & N & \#1010 \\
1225 & 0 & 0 & 0 & 0 & 0 & 0 & 1 & 1 & 0 & 2 & 0 & 4 & 8 & 10 & 9 & 14 & 8 & 4 & $w_{370}$ & N & can. \\
1226 & 0 & 0 & 0 & 0 & 0 & 0 & 1 & 1 & 0 & 2 & 0 & 4 & 8 & 10 & 9 & 15 & 8 & 4 & $w_{271}$ & N & can. \\
1227 & 0 & 0 & 0 & 0 & 0 & 0 & 1 & 1 & 0 & 2 & 0 & 4 & 8 & 10 & 10 & 12 & 8 & 4 & $w_{199}$ & N & can. \\
1228 & 0 & 0 & 0 & 0 & 0 & 0 & 1 & 1 & 0 & 2 & 0 & 4 & 8 & 10 & 10 & 13 & 8 & 4 & $w_{370}$ & N & can. \\
1229 & 0 & 0 & 0 & 0 & 0 & 0 & 1 & 1 & 0 & 2 & 0 & 4 & 8 & 10 & 10 & 14 & 16 & 3 & $w_{328}$ & N & can. \\
1230 & 0 & 0 & 0 & 0 & 0 & 0 & 1 & 1 & 0 & 2 & 0 & 4 & 8 & 10 & 10 & 15 & 8 & 4 & $w_{224}$ & N & can. \\
1231 & 0 & 0 & 0 & 0 & 0 & 0 & 1 & 1 & 0 & 2 & 0 & 4 & 8 & 10 & 11 & 12 & 8 & 4 & $w_{242}$ & N & can. \\
1232 & 0 & 0 & 0 & 0 & 0 & 0 & 1 & 1 & 0 & 2 & 0 & 4 & 8 & 10 & 11 & 13 & 8 & 4 & $w_{271}$ & N & \#1226 \\
1233 & 0 & 0 & 0 & 0 & 0 & 0 & 1 & 1 & 0 & 2 & 0 & 4 & 8 & 10 & 11 & 14 & 8 & 4 & $w_{224}$ & N & can. \\
1234 & 0 & 0 & 0 & 0 & 0 & 0 & 1 & 1 & 0 & 2 & 0 & 4 & 8 & 10 & 11 & 15 & 16 & 3 & $w_{252}$ & Y & \#1010 \\
1235 & 0 & 0 & 0 & 0 & 0 & 0 & 1 & 1 & 0 & 2 & 0 & 4 & 8 & 10 & 12 & 14 & 16 & 4 & $w_{338}$ & N & can. \\
1236 & 0 & 0 & 0 & 0 & 0 & 0 & 1 & 1 & 0 & 2 & 0 & 4 & 8 & 10 & 12 & 15 & 4 & 4 & $w_{265}$ & N & can. \\
1237 & 0 & 0 & 0 & 0 & 0 & 0 & 1 & 1 & 0 & 2 & 0 & 4 & 8 & 10 & 13 & 15 & 16 & 4 & $w_{270}$ & N & can. \\
1238 & 0 & 0 & 0 & 0 & 0 & 0 & 1 & 1 & 0 & 2 & 0 & 4 & 8 & 10 & 14 & 13 & 4 & 4 & $w_{270}$ & N & \#1237 \\
1239 & 0 & 0 & 0 & 0 & 0 & 0 & 1 & 1 & 0 & 2 & 0 & 4 & 8 & 10 & 15 & 13 & 16 & 4 & $w_{270}$ & N & \#1237 \\
1240 & 0 & 0 & 0 & 0 & 0 & 0 & 1 & 1 & 0 & 2 & 0 & 4 & 8 & 11 & 8 & 13 & 16 & 3 & $w_{252}$ & Y & \#962 \\
1241 & 0 & 0 & 0 & 0 & 0 & 0 & 1 & 1 & 0 & 2 & 0 & 4 & 8 & 11 & 8 & 14 & 4 & 4 & $w_{239}$ & N & can. \\
1242 & 0 & 0 & 0 & 0 & 0 & 0 & 1 & 1 & 0 & 2 & 0 & 4 & 8 & 11 & 9 & 12 & 16 & 3 & $w_{252}$ & Y & \#1010 \\
1243 & 0 & 0 & 0 & 0 & 0 & 0 & 1 & 1 & 0 & 2 & 0 & 4 & 8 & 11 & 9 & 14 & 4 & 4 & $w_{239}$ & N & can. \\
1244 & 0 & 0 & 0 & 0 & 0 & 0 & 1 & 1 & 0 & 2 & 0 & 4 & 8 & 11 & 10 & 12 & 4 & 4 & $w_{239}$ & N & can. \\
1245 & 0 & 0 & 0 & 0 & 0 & 0 & 1 & 1 & 0 & 2 & 0 & 4 & 8 & 11 & 10 & 15 & 16 & 3 & $w_{251}$ & N & \#1017 \\
1246 & 0 & 0 & 0 & 0 & 0 & 0 & 1 & 1 & 0 & 2 & 0 & 4 & 8 & 11 & 11 & 12 & 4 & 4 & $w_{239}$ & N & can. \\
1247 & 0 & 0 & 0 & 0 & 0 & 0 & 1 & 1 & 0 & 2 & 0 & 4 & 8 & 11 & 11 & 14 & 16 & 3 & $w_{251}$ & N & can. \\
1248 & 0 & 0 & 0 & 0 & 0 & 0 & 1 & 1 & 0 & 2 & 0 & 4 & 8 & 11 & 12 & 15 & 16 & 4 & $w_{265}$ & N & can. \\
1249 & 0 & 0 & 0 & 0 & 0 & 0 & 1 & 1 & 0 & 2 & 0 & 4 & 8 & 11 & 13 & 14 & 16 & 4 & $w_{265}$ & N & can. \\
1250 & 0 & 0 & 0 & 0 & 0 & 0 & 1 & 1 & 0 & 2 & 0 & 4 & 8 & 11 & 14 & 13 & 16 & 4 & $w_{270}$ & N & can. \\
1251 & 0 & 0 & 0 & 0 & 0 & 0 & 1 & 1 & 0 & 2 & 0 & 4 & 8 & 11 & 15 & 12 & 16 & 4 & $w_{270}$ & N & \#1237 \\
1252 & 0 & 0 & 0 & 0 & 0 & 0 & 1 & 1 & 0 & 2 & 0 & 4 & 8 & 14 & 10 & 12 & 16 & 4 & $w_{224}$ & N & can. \\
1253 & 0 & 0 & 0 & 0 & 0 & 0 & 1 & 1 & 0 & 2 & 0 & 4 & 8 & 14 & 10 & 13 & 4 & 4 & $w_{265}$ & N & can. \\
1254 & 0 & 0 & 0 & 0 & 0 & 0 & 1 & 1 & 0 & 2 & 0 & 4 & 8 & 14 & 11 & 13 & 16 & 4 & $w_{275}$ & N & can. \\
1255 & 0 & 0 & 0 & 0 & 0 & 0 & 1 & 1 & 0 & 2 & 0 & 4 & 8 & 15 & 10 & 13 & 16 & 4 & $w_{356}$ & N & can. \\
1256 & 0 & 0 & 0 & 0 & 0 & 0 & 1 & 1 & 0 & 2 & 0 & 4 & 8 & 15 & 11 & 12 & 16 & 4 & $w_{243}$ & N & can. \\
1257 & 0 & 0 & 0 & 0 & 0 & 0 & 1 & 1 & 0 & 2 & 4 & 6 & 8 & 10 & 13 & 15 & 3072 & 2 & $w_{375}$ & N & can. \\
1258 & 0 & 0 & 0 & 0 & 0 & 0 & 1 & 1 & 0 & 2 & 4 & 6 & 8 & 10 & 14 & 13 & 192 & 4 & $w_{275}$ & N & can. \\
1259 & 0 & 0 & 0 & 0 & 0 & 0 & 1 & 1 & 0 & 2 & 4 & 6 & 8 & 10 & 15 & 13 & 3072 & 3 & $w_{375}$ & N & \#1257 \\
1260 & 0 & 0 & 0 & 0 & 0 & 0 & 1 & 1 & 0 & 2 & 4 & 6 & 8 & 11 & 12 & 15 & 512 & 3 & $w_{375}$ & Y & \#1257 \\
1261 & 0 & 0 & 0 & 0 & 0 & 0 & 1 & 1 & 0 & 2 & 4 & 6 & 8 & 11 & 14 & 13 & 256 & 3 & $w_{376}$ & N & can. \\
1262 & 0 & 0 & 0 & 0 & 0 & 0 & 1 & 1 & 0 & 2 & 4 & 6 & 8 & 12 & 10 & 15 & 32 & 4 & $w_{275}$ & N & can. \\
1263 & 0 & 0 & 0 & 0 & 0 & 0 & 1 & 1 & 0 & 2 & 4 & 6 & 8 & 12 & 11 & 15 & 128 & 3 & $w_{376}$ & N & can. \\
1264 & 0 & 0 & 0 & 0 & 0 & 0 & 1 & 1 & 0 & 2 & 4 & 6 & 8 & 12 & 14 & 10 & 128 & 3 & $w_{345}$ & N & can. \\
1265 & 0 & 0 & 0 & 0 & 0 & 0 & 1 & 1 & 0 & 2 & 4 & 6 & 8 & 12 & 14 & 11 & 32 & 4 & $w_{278}$ & N & can. \\
1266 & 0 & 0 & 0 & 0 & 0 & 0 & 1 & 1 & 0 & 2 & 4 & 6 & 8 & 12 & 15 & 11 & 128 & 3 & $w_{376}$ & N & \#1263 \\
1267 & 0 & 0 & 0 & 0 & 0 & 0 & 1 & 1 & 0 & 2 & 4 & 7 & 8 & 12 & 10 & 15 & 64 & 3 & $w_{376}$ & N & \#1263 \\
1268 & 0 & 0 & 0 & 0 & 0 & 0 & 1 & 1 & 0 & 2 & 4 & 7 & 8 & 12 & 14 & 11 & 64 & 3 & $w_{377}$ & N & can. \\
1269 & 0 & 0 & 0 & 0 & 0 & 0 & 1 & 1 & 0 & 2 & 4 & 8 & 6 & 12 & 10 & 14 & 48 & 3 & $w_{277}$ & N & can. \\
1270 & 0 & 0 & 0 & 0 & 0 & 0 & 1 & 1 & 0 & 2 & 4 & 8 & 6 & 12 & 10 & 15 & 24 & 4 & $w_{278}$ & N & can. \\
1271 & 0 & 0 & 0 & 0 & 0 & 0 & 1 & 1 & 0 & 2 & 4 & 8 & 6 & 12 & 11 & 15 & 48 & 3 & $w_{377}$ & N & can. \\
1272 & 0 & 0 & 0 & 0 & 0 & 0 & 1 & 1 & 0 & 2 & 4 & 8 & 6 & 12 & 14 & 10 & 48 & 3 & $w_{378}$ & N & can. \\
1273 & 0 & 0 & 0 & 0 & 0 & 0 & 1 & 1 & 0 & 2 & 4 & 8 & 6 & 12 & 14 & 11 & 24 & 4 & $w_{379}$ & N & can. \\
1274 & 0 & 0 & 0 & 0 & 0 & 0 & 1 & 1 & 0 & 2 & 4 & 8 & 6 & 12 & 15 & 11 & 48 & 3 & $w_{377}$ & N & \#1271 \\
1275 & 0 & 0 & 0 & 0 & 0 & 0 & 1 & 2 & 0 & 0 & 1 & 3 & 1 & 4 & 2 & 3 & 16 & 4 & $w_{232}$ & N & can. \\
1276 & 0 & 0 & 0 & 0 & 0 & 0 & 1 & 2 & 0 & 0 & 1 & 3 & 1 & 4 & 2 & 4 & 8 & 4 & $w_{153}$ & N & can. \\
1277 & 0 & 0 & 0 & 0 & 0 & 0 & 1 & 2 & 0 & 0 & 1 & 3 & 1 & 4 & 2 & 5 & 8 & 4 & $w_{153}$ & N & can. \\
1278 & 0 & 0 & 0 & 0 & 0 & 0 & 1 & 2 & 0 & 0 & 1 & 3 & 1 & 4 & 2 & 6 & 8 & 3 & $w_{165}$ & N & can. \\
1279 & 0 & 0 & 0 & 0 & 0 & 0 & 1 & 2 & 0 & 0 & 1 & 3 & 1 & 4 & 2 & 7 & 8 & 4 & $w_{153}$ & N & can. \\
1280 & 0 & 0 & 0 & 0 & 0 & 0 & 1 & 2 & 0 & 0 & 1 & 3 & 1 & 4 & 2 & 8 & 1 & 4 & $w_{166}$ & N & can. \\
1281 & 0 & 0 & 0 & 0 & 0 & 0 & 1 & 2 & 0 & 0 & 1 & 3 & 1 & 4 & 4 & 5 & 16 & 4 & $w_{164}$ & N & can. \\
1282 & 0 & 0 & 0 & 0 & 0 & 0 & 1 & 2 & 0 & 0 & 1 & 3 & 1 & 4 & 4 & 6 & 8 & 4 & $w_{168}$ & N & can. \\
1283 & 0 & 0 & 0 & 0 & 0 & 0 & 1 & 2 & 0 & 0 & 1 & 3 & 1 & 4 & 4 & 8 & 2 & 4 & $w_{171}$ & N & can. \\
1284 & 0 & 0 & 0 & 0 & 0 & 0 & 1 & 2 & 0 & 0 & 1 & 3 & 1 & 4 & 5 & 6 & 8 & 4 & $w_{168}$ & N & can. \\
1285 & 0 & 0 & 0 & 0 & 0 & 0 & 1 & 2 & 0 & 0 & 1 & 3 & 1 & 4 & 5 & 8 & 2 & 4 & $w_{171}$ & N & can. \\
1286 & 0 & 0 & 0 & 0 & 0 & 0 & 1 & 2 & 0 & 0 & 1 & 3 & 1 & 4 & 6 & 7 & 16 & 4 & $w_{176}$ & N & can. \\
1287 & 0 & 0 & 0 & 0 & 0 & 0 & 1 & 2 & 0 & 0 & 1 & 3 & 1 & 4 & 6 & 8 & 1 & 4 & $w_{177}$ & N & can. \\
1288 & 0 & 0 & 0 & 0 & 0 & 0 & 1 & 2 & 0 & 0 & 1 & 3 & 1 & 4 & 8 & 9 & 4 & 4 & $w_{233}$ & N & can. \\
1289 & 0 & 0 & 0 & 0 & 0 & 0 & 1 & 2 & 0 & 0 & 1 & 3 & 1 & 4 & 8 & 10 & 2 & 4 & $w_{177}$ & N & can. \\
1290 & 0 & 0 & 0 & 0 & 0 & 0 & 1 & 2 & 0 & 0 & 1 & 3 & 1 & 4 & 8 & 12 & 4 & 3 & $w_{181}$ & N & can. \\
1291 & 0 & 0 & 0 & 0 & 0 & 0 & 1 & 2 & 0 & 0 & 1 & 3 & 1 & 4 & 8 & 13 & 4 & 4 & $w_{227}$ & N & can. \\
1292 & 0 & 0 & 0 & 0 & 0 & 0 & 1 & 2 & 0 & 0 & 1 & 3 & 1 & 4 & 8 & 14 & 2 & 4 & $w_{182}$ & N & can. \\
1293 & 0 & 0 & 0 & 0 & 0 & 0 & 1 & 2 & 0 & 0 & 1 & 3 & 2 & 3 & 4 & 5 & 32 & 4 & $w_{317}$ & N & can. \\
1294 & 0 & 0 & 0 & 0 & 0 & 0 & 1 & 2 & 0 & 0 & 1 & 3 & 2 & 3 & 4 & 8 & 12 & 4 & $w_{380}$ & N & can. \\
1295 & 0 & 0 & 0 & 0 & 0 & 0 & 1 & 2 & 0 & 0 & 1 & 3 & 2 & 4 & 2 & 5 & 16 & 3 & $w_{165}$ & N & can. \\
1296 & 0 & 0 & 0 & 0 & 0 & 0 & 1 & 2 & 0 & 0 & 1 & 3 & 2 & 4 & 2 & 7 & 16 & 4 & $w_{164}$ & N & can. \\
1297 & 0 & 0 & 0 & 0 & 0 & 0 & 1 & 2 & 0 & 0 & 1 & 3 & 2 & 4 & 2 & 8 & 2 & 4 & $w_{381}$ & N & can. \\
1298 & 0 & 0 & 0 & 0 & 0 & 0 & 1 & 2 & 0 & 0 & 1 & 3 & 2 & 4 & 3 & 4 & 16 & 3 & $w_{165}$ & N & can. \\
1299 & 0 & 0 & 0 & 0 & 0 & 0 & 1 & 2 & 0 & 0 & 1 & 3 & 2 & 4 & 3 & 5 & 16 & 4 & $w_{153}$ & N & can. \\
1300 & 0 & 0 & 0 & 0 & 0 & 0 & 1 & 2 & 0 & 0 & 1 & 3 & 2 & 4 & 3 & 6 & 8 & 4 & $w_{164}$ & N & can. \\
1301 & 0 & 0 & 0 & 0 & 0 & 0 & 1 & 2 & 0 & 0 & 1 & 3 & 2 & 4 & 3 & 8 & 2 & 4 & $w_{381}$ & N & can. \\
1302 & 0 & 0 & 0 & 0 & 0 & 0 & 1 & 2 & 0 & 0 & 1 & 3 & 2 & 4 & 4 & 5 & 8 & 4 & $w_{176}$ & N & can. \\
1303 & 0 & 0 & 0 & 0 & 0 & 0 & 1 & 2 & 0 & 0 & 1 & 3 & 2 & 4 & 4 & 6 & 16 & 4 & $w_{176}$ & N & can. \\
1304 & 0 & 0 & 0 & 0 & 0 & 0 & 1 & 2 & 0 & 0 & 1 & 3 & 2 & 4 & 4 & 7 & 8 & 4 & $w_{176}$ & N & can. \\
1305 & 0 & 0 & 0 & 0 & 0 & 0 & 1 & 2 & 0 & 0 & 1 & 3 & 2 & 4 & 4 & 8 & 1 & 4 & $w_{178}$ & N & can. \\
1306 & 0 & 0 & 0 & 0 & 0 & 0 & 1 & 2 & 0 & 0 & 1 & 3 & 2 & 4 & 5 & 6 & 8 & 4 & $w_{176}$ & N & can. \\
1307 & 0 & 0 & 0 & 0 & 0 & 0 & 1 & 2 & 0 & 0 & 1 & 3 & 2 & 4 & 5 & 7 & 8 & 4 & $w_{176}$ & N & can. \\
1308 & 0 & 0 & 0 & 0 & 0 & 0 & 1 & 2 & 0 & 0 & 1 & 3 & 2 & 4 & 5 & 8 & 1 & 4 & $w_{178}$ & N & can. \\
1309 & 0 & 0 & 0 & 0 & 0 & 0 & 1 & 2 & 0 & 0 & 1 & 3 & 2 & 4 & 6 & 7 & 8 & 4 & $w_{255}$ & N & can. \\
1310 & 0 & 0 & 0 & 0 & 0 & 0 & 1 & 2 & 0 & 0 & 1 & 3 & 2 & 4 & 6 & 8 & 1 & 4 & $w_{326}$ & N & can. \\
1311 & 0 & 0 & 0 & 0 & 0 & 0 & 1 & 2 & 0 & 0 & 1 & 3 & 2 & 4 & 7 & 8 & 1 & 4 & $w_{326}$ & N & can. \\
1312 & 0 & 0 & 0 & 0 & 0 & 0 & 1 & 2 & 0 & 0 & 1 & 3 & 2 & 4 & 8 & 9 & 2 & 4 & $w_{382}$ & N & can. \\
1313 & 0 & 0 & 0 & 0 & 0 & 0 & 1 & 2 & 0 & 0 & 1 & 3 & 2 & 4 & 8 & 10 & 2 & 4 & $w_{326}$ & N & can. \\
1314 & 0 & 0 & 0 & 0 & 0 & 0 & 1 & 2 & 0 & 0 & 1 & 3 & 2 & 4 & 8 & 11 & 2 & 4 & $w_{326}$ & N & can. \\
1315 & 0 & 0 & 0 & 0 & 0 & 0 & 1 & 2 & 0 & 0 & 1 & 3 & 2 & 4 & 8 & 12 & 2 & 4 & $w_{383}$ & N & can. \\
1316 & 0 & 0 & 0 & 0 & 0 & 0 & 1 & 2 & 0 & 0 & 1 & 3 & 2 & 4 & 8 & 13 & 2 & 4 & $w_{383}$ & N & can. \\
1317 & 0 & 0 & 0 & 0 & 0 & 0 & 1 & 2 & 0 & 0 & 1 & 3 & 2 & 4 & 8 & 14 & 2 & 4 & $w_{201}$ & N & can. \\
1318 & 0 & 0 & 0 & 0 & 0 & 0 & 1 & 2 & 0 & 0 & 1 & 3 & 2 & 4 & 8 & 15 & 2 & 3 & $w_{237}$ & Y & can. \\
1319 & 0 & 0 & 0 & 0 & 0 & 0 & 1 & 2 & 0 & 0 & 1 & 3 & 4 & 5 & 4 & 6 & 8 & 4 & $w_{384}$ & N & can. \\
1320 & 0 & 0 & 0 & 0 & 0 & 0 & 1 & 2 & 0 & 0 & 1 & 3 & 4 & 5 & 4 & 8 & 2 & 4 & $w_{385}$ & N & can. \\
1321 & 0 & 0 & 0 & 0 & 0 & 0 & 1 & 2 & 0 & 0 & 1 & 3 & 4 & 5 & 6 & 7 & 64 & 4 & $w_{249}$ & N & can. \\
1322 & 0 & 0 & 0 & 0 & 0 & 0 & 1 & 2 & 0 & 0 & 1 & 3 & 4 & 5 & 6 & 8 & 2 & 4 & $w_{386}$ & N & can. \\
1323 & 0 & 0 & 0 & 0 & 0 & 0 & 1 & 2 & 0 & 0 & 1 & 3 & 4 & 5 & 8 & 9 & 16 & 4 & $w_{387}$ & N & can. \\
1324 & 0 & 0 & 0 & 0 & 0 & 0 & 1 & 2 & 0 & 0 & 1 & 3 & 4 & 5 & 8 & 10 & 4 & 4 & $w_{263}$ & N & can. \\
1325 & 0 & 0 & 0 & 0 & 0 & 0 & 1 & 2 & 0 & 0 & 1 & 3 & 4 & 5 & 8 & 12 & 4 & 4 & $w_{351}$ & N & can. \\
1326 & 0 & 0 & 0 & 0 & 0 & 0 & 1 & 2 & 0 & 0 & 1 & 3 & 4 & 5 & 8 & 14 & 4 & 4 & $w_{266}$ & N & can. \\
1327 & 0 & 0 & 0 & 0 & 0 & 0 & 1 & 2 & 0 & 0 & 1 & 3 & 4 & 6 & 4 & 7 & 16 & 3 & $w_{84}$ & N & can. \\
1328 & 0 & 0 & 0 & 0 & 0 & 0 & 1 & 2 & 0 & 0 & 1 & 3 & 4 & 6 & 4 & 8 & 1 & 4 & $w_{368}$ & N & can. \\
1329 & 0 & 0 & 0 & 0 & 0 & 0 & 1 & 2 & 0 & 0 & 1 & 3 & 4 & 6 & 5 & 8 & 1 & 4 & $w_{368}$ & N & can. \\
1330 & 0 & 0 & 0 & 0 & 0 & 0 & 1 & 2 & 0 & 0 & 1 & 3 & 4 & 6 & 8 & 10 & 8 & 4 & $w_{224}$ & N & can. \\
1331 & 0 & 0 & 0 & 0 & 0 & 0 & 1 & 2 & 0 & 0 & 1 & 3 & 4 & 6 & 8 & 11 & 8 & 3 & $w_{251}$ & N & can. \\
1332 & 0 & 0 & 0 & 0 & 0 & 0 & 1 & 2 & 0 & 0 & 1 & 3 & 4 & 6 & 8 & 12 & 2 & 4 & $w_{260}$ & N & can. \\
1333 & 0 & 0 & 0 & 0 & 0 & 0 & 1 & 2 & 0 & 0 & 1 & 3 & 4 & 6 & 8 & 13 & 2 & 4 & $w_{260}$ & N & can. \\
1334 & 0 & 0 & 0 & 0 & 0 & 0 & 1 & 2 & 0 & 0 & 1 & 3 & 4 & 8 & 4 & 9 & 4 & 3 & $w_{223}$ & N & can. \\
1335 & 0 & 0 & 0 & 0 & 0 & 0 & 1 & 2 & 0 & 0 & 1 & 3 & 4 & 8 & 4 & 10 & 2 & 4 & $w_{224}$ & N & can. \\
1336 & 0 & 0 & 0 & 0 & 0 & 0 & 1 & 2 & 0 & 0 & 1 & 3 & 4 & 8 & 4 & 12 & 8 & 4 & $w_{262}$ & N & can. \\
1337 & 0 & 0 & 0 & 0 & 0 & 0 & 1 & 2 & 0 & 0 & 1 & 3 & 4 & 8 & 4 & 13 & 4 & 4 & $w_{262}$ & N & can. \\
1338 & 0 & 0 & 0 & 0 & 0 & 0 & 1 & 2 & 0 & 0 & 1 & 3 & 4 & 8 & 4 & 14 & 2 & 4 & $w_{241}$ & N & can. \\
1339 & 0 & 0 & 0 & 0 & 0 & 0 & 1 & 2 & 0 & 0 & 1 & 3 & 4 & 8 & 5 & 9 & 8 & 4 & $w_{199}$ & N & can. \\
1340 & 0 & 0 & 0 & 0 & 0 & 0 & 1 & 2 & 0 & 0 & 1 & 3 & 4 & 8 & 5 & 10 & 2 & 4 & $w_{224}$ & N & can. \\
1341 & 0 & 0 & 0 & 0 & 0 & 0 & 1 & 2 & 0 & 0 & 1 & 3 & 4 & 8 & 5 & 12 & 2 & 4 & $w_{262}$ & N & can. \\
1342 & 0 & 0 & 0 & 0 & 0 & 0 & 1 & 2 & 0 & 0 & 1 & 3 & 4 & 8 & 5 & 14 & 2 & 4 & $w_{241}$ & N & can. \\
1343 & 0 & 0 & 0 & 0 & 0 & 0 & 1 & 2 & 0 & 0 & 1 & 3 & 4 & 8 & 6 & 10 & 4 & 4 & $w_{224}$ & N & can. \\
1344 & 0 & 0 & 0 & 0 & 0 & 0 & 1 & 2 & 0 & 0 & 1 & 3 & 4 & 8 & 6 & 11 & 4 & 3 & $w_{251}$ & Y & can. \\
1345 & 0 & 0 & 0 & 0 & 0 & 0 & 1 & 2 & 0 & 0 & 1 & 3 & 4 & 8 & 6 & 12 & 1 & 4 & $w_{260}$ & N & can. \\
1346 & 0 & 0 & 0 & 0 & 0 & 0 & 1 & 2 & 0 & 0 & 1 & 3 & 4 & 8 & 6 & 13 & 1 & 4 & $w_{260}$ & N & can. \\
1347 & 0 & 0 & 0 & 0 & 0 & 0 & 1 & 2 & 0 & 0 & 1 & 4 & 1 & 6 & 2 & 4 & 16 & 4 & $w_{68}$ & N & can. \\
1348 & 0 & 0 & 0 & 0 & 0 & 0 & 1 & 2 & 0 & 0 & 1 & 4 & 1 & 6 & 2 & 5 & 8 & 4 & $w_{68}$ & N & can. \\
1349 & 0 & 0 & 0 & 0 & 0 & 0 & 1 & 2 & 0 & 0 & 1 & 4 & 1 & 6 & 2 & 8 & 2 & 4 & $w_{172}$ & N & can. \\
1350 & 0 & 0 & 0 & 0 & 0 & 0 & 1 & 2 & 0 & 0 & 1 & 4 & 1 & 6 & 3 & 5 & 16 & 4 & $w_{168}$ & N & can. \\
1351 & 0 & 0 & 0 & 0 & 0 & 0 & 1 & 2 & 0 & 0 & 1 & 4 & 1 & 6 & 3 & 8 & 2 & 4 & $w_{170}$ & N & can. \\
1352 & 0 & 0 & 0 & 0 & 0 & 0 & 1 & 2 & 0 & 0 & 1 & 4 & 1 & 6 & 8 & 9 & 12 & 3 & $w_{181}$ & N & can. \\
1353 & 0 & 0 & 0 & 0 & 0 & 0 & 1 & 2 & 0 & 0 & 1 & 4 & 1 & 6 & 8 & 10 & 4 & 4 & $w_{203}$ & N & can. \\
1354 & 0 & 0 & 0 & 0 & 0 & 0 & 1 & 2 & 0 & 0 & 1 & 4 & 1 & 6 & 8 & 11 & 4 & 4 & $w_{91}$ & N & can. \\
1355 & 0 & 0 & 0 & 0 & 0 & 0 & 1 & 2 & 0 & 0 & 1 & 4 & 1 & 7 & 2 & 4 & 16 & 4 & $w_{193}$ & N & \#1031 \\
1356 & 0 & 0 & 0 & 0 & 0 & 0 & 1 & 2 & 0 & 0 & 1 & 4 & 1 & 7 & 2 & 5 & 8 & 4 & $w_{168}$ & N & can. \\
1357 & 0 & 0 & 0 & 0 & 0 & 0 & 1 & 2 & 0 & 0 & 1 & 4 & 1 & 7 & 2 & 8 & 2 & 4 & $w_{194}$ & N & \#1032 \\
1358 & 0 & 0 & 0 & 0 & 0 & 0 & 1 & 2 & 0 & 0 & 1 & 4 & 1 & 7 & 3 & 5 & 16 & 4 & $w_{153}$ & N & can. \\
1359 & 0 & 0 & 0 & 0 & 0 & 0 & 1 & 2 & 0 & 0 & 1 & 4 & 1 & 7 & 3 & 8 & 2 & 4 & $w_{171}$ & N & can. \\
1360 & 0 & 0 & 0 & 0 & 0 & 0 & 1 & 2 & 0 & 0 & 1 & 4 & 1 & 7 & 8 & 9 & 12 & 4 & $w_{327}$ & N & can. \\
1361 & 0 & 0 & 0 & 0 & 0 & 0 & 1 & 2 & 0 & 0 & 1 & 4 & 1 & 7 & 8 & 10 & 4 & 4 & $w_{202}$ & N & can. \\
1362 & 0 & 0 & 0 & 0 & 0 & 0 & 1 & 2 & 0 & 0 & 1 & 4 & 1 & 7 & 8 & 11 & 4 & 4 & $w_{199}$ & N & \#1037 \\
1363 & 0 & 0 & 0 & 0 & 0 & 0 & 1 & 2 & 0 & 0 & 1 & 4 & 1 & 8 & 2 & 4 & 2 & 4 & $w_{198}$ & N & \#1063 \\
1364 & 0 & 0 & 0 & 0 & 0 & 0 & 1 & 2 & 0 & 0 & 1 & 4 & 1 & 8 & 2 & 5 & 1 & 4 & $w_{169}$ & N & can. \\
1365 & 0 & 0 & 0 & 0 & 0 & 0 & 1 & 2 & 0 & 0 & 1 & 4 & 1 & 8 & 2 & 6 & 1 & 4 & $w_{170}$ & N & can. \\
1366 & 0 & 0 & 0 & 0 & 0 & 0 & 1 & 2 & 0 & 0 & 1 & 4 & 1 & 8 & 2 & 7 & 1 & 4 & $w_{215}$ & N & \#994 \\
1367 & 0 & 0 & 0 & 0 & 0 & 0 & 1 & 2 & 0 & 0 & 1 & 4 & 1 & 8 & 2 & 12 & 2 & 4 & $w_{174}$ & N & can. \\
1368 & 0 & 0 & 0 & 0 & 0 & 0 & 1 & 2 & 0 & 0 & 1 & 4 & 1 & 8 & 2 & 13 & 2 & 3 & $w_{90}$ & Y & \#1003 \\
1369 & 0 & 0 & 0 & 0 & 0 & 0 & 1 & 2 & 0 & 0 & 1 & 4 & 1 & 8 & 2 & 14 & 2 & 4 & $w_{234}$ & N & \#1069 \\
1370 & 0 & 0 & 0 & 0 & 0 & 0 & 1 & 2 & 0 & 0 & 1 & 4 & 1 & 8 & 2 & 15 & 2 & 4 & $w_{183}$ & N & can. \\
1371 & 0 & 0 & 0 & 0 & 0 & 0 & 1 & 2 & 0 & 0 & 1 & 4 & 1 & 8 & 3 & 5 & 2 & 4 & $w_{314}$ & N & can. \\
1372 & 0 & 0 & 0 & 0 & 0 & 0 & 1 & 2 & 0 & 0 & 1 & 4 & 1 & 8 & 3 & 6 & 1 & 4 & $w_{169}$ & N & can. \\
1373 & 0 & 0 & 0 & 0 & 0 & 0 & 1 & 2 & 0 & 0 & 1 & 4 & 1 & 8 & 3 & 7 & 1 & 4 & $w_{177}$ & N & can. \\
1374 & 0 & 0 & 0 & 0 & 0 & 0 & 1 & 2 & 0 & 0 & 1 & 4 & 1 & 8 & 3 & 12 & 2 & 3 & $w_{173}$ & Y & can. \\
1375 & 0 & 0 & 0 & 0 & 0 & 0 & 1 & 2 & 0 & 0 & 1 & 4 & 1 & 8 & 3 & 13 & 2 & 4 & $w_{227}$ & N & can. \\
1376 & 0 & 0 & 0 & 0 & 0 & 0 & 1 & 2 & 0 & 0 & 1 & 4 & 1 & 8 & 3 & 14 & 2 & 4 & $w_{183}$ & N & can. \\
1377 & 0 & 0 & 0 & 0 & 0 & 0 & 1 & 2 & 0 & 0 & 1 & 4 & 1 & 8 & 3 & 15 & 2 & 4 & $w_{227}$ & N & can. \\
1378 & 0 & 0 & 0 & 0 & 0 & 0 & 1 & 2 & 0 & 0 & 1 & 4 & 1 & 8 & 6 & 7 & 2 & 4 & $w_{178}$ & N & can. \\
1379 & 0 & 0 & 0 & 0 & 0 & 0 & 1 & 2 & 0 & 0 & 1 & 4 & 1 & 8 & 6 & 10 & 2 & 4 & $w_{183}$ & N & can. \\
1380 & 0 & 0 & 0 & 0 & 0 & 0 & 1 & 2 & 0 & 0 & 1 & 4 & 1 & 8 & 6 & 11 & 1 & 4 & $w_{203}$ & N & can. \\
1381 & 0 & 0 & 0 & 0 & 0 & 0 & 1 & 2 & 0 & 0 & 1 & 4 & 1 & 8 & 6 & 14 & 2 & 4 & $w_{203}$ & N & can. \\
1382 & 0 & 0 & 0 & 0 & 0 & 0 & 1 & 2 & 0 & 0 & 1 & 4 & 1 & 8 & 6 & 15 & 2 & 4 & $w_{183}$ & N & can. \\
1383 & 0 & 0 & 0 & 0 & 0 & 0 & 1 & 2 & 0 & 0 & 1 & 4 & 1 & 8 & 7 & 11 & 2 & 4 & $w_{230}$ & N & \#1083 \\
1384 & 0 & 0 & 0 & 0 & 0 & 0 & 1 & 2 & 0 & 0 & 1 & 4 & 1 & 8 & 7 & 14 & 2 & 4 & $w_{200}$ & N & \#1077 \\
1385 & 0 & 0 & 0 & 0 & 0 & 0 & 1 & 2 & 0 & 0 & 1 & 4 & 1 & 8 & 7 & 15 & 2 & 4 & $w_{202}$ & N & can. \\
1386 & 0 & 0 & 0 & 0 & 0 & 0 & 1 & 2 & 0 & 0 & 1 & 4 & 1 & 8 & 14 & 15 & 6 & 4 & $w_{201}$ & N & can. \\
1387 & 0 & 0 & 0 & 0 & 0 & 0 & 1 & 2 & 0 & 0 & 1 & 4 & 2 & 4 & 3 & 7 & 16 & 4 & $w_{176}$ & N & can. \\
1388 & 0 & 0 & 0 & 0 & 0 & 0 & 1 & 2 & 0 & 0 & 1 & 4 & 2 & 4 & 3 & 8 & 2 & 4 & $w_{171}$ & N & can. \\
1389 & 0 & 0 & 0 & 0 & 0 & 0 & 1 & 2 & 0 & 0 & 1 & 4 & 2 & 4 & 7 & 8 & 6 & 4 & $w_{368}$ & N & \#1099 \\
1390 & 0 & 0 & 0 & 0 & 0 & 0 & 1 & 2 & 0 & 0 & 1 & 4 & 2 & 4 & 8 & 9 & 4 & 4 & $w_{201}$ & N & can. \\
1391 & 0 & 0 & 0 & 0 & 0 & 0 & 1 & 2 & 0 & 0 & 1 & 4 & 2 & 4 & 8 & 11 & 4 & 4 & $w_{199}$ & N & \#1100 \\
1392 & 0 & 0 & 0 & 0 & 0 & 0 & 1 & 2 & 0 & 0 & 1 & 4 & 2 & 4 & 8 & 15 & 12 & 4 & $w_{238}$ & N & can. \\
1393 & 0 & 0 & 0 & 0 & 0 & 0 & 1 & 2 & 0 & 0 & 1 & 4 & 2 & 5 & 3 & 4 & 32 & 4 & $w_{68}$ & N & can. \\
1394 & 0 & 0 & 0 & 0 & 0 & 0 & 1 & 2 & 0 & 0 & 1 & 4 & 2 & 5 & 3 & 5 & 8 & 4 & $w_{168}$ & N & can. \\
1395 & 0 & 0 & 0 & 0 & 0 & 0 & 1 & 2 & 0 & 0 & 1 & 4 & 2 & 5 & 3 & 6 & 16 & 4 & $w_{168}$ & N & can. \\
1396 & 0 & 0 & 0 & 0 & 0 & 0 & 1 & 2 & 0 & 0 & 1 & 4 & 2 & 5 & 3 & 7 & 8 & 4 & $w_{176}$ & N & can. \\
1397 & 0 & 0 & 0 & 0 & 0 & 0 & 1 & 2 & 0 & 0 & 1 & 4 & 2 & 5 & 3 & 8 & 1 & 4 & $w_{177}$ & N & can. \\
1398 & 0 & 0 & 0 & 0 & 0 & 0 & 1 & 2 & 0 & 0 & 1 & 4 & 2 & 5 & 4 & 7 & 8 & 4 & $w_{168}$ & N & can. \\
1399 & 0 & 0 & 0 & 0 & 0 & 0 & 1 & 2 & 0 & 0 & 1 & 4 & 2 & 5 & 4 & 8 & 1 & 4 & $w_{170}$ & N & can. \\
1400 & 0 & 0 & 0 & 0 & 0 & 0 & 1 & 2 & 0 & 0 & 1 & 4 & 2 & 5 & 5 & 6 & 8 & 4 & $w_{168}$ & N & can. \\
1401 & 0 & 0 & 0 & 0 & 0 & 0 & 1 & 2 & 0 & 0 & 1 & 4 & 2 & 5 & 5 & 8 & 1 & 4 & $w_{177}$ & N & can. \\
1402 & 0 & 0 & 0 & 0 & 0 & 0 & 1 & 2 & 0 & 0 & 1 & 4 & 2 & 5 & 6 & 7 & 8 & 3 & $w_{88}$ & N & can. \\
1403 & 0 & 0 & 0 & 0 & 0 & 0 & 1 & 2 & 0 & 0 & 1 & 4 & 2 & 5 & 6 & 8 & 1 & 4 & $w_{236}$ & N & can. \\
1404 & 0 & 0 & 0 & 0 & 0 & 0 & 1 & 2 & 0 & 0 & 1 & 4 & 2 & 5 & 7 & 8 & 1 & 4 & $w_{235}$ & N & can. \\
1405 & 0 & 0 & 0 & 0 & 0 & 0 & 1 & 2 & 0 & 0 & 1 & 4 & 2 & 5 & 8 & 9 & 2 & 3 & $w_{237}$ & N & can. \\
1406 & 0 & 0 & 0 & 0 & 0 & 0 & 1 & 2 & 0 & 0 & 1 & 4 & 2 & 5 & 8 & 10 & 2 & 4 & $w_{225}$ & N & can. \\
1407 & 0 & 0 & 0 & 0 & 0 & 0 & 1 & 2 & 0 & 0 & 1 & 4 & 2 & 5 & 8 & 11 & 2 & 4 & $w_{202}$ & N & can. \\
1408 & 0 & 0 & 0 & 0 & 0 & 0 & 1 & 2 & 0 & 0 & 1 & 4 & 2 & 5 & 8 & 12 & 2 & 4 & $w_{225}$ & N & can. \\
1409 & 0 & 0 & 0 & 0 & 0 & 0 & 1 & 2 & 0 & 0 & 1 & 4 & 2 & 5 & 8 & 13 & 2 & 4 & $w_{202}$ & N & can. \\
1410 & 0 & 0 & 0 & 0 & 0 & 0 & 1 & 2 & 0 & 0 & 1 & 4 & 2 & 5 & 8 & 14 & 2 & 4 & $w_{226}$ & N & can. \\
1411 & 0 & 0 & 0 & 0 & 0 & 0 & 1 & 2 & 0 & 0 & 1 & 4 & 2 & 5 & 8 & 15 & 2 & 4 & $w_{238}$ & N & can. \\
1412 & 0 & 0 & 0 & 0 & 0 & 0 & 1 & 2 & 0 & 0 & 1 & 4 & 2 & 7 & 3 & 5 & 16 & 4 & $w_{164}$ & N & can. \\
1413 & 0 & 0 & 0 & 0 & 0 & 0 & 1 & 2 & 0 & 0 & 1 & 4 & 2 & 7 & 3 & 8 & 2 & 4 & $w_{171}$ & N & can. \\
1414 & 0 & 0 & 0 & 0 & 0 & 0 & 1 & 2 & 0 & 0 & 1 & 4 & 2 & 7 & 4 & 7 & 64 & 3 & $w_{84}$ & Y & \#1108 \\
1415 & 0 & 0 & 0 & 0 & 0 & 0 & 1 & 2 & 0 & 0 & 1 & 4 & 2 & 7 & 4 & 8 & 1 & 4 & $w_{222}$ & N & \#1109 \\
1416 & 0 & 0 & 0 & 0 & 0 & 0 & 1 & 2 & 0 & 0 & 1 & 4 & 2 & 7 & 5 & 6 & 16 & 3 & $w_{165}$ & N & can. \\
1417 & 0 & 0 & 0 & 0 & 0 & 0 & 1 & 2 & 0 & 0 & 1 & 4 & 2 & 7 & 5 & 8 & 2 & 4 & $w_{178}$ & N & can. \\
1418 & 0 & 0 & 0 & 0 & 0 & 0 & 1 & 2 & 0 & 0 & 1 & 4 & 2 & 7 & 6 & 8 & 2 & 4 & $w_{170}$ & N & can. \\
1419 & 0 & 0 & 0 & 0 & 0 & 0 & 1 & 2 & 0 & 0 & 1 & 4 & 2 & 7 & 8 & 9 & 2 & 4 & $w_{225}$ & N & can. \\
1420 & 0 & 0 & 0 & 0 & 0 & 0 & 1 & 2 & 0 & 0 & 1 & 4 & 2 & 7 & 8 & 11 & 4 & 3 & $w_{223}$ & Y & \#1115 \\
1421 & 0 & 0 & 0 & 0 & 0 & 0 & 1 & 2 & 0 & 0 & 1 & 4 & 2 & 7 & 8 & 12 & 2 & 4 & $w_{226}$ & N & can. \\
1422 & 0 & 0 & 0 & 0 & 0 & 0 & 1 & 2 & 0 & 0 & 1 & 4 & 2 & 7 & 8 & 13 & 4 & 4 & $w_{200}$ & N & \#1117 \\
1423 & 0 & 0 & 0 & 0 & 0 & 0 & 1 & 2 & 0 & 0 & 1 & 4 & 2 & 7 & 8 & 14 & 4 & 4 & $w_{224}$ & N & \#1119 \\
1424 & 0 & 0 & 0 & 0 & 0 & 0 & 1 & 2 & 0 & 0 & 1 & 4 & 2 & 8 & 3 & 5 & 1 & 4 & $w_{179}$ & N & can. \\
1425 & 0 & 0 & 0 & 0 & 0 & 0 & 1 & 2 & 0 & 0 & 1 & 4 & 2 & 8 & 3 & 6 & 2 & 4 & $w_{169}$ & N & can. \\
1426 & 0 & 0 & 0 & 0 & 0 & 0 & 1 & 2 & 0 & 0 & 1 & 4 & 2 & 8 & 3 & 7 & 1 & 4 & $w_{178}$ & N & can. \\
1427 & 0 & 0 & 0 & 0 & 0 & 0 & 1 & 2 & 0 & 0 & 1 & 4 & 2 & 8 & 3 & 12 & 2 & 4 & $w_{227}$ & N & can. \\
1428 & 0 & 0 & 0 & 0 & 0 & 0 & 1 & 2 & 0 & 0 & 1 & 4 & 2 & 8 & 3 & 13 & 1 & 4 & $w_{182}$ & N & can. \\
1429 & 0 & 0 & 0 & 0 & 0 & 0 & 1 & 2 & 0 & 0 & 1 & 4 & 2 & 8 & 3 & 15 & 2 & 3 & $w_{181}$ & N & can. \\
1430 & 0 & 0 & 0 & 0 & 0 & 0 & 1 & 2 & 0 & 0 & 1 & 4 & 2 & 8 & 4 & 8 & 8 & 3 & $w_{90}$ & Y & \#1139 \\
1431 & 0 & 0 & 0 & 0 & 0 & 0 & 1 & 2 & 0 & 0 & 1 & 4 & 2 & 8 & 4 & 9 & 2 & 4 & $w_{183}$ & N & can. \\
1432 & 0 & 0 & 0 & 0 & 0 & 0 & 1 & 2 & 0 & 0 & 1 & 4 & 2 & 8 & 4 & 10 & 1 & 4 & $w_{182}$ & N & can. \\
1433 & 0 & 0 & 0 & 0 & 0 & 0 & 1 & 2 & 0 & 0 & 1 & 4 & 2 & 8 & 4 & 11 & 1 & 4 & $w_{230}$ & N & \#1141 \\
1434 & 0 & 0 & 0 & 0 & 0 & 0 & 1 & 2 & 0 & 0 & 1 & 4 & 2 & 8 & 4 & 14 & 2 & 4 & $w_{200}$ & N & \#1142 \\
1435 & 0 & 0 & 0 & 0 & 0 & 0 & 1 & 2 & 0 & 0 & 1 & 4 & 2 & 8 & 4 & 15 & 2 & 4 & $w_{238}$ & N & can. \\
1436 & 0 & 0 & 0 & 0 & 0 & 0 & 1 & 2 & 0 & 0 & 1 & 4 & 2 & 8 & 5 & 6 & 1 & 4 & $w_{177}$ & N & can. \\
1437 & 0 & 0 & 0 & 0 & 0 & 0 & 1 & 2 & 0 & 0 & 1 & 4 & 2 & 8 & 5 & 7 & 1 & 4 & $w_{240}$ & N & can. \\
1438 & 0 & 0 & 0 & 0 & 0 & 0 & 1 & 2 & 0 & 0 & 1 & 4 & 2 & 8 & 5 & 9 & 1 & 3 & $w_{181}$ & Y & can. \\
1439 & 0 & 0 & 0 & 0 & 0 & 0 & 1 & 2 & 0 & 0 & 1 & 4 & 2 & 8 & 5 & 10 & 2 & 4 & $w_{201}$ & N & can. \\
1440 & 0 & 0 & 0 & 0 & 0 & 0 & 1 & 2 & 0 & 0 & 1 & 4 & 2 & 8 & 5 & 11 & 1 & 4 & $w_{225}$ & N & can. \\
1441 & 0 & 0 & 0 & 0 & 0 & 0 & 1 & 2 & 0 & 0 & 1 & 4 & 2 & 8 & 5 & 12 & 1 & 4 & $w_{201}$ & N & can. \\
1442 & 0 & 0 & 0 & 0 & 0 & 0 & 1 & 2 & 0 & 0 & 1 & 4 & 2 & 8 & 5 & 13 & 1 & 4 & $w_{225}$ & N & can. \\
1443 & 0 & 0 & 0 & 0 & 0 & 0 & 1 & 2 & 0 & 0 & 1 & 4 & 2 & 8 & 5 & 14 & 1 & 4 & $w_{226}$ & N & can. \\
1444 & 0 & 0 & 0 & 0 & 0 & 0 & 1 & 2 & 0 & 0 & 1 & 4 & 2 & 8 & 5 & 15 & 1 & 4 & $w_{202}$ & N & can. \\
1445 & 0 & 0 & 0 & 0 & 0 & 0 & 1 & 2 & 0 & 0 & 1 & 4 & 2 & 8 & 6 & 7 & 1 & 4 & $w_{235}$ & N & can. \\
1446 & 0 & 0 & 0 & 0 & 0 & 0 & 1 & 2 & 0 & 0 & 1 & 4 & 2 & 8 & 6 & 9 & 2 & 4 & $w_{183}$ & N & can. \\
1447 & 0 & 0 & 0 & 0 & 0 & 0 & 1 & 2 & 0 & 0 & 1 & 4 & 2 & 8 & 6 & 11 & 1 & 4 & $w_{202}$ & N & can. \\
1448 & 0 & 0 & 0 & 0 & 0 & 0 & 1 & 2 & 0 & 0 & 1 & 4 & 2 & 8 & 6 & 12 & 1 & 4 & $w_{183}$ & N & can. \\
1449 & 0 & 0 & 0 & 0 & 0 & 0 & 1 & 2 & 0 & 0 & 1 & 4 & 2 & 8 & 6 & 13 & 1 & 4 & $w_{238}$ & N & can. \\
1450 & 0 & 0 & 0 & 0 & 0 & 0 & 1 & 2 & 0 & 0 & 1 & 4 & 2 & 8 & 6 & 14 & 1 & 4 & $w_{202}$ & N & can. \\
1451 & 0 & 0 & 0 & 0 & 0 & 0 & 1 & 2 & 0 & 0 & 1 & 4 & 2 & 8 & 6 & 15 & 1 & 4 & $w_{202}$ & N & can. \\
1452 & 0 & 0 & 0 & 0 & 0 & 0 & 1 & 2 & 0 & 0 & 1 & 4 & 2 & 8 & 7 & 11 & 2 & 3 & $w_{251}$ & Y & \#1130 \\
1453 & 0 & 0 & 0 & 0 & 0 & 0 & 1 & 2 & 0 & 0 & 1 & 4 & 2 & 8 & 7 & 12 & 1 & 4 & $w_{226}$ & N & can. \\
1454 & 0 & 0 & 0 & 0 & 0 & 0 & 1 & 2 & 0 & 0 & 1 & 4 & 2 & 8 & 7 & 13 & 1 & 4 & $w_{239}$ & N & \#1131 \\
1455 & 0 & 0 & 0 & 0 & 0 & 0 & 1 & 2 & 0 & 0 & 1 & 4 & 2 & 8 & 7 & 14 & 1 & 4 & $w_{254}$ & N & \#1154 \\
1456 & 0 & 0 & 0 & 0 & 0 & 0 & 1 & 2 & 0 & 0 & 1 & 4 & 2 & 8 & 7 & 15 & 1 & 4 & $w_{241}$ & N & can. \\
1457 & 0 & 0 & 0 & 0 & 0 & 0 & 1 & 2 & 0 & 0 & 1 & 4 & 2 & 8 & 12 & 13 & 1 & 4 & $w_{225}$ & N & can. \\
1458 & 0 & 0 & 0 & 0 & 0 & 0 & 1 & 2 & 0 & 0 & 1 & 4 & 2 & 8 & 12 & 15 & 2 & 4 & $w_{202}$ & N & can. \\
1459 & 0 & 0 & 0 & 0 & 0 & 0 & 1 & 2 & 0 & 0 & 1 & 4 & 2 & 8 & 13 & 14 & 2 & 4 & $w_{239}$ & N & \#1118 \\
1460 & 0 & 0 & 0 & 0 & 0 & 0 & 1 & 2 & 0 & 0 & 1 & 4 & 2 & 8 & 13 & 15 & 1 & 4 & $w_{241}$ & N & can. \\
1461 & 0 & 0 & 0 & 0 & 0 & 0 & 1 & 2 & 0 & 0 & 1 & 4 & 3 & 5 & 3 & 7 & 8 & 4 & $w_{176}$ & N & can. \\
1462 & 0 & 0 & 0 & 0 & 0 & 0 & 1 & 2 & 0 & 0 & 1 & 4 & 3 & 5 & 3 & 8 & 1 & 4 & $w_{333}$ & N & can. \\
1463 & 0 & 0 & 0 & 0 & 0 & 0 & 1 & 2 & 0 & 0 & 1 & 4 & 3 & 5 & 6 & 7 & 16 & 4 & $w_{255}$ & N & can. \\
1464 & 0 & 0 & 0 & 0 & 0 & 0 & 1 & 2 & 0 & 0 & 1 & 4 & 3 & 5 & 6 & 8 & 2 & 4 & $w_{250}$ & N & can. \\
1465 & 0 & 0 & 0 & 0 & 0 & 0 & 1 & 2 & 0 & 0 & 1 & 4 & 3 & 5 & 7 & 8 & 2 & 4 & $w_{326}$ & N & can. \\
1466 & 0 & 0 & 0 & 0 & 0 & 0 & 1 & 2 & 0 & 0 & 1 & 4 & 3 & 5 & 8 & 9 & 4 & 4 & $w_{388}$ & N & can. \\
1467 & 0 & 0 & 0 & 0 & 0 & 0 & 1 & 2 & 0 & 0 & 1 & 4 & 3 & 5 & 8 & 10 & 2 & 4 & $w_{262}$ & N & can. \\
1468 & 0 & 0 & 0 & 0 & 0 & 0 & 1 & 2 & 0 & 0 & 1 & 4 & 3 & 5 & 8 & 11 & 2 & 4 & $w_{389}$ & N & can. \\
1469 & 0 & 0 & 0 & 0 & 0 & 0 & 1 & 2 & 0 & 0 & 1 & 4 & 3 & 5 & 8 & 14 & 4 & 4 & $w_{224}$ & N & can. \\
1470 & 0 & 0 & 0 & 0 & 0 & 0 & 1 & 2 & 0 & 0 & 1 & 4 & 3 & 5 & 8 & 15 & 4 & 4 & $w_{241}$ & N & can. \\
1471 & 0 & 0 & 0 & 0 & 0 & 0 & 1 & 2 & 0 & 0 & 1 & 4 & 3 & 6 & 3 & 7 & 8 & 4 & $w_{168}$ & N & can. \\
1472 & 0 & 0 & 0 & 0 & 0 & 0 & 1 & 2 & 0 & 0 & 1 & 4 & 3 & 6 & 3 & 8 & 1 & 4 & $w_{194}$ & N & can. \\
1473 & 0 & 0 & 0 & 0 & 0 & 0 & 1 & 2 & 0 & 0 & 1 & 4 & 3 & 6 & 5 & 7 & 8 & 4 & $w_{176}$ & N & can. \\
1474 & 0 & 0 & 0 & 0 & 0 & 0 & 1 & 2 & 0 & 0 & 1 & 4 & 3 & 6 & 5 & 8 & 1 & 4 & $w_{368}$ & N & can. \\
1475 & 0 & 0 & 0 & 0 & 0 & 0 & 1 & 2 & 0 & 0 & 1 & 4 & 3 & 6 & 6 & 8 & 1 & 4 & $w_{285}$ & N & can. \\
1476 & 0 & 0 & 0 & 0 & 0 & 0 & 1 & 2 & 0 & 0 & 1 & 4 & 3 & 6 & 7 & 8 & 1 & 4 & $w_{177}$ & N & can. \\
1477 & 0 & 0 & 0 & 0 & 0 & 0 & 1 & 2 & 0 & 0 & 1 & 4 & 3 & 6 & 8 & 9 & 2 & 4 & $w_{225}$ & N & can. \\
1478 & 0 & 0 & 0 & 0 & 0 & 0 & 1 & 2 & 0 & 0 & 1 & 4 & 3 & 6 & 8 & 10 & 2 & 4 & $w_{225}$ & N & can. \\
1479 & 0 & 0 & 0 & 0 & 0 & 0 & 1 & 2 & 0 & 0 & 1 & 4 & 3 & 6 & 8 & 11 & 2 & 3 & $w_{223}$ & N & can. \\
1480 & 0 & 0 & 0 & 0 & 0 & 0 & 1 & 2 & 0 & 0 & 1 & 4 & 3 & 6 & 8 & 12 & 2 & 4 & $w_{226}$ & N & can. \\
1481 & 0 & 0 & 0 & 0 & 0 & 0 & 1 & 2 & 0 & 0 & 1 & 4 & 3 & 6 & 8 & 13 & 2 & 4 & $w_{200}$ & N & can. \\
1482 & 0 & 0 & 0 & 0 & 0 & 0 & 1 & 2 & 0 & 0 & 1 & 4 & 3 & 6 & 8 & 14 & 2 & 4 & $w_{224}$ & N & can. \\
1483 & 0 & 0 & 0 & 0 & 0 & 0 & 1 & 2 & 0 & 0 & 1 & 4 & 3 & 6 & 8 & 15 & 2 & 4 & $w_{226}$ & N & can. \\
1484 & 0 & 0 & 0 & 0 & 0 & 0 & 1 & 2 & 0 & 0 & 1 & 4 & 3 & 7 & 3 & 8 & 1 & 4 & $w_{178}$ & N & can. \\
1485 & 0 & 0 & 0 & 0 & 0 & 0 & 1 & 2 & 0 & 0 & 1 & 4 & 3 & 7 & 5 & 7 & 32 & 3 & $w_{88}$ & Y & can. \\
1486 & 0 & 0 & 0 & 0 & 0 & 0 & 1 & 2 & 0 & 0 & 1 & 4 & 3 & 7 & 5 & 8 & 1 & 4 & $w_{240}$ & N & can. \\
1487 & 0 & 0 & 0 & 0 & 0 & 0 & 1 & 2 & 0 & 0 & 1 & 4 & 3 & 7 & 6 & 8 & 1 & 4 & $w_{235}$ & N & can. \\
1488 & 0 & 0 & 0 & 0 & 0 & 0 & 1 & 2 & 0 & 0 & 1 & 4 & 3 & 7 & 7 & 8 & 1 & 4 & $w_{235}$ & N & can. \\
1489 & 0 & 0 & 0 & 0 & 0 & 0 & 1 & 2 & 0 & 0 & 1 & 4 & 3 & 7 & 8 & 9 & 2 & 4 & $w_{258}$ & N & can. \\
1490 & 0 & 0 & 0 & 0 & 0 & 0 & 1 & 2 & 0 & 0 & 1 & 4 & 3 & 7 & 8 & 10 & 2 & 3 & $w_{259}$ & Y & can. \\
1491 & 0 & 0 & 0 & 0 & 0 & 0 & 1 & 2 & 0 & 0 & 1 & 4 & 3 & 7 & 8 & 11 & 2 & 4 & $w_{262}$ & N & can. \\
1492 & 0 & 0 & 0 & 0 & 0 & 0 & 1 & 2 & 0 & 0 & 1 & 4 & 3 & 7 & 8 & 12 & 2 & 4 & $w_{241}$ & N & can. \\
1493 & 0 & 0 & 0 & 0 & 0 & 0 & 1 & 2 & 0 & 0 & 1 & 4 & 3 & 7 & 8 & 13 & 2 & 4 & $w_{241}$ & N & can. \\
1494 & 0 & 0 & 0 & 0 & 0 & 0 & 1 & 2 & 0 & 0 & 1 & 4 & 3 & 7 & 8 & 14 & 2 & 4 & $w_{260}$ & N & can. \\
1495 & 0 & 0 & 0 & 0 & 0 & 0 & 1 & 2 & 0 & 0 & 1 & 4 & 3 & 7 & 8 & 15 & 2 & 4 & $w_{260}$ & N & can. \\
1496 & 0 & 0 & 0 & 0 & 0 & 0 & 1 & 2 & 0 & 0 & 1 & 4 & 3 & 8 & 3 & 9 & 2 & 4 & $w_{177}$ & N & can. \\
1497 & 0 & 0 & 0 & 0 & 0 & 0 & 1 & 2 & 0 & 0 & 1 & 4 & 3 & 8 & 3 & 10 & 2 & 4 & $w_{178}$ & N & can. \\
1498 & 0 & 0 & 0 & 0 & 0 & 0 & 1 & 2 & 0 & 0 & 1 & 4 & 3 & 8 & 3 & 12 & 2 & 4 & $w_{182}$ & N & can. \\
1499 & 0 & 0 & 0 & 0 & 0 & 0 & 1 & 2 & 0 & 0 & 1 & 4 & 3 & 8 & 3 & 13 & 2 & 4 & $w_{390}$ & N & can. \\
1500 & 0 & 0 & 0 & 0 & 0 & 0 & 1 & 2 & 0 & 0 & 1 & 4 & 3 & 8 & 3 & 14 & 2 & 3 & $w_{223}$ & N & can. \\
1501 & 0 & 0 & 0 & 0 & 0 & 0 & 1 & 2 & 0 & 0 & 1 & 4 & 3 & 8 & 3 & 15 & 2 & 4 & $w_{201}$ & N & can. \\
1502 & 0 & 0 & 0 & 0 & 0 & 0 & 1 & 2 & 0 & 0 & 1 & 4 & 3 & 8 & 5 & 8 & 2 & 3 & $w_{223}$ & Y & can. \\
1503 & 0 & 0 & 0 & 0 & 0 & 0 & 1 & 2 & 0 & 0 & 1 & 4 & 3 & 8 & 5 & 9 & 2 & 4 & $w_{201}$ & N & can. \\
1504 & 0 & 0 & 0 & 0 & 0 & 0 & 1 & 2 & 0 & 0 & 1 & 4 & 3 & 8 & 5 & 10 & 1 & 4 & $w_{225}$ & N & can. \\
1505 & 0 & 0 & 0 & 0 & 0 & 0 & 1 & 2 & 0 & 0 & 1 & 4 & 3 & 8 & 5 & 11 & 1 & 4 & $w_{391}$ & N & can. \\
1506 & 0 & 0 & 0 & 0 & 0 & 0 & 1 & 2 & 0 & 0 & 1 & 4 & 3 & 8 & 5 & 14 & 2 & 4 & $w_{239}$ & N & can. \\
1507 & 0 & 0 & 0 & 0 & 0 & 0 & 1 & 2 & 0 & 0 & 1 & 4 & 3 & 8 & 5 & 15 & 2 & 4 & $w_{241}$ & N & can. \\
1508 & 0 & 0 & 0 & 0 & 0 & 0 & 1 & 2 & 0 & 0 & 1 & 4 & 3 & 8 & 6 & 7 & 1 & 4 & $w_{240}$ & N & can. \\
1509 & 0 & 0 & 0 & 0 & 0 & 0 & 1 & 2 & 0 & 0 & 1 & 4 & 3 & 8 & 6 & 8 & 1 & 4 & $w_{230}$ & N & can. \\
1510 & 0 & 0 & 0 & 0 & 0 & 0 & 1 & 2 & 0 & 0 & 1 & 4 & 3 & 8 & 6 & 9 & 1 & 4 & $w_{202}$ & N & can. \\
1511 & 0 & 0 & 0 & 0 & 0 & 0 & 1 & 2 & 0 & 0 & 1 & 4 & 3 & 8 & 6 & 10 & 1 & 4 & $w_{225}$ & N & can. \\
1512 & 0 & 0 & 0 & 0 & 0 & 0 & 1 & 2 & 0 & 0 & 1 & 4 & 3 & 8 & 6 & 11 & 1 & 3 & $w_{251}$ & N & can. \\
1513 & 0 & 0 & 0 & 0 & 0 & 0 & 1 & 2 & 0 & 0 & 1 & 4 & 3 & 8 & 6 & 12 & 1 & 4 & $w_{226}$ & N & can. \\
1514 & 0 & 0 & 0 & 0 & 0 & 0 & 1 & 2 & 0 & 0 & 1 & 4 & 3 & 8 & 6 & 13 & 1 & 4 & $w_{239}$ & N & can. \\
1515 & 0 & 0 & 0 & 0 & 0 & 0 & 1 & 2 & 0 & 0 & 1 & 4 & 3 & 8 & 6 & 14 & 1 & 4 & $w_{254}$ & N & can. \\
1516 & 0 & 0 & 0 & 0 & 0 & 0 & 1 & 2 & 0 & 0 & 1 & 4 & 3 & 8 & 6 & 15 & 1 & 4 & $w_{241}$ & N & can. \\
1517 & 0 & 0 & 0 & 0 & 0 & 0 & 1 & 2 & 0 & 0 & 1 & 4 & 3 & 8 & 7 & 8 & 1 & 4 & $w_{202}$ & N & can. \\
1518 & 0 & 0 & 0 & 0 & 0 & 0 & 1 & 2 & 0 & 0 & 1 & 4 & 3 & 8 & 7 & 9 & 1 & 4 & $w_{201}$ & N & can. \\
1519 & 0 & 0 & 0 & 0 & 0 & 0 & 1 & 2 & 0 & 0 & 1 & 4 & 3 & 8 & 7 & 10 & 1 & 3 & $w_{237}$ & Y & can. \\
1520 & 0 & 0 & 0 & 0 & 0 & 0 & 1 & 2 & 0 & 0 & 1 & 4 & 3 & 8 & 7 & 11 & 1 & 4 & $w_{262}$ & N & can. \\
1521 & 0 & 0 & 0 & 0 & 0 & 0 & 1 & 2 & 0 & 0 & 1 & 4 & 3 & 8 & 7 & 12 & 1 & 4 & $w_{202}$ & N & can. \\
1522 & 0 & 0 & 0 & 0 & 0 & 0 & 1 & 2 & 0 & 0 & 1 & 4 & 3 & 8 & 7 & 13 & 1 & 4 & $w_{241}$ & N & can. \\
1523 & 0 & 0 & 0 & 0 & 0 & 0 & 1 & 2 & 0 & 0 & 1 & 4 & 3 & 8 & 7 & 14 & 1 & 4 & $w_{241}$ & N & can. \\
1524 & 0 & 0 & 0 & 0 & 0 & 0 & 1 & 2 & 0 & 0 & 1 & 4 & 3 & 8 & 7 & 15 & 1 & 4 & $w_{262}$ & N & can. \\
1525 & 0 & 0 & 0 & 0 & 0 & 0 & 1 & 2 & 0 & 0 & 1 & 4 & 3 & 8 & 8 & 12 & 2 & 4 & $w_{202}$ & N & can. \\
1526 & 0 & 0 & 0 & 0 & 0 & 0 & 1 & 2 & 0 & 0 & 1 & 4 & 3 & 8 & 8 & 13 & 1 & 4 & $w_{230}$ & N & can. \\
1527 & 0 & 0 & 0 & 0 & 0 & 0 & 1 & 2 & 0 & 0 & 1 & 4 & 3 & 8 & 8 & 14 & 1 & 4 & $w_{239}$ & N & can. \\
1528 & 0 & 0 & 0 & 0 & 0 & 0 & 1 & 2 & 0 & 0 & 1 & 4 & 3 & 8 & 8 & 15 & 1 & 4 & $w_{226}$ & N & can. \\
1529 & 0 & 0 & 0 & 0 & 0 & 0 & 1 & 2 & 0 & 0 & 1 & 4 & 3 & 8 & 9 & 10 & 1 & 4 & $w_{178}$ & N & can. \\
1530 & 0 & 0 & 0 & 0 & 0 & 0 & 1 & 2 & 0 & 0 & 1 & 4 & 3 & 8 & 9 & 11 & 1 & 4 & $w_{240}$ & N & can. \\
1531 & 0 & 0 & 0 & 0 & 0 & 0 & 1 & 2 & 0 & 0 & 1 & 4 & 3 & 8 & 9 & 12 & 1 & 4 & $w_{182}$ & N & can. \\
1532 & 0 & 0 & 0 & 0 & 0 & 0 & 1 & 2 & 0 & 0 & 1 & 4 & 3 & 8 & 9 & 13 & 1 & 4 & $w_{225}$ & N & can. \\
1533 & 0 & 0 & 0 & 0 & 0 & 0 & 1 & 2 & 0 & 0 & 1 & 4 & 3 & 8 & 9 & 14 & 1 & 4 & $w_{226}$ & N & can. \\
1534 & 0 & 0 & 0 & 0 & 0 & 0 & 1 & 2 & 0 & 0 & 1 & 4 & 3 & 8 & 9 & 15 & 1 & 4 & $w_{225}$ & N & can. \\
1535 & 0 & 0 & 0 & 0 & 0 & 0 & 1 & 2 & 0 & 0 & 1 & 4 & 3 & 8 & 10 & 11 & 1 & 4 & $w_{257}$ & N & can. \\
1536 & 0 & 0 & 0 & 0 & 0 & 0 & 1 & 2 & 0 & 0 & 1 & 4 & 3 & 8 & 10 & 12 & 1 & 4 & $w_{225}$ & N & can. \\
1537 & 0 & 0 & 0 & 0 & 0 & 0 & 1 & 2 & 0 & 0 & 1 & 4 & 3 & 8 & 10 & 13 & 1 & 4 & $w_{241}$ & N & \#1522 \\
1538 & 0 & 0 & 0 & 0 & 0 & 0 & 1 & 2 & 0 & 0 & 1 & 4 & 3 & 8 & 10 & 14 & 1 & 4 & $w_{241}$ & N & can. \\
1539 & 0 & 0 & 0 & 0 & 0 & 0 & 1 & 2 & 0 & 0 & 1 & 4 & 3 & 8 & 10 & 15 & 1 & 4 & $w_{225}$ & N & can. \\
1540 & 0 & 0 & 0 & 0 & 0 & 0 & 1 & 2 & 0 & 0 & 1 & 4 & 3 & 8 & 11 & 12 & 1 & 4 & $w_{241}$ & N & can. \\
1541 & 0 & 0 & 0 & 0 & 0 & 0 & 1 & 2 & 0 & 0 & 1 & 4 & 3 & 8 & 11 & 13 & 1 & 4 & $w_{264}$ & N & can. \\
1542 & 0 & 0 & 0 & 0 & 0 & 0 & 1 & 2 & 0 & 0 & 1 & 4 & 3 & 8 & 11 & 14 & 1 & 4 & $w_{254}$ & N & can. \\
1543 & 0 & 0 & 0 & 0 & 0 & 0 & 1 & 2 & 0 & 0 & 1 & 4 & 3 & 8 & 11 & 15 & 1 & 4 & $w_{260}$ & N & can. \\
1544 & 0 & 0 & 0 & 0 & 0 & 0 & 1 & 2 & 0 & 0 & 1 & 4 & 3 & 8 & 12 & 13 & 1 & 4 & $w_{262}$ & N & can. \\
1545 & 0 & 0 & 0 & 0 & 0 & 0 & 1 & 2 & 0 & 0 & 1 & 4 & 3 & 8 & 12 & 14 & 1 & 4 & $w_{241}$ & N & can. \\
1546 & 0 & 0 & 0 & 0 & 0 & 0 & 1 & 2 & 0 & 0 & 1 & 4 & 3 & 8 & 12 & 15 & 1 & 4 & $w_{225}$ & N & can. \\
1547 & 0 & 0 & 0 & 0 & 0 & 0 & 1 & 2 & 0 & 0 & 1 & 4 & 3 & 8 & 13 & 14 & 1 & 4 & $w_{254}$ & N & can. \\
1548 & 0 & 0 & 0 & 0 & 0 & 0 & 1 & 2 & 0 & 0 & 1 & 4 & 3 & 8 & 13 & 15 & 1 & 4 & $w_{260}$ & N & can. \\
1549 & 0 & 0 & 0 & 0 & 0 & 0 & 1 & 2 & 0 & 0 & 1 & 4 & 3 & 8 & 14 & 15 & 1 & 4 & $w_{260}$ & N & can. \\
1550 & 0 & 0 & 0 & 0 & 0 & 0 & 1 & 2 & 0 & 0 & 1 & 4 & 6 & 7 & 8 & 9 & 4 & 4 & $w_{262}$ & N & can. \\
1551 & 0 & 0 & 0 & 0 & 0 & 0 & 1 & 2 & 0 & 0 & 1 & 4 & 6 & 7 & 8 & 10 & 2 & 4 & $w_{260}$ & N & can. \\
1552 & 0 & 0 & 0 & 0 & 0 & 0 & 1 & 2 & 0 & 0 & 1 & 4 & 6 & 7 & 8 & 11 & 2 & 4 & $w_{241}$ & N & can. \\
1553 & 0 & 0 & 0 & 0 & 0 & 0 & 1 & 2 & 0 & 0 & 1 & 4 & 6 & 7 & 8 & 14 & 4 & 4 & $w_{258}$ & N & can. \\
1554 & 0 & 0 & 0 & 0 & 0 & 0 & 1 & 2 & 0 & 0 & 1 & 4 & 6 & 7 & 8 & 15 & 4 & 3 & $w_{259}$ & N & can. \\
1555 & 0 & 0 & 0 & 0 & 0 & 0 & 1 & 2 & 0 & 0 & 1 & 4 & 6 & 8 & 6 & 9 & 8 & 4 & $w_{91}$ & N & can. \\
1556 & 0 & 0 & 0 & 0 & 0 & 0 & 1 & 2 & 0 & 0 & 1 & 4 & 6 & 8 & 6 & 10 & 2 & 4 & $w_{203}$ & N & can. \\
1557 & 0 & 0 & 0 & 0 & 0 & 0 & 1 & 2 & 0 & 0 & 1 & 4 & 6 & 8 & 6 & 11 & 2 & 4 & $w_{231}$ & N & can. \\
1558 & 0 & 0 & 0 & 0 & 0 & 0 & 1 & 2 & 0 & 0 & 1 & 4 & 6 & 8 & 6 & 15 & 4 & 4 & $w_{202}$ & N & can. \\
1559 & 0 & 0 & 0 & 0 & 0 & 0 & 1 & 2 & 0 & 0 & 1 & 4 & 6 & 8 & 7 & 9 & 4 & 4 & $w_{183}$ & N & can. \\
1560 & 0 & 0 & 0 & 0 & 0 & 0 & 1 & 2 & 0 & 0 & 1 & 4 & 6 & 8 & 7 & 10 & 1 & 4 & $w_{202}$ & N & can. \\
1561 & 0 & 0 & 0 & 0 & 0 & 0 & 1 & 2 & 0 & 0 & 1 & 4 & 6 & 8 & 7 & 11 & 1 & 4 & $w_{226}$ & N & can. \\
1562 & 0 & 0 & 0 & 0 & 0 & 0 & 1 & 2 & 0 & 0 & 1 & 4 & 6 & 8 & 7 & 14 & 2 & 4 & $w_{225}$ & N & can. \\
1563 & 0 & 0 & 0 & 0 & 0 & 0 & 1 & 2 & 0 & 0 & 1 & 4 & 6 & 8 & 7 & 15 & 2 & 3 & $w_{237}$ & N & can. \\
1564 & 0 & 0 & 0 & 0 & 0 & 0 & 1 & 2 & 0 & 0 & 1 & 4 & 6 & 8 & 8 & 11 & 1 & 4 & $w_{231}$ & N & can. \\
1565 & 0 & 0 & 0 & 0 & 0 & 0 & 1 & 2 & 0 & 0 & 1 & 4 & 6 & 8 & 8 & 15 & 2 & 4 & $w_{226}$ & N & can. \\
1566 & 0 & 0 & 0 & 0 & 0 & 0 & 1 & 2 & 0 & 0 & 1 & 4 & 6 & 8 & 9 & 10 & 1 & 4 & $w_{203}$ & N & can. \\
1567 & 0 & 0 & 0 & 0 & 0 & 0 & 1 & 2 & 0 & 0 & 1 & 4 & 6 & 8 & 9 & 11 & 1 & 4 & $w_{226}$ & N & can. \\
1568 & 0 & 0 & 0 & 0 & 0 & 0 & 1 & 2 & 0 & 0 & 1 & 4 & 6 & 8 & 9 & 14 & 2 & 4 & $w_{226}$ & N & can. \\
1569 & 0 & 0 & 0 & 0 & 0 & 0 & 1 & 2 & 0 & 0 & 1 & 4 & 6 & 8 & 9 & 15 & 2 & 4 & $w_{225}$ & N & can. \\
1570 & 0 & 0 & 0 & 0 & 0 & 0 & 1 & 2 & 0 & 0 & 1 & 4 & 6 & 8 & 10 & 11 & 1 & 4 & $w_{241}$ & N & can. \\
1571 & 0 & 0 & 0 & 0 & 0 & 0 & 1 & 2 & 0 & 0 & 1 & 4 & 6 & 8 & 10 & 12 & 2 & 4 & $w_{225}$ & N & can. \\
1572 & 0 & 0 & 0 & 0 & 0 & 0 & 1 & 2 & 0 & 0 & 1 & 4 & 6 & 8 & 10 & 13 & 1 & 4 & $w_{244}$ & N & can. \\
1573 & 0 & 0 & 0 & 0 & 0 & 0 & 1 & 2 & 0 & 0 & 1 & 4 & 6 & 8 & 10 & 14 & 1 & 4 & $w_{241}$ & N & can. \\
1574 & 0 & 0 & 0 & 0 & 0 & 0 & 1 & 2 & 0 & 0 & 1 & 4 & 6 & 8 & 10 & 15 & 1 & 4 & $w_{226}$ & N & can. \\
1575 & 0 & 0 & 0 & 0 & 0 & 0 & 1 & 2 & 0 & 0 & 1 & 4 & 6 & 8 & 11 & 13 & 2 & 4 & $w_{265}$ & N & can. \\
1576 & 0 & 0 & 0 & 0 & 0 & 0 & 1 & 2 & 0 & 0 & 1 & 4 & 6 & 8 & 11 & 14 & 1 & 4 & $w_{286}$ & N & can. \\
1577 & 0 & 0 & 0 & 0 & 0 & 0 & 1 & 2 & 0 & 0 & 1 & 4 & 6 & 8 & 11 & 15 & 1 & 4 & $w_{268}$ & N & can. \\
1578 & 0 & 0 & 0 & 0 & 0 & 0 & 1 & 2 & 0 & 0 & 1 & 4 & 6 & 8 & 14 & 15 & 2 & 4 & $w_{260}$ & N & can. \\
1579 & 0 & 0 & 0 & 0 & 0 & 0 & 1 & 2 & 0 & 0 & 1 & 4 & 7 & 8 & 7 & 11 & 2 & 4 & $w_{254}$ & N & \#1188 \\
1580 & 0 & 0 & 0 & 0 & 0 & 0 & 1 & 2 & 0 & 0 & 1 & 4 & 7 & 8 & 7 & 14 & 8 & 3 & $w_{251}$ & Y & \#1190 \\
1581 & 0 & 0 & 0 & 0 & 0 & 0 & 1 & 2 & 0 & 0 & 1 & 4 & 7 & 8 & 8 & 11 & 2 & 4 & $w_{239}$ & N & \#1011 \\
1582 & 0 & 0 & 0 & 0 & 0 & 0 & 1 & 2 & 0 & 0 & 1 & 4 & 7 & 8 & 9 & 10 & 1 & 4 & $w_{202}$ & N & can. \\
1583 & 0 & 0 & 0 & 0 & 0 & 0 & 1 & 2 & 0 & 0 & 1 & 4 & 7 & 8 & 9 & 11 & 1 & 4 & $w_{241}$ & N & can. \\
1584 & 0 & 0 & 0 & 0 & 0 & 0 & 1 & 2 & 0 & 0 & 1 & 4 & 7 & 8 & 9 & 14 & 2 & 4 & $w_{241}$ & N & can. \\
1585 & 0 & 0 & 0 & 0 & 0 & 0 & 1 & 2 & 0 & 0 & 1 & 4 & 7 & 8 & 9 & 15 & 2 & 4 & $w_{262}$ & N & can. \\
1586 & 0 & 0 & 0 & 0 & 0 & 0 & 1 & 2 & 0 & 0 & 1 & 4 & 7 & 8 & 10 & 11 & 1 & 4 & $w_{260}$ & N & \#1346 \\
1587 & 0 & 0 & 0 & 0 & 0 & 0 & 1 & 2 & 0 & 0 & 1 & 4 & 7 & 8 & 10 & 12 & 2 & 4 & $w_{241}$ & N & can. \\
1588 & 0 & 0 & 0 & 0 & 0 & 0 & 1 & 2 & 0 & 0 & 1 & 4 & 7 & 8 & 10 & 13 & 1 & 4 & $w_{268}$ & N & can. \\
1589 & 0 & 0 & 0 & 0 & 0 & 0 & 1 & 2 & 0 & 0 & 1 & 4 & 7 & 8 & 10 & 14 & 1 & 4 & $w_{268}$ & N & can. \\
1590 & 0 & 0 & 0 & 0 & 0 & 0 & 1 & 2 & 0 & 0 & 1 & 4 & 7 & 8 & 10 & 15 & 1 & 4 & $w_{241}$ & N & can. \\
1591 & 0 & 0 & 0 & 0 & 0 & 0 & 1 & 2 & 0 & 0 & 1 & 4 & 7 & 8 & 11 & 13 & 2 & 4 & $w_{374}$ & N & \#1208 \\
1592 & 0 & 0 & 0 & 0 & 0 & 0 & 1 & 2 & 0 & 0 & 1 & 4 & 7 & 8 & 11 & 14 & 1 & 4 & $w_{265}$ & N & \#1204 \\
1593 & 0 & 0 & 0 & 0 & 0 & 0 & 1 & 2 & 0 & 0 & 1 & 4 & 7 & 8 & 11 & 15 & 1 & 4 & $w_{267}$ & N & can. \\
1594 & 0 & 0 & 0 & 0 & 0 & 0 & 1 & 2 & 0 & 0 & 1 & 4 & 7 & 8 & 14 & 15 & 2 & 4 & $w_{266}$ & N & can. \\
1595 & 0 & 0 & 0 & 0 & 0 & 0 & 1 & 2 & 0 & 0 & 1 & 4 & 8 & 9 & 10 & 11 & 4 & 4 & $w_{250}$ & N & can. \\
1596 & 0 & 0 & 0 & 0 & 0 & 0 & 1 & 2 & 0 & 0 & 1 & 4 & 8 & 9 & 10 & 12 & 2 & 4 & $w_{391}$ & N & can. \\
1597 & 0 & 0 & 0 & 0 & 0 & 0 & 1 & 2 & 0 & 0 & 1 & 4 & 8 & 9 & 10 & 13 & 2 & 3 & $w_{251}$ & Y & \#1344 \\
1598 & 0 & 0 & 0 & 0 & 0 & 0 & 1 & 2 & 0 & 0 & 1 & 4 & 8 & 9 & 10 & 14 & 1 & 4 & $w_{264}$ & N & can. \\
1599 & 0 & 0 & 0 & 0 & 0 & 0 & 1 & 2 & 0 & 0 & 1 & 4 & 8 & 9 & 10 & 15 & 1 & 4 & $w_{254}$ & N & can. \\
1600 & 0 & 0 & 0 & 0 & 0 & 0 & 1 & 2 & 0 & 0 & 1 & 4 & 8 & 9 & 14 & 15 & 8 & 4 & $w_{264}$ & N & can. \\
1601 & 0 & 0 & 0 & 0 & 0 & 0 & 1 & 2 & 0 & 0 & 1 & 4 & 8 & 10 & 9 & 11 & 8 & 4 & $w_{222}$ & N & can. \\
1602 & 0 & 0 & 0 & 0 & 0 & 0 & 1 & 2 & 0 & 0 & 1 & 4 & 8 & 10 & 9 & 12 & 1 & 4 & $w_{230}$ & N & can. \\
1603 & 0 & 0 & 0 & 0 & 0 & 0 & 1 & 2 & 0 & 0 & 1 & 4 & 8 & 10 & 9 & 13 & 2 & 3 & $w_{251}$ & N & can. \\
1604 & 0 & 0 & 0 & 0 & 0 & 0 & 1 & 2 & 0 & 0 & 1 & 4 & 8 & 10 & 9 & 14 & 1 & 4 & $w_{239}$ & N & can. \\
1605 & 0 & 0 & 0 & 0 & 0 & 0 & 1 & 2 & 0 & 0 & 1 & 4 & 8 & 10 & 9 & 15 & 1 & 4 & $w_{254}$ & N & can. \\
1606 & 0 & 0 & 0 & 0 & 0 & 0 & 1 & 2 & 0 & 0 & 1 & 4 & 8 & 10 & 12 & 14 & 4 & 4 & $w_{239}$ & N & can. \\
1607 & 0 & 0 & 0 & 0 & 0 & 0 & 1 & 2 & 0 & 0 & 1 & 4 & 8 & 10 & 12 & 15 & 1 & 4 & $w_{286}$ & N & can. \\
1608 & 0 & 0 & 0 & 0 & 0 & 0 & 1 & 2 & 0 & 0 & 1 & 4 & 8 & 10 & 13 & 15 & 4 & 4 & $w_{265}$ & N & can. \\
1609 & 0 & 0 & 0 & 0 & 0 & 0 & 1 & 2 & 0 & 0 & 1 & 4 & 8 & 11 & 8 & 13 & 4 & 3 & $w_{252}$ & Y & \#1010 \\
1610 & 0 & 0 & 0 & 0 & 0 & 0 & 1 & 2 & 0 & 0 & 1 & 4 & 8 & 11 & 8 & 14 & 2 & 4 & $w_{271}$ & N & \#1226 \\
1611 & 0 & 0 & 0 & 0 & 0 & 0 & 1 & 2 & 0 & 0 & 1 & 4 & 8 & 11 & 9 & 12 & 2 & 3 & $w_{90}$ & Y & can. \\
1612 & 0 & 0 & 0 & 0 & 0 & 0 & 1 & 2 & 0 & 0 & 1 & 4 & 8 & 11 & 9 & 14 & 1 & 4 & $w_{231}$ & N & can. \\
1613 & 0 & 0 & 0 & 0 & 0 & 0 & 1 & 2 & 0 & 0 & 1 & 4 & 8 & 11 & 9 & 15 & 1 & 4 & $w_{199}$ & N & can. \\
1614 & 0 & 0 & 0 & 0 & 0 & 0 & 1 & 2 & 0 & 0 & 1 & 4 & 8 & 11 & 12 & 15 & 4 & 4 & $w_{200}$ & N & can. \\
1615 & 0 & 0 & 0 & 0 & 0 & 0 & 1 & 2 & 0 & 0 & 1 & 4 & 8 & 11 & 13 & 14 & 4 & 4 & $w_{270}$ & N & \#1237 \\
1616 & 0 & 0 & 0 & 0 & 0 & 0 & 1 & 2 & 0 & 0 & 1 & 4 & 8 & 14 & 10 & 12 & 8 & 4 & $w_{199}$ & N & can. \\
1617 & 0 & 0 & 0 & 0 & 0 & 0 & 1 & 2 & 0 & 0 & 1 & 4 & 8 & 14 & 10 & 13 & 2 & 4 & $w_{286}$ & N & can. \\
1618 & 0 & 0 & 0 & 0 & 0 & 0 & 1 & 2 & 0 & 0 & 1 & 4 & 8 & 14 & 11 & 13 & 8 & 4 & $w_{275}$ & N & \#1254 \\
1619 & 0 & 0 & 0 & 0 & 0 & 0 & 1 & 2 & 0 & 0 & 1 & 4 & 8 & 15 & 9 & 14 & 16 & 4 & $w_{242}$ & N & can. \\
1620 & 0 & 0 & 0 & 0 & 0 & 0 & 1 & 2 & 0 & 0 & 1 & 4 & 8 & 15 & 10 & 13 & 4 & 4 & $w_{243}$ & N & can. \\
1621 & 0 & 0 & 0 & 0 & 0 & 0 & 1 & 2 & 0 & 0 & 3 & 4 & 5 & 6 & 7 & 8 & 4 & 4 & $w_{250}$ & N & can. \\
1622 & 0 & 0 & 0 & 0 & 0 & 0 & 1 & 2 & 0 & 0 & 3 & 4 & 5 & 6 & 8 & 9 & 2 & 4 & $w_{260}$ & N & can. \\
1623 & 0 & 0 & 0 & 0 & 0 & 0 & 1 & 2 & 0 & 0 & 3 & 4 & 5 & 6 & 8 & 11 & 8 & 4 & $w_{224}$ & N & can. \\
1624 & 0 & 0 & 0 & 0 & 0 & 0 & 1 & 2 & 0 & 0 & 3 & 4 & 5 & 6 & 8 & 12 & 8 & 4 & $w_{263}$ & N & can. \\
1625 & 0 & 0 & 0 & 0 & 0 & 0 & 1 & 2 & 0 & 0 & 3 & 4 & 5 & 6 & 8 & 15 & 8 & 3 & $w_{251}$ & N & can. \\
1626 & 0 & 0 & 0 & 0 & 0 & 0 & 1 & 2 & 0 & 0 & 3 & 4 & 5 & 7 & 6 & 8 & 3 & 4 & $w_{257}$ & N & can. \\
1627 & 0 & 0 & 0 & 0 & 0 & 0 & 1 & 2 & 0 & 0 & 3 & 4 & 5 & 7 & 8 & 9 & 2 & 4 & $w_{392}$ & N & can. \\
1628 & 0 & 0 & 0 & 0 & 0 & 0 & 1 & 2 & 0 & 0 & 3 & 4 & 5 & 7 & 8 & 10 & 2 & 4 & $w_{266}$ & N & can. \\
1629 & 0 & 0 & 0 & 0 & 0 & 0 & 1 & 2 & 0 & 0 & 3 & 4 & 5 & 7 & 8 & 14 & 6 & 3 & $w_{280}$ & N & can. \\
1630 & 0 & 0 & 0 & 0 & 0 & 0 & 1 & 2 & 0 & 0 & 3 & 4 & 5 & 8 & 6 & 9 & 1 & 4 & $w_{241}$ & N & can. \\
1631 & 0 & 0 & 0 & 0 & 0 & 0 & 1 & 2 & 0 & 0 & 3 & 4 & 5 & 8 & 6 & 11 & 2 & 4 & $w_{225}$ & N & can. \\
1632 & 0 & 0 & 0 & 0 & 0 & 0 & 1 & 2 & 0 & 0 & 3 & 4 & 5 & 8 & 6 & 12 & 2 & 4 & $w_{262}$ & N & can. \\
1633 & 0 & 0 & 0 & 0 & 0 & 0 & 1 & 2 & 0 & 0 & 3 & 4 & 5 & 8 & 6 & 13 & 1 & 4 & $w_{260}$ & N & can. \\
1634 & 0 & 0 & 0 & 0 & 0 & 0 & 1 & 2 & 0 & 0 & 3 & 4 & 5 & 8 & 6 & 15 & 2 & 3 & $w_{259}$ & Y & can. \\
1635 & 0 & 0 & 0 & 0 & 0 & 0 & 1 & 2 & 0 & 0 & 3 & 4 & 5 & 8 & 7 & 9 & 1 & 4 & $w_{264}$ & N & can. \\
1636 & 0 & 0 & 0 & 0 & 0 & 0 & 1 & 2 & 0 & 0 & 3 & 4 & 5 & 8 & 7 & 10 & 1 & 4 & $w_{254}$ & N & can. \\
1637 & 0 & 0 & 0 & 0 & 0 & 0 & 1 & 2 & 0 & 0 & 3 & 4 & 5 & 8 & 7 & 11 & 1 & 4 & $w_{260}$ & N & can. \\
1638 & 0 & 0 & 0 & 0 & 0 & 0 & 1 & 2 & 0 & 0 & 3 & 4 & 5 & 8 & 7 & 12 & 1 & 4 & $w_{260}$ & N & can. \\
1639 & 0 & 0 & 0 & 0 & 0 & 0 & 1 & 2 & 0 & 0 & 3 & 4 & 5 & 8 & 7 & 13 & 1 & 4 & $w_{393}$ & N & can. \\
1640 & 0 & 0 & 0 & 0 & 0 & 0 & 1 & 2 & 0 & 0 & 3 & 4 & 5 & 8 & 7 & 14 & 1 & 3 & $w_{277}$ & Y & can. \\
1641 & 0 & 0 & 0 & 0 & 0 & 0 & 1 & 2 & 0 & 0 & 3 & 4 & 5 & 8 & 7 & 15 & 1 & 4 & $w_{266}$ & N & can. \\
1642 & 0 & 0 & 0 & 0 & 0 & 0 & 1 & 2 & 0 & 0 & 3 & 4 & 5 & 8 & 9 & 10 & 1 & 4 & $w_{254}$ & N & can. \\
1643 & 0 & 0 & 0 & 0 & 0 & 0 & 1 & 2 & 0 & 0 & 3 & 4 & 5 & 8 & 9 & 14 & 1 & 4 & $w_{265}$ & N & can. \\
1644 & 0 & 0 & 0 & 0 & 0 & 0 & 1 & 2 & 0 & 0 & 3 & 4 & 5 & 8 & 9 & 15 & 1 & 4 & $w_{267}$ & N & can. \\
1645 & 0 & 0 & 0 & 0 & 0 & 0 & 1 & 2 & 0 & 0 & 3 & 4 & 5 & 8 & 10 & 12 & 1 & 4 & $w_{268}$ & N & can. \\
1646 & 0 & 0 & 0 & 0 & 0 & 0 & 1 & 2 & 0 & 0 & 3 & 4 & 5 & 8 & 10 & 13 & 2 & 4 & $w_{265}$ & N & can. \\
1647 & 0 & 0 & 0 & 0 & 0 & 0 & 1 & 2 & 0 & 0 & 3 & 4 & 5 & 8 & 10 & 14 & 1 & 4 & $w_{374}$ & N & can. \\
1648 & 0 & 0 & 0 & 0 & 0 & 0 & 1 & 2 & 0 & 0 & 3 & 4 & 5 & 8 & 10 & 15 & 1 & 4 & $w_{267}$ & N & can. \\
1649 & 0 & 0 & 0 & 0 & 0 & 0 & 1 & 2 & 0 & 0 & 3 & 4 & 5 & 8 & 11 & 12 & 1 & 4 & $w_{241}$ & N & can. \\
1650 & 0 & 0 & 0 & 0 & 0 & 0 & 1 & 2 & 0 & 0 & 3 & 4 & 5 & 8 & 11 & 14 & 2 & 4 & $w_{267}$ & N & can. \\
1651 & 0 & 0 & 0 & 0 & 0 & 0 & 1 & 2 & 0 & 0 & 3 & 4 & 5 & 8 & 11 & 15 & 2 & 4 & $w_{266}$ & N & can. \\
1652 & 0 & 0 & 0 & 0 & 0 & 0 & 1 & 2 & 0 & 0 & 3 & 4 & 5 & 8 & 12 & 14 & 1 & 4 & $w_{267}$ & N & can. \\
1653 & 0 & 0 & 0 & 0 & 0 & 0 & 1 & 2 & 0 & 0 & 3 & 4 & 5 & 8 & 12 & 15 & 2 & 4 & $w_{260}$ & N & can. \\
1654 & 0 & 0 & 0 & 0 & 0 & 0 & 1 & 2 & 0 & 0 & 3 & 4 & 5 & 8 & 13 & 14 & 1 & 4 & $w_{374}$ & N & can. \\
1655 & 0 & 0 & 0 & 0 & 0 & 0 & 1 & 2 & 0 & 0 & 3 & 4 & 5 & 8 & 13 & 15 & 1 & 4 & $w_{281}$ & N & can. \\
1656 & 0 & 0 & 0 & 0 & 0 & 0 & 1 & 2 & 0 & 0 & 3 & 4 & 5 & 8 & 14 & 15 & 1 & 4 & $w_{281}$ & N & can. \\
1657 & 0 & 0 & 0 & 0 & 0 & 0 & 1 & 2 & 0 & 0 & 3 & 4 & 7 & 8 & 9 & 10 & 4 & 4 & $w_{264}$ & N & can. \\
1658 & 0 & 0 & 0 & 0 & 0 & 0 & 1 & 2 & 0 & 0 & 3 & 4 & 7 & 8 & 9 & 13 & 2 & 4 & $w_{278}$ & N & can. \\
1659 & 0 & 0 & 0 & 0 & 0 & 0 & 1 & 2 & 0 & 0 & 3 & 4 & 7 & 8 & 13 & 14 & 4 & 4 & $w_{278}$ & N & can. \\
1660 & 0 & 0 & 0 & 0 & 0 & 0 & 1 & 2 & 0 & 0 & 3 & 4 & 7 & 8 & 13 & 15 & 4 & 4 & $w_{379}$ & N & can. \\
1661 & 0 & 0 & 0 & 0 & 0 & 0 & 1 & 2 & 0 & 0 & 3 & 4 & 8 & 11 & 9 & 13 & 4 & 4 & $w_{394}$ & N & can. \\
1662 & 0 & 0 & 0 & 0 & 0 & 0 & 1 & 2 & 0 & 0 & 3 & 4 & 8 & 11 & 9 & 14 & 4 & 3 & $w_{274}$ & N & can. \\
1663 & 0 & 0 & 0 & 0 & 0 & 0 & 1 & 2 & 0 & 0 & 3 & 4 & 8 & 11 & 13 & 14 & 24 & 4 & $w_{275}$ & N & can. \\
1664 & 0 & 0 & 0 & 0 & 0 & 0 & 1 & 2 & 0 & 0 & 3 & 4 & 8 & 12 & 9 & 14 & 4 & 4 & $w_{275}$ & N & can. \\
1665 & 0 & 0 & 0 & 0 & 0 & 0 & 1 & 2 & 0 & 0 & 3 & 4 & 8 & 13 & 9 & 12 & 4 & 4 & $w_{239}$ & N & can. \\
1666 & 0 & 0 & 0 & 0 & 0 & 0 & 1 & 2 & 0 & 0 & 3 & 4 & 8 & 13 & 9 & 14 & 1 & 4 & $w_{286}$ & N & can. \\
1667 & 0 & 0 & 0 & 0 & 0 & 0 & 1 & 2 & 0 & 0 & 3 & 4 & 8 & 13 & 9 & 15 & 2 & 4 & $w_{265}$ & N & can. \\
1668 & 0 & 0 & 0 & 0 & 0 & 0 & 1 & 2 & 0 & 0 & 3 & 4 & 8 & 13 & 10 & 15 & 4 & 4 & $w_{265}$ & N & can. \\
1669 & 0 & 0 & 0 & 0 & 0 & 0 & 1 & 2 & 0 & 0 & 3 & 4 & 8 & 13 & 11 & 14 & 4 & 4 & $w_{265}$ & N & can. \\
1670 & 0 & 0 & 0 & 0 & 0 & 0 & 1 & 2 & 0 & 0 & 3 & 4 & 8 & 15 & 9 & 14 & 4 & 4 & $w_{270}$ & N & can. \\
1671 & 0 & 0 & 0 & 0 & 0 & 0 & 1 & 2 & 0 & 0 & 4 & 7 & 5 & 8 & 6 & 9 & 2 & 4 & $w_{202}$ & N & can. \\
1672 & 0 & 0 & 0 & 0 & 0 & 0 & 1 & 2 & 0 & 0 & 4 & 7 & 5 & 8 & 6 & 11 & 8 & 3 & $w_{223}$ & N & can. \\
1673 & 0 & 0 & 0 & 0 & 0 & 0 & 1 & 2 & 0 & 0 & 4 & 7 & 5 & 8 & 9 & 10 & 4 & 4 & $w_{200}$ & N & can. \\
1674 & 0 & 0 & 0 & 0 & 0 & 0 & 1 & 2 & 0 & 0 & 4 & 7 & 5 & 8 & 9 & 15 & 2 & 4 & $w_{226}$ & N & can. \\
1675 & 0 & 0 & 0 & 0 & 0 & 0 & 1 & 2 & 0 & 0 & 4 & 7 & 8 & 11 & 13 & 14 & 384 & 3 & $w_{375}$ & Y & \#1257 \\
1676 & 0 & 0 & 0 & 0 & 0 & 0 & 1 & 2 & 0 & 0 & 4 & 7 & 8 & 13 & 9 & 15 & 8 & 4 & $w_{224}$ & N & can. \\
1677 & 0 & 0 & 0 & 0 & 0 & 0 & 1 & 2 & 0 & 0 & 4 & 7 & 8 & 13 & 11 & 14 & 32 & 3 & $w_{376}$ & Y & \#1263 \\
1678 & 0 & 0 & 0 & 0 & 0 & 0 & 1 & 2 & 0 & 0 & 4 & 8 & 5 & 10 & 6 & 11 & 2 & 4 & $w_{241}$ & N & can. \\
1679 & 0 & 0 & 0 & 0 & 0 & 0 & 1 & 2 & 0 & 0 & 4 & 8 & 5 & 11 & 6 & 15 & 3 & 4 & $w_{268}$ & N & can. \\
1680 & 0 & 0 & 0 & 0 & 0 & 0 & 1 & 2 & 0 & 0 & 4 & 8 & 5 & 11 & 9 & 15 & 4 & 4 & $w_{244}$ & N & can. \\
1681 & 0 & 0 & 0 & 0 & 0 & 0 & 1 & 2 & 0 & 0 & 4 & 8 & 7 & 13 & 11 & 14 & 24 & 3 & $w_{377}$ & N & \#1271 \\
1682 & 0 & 0 & 0 & 0 & 0 & 0 & 1 & 2 & 0 & 1 & 0 & 2 & 0 & 3 & 3 & 4 & 192 & 4 & $w_{232}$ & N & can. \\
1683 & 0 & 0 & 0 & 0 & 0 & 0 & 1 & 2 & 0 & 1 & 0 & 2 & 0 & 3 & 4 & 5 & 8 & 4 & $w_{164}$ & N & can. \\
1684 & 0 & 0 & 0 & 0 & 0 & 0 & 1 & 2 & 0 & 1 & 0 & 2 & 0 & 3 & 4 & 7 & 16 & 3 & $w_{165}$ & Y & can. \\
1685 & 0 & 0 & 0 & 0 & 0 & 0 & 1 & 2 & 0 & 1 & 0 & 2 & 0 & 3 & 4 & 8 & 2 & 4 & $w_{381}$ & N & can. \\
1686 & 0 & 0 & 0 & 0 & 0 & 0 & 1 & 2 & 0 & 1 & 0 & 2 & 0 & 4 & 4 & 3 & 32 & 4 & $w_{299}$ & N & can. \\
1687 & 0 & 0 & 0 & 0 & 0 & 0 & 1 & 2 & 0 & 1 & 0 & 2 & 0 & 4 & 4 & 7 & 96 & 4 & $w_{168}$ & N & can. \\
1688 & 0 & 0 & 0 & 0 & 0 & 0 & 1 & 2 & 0 & 1 & 0 & 2 & 0 & 4 & 4 & 8 & 12 & 4 & $w_{301}$ & N & can. \\
1689 & 0 & 0 & 0 & 0 & 0 & 0 & 1 & 2 & 0 & 1 & 0 & 2 & 0 & 4 & 5 & 3 & 16 & 4 & $w_{164}$ & N & can. \\
1690 & 0 & 0 & 0 & 0 & 0 & 0 & 1 & 2 & 0 & 1 & 0 & 2 & 0 & 4 & 5 & 6 & 8 & 4 & $w_{168}$ & N & can. \\
1691 & 0 & 0 & 0 & 0 & 0 & 0 & 1 & 2 & 0 & 1 & 0 & 2 & 0 & 4 & 5 & 8 & 2 & 4 & $w_{171}$ & N & can. \\
1692 & 0 & 0 & 0 & 0 & 0 & 0 & 1 & 2 & 0 & 1 & 0 & 2 & 0 & 4 & 7 & 3 & 32 & 3 & $w_{316}$ & N & can. \\
1693 & 0 & 0 & 0 & 0 & 0 & 0 & 1 & 2 & 0 & 1 & 0 & 2 & 0 & 4 & 7 & 5 & 16 & 4 & $w_{320}$ & N & can. \\
1694 & 0 & 0 & 0 & 0 & 0 & 0 & 1 & 2 & 0 & 1 & 0 & 2 & 0 & 4 & 7 & 8 & 4 & 4 & $w_{325}$ & N & can. \\
1695 & 0 & 0 & 0 & 0 & 0 & 0 & 1 & 2 & 0 & 1 & 0 & 2 & 0 & 4 & 8 & 3 & 4 & 4 & $w_{318}$ & N & can. \\
1696 & 0 & 0 & 0 & 0 & 0 & 0 & 1 & 2 & 0 & 1 & 0 & 2 & 0 & 4 & 8 & 5 & 1 & 4 & $w_{321}$ & N & can. \\
1697 & 0 & 0 & 0 & 0 & 0 & 0 & 1 & 2 & 0 & 1 & 0 & 2 & 0 & 4 & 8 & 7 & 2 & 4 & $w_{177}$ & N & can. \\
1698 & 0 & 0 & 0 & 0 & 0 & 0 & 1 & 2 & 0 & 1 & 0 & 2 & 0 & 4 & 8 & 12 & 4 & 3 & $w_{363}$ & N & can. \\
1699 & 0 & 0 & 0 & 0 & 0 & 0 & 1 & 2 & 0 & 1 & 0 & 2 & 0 & 4 & 8 & 13 & 2 & 4 & $w_{180}$ & N & can. \\
1700 & 0 & 0 & 0 & 0 & 0 & 0 & 1 & 2 & 0 & 1 & 0 & 2 & 0 & 4 & 8 & 15 & 4 & 4 & $w_{329}$ & N & can. \\
1701 & 0 & 0 & 0 & 0 & 0 & 0 & 1 & 2 & 0 & 1 & 0 & 2 & 1 & 3 & 3 & 2 & 11520 & 4 & $w_{395}$ & N & can. \\
1702 & 0 & 0 & 0 & 0 & 0 & 0 & 1 & 2 & 0 & 1 & 0 & 2 & 1 & 3 & 3 & 4 & 64 & 4 & $w_{396}$ & N & can. \\
1703 & 0 & 0 & 0 & 0 & 0 & 0 & 1 & 2 & 0 & 1 & 0 & 2 & 1 & 3 & 4 & 5 & 16 & 4 & $w_{397}$ & N & can. \\
1704 & 0 & 0 & 0 & 0 & 0 & 0 & 1 & 2 & 0 & 1 & 0 & 2 & 1 & 3 & 4 & 6 & 16 & 3 & $w_{316}$ & N & can. \\
1705 & 0 & 0 & 0 & 0 & 0 & 0 & 1 & 2 & 0 & 1 & 0 & 2 & 1 & 3 & 4 & 7 & 16 & 4 & $w_{317}$ & N & can. \\
1706 & 0 & 0 & 0 & 0 & 0 & 0 & 1 & 2 & 0 & 1 & 0 & 2 & 1 & 3 & 4 & 8 & 2 & 4 & $w_{398}$ & N & can. \\
1707 & 0 & 0 & 0 & 0 & 0 & 0 & 1 & 2 & 0 & 1 & 0 & 2 & 1 & 4 & 4 & 2 & 32 & 4 & $w_{299}$ & N & can. \\
1708 & 0 & 0 & 0 & 0 & 0 & 0 & 1 & 2 & 0 & 1 & 0 & 2 & 1 & 4 & 4 & 3 & 16 & 4 & $w_{164}$ & N & can. \\
1709 & 0 & 0 & 0 & 0 & 0 & 0 & 1 & 2 & 0 & 1 & 0 & 2 & 1 & 4 & 4 & 6 & 16 & 4 & $w_{320}$ & N & can. \\
1710 & 0 & 0 & 0 & 0 & 0 & 0 & 1 & 2 & 0 & 1 & 0 & 2 & 1 & 4 & 4 & 7 & 16 & 4 & $w_{176}$ & N & can. \\
1711 & 0 & 0 & 0 & 0 & 0 & 0 & 1 & 2 & 0 & 1 & 0 & 2 & 1 & 4 & 4 & 8 & 2 & 4 & $w_{324}$ & N & can. \\
1712 & 0 & 0 & 0 & 0 & 0 & 0 & 1 & 2 & 0 & 1 & 0 & 2 & 1 & 4 & 5 & 2 & 16 & 4 & $w_{164}$ & N & can. \\
1713 & 0 & 0 & 0 & 0 & 0 & 0 & 1 & 2 & 0 & 1 & 0 & 2 & 1 & 4 & 5 & 3 & 16 & 4 & $w_{164}$ & N & can. \\
1714 & 0 & 0 & 0 & 0 & 0 & 0 & 1 & 2 & 0 & 1 & 0 & 2 & 1 & 4 & 5 & 6 & 8 & 4 & $w_{176}$ & N & can. \\
1715 & 0 & 0 & 0 & 0 & 0 & 0 & 1 & 2 & 0 & 1 & 0 & 2 & 1 & 4 & 5 & 8 & 2 & 4 & $w_{233}$ & N & can. \\
1716 & 0 & 0 & 0 & 0 & 0 & 0 & 1 & 2 & 0 & 1 & 0 & 2 & 1 & 4 & 6 & 5 & 16 & 4 & $w_{176}$ & N & can. \\
1717 & 0 & 0 & 0 & 0 & 0 & 0 & 1 & 2 & 0 & 1 & 0 & 2 & 1 & 4 & 6 & 8 & 2 & 4 & $w_{178}$ & N & can. \\
1718 & 0 & 0 & 0 & 0 & 0 & 0 & 1 & 2 & 0 & 1 & 0 & 2 & 1 & 4 & 7 & 2 & 32 & 3 & $w_{316}$ & N & can. \\
1719 & 0 & 0 & 0 & 0 & 0 & 0 & 1 & 2 & 0 & 1 & 0 & 2 & 1 & 4 & 7 & 3 & 16 & 4 & $w_{317}$ & N & can. \\
1720 & 0 & 0 & 0 & 0 & 0 & 0 & 1 & 2 & 0 & 1 & 0 & 2 & 1 & 4 & 7 & 5 & 16 & 4 & $w_{322}$ & N & can. \\
1721 & 0 & 0 & 0 & 0 & 0 & 0 & 1 & 2 & 0 & 1 & 0 & 2 & 1 & 4 & 7 & 8 & 2 & 4 & $w_{332}$ & N & can. \\
1722 & 0 & 0 & 0 & 0 & 0 & 0 & 1 & 2 & 0 & 1 & 0 & 2 & 1 & 4 & 8 & 2 & 4 & 4 & $w_{318}$ & N & can. \\
1723 & 0 & 0 & 0 & 0 & 0 & 0 & 1 & 2 & 0 & 1 & 0 & 2 & 1 & 4 & 8 & 3 & 2 & 4 & $w_{399}$ & N & can. \\
1724 & 0 & 0 & 0 & 0 & 0 & 0 & 1 & 2 & 0 & 1 & 0 & 2 & 1 & 4 & 8 & 5 & 2 & 4 & $w_{400}$ & N & can. \\
1725 & 0 & 0 & 0 & 0 & 0 & 0 & 1 & 2 & 0 & 1 & 0 & 2 & 1 & 4 & 8 & 6 & 1 & 4 & $w_{323}$ & N & can. \\
1726 & 0 & 0 & 0 & 0 & 0 & 0 & 1 & 2 & 0 & 1 & 0 & 2 & 1 & 4 & 8 & 7 & 1 & 4 & $w_{326}$ & N & can. \\
1727 & 0 & 0 & 0 & 0 & 0 & 0 & 1 & 2 & 0 & 1 & 0 & 2 & 1 & 4 & 8 & 12 & 2 & 4 & $w_{261}$ & N & can. \\
1728 & 0 & 0 & 0 & 0 & 0 & 0 & 1 & 2 & 0 & 1 & 0 & 2 & 1 & 4 & 8 & 13 & 2 & 3 & $w_{328}$ & N & can. \\
1729 & 0 & 0 & 0 & 0 & 0 & 0 & 1 & 2 & 0 & 1 & 0 & 2 & 1 & 4 & 8 & 14 & 2 & 4 & $w_{335}$ & N & can. \\
1730 & 0 & 0 & 0 & 0 & 0 & 0 & 1 & 2 & 0 & 1 & 0 & 2 & 1 & 4 & 8 & 15 & 2 & 4 & $w_{383}$ & N & can. \\
1731 & 0 & 0 & 0 & 0 & 0 & 0 & 1 & 2 & 0 & 1 & 0 & 2 & 3 & 4 & 4 & 5 & 16 & 4 & $w_{322}$ & N & can. \\
1732 & 0 & 0 & 0 & 0 & 0 & 0 & 1 & 2 & 0 & 1 & 0 & 2 & 3 & 4 & 4 & 7 & 32 & 4 & $w_{255}$ & N & can. \\
1733 & 0 & 0 & 0 & 0 & 0 & 0 & 1 & 2 & 0 & 1 & 0 & 2 & 3 & 4 & 4 & 8 & 4 & 4 & $w_{332}$ & N & can. \\
1734 & 0 & 0 & 0 & 0 & 0 & 0 & 1 & 2 & 0 & 1 & 0 & 2 & 3 & 4 & 5 & 3 & 64 & 4 & $w_{317}$ & N & can. \\
1735 & 0 & 0 & 0 & 0 & 0 & 0 & 1 & 2 & 0 & 1 & 0 & 2 & 3 & 4 & 5 & 6 & 8 & 4 & $w_{255}$ & N & can. \\
1736 & 0 & 0 & 0 & 0 & 0 & 0 & 1 & 2 & 0 & 1 & 0 & 2 & 3 & 4 & 5 & 8 & 2 & 4 & $w_{382}$ & N & can. \\
1737 & 0 & 0 & 0 & 0 & 0 & 0 & 1 & 2 & 0 & 1 & 0 & 2 & 3 & 4 & 7 & 3 & 128 & 4 & $w_{401}$ & N & can. \\
1738 & 0 & 0 & 0 & 0 & 0 & 0 & 1 & 2 & 0 & 1 & 0 & 2 & 3 & 4 & 7 & 5 & 16 & 4 & $w_{402}$ & N & can. \\
1739 & 0 & 0 & 0 & 0 & 0 & 0 & 1 & 2 & 0 & 1 & 0 & 2 & 3 & 4 & 7 & 8 & 4 & 4 & $w_{403}$ & N & can. \\
1740 & 0 & 0 & 0 & 0 & 0 & 0 & 1 & 2 & 0 & 1 & 0 & 2 & 3 & 4 & 8 & 3 & 16 & 4 & $w_{404}$ & N & can. \\
1741 & 0 & 0 & 0 & 0 & 0 & 0 & 1 & 2 & 0 & 1 & 0 & 2 & 3 & 4 & 8 & 5 & 1 & 4 & $w_{405}$ & N & can. \\
1742 & 0 & 0 & 0 & 0 & 0 & 0 & 1 & 2 & 0 & 1 & 0 & 2 & 3 & 4 & 8 & 7 & 2 & 4 & $w_{406}$ & N & can. \\
1743 & 0 & 0 & 0 & 0 & 0 & 0 & 1 & 2 & 0 & 1 & 0 & 2 & 3 & 4 & 8 & 12 & 4 & 4 & $w_{407}$ & N & can. \\
1744 & 0 & 0 & 0 & 0 & 0 & 0 & 1 & 2 & 0 & 1 & 0 & 2 & 3 & 4 & 8 & 13 & 2 & 4 & $w_{408}$ & N & can. \\
1745 & 0 & 0 & 0 & 0 & 0 & 0 & 1 & 2 & 0 & 1 & 0 & 2 & 3 & 4 & 8 & 15 & 4 & 3 & $w_{371}$ & N & can. \\
1746 & 0 & 0 & 0 & 0 & 0 & 0 & 1 & 2 & 0 & 1 & 0 & 2 & 4 & 5 & 5 & 6 & 16 & 4 & $w_{384}$ & N & can. \\
1747 & 0 & 0 & 0 & 0 & 0 & 0 & 1 & 2 & 0 & 1 & 0 & 2 & 4 & 5 & 5 & 7 & 96 & 4 & $w_{322}$ & N & can. \\
1748 & 0 & 0 & 0 & 0 & 0 & 0 & 1 & 2 & 0 & 1 & 0 & 2 & 4 & 5 & 5 & 8 & 2 & 4 & $w_{332}$ & N & can. \\
1749 & 0 & 0 & 0 & 0 & 0 & 0 & 1 & 2 & 0 & 1 & 0 & 2 & 4 & 5 & 6 & 7 & 128 & 3 & $w_{342}$ & N & can. \\
1750 & 0 & 0 & 0 & 0 & 0 & 0 & 1 & 2 & 0 & 1 & 0 & 2 & 4 & 5 & 6 & 8 & 2 & 4 & $w_{343}$ & N & can. \\
1751 & 0 & 0 & 0 & 0 & 0 & 0 & 1 & 2 & 0 & 1 & 0 & 2 & 4 & 5 & 7 & 8 & 1 & 4 & $w_{250}$ & N & can. \\
1752 & 0 & 0 & 0 & 0 & 0 & 0 & 1 & 2 & 0 & 1 & 0 & 2 & 4 & 5 & 8 & 9 & 4 & 3 & $w_{371}$ & N & can. \\
1753 & 0 & 0 & 0 & 0 & 0 & 0 & 1 & 2 & 0 & 1 & 0 & 2 & 4 & 5 & 8 & 10 & 4 & 4 & $w_{344}$ & N & can. \\
1754 & 0 & 0 & 0 & 0 & 0 & 0 & 1 & 2 & 0 & 1 & 0 & 2 & 4 & 5 & 8 & 11 & 2 & 4 & $w_{263}$ & N & can. \\
1755 & 0 & 0 & 0 & 0 & 0 & 0 & 1 & 2 & 0 & 1 & 0 & 2 & 4 & 5 & 8 & 12 & 1 & 4 & $w_{409}$ & N & can. \\
1756 & 0 & 0 & 0 & 0 & 0 & 0 & 1 & 2 & 0 & 1 & 0 & 2 & 4 & 5 & 8 & 14 & 1 & 4 & $w_{353}$ & N & can. \\
1757 & 0 & 0 & 0 & 0 & 0 & 0 & 1 & 2 & 0 & 1 & 0 & 2 & 4 & 7 & 7 & 8 & 4 & 4 & $w_{250}$ & N & can. \\
1758 & 0 & 0 & 0 & 0 & 0 & 0 & 1 & 2 & 0 & 1 & 0 & 2 & 4 & 7 & 8 & 11 & 8 & 3 & $w_{251}$ & N & can. \\
1759 & 0 & 0 & 0 & 0 & 0 & 0 & 1 & 2 & 0 & 1 & 0 & 2 & 4 & 7 & 8 & 12 & 2 & 4 & $w_{266}$ & N & can. \\
1760 & 0 & 0 & 0 & 0 & 0 & 0 & 1 & 2 & 0 & 1 & 0 & 2 & 4 & 7 & 8 & 13 & 2 & 4 & $w_{266}$ & N & can. \\
1761 & 0 & 0 & 0 & 0 & 0 & 0 & 1 & 2 & 0 & 1 & 0 & 2 & 4 & 8 & 8 & 5 & 4 & 4 & $w_{389}$ & N & can. \\
1762 & 0 & 0 & 0 & 0 & 0 & 0 & 1 & 2 & 0 & 1 & 0 & 2 & 4 & 8 & 8 & 7 & 8 & 4 & $w_{344}$ & N & can. \\
1763 & 0 & 0 & 0 & 0 & 0 & 0 & 1 & 2 & 0 & 1 & 0 & 2 & 4 & 8 & 8 & 12 & 8 & 4 & $w_{407}$ & N & can. \\
1764 & 0 & 0 & 0 & 0 & 0 & 0 & 1 & 2 & 0 & 1 & 0 & 2 & 4 & 8 & 8 & 13 & 4 & 4 & $w_{344}$ & N & can. \\
1765 & 0 & 0 & 0 & 0 & 0 & 0 & 1 & 2 & 0 & 1 & 0 & 2 & 4 & 8 & 8 & 15 & 8 & 4 & $w_{352}$ & N & can. \\
1766 & 0 & 0 & 0 & 0 & 0 & 0 & 1 & 2 & 0 & 1 & 0 & 2 & 4 & 8 & 9 & 6 & 8 & 4 & $w_{224}$ & N & can. \\
1767 & 0 & 0 & 0 & 0 & 0 & 0 & 1 & 2 & 0 & 1 & 0 & 2 & 4 & 8 & 9 & 7 & 4 & 4 & $w_{263}$ & N & can. \\
1768 & 0 & 0 & 0 & 0 & 0 & 0 & 1 & 2 & 0 & 1 & 0 & 2 & 4 & 8 & 9 & 12 & 2 & 4 & $w_{258}$ & N & can. \\
1769 & 0 & 0 & 0 & 0 & 0 & 0 & 1 & 2 & 0 & 1 & 0 & 2 & 4 & 8 & 9 & 14 & 2 & 4 & $w_{260}$ & N & can. \\
1770 & 0 & 0 & 0 & 0 & 0 & 0 & 1 & 2 & 0 & 1 & 0 & 2 & 4 & 8 & 11 & 7 & 16 & 3 & $w_{345}$ & N & can. \\
1771 & 0 & 0 & 0 & 0 & 0 & 0 & 1 & 2 & 0 & 1 & 0 & 2 & 4 & 8 & 11 & 12 & 4 & 4 & $w_{347}$ & N & can. \\
1772 & 0 & 0 & 0 & 0 & 0 & 0 & 1 & 2 & 0 & 1 & 0 & 2 & 4 & 8 & 11 & 13 & 4 & 4 & $w_{347}$ & N & can. \\
1773 & 0 & 0 & 0 & 0 & 0 & 0 & 1 & 2 & 0 & 1 & 0 & 3 & 0 & 4 & 5 & 6 & 64 & 4 & $w_{68}$ & N & can. \\
1774 & 0 & 0 & 0 & 0 & 0 & 0 & 1 & 2 & 0 & 1 & 0 & 3 & 0 & 4 & 5 & 7 & 32 & 4 & $w_{176}$ & N & can. \\
1775 & 0 & 0 & 0 & 0 & 0 & 0 & 1 & 2 & 0 & 1 & 0 & 3 & 0 & 4 & 5 & 8 & 4 & 4 & $w_{171}$ & N & can. \\
1776 & 0 & 0 & 0 & 0 & 0 & 0 & 1 & 2 & 0 & 1 & 0 & 3 & 0 & 4 & 6 & 5 & 8 & 4 & $w_{168}$ & N & can. \\
1777 & 0 & 0 & 0 & 0 & 0 & 0 & 1 & 2 & 0 & 1 & 0 & 3 & 0 & 4 & 6 & 7 & 8 & 4 & $w_{176}$ & N & can. \\
1778 & 0 & 0 & 0 & 0 & 0 & 0 & 1 & 2 & 0 & 1 & 0 & 3 & 0 & 4 & 6 & 8 & 1 & 4 & $w_{177}$ & N & can. \\
1779 & 0 & 0 & 0 & 0 & 0 & 0 & 1 & 2 & 0 & 1 & 0 & 3 & 0 & 4 & 8 & 5 & 1 & 4 & $w_{179}$ & N & can. \\
1780 & 0 & 0 & 0 & 0 & 0 & 0 & 1 & 2 & 0 & 1 & 0 & 3 & 0 & 4 & 8 & 6 & 1 & 4 & $w_{170}$ & N & can. \\
1781 & 0 & 0 & 0 & 0 & 0 & 0 & 1 & 2 & 0 & 1 & 0 & 3 & 0 & 4 & 8 & 7 & 1 & 4 & $w_{178}$ & N & can. \\
1782 & 0 & 0 & 0 & 0 & 0 & 0 & 1 & 2 & 0 & 1 & 0 & 3 & 0 & 4 & 8 & 12 & 2 & 4 & $w_{227}$ & N & can. \\
1783 & 0 & 0 & 0 & 0 & 0 & 0 & 1 & 2 & 0 & 1 & 0 & 3 & 0 & 4 & 8 & 13 & 2 & 3 & $w_{181}$ & Y & can. \\
1784 & 0 & 0 & 0 & 0 & 0 & 0 & 1 & 2 & 0 & 1 & 0 & 3 & 0 & 4 & 8 & 14 & 1 & 4 & $w_{182}$ & N & can. \\
1785 & 0 & 0 & 0 & 0 & 0 & 0 & 1 & 2 & 0 & 1 & 0 & 3 & 2 & 4 & 4 & 5 & 8 & 4 & $w_{176}$ & N & can. \\
1786 & 0 & 0 & 0 & 0 & 0 & 0 & 1 & 2 & 0 & 1 & 0 & 3 & 2 & 4 & 4 & 6 & 8 & 4 & $w_{176}$ & N & can. \\
1787 & 0 & 0 & 0 & 0 & 0 & 0 & 1 & 2 & 0 & 1 & 0 & 3 & 2 & 4 & 4 & 7 & 16 & 4 & $w_{176}$ & N & can. \\
1788 & 0 & 0 & 0 & 0 & 0 & 0 & 1 & 2 & 0 & 1 & 0 & 3 & 2 & 4 & 4 & 8 & 1 & 4 & $w_{178}$ & N & can. \\
1789 & 0 & 0 & 0 & 0 & 0 & 0 & 1 & 2 & 0 & 1 & 0 & 3 & 2 & 4 & 5 & 7 & 8 & 4 & $w_{255}$ & N & can. \\
1790 & 0 & 0 & 0 & 0 & 0 & 0 & 1 & 2 & 0 & 1 & 0 & 3 & 2 & 4 & 5 & 8 & 1 & 4 & $w_{178}$ & N & can. \\
1791 & 0 & 0 & 0 & 0 & 0 & 0 & 1 & 2 & 0 & 1 & 0 & 3 & 2 & 4 & 6 & 5 & 8 & 4 & $w_{176}$ & N & can. \\
1792 & 0 & 0 & 0 & 0 & 0 & 0 & 1 & 2 & 0 & 1 & 0 & 3 & 2 & 4 & 6 & 7 & 8 & 4 & $w_{255}$ & N & can. \\
1793 & 0 & 0 & 0 & 0 & 0 & 0 & 1 & 2 & 0 & 1 & 0 & 3 & 2 & 4 & 6 & 8 & 1 & 4 & $w_{326}$ & N & can. \\
1794 & 0 & 0 & 0 & 0 & 0 & 0 & 1 & 2 & 0 & 1 & 0 & 3 & 2 & 4 & 7 & 5 & 8 & 4 & $w_{255}$ & N & can. \\
1795 & 0 & 0 & 0 & 0 & 0 & 0 & 1 & 2 & 0 & 1 & 0 & 3 & 2 & 4 & 7 & 6 & 16 & 4 & $w_{176}$ & N & can. \\
1796 & 0 & 0 & 0 & 0 & 0 & 0 & 1 & 2 & 0 & 1 & 0 & 3 & 2 & 4 & 7 & 8 & 1 & 4 & $w_{326}$ & N & can. \\
1797 & 0 & 0 & 0 & 0 & 0 & 0 & 1 & 2 & 0 & 1 & 0 & 3 & 2 & 4 & 8 & 5 & 1 & 4 & $w_{326}$ & N & can. \\
1798 & 0 & 0 & 0 & 0 & 0 & 0 & 1 & 2 & 0 & 1 & 0 & 3 & 2 & 4 & 8 & 6 & 1 & 4 & $w_{178}$ & N & can. \\
1799 & 0 & 0 & 0 & 0 & 0 & 0 & 1 & 2 & 0 & 1 & 0 & 3 & 2 & 4 & 8 & 7 & 1 & 4 & $w_{382}$ & N & can. \\
1800 & 0 & 0 & 0 & 0 & 0 & 0 & 1 & 2 & 0 & 1 & 0 & 3 & 2 & 4 & 8 & 9 & 1 & 4 & $w_{326}$ & N & can. \\
1801 & 0 & 0 & 0 & 0 & 0 & 0 & 1 & 2 & 0 & 1 & 0 & 3 & 2 & 4 & 8 & 10 & 1 & 4 & $w_{382}$ & N & can. \\
1802 & 0 & 0 & 0 & 0 & 0 & 0 & 1 & 2 & 0 & 1 & 0 & 3 & 2 & 4 & 8 & 11 & 1 & 4 & $w_{178}$ & N & \#1788 \\
1803 & 0 & 0 & 0 & 0 & 0 & 0 & 1 & 2 & 0 & 1 & 0 & 3 & 2 & 4 & 8 & 12 & 1 & 4 & $w_{383}$ & N & can. \\
1804 & 0 & 0 & 0 & 0 & 0 & 0 & 1 & 2 & 0 & 1 & 0 & 3 & 2 & 4 & 8 & 13 & 1 & 4 & $w_{383}$ & N & can. \\
1805 & 0 & 0 & 0 & 0 & 0 & 0 & 1 & 2 & 0 & 1 & 0 & 3 & 2 & 4 & 8 & 14 & 1 & 4 & $w_{201}$ & N & can. \\
1806 & 0 & 0 & 0 & 0 & 0 & 0 & 1 & 2 & 0 & 1 & 0 & 3 & 2 & 4 & 8 & 15 & 1 & 3 & $w_{237}$ & Y & can. \\
1807 & 0 & 0 & 0 & 0 & 0 & 0 & 1 & 2 & 0 & 1 & 0 & 3 & 4 & 5 & 5 & 8 & 2 & 4 & $w_{385}$ & N & can. \\
1808 & 0 & 0 & 0 & 0 & 0 & 0 & 1 & 2 & 0 & 1 & 0 & 3 & 4 & 5 & 6 & 8 & 1 & 4 & $w_{368}$ & N & can. \\
1809 & 0 & 0 & 0 & 0 & 0 & 0 & 1 & 2 & 0 & 1 & 0 & 3 & 4 & 5 & 7 & 6 & 64 & 4 & $w_{249}$ & N & can. \\
1810 & 0 & 0 & 0 & 0 & 0 & 0 & 1 & 2 & 0 & 1 & 0 & 3 & 4 & 5 & 7 & 8 & 1 & 4 & $w_{386}$ & N & can. \\
1811 & 0 & 0 & 0 & 0 & 0 & 0 & 1 & 2 & 0 & 1 & 0 & 3 & 4 & 5 & 8 & 9 & 4 & 4 & $w_{389}$ & N & can. \\
1812 & 0 & 0 & 0 & 0 & 0 & 0 & 1 & 2 & 0 & 1 & 0 & 3 & 4 & 5 & 8 & 10 & 2 & 4 & $w_{263}$ & N & can. \\
1813 & 0 & 0 & 0 & 0 & 0 & 0 & 1 & 2 & 0 & 1 & 0 & 3 & 4 & 5 & 8 & 11 & 2 & 4 & $w_{224}$ & N & can. \\
1814 & 0 & 0 & 0 & 0 & 0 & 0 & 1 & 2 & 0 & 1 & 0 & 3 & 4 & 5 & 8 & 12 & 1 & 4 & $w_{258}$ & N & can. \\
1815 & 0 & 0 & 0 & 0 & 0 & 0 & 1 & 2 & 0 & 1 & 0 & 3 & 4 & 5 & 8 & 14 & 1 & 4 & $w_{260}$ & N & can. \\
1816 & 0 & 0 & 0 & 0 & 0 & 0 & 1 & 2 & 0 & 1 & 0 & 3 & 4 & 6 & 6 & 5 & 48 & 3 & $w_{84}$ & Y & can. \\
1817 & 0 & 0 & 0 & 0 & 0 & 0 & 1 & 2 & 0 & 1 & 0 & 3 & 4 & 6 & 6 & 8 & 1 & 4 & $w_{368}$ & N & can. \\
1818 & 0 & 0 & 0 & 0 & 0 & 0 & 1 & 2 & 0 & 1 & 0 & 3 & 4 & 6 & 7 & 8 & 2 & 4 & $w_{386}$ & N & can. \\
1819 & 0 & 0 & 0 & 0 & 0 & 0 & 1 & 2 & 0 & 1 & 0 & 3 & 4 & 6 & 8 & 10 & 4 & 4 & $w_{263}$ & N & can. \\
1820 & 0 & 0 & 0 & 0 & 0 & 0 & 1 & 2 & 0 & 1 & 0 & 3 & 4 & 6 & 8 & 11 & 2 & 3 & $w_{251}$ & Y & can. \\
1821 & 0 & 0 & 0 & 0 & 0 & 0 & 1 & 2 & 0 & 1 & 0 & 3 & 4 & 6 & 8 & 12 & 1 & 4 & $w_{266}$ & N & can. \\
1822 & 0 & 0 & 0 & 0 & 0 & 0 & 1 & 2 & 0 & 1 & 0 & 3 & 4 & 6 & 8 & 13 & 1 & 4 & $w_{266}$ & N & can. \\
1823 & 0 & 0 & 0 & 0 & 0 & 0 & 1 & 2 & 0 & 1 & 0 & 3 & 4 & 7 & 6 & 8 & 2 & 4 & $w_{285}$ & N & can. \\
1824 & 0 & 0 & 0 & 0 & 0 & 0 & 1 & 2 & 0 & 1 & 0 & 3 & 4 & 7 & 8 & 12 & 1 & 4 & $w_{241}$ & N & can. \\
1825 & 0 & 0 & 0 & 0 & 0 & 0 & 1 & 2 & 0 & 1 & 0 & 3 & 4 & 7 & 8 & 13 & 1 & 4 & $w_{241}$ & N & can. \\
1826 & 0 & 0 & 0 & 0 & 0 & 0 & 1 & 2 & 0 & 1 & 0 & 3 & 4 & 8 & 8 & 5 & 4 & 3 & $w_{223}$ & Y & can. \\
1827 & 0 & 0 & 0 & 0 & 0 & 0 & 1 & 2 & 0 & 1 & 0 & 3 & 4 & 8 & 8 & 6 & 2 & 4 & $w_{224}$ & N & can. \\
1828 & 0 & 0 & 0 & 0 & 0 & 0 & 1 & 2 & 0 & 1 & 0 & 3 & 4 & 8 & 8 & 12 & 4 & 4 & $w_{262}$ & N & can. \\
1829 & 0 & 0 & 0 & 0 & 0 & 0 & 1 & 2 & 0 & 1 & 0 & 3 & 4 & 8 & 8 & 13 & 4 & 4 & $w_{262}$ & N & can. \\
1830 & 0 & 0 & 0 & 0 & 0 & 0 & 1 & 2 & 0 & 1 & 0 & 3 & 4 & 8 & 8 & 14 & 2 & 4 & $w_{241}$ & N & can. \\
1831 & 0 & 0 & 0 & 0 & 0 & 0 & 1 & 2 & 0 & 1 & 0 & 3 & 4 & 8 & 9 & 6 & 2 & 4 & $w_{224}$ & N & can. \\
1832 & 0 & 0 & 0 & 0 & 0 & 0 & 1 & 2 & 0 & 1 & 0 & 3 & 4 & 8 & 9 & 12 & 2 & 4 & $w_{262}$ & N & can. \\
1833 & 0 & 0 & 0 & 0 & 0 & 0 & 1 & 2 & 0 & 1 & 0 & 3 & 4 & 8 & 9 & 14 & 2 & 4 & $w_{241}$ & N & can. \\
1834 & 0 & 0 & 0 & 0 & 0 & 0 & 1 & 2 & 0 & 1 & 0 & 3 & 4 & 8 & 10 & 6 & 4 & 4 & $w_{224}$ & N & can. \\
1835 & 0 & 0 & 0 & 0 & 0 & 0 & 1 & 2 & 0 & 1 & 0 & 3 & 4 & 8 & 10 & 7 & 4 & 3 & $w_{251}$ & N & can. \\
1836 & 0 & 0 & 0 & 0 & 0 & 0 & 1 & 2 & 0 & 1 & 0 & 3 & 4 & 8 & 10 & 12 & 1 & 4 & $w_{260}$ & N & can. \\
1837 & 0 & 0 & 0 & 0 & 0 & 0 & 1 & 2 & 0 & 1 & 0 & 3 & 4 & 8 & 10 & 13 & 1 & 4 & $w_{260}$ & N & can. \\
1838 & 0 & 0 & 0 & 0 & 0 & 0 & 1 & 2 & 0 & 1 & 0 & 4 & 0 & 5 & 3 & 6 & 32 & 4 & $w_{176}$ & N & can. \\
1839 & 0 & 0 & 0 & 0 & 0 & 0 & 1 & 2 & 0 & 1 & 0 & 4 & 0 & 5 & 3 & 8 & 8 & 4 & $w_{233}$ & N & can. \\
1840 & 0 & 0 & 0 & 0 & 0 & 0 & 1 & 2 & 0 & 1 & 0 & 4 & 0 & 5 & 6 & 3 & 8 & 4 & $w_{176}$ & N & can. \\
1841 & 0 & 0 & 0 & 0 & 0 & 0 & 1 & 2 & 0 & 1 & 0 & 4 & 0 & 5 & 6 & 8 & 2 & 4 & $w_{178}$ & N & can. \\
1842 & 0 & 0 & 0 & 0 & 0 & 0 & 1 & 2 & 0 & 1 & 0 & 4 & 0 & 5 & 8 & 3 & 2 & 4 & $w_{233}$ & N & can. \\
1843 & 0 & 0 & 0 & 0 & 0 & 0 & 1 & 2 & 0 & 1 & 0 & 4 & 0 & 5 & 8 & 6 & 1 & 4 & $w_{178}$ & N & can. \\
1844 & 0 & 0 & 0 & 0 & 0 & 0 & 1 & 2 & 0 & 1 & 0 & 4 & 0 & 5 & 8 & 10 & 4 & 4 & $w_{327}$ & N & can. \\
1845 & 0 & 0 & 0 & 0 & 0 & 0 & 1 & 2 & 0 & 1 & 0 & 4 & 0 & 5 & 8 & 11 & 4 & 3 & $w_{181}$ & Y & can. \\
1846 & 0 & 0 & 0 & 0 & 0 & 0 & 1 & 2 & 0 & 1 & 0 & 4 & 0 & 5 & 8 & 14 & 2 & 4 & $w_{201}$ & N & can. \\
1847 & 0 & 0 & 0 & 0 & 0 & 0 & 1 & 2 & 0 & 1 & 0 & 4 & 0 & 6 & 8 & 3 & 2 & 4 & $w_{177}$ & N & can. \\
1848 & 0 & 0 & 0 & 0 & 0 & 0 & 1 & 2 & 0 & 1 & 0 & 4 & 0 & 6 & 8 & 5 & 1 & 4 & $w_{178}$ & N & can. \\
1849 & 0 & 0 & 0 & 0 & 0 & 0 & 1 & 2 & 0 & 1 & 0 & 4 & 0 & 6 & 8 & 9 & 2 & 4 & $w_{180}$ & N & can. \\
1850 & 0 & 0 & 0 & 0 & 0 & 0 & 1 & 2 & 0 & 1 & 0 & 4 & 0 & 6 & 8 & 10 & 2 & 4 & $w_{201}$ & N & can. \\
1851 & 0 & 0 & 0 & 0 & 0 & 0 & 1 & 2 & 0 & 1 & 0 & 4 & 0 & 6 & 8 & 11 & 2 & 4 & $w_{203}$ & N & can. \\
1852 & 0 & 0 & 0 & 0 & 0 & 0 & 1 & 2 & 0 & 1 & 0 & 4 & 0 & 6 & 8 & 12 & 1 & 4 & $w_{182}$ & N & can. \\
1853 & 0 & 0 & 0 & 0 & 0 & 0 & 1 & 2 & 0 & 1 & 0 & 4 & 0 & 6 & 8 & 13 & 1 & 4 & $w_{202}$ & N & can. \\
1854 & 0 & 0 & 0 & 0 & 0 & 0 & 1 & 2 & 0 & 1 & 0 & 4 & 0 & 7 & 8 & 3 & 2 & 4 & $w_{323}$ & N & can. \\
1855 & 0 & 0 & 0 & 0 & 0 & 0 & 1 & 2 & 0 & 1 & 0 & 4 & 0 & 7 & 8 & 5 & 1 & 4 & $w_{325}$ & N & can. \\
1856 & 0 & 0 & 0 & 0 & 0 & 0 & 1 & 2 & 0 & 1 & 0 & 4 & 0 & 7 & 8 & 9 & 2 & 3 & $w_{328}$ & N & can. \\
1857 & 0 & 0 & 0 & 0 & 0 & 0 & 1 & 2 & 0 & 1 & 0 & 4 & 0 & 7 & 8 & 10 & 2 & 4 & $w_{338}$ & N & can. \\
1858 & 0 & 0 & 0 & 0 & 0 & 0 & 1 & 2 & 0 & 1 & 0 & 4 & 0 & 7 & 8 & 11 & 2 & 4 & $w_{202}$ & N & can. \\
1859 & 0 & 0 & 0 & 0 & 0 & 0 & 1 & 2 & 0 & 1 & 0 & 4 & 0 & 7 & 8 & 12 & 1 & 4 & $w_{365}$ & N & can. \\
1860 & 0 & 0 & 0 & 0 & 0 & 0 & 1 & 2 & 0 & 1 & 0 & 4 & 0 & 7 & 8 & 13 & 1 & 4 & $w_{225}$ & N & can. \\
1861 & 0 & 0 & 0 & 0 & 0 & 0 & 1 & 2 & 0 & 1 & 0 & 4 & 0 & 8 & 14 & 3 & 2 & 4 & $w_{340}$ & N & can. \\
1862 & 0 & 0 & 0 & 0 & 0 & 0 & 1 & 2 & 0 & 1 & 0 & 4 & 0 & 8 & 14 & 6 & 4 & 4 & $w_{329}$ & N & can. \\
1863 & 0 & 0 & 0 & 0 & 0 & 0 & 1 & 2 & 0 & 1 & 0 & 4 & 0 & 8 & 14 & 7 & 4 & 4 & $w_{225}$ & N & can. \\
1864 & 0 & 0 & 0 & 0 & 0 & 0 & 1 & 2 & 0 & 1 & 0 & 4 & 0 & 8 & 14 & 12 & 8 & 4 & $w_{335}$ & N & can. \\
1865 & 0 & 0 & 0 & 0 & 0 & 0 & 1 & 2 & 0 & 1 & 0 & 4 & 0 & 8 & 14 & 13 & 8 & 4 & $w_{203}$ & N & can. \\
1866 & 0 & 0 & 0 & 0 & 0 & 0 & 1 & 2 & 0 & 1 & 0 & 4 & 0 & 8 & 15 & 3 & 2 & 4 & $w_{201}$ & N & can. \\
1867 & 0 & 0 & 0 & 0 & 0 & 0 & 1 & 2 & 0 & 1 & 0 & 4 & 0 & 8 & 15 & 6 & 2 & 4 & $w_{202}$ & N & can. \\
1868 & 0 & 0 & 0 & 0 & 0 & 0 & 1 & 2 & 0 & 1 & 0 & 4 & 0 & 8 & 15 & 12 & 4 & 4 & $w_{202}$ & N & can. \\
1869 & 0 & 0 & 0 & 0 & 0 & 0 & 1 & 2 & 0 & 1 & 0 & 4 & 1 & 4 & 2 & 6 & 16 & 4 & $w_{168}$ & N & can. \\
1870 & 0 & 0 & 0 & 0 & 0 & 0 & 1 & 2 & 0 & 1 & 0 & 4 & 1 & 4 & 2 & 7 & 16 & 4 & $w_{176}$ & N & can. \\
1871 & 0 & 0 & 0 & 0 & 0 & 0 & 1 & 2 & 0 & 1 & 0 & 4 & 1 & 4 & 2 & 8 & 2 & 4 & $w_{179}$ & N & can. \\
1872 & 0 & 0 & 0 & 0 & 0 & 0 & 1 & 2 & 0 & 1 & 0 & 4 & 1 & 4 & 3 & 6 & 8 & 4 & $w_{168}$ & N & can. \\
1873 & 0 & 0 & 0 & 0 & 0 & 0 & 1 & 2 & 0 & 1 & 0 & 4 & 1 & 4 & 3 & 7 & 8 & 4 & $w_{176}$ & N & can. \\
1874 & 0 & 0 & 0 & 0 & 0 & 0 & 1 & 2 & 0 & 1 & 0 & 4 & 1 & 4 & 3 & 8 & 1 & 4 & $w_{179}$ & N & can. \\
1875 & 0 & 0 & 0 & 0 & 0 & 0 & 1 & 2 & 0 & 1 & 0 & 4 & 1 & 4 & 6 & 8 & 1 & 4 & $w_{177}$ & N & can. \\
1876 & 0 & 0 & 0 & 0 & 0 & 0 & 1 & 2 & 0 & 1 & 0 & 4 & 1 & 4 & 7 & 8 & 1 & 4 & $w_{326}$ & N & can. \\
1877 & 0 & 0 & 0 & 0 & 0 & 0 & 1 & 2 & 0 & 1 & 0 & 4 & 1 & 4 & 8 & 10 & 2 & 4 & $w_{327}$ & N & can. \\
1878 & 0 & 0 & 0 & 0 & 0 & 0 & 1 & 2 & 0 & 1 & 0 & 4 & 1 & 4 & 8 & 11 & 2 & 3 & $w_{181}$ & N & can. \\
1879 & 0 & 0 & 0 & 0 & 0 & 0 & 1 & 2 & 0 & 1 & 0 & 4 & 1 & 4 & 8 & 12 & 1 & 4 & $w_{178}$ & N & can. \\
1880 & 0 & 0 & 0 & 0 & 0 & 0 & 1 & 2 & 0 & 1 & 0 & 4 & 1 & 4 & 8 & 14 & 1 & 4 & $w_{201}$ & N & can. \\
1881 & 0 & 0 & 0 & 0 & 0 & 0 & 1 & 2 & 0 & 1 & 0 & 4 & 1 & 5 & 3 & 6 & 16 & 4 & $w_{176}$ & N & can. \\
1882 & 0 & 0 & 0 & 0 & 0 & 0 & 1 & 2 & 0 & 1 & 0 & 4 & 1 & 5 & 3 & 7 & 32 & 4 & $w_{336}$ & N & can. \\
1883 & 0 & 0 & 0 & 0 & 0 & 0 & 1 & 2 & 0 & 1 & 0 & 4 & 1 & 5 & 3 & 8 & 2 & 4 & $w_{321}$ & N & can. \\
1884 & 0 & 0 & 0 & 0 & 0 & 0 & 1 & 2 & 0 & 1 & 0 & 4 & 1 & 5 & 6 & 8 & 1 & 4 & $w_{325}$ & N & can. \\
1885 & 0 & 0 & 0 & 0 & 0 & 0 & 1 & 2 & 0 & 1 & 0 & 4 & 1 & 5 & 7 & 8 & 1 & 4 & $w_{178}$ & N & can. \\
1886 & 0 & 0 & 0 & 0 & 0 & 0 & 1 & 2 & 0 & 1 & 0 & 4 & 1 & 5 & 8 & 10 & 2 & 3 & $w_{328}$ & N & can. \\
1887 & 0 & 0 & 0 & 0 & 0 & 0 & 1 & 2 & 0 & 1 & 0 & 4 & 1 & 5 & 8 & 11 & 2 & 4 & $w_{180}$ & N & can. \\
1888 & 0 & 0 & 0 & 0 & 0 & 0 & 1 & 2 & 0 & 1 & 0 & 4 & 1 & 5 & 8 & 12 & 2 & 4 & $w_{361}$ & N & can. \\
1889 & 0 & 0 & 0 & 0 & 0 & 0 & 1 & 2 & 0 & 1 & 0 & 4 & 1 & 5 & 8 & 13 & 2 & 4 & $w_{326}$ & N & can. \\
1890 & 0 & 0 & 0 & 0 & 0 & 0 & 1 & 2 & 0 & 1 & 0 & 4 & 1 & 5 & 8 & 14 & 2 & 4 & $w_{201}$ & N & can. \\
1891 & 0 & 0 & 0 & 0 & 0 & 0 & 1 & 2 & 0 & 1 & 0 & 4 & 1 & 5 & 8 & 15 & 2 & 4 & $w_{340}$ & N & can. \\
1892 & 0 & 0 & 0 & 0 & 0 & 0 & 1 & 2 & 0 & 1 & 0 & 4 & 1 & 6 & 8 & 10 & 2 & 4 & $w_{338}$ & N & can. \\
1893 & 0 & 0 & 0 & 0 & 0 & 0 & 1 & 2 & 0 & 1 & 0 & 4 & 1 & 6 & 8 & 11 & 2 & 4 & $w_{202}$ & N & can. \\
1894 & 0 & 0 & 0 & 0 & 0 & 0 & 1 & 2 & 0 & 1 & 0 & 4 & 1 & 6 & 8 & 12 & 2 & 4 & $w_{365}$ & N & can. \\
1895 & 0 & 0 & 0 & 0 & 0 & 0 & 1 & 2 & 0 & 1 & 0 & 4 & 1 & 6 & 8 & 13 & 1 & 4 & $w_{225}$ & N & can. \\
1896 & 0 & 0 & 0 & 0 & 0 & 0 & 1 & 2 & 0 & 1 & 0 & 4 & 1 & 7 & 8 & 10 & 2 & 4 & $w_{262}$ & N & can. \\
1897 & 0 & 0 & 0 & 0 & 0 & 0 & 1 & 2 & 0 & 1 & 0 & 4 & 1 & 7 & 8 & 11 & 2 & 4 & $w_{225}$ & N & can. \\
1898 & 0 & 0 & 0 & 0 & 0 & 0 & 1 & 2 & 0 & 1 & 0 & 4 & 1 & 7 & 8 & 13 & 2 & 4 & $w_{262}$ & N & can. \\
1899 & 0 & 0 & 0 & 0 & 0 & 0 & 1 & 2 & 0 & 1 & 0 & 4 & 1 & 8 & 10 & 6 & 2 & 4 & $w_{202}$ & N & can. \\
1900 & 0 & 0 & 0 & 0 & 0 & 0 & 1 & 2 & 0 & 1 & 0 & 4 & 1 & 8 & 10 & 7 & 1 & 4 & $w_{225}$ & N & can. \\
1901 & 0 & 0 & 0 & 0 & 0 & 0 & 1 & 2 & 0 & 1 & 0 & 4 & 1 & 8 & 11 & 7 & 2 & 4 & $w_{341}$ & N & can. \\
1902 & 0 & 0 & 0 & 0 & 0 & 0 & 1 & 2 & 0 & 1 & 0 & 4 & 2 & 3 & 4 & 5 & 8 & 4 & $w_{176}$ & N & can. \\
1903 & 0 & 0 & 0 & 0 & 0 & 0 & 1 & 2 & 0 & 1 & 0 & 4 & 2 & 3 & 4 & 7 & 8 & 4 & $w_{255}$ & N & can. \\
1904 & 0 & 0 & 0 & 0 & 0 & 0 & 1 & 2 & 0 & 1 & 0 & 4 & 2 & 3 & 4 & 8 & 1 & 4 & $w_{326}$ & N & can. \\
1905 & 0 & 0 & 0 & 0 & 0 & 0 & 1 & 2 & 0 & 1 & 0 & 4 & 2 & 3 & 5 & 4 & 16 & 4 & $w_{320}$ & N & can. \\
1906 & 0 & 0 & 0 & 0 & 0 & 0 & 1 & 2 & 0 & 1 & 0 & 4 & 2 & 3 & 5 & 7 & 16 & 4 & $w_{402}$ & N & can. \\
1907 & 0 & 0 & 0 & 0 & 0 & 0 & 1 & 2 & 0 & 1 & 0 & 4 & 2 & 3 & 5 & 8 & 2 & 4 & $w_{332}$ & N & can. \\
1908 & 0 & 0 & 0 & 0 & 0 & 0 & 1 & 2 & 0 & 1 & 0 & 4 & 2 & 3 & 6 & 4 & 16 & 4 & $w_{322}$ & N & can. \\
1909 & 0 & 0 & 0 & 0 & 0 & 0 & 1 & 2 & 0 & 1 & 0 & 4 & 2 & 3 & 6 & 5 & 16 & 4 & $w_{255}$ & N & can. \\
1910 & 0 & 0 & 0 & 0 & 0 & 0 & 1 & 2 & 0 & 1 & 0 & 4 & 2 & 3 & 6 & 8 & 2 & 4 & $w_{332}$ & N & can. \\
1911 & 0 & 0 & 0 & 0 & 0 & 0 & 1 & 2 & 0 & 1 & 0 & 4 & 2 & 3 & 8 & 4 & 1 & 4 & $w_{323}$ & N & can. \\
1912 & 0 & 0 & 0 & 0 & 0 & 0 & 1 & 2 & 0 & 1 & 0 & 4 & 2 & 3 & 8 & 5 & 1 & 4 & $w_{382}$ & N & can. \\
1913 & 0 & 0 & 0 & 0 & 0 & 0 & 1 & 2 & 0 & 1 & 0 & 4 & 2 & 3 & 8 & 7 & 1 & 4 & $w_{405}$ & N & can. \\
1914 & 0 & 0 & 0 & 0 & 0 & 0 & 1 & 2 & 0 & 1 & 0 & 4 & 2 & 3 & 8 & 12 & 2 & 4 & $w_{335}$ & N & can. \\
1915 & 0 & 0 & 0 & 0 & 0 & 0 & 1 & 2 & 0 & 1 & 0 & 4 & 2 & 3 & 8 & 13 & 1 & 4 & $w_{388}$ & N & can. \\
1916 & 0 & 0 & 0 & 0 & 0 & 0 & 1 & 2 & 0 & 1 & 0 & 4 & 2 & 3 & 8 & 15 & 2 & 3 & $w_{371}$ & N & can. \\
1917 & 0 & 0 & 0 & 0 & 0 & 0 & 1 & 2 & 0 & 1 & 0 & 4 & 2 & 4 & 3 & 6 & 16 & 4 & $w_{176}$ & N & can. \\
1918 & 0 & 0 & 0 & 0 & 0 & 0 & 1 & 2 & 0 & 1 & 0 & 4 & 2 & 4 & 3 & 7 & 8 & 4 & $w_{255}$ & N & can. \\
1919 & 0 & 0 & 0 & 0 & 0 & 0 & 1 & 2 & 0 & 1 & 0 & 4 & 2 & 4 & 3 & 8 & 1 & 4 & $w_{326}$ & N & can. \\
1920 & 0 & 0 & 0 & 0 & 0 & 0 & 1 & 2 & 0 & 1 & 0 & 4 & 2 & 4 & 7 & 6 & 8 & 4 & $w_{176}$ & N & can. \\
1921 & 0 & 0 & 0 & 0 & 0 & 0 & 1 & 2 & 0 & 1 & 0 & 4 & 2 & 4 & 7 & 8 & 1 & 4 & $w_{240}$ & N & can. \\
1922 & 0 & 0 & 0 & 0 & 0 & 0 & 1 & 2 & 0 & 1 & 0 & 4 & 2 & 4 & 8 & 6 & 1 & 4 & $w_{178}$ & N & can. \\
1923 & 0 & 0 & 0 & 0 & 0 & 0 & 1 & 2 & 0 & 1 & 0 & 4 & 2 & 4 & 8 & 7 & 1 & 4 & $w_{257}$ & N & can. \\
1924 & 0 & 0 & 0 & 0 & 0 & 0 & 1 & 2 & 0 & 1 & 0 & 4 & 2 & 4 & 8 & 9 & 1 & 4 & $w_{383}$ & N & can. \\
1925 & 0 & 0 & 0 & 0 & 0 & 0 & 1 & 2 & 0 & 1 & 0 & 4 & 2 & 4 & 8 & 10 & 1 & 4 & $w_{388}$ & N & can. \\
1926 & 0 & 0 & 0 & 0 & 0 & 0 & 1 & 2 & 0 & 1 & 0 & 4 & 2 & 4 & 8 & 11 & 1 & 4 & $w_{225}$ & N & can. \\
1927 & 0 & 0 & 0 & 0 & 0 & 0 & 1 & 2 & 0 & 1 & 0 & 4 & 2 & 4 & 8 & 12 & 1 & 4 & $w_{383}$ & N & can. \\
1928 & 0 & 0 & 0 & 0 & 0 & 0 & 1 & 2 & 0 & 1 & 0 & 4 & 2 & 4 & 8 & 13 & 1 & 4 & $w_{262}$ & N & can. \\
1929 & 0 & 0 & 0 & 0 & 0 & 0 & 1 & 2 & 0 & 1 & 0 & 4 & 2 & 4 & 8 & 14 & 1 & 4 & $w_{225}$ & N & can. \\
1930 & 0 & 0 & 0 & 0 & 0 & 0 & 1 & 2 & 0 & 1 & 0 & 4 & 2 & 4 & 8 & 15 & 1 & 4 & $w_{241}$ & N & can. \\
1931 & 0 & 0 & 0 & 0 & 0 & 0 & 1 & 2 & 0 & 1 & 0 & 4 & 2 & 5 & 3 & 7 & 16 & 4 & $w_{322}$ & N & can. \\
1932 & 0 & 0 & 0 & 0 & 0 & 0 & 1 & 2 & 0 & 1 & 0 & 4 & 2 & 5 & 3 & 8 & 1 & 4 & $w_{332}$ & N & can. \\
1933 & 0 & 0 & 0 & 0 & 0 & 0 & 1 & 2 & 0 & 1 & 0 & 4 & 2 & 5 & 5 & 6 & 8 & 4 & $w_{176}$ & N & can. \\
1934 & 0 & 0 & 0 & 0 & 0 & 0 & 1 & 2 & 0 & 1 & 0 & 4 & 2 & 5 & 5 & 7 & 16 & 4 & $w_{322}$ & N & can. \\
1935 & 0 & 0 & 0 & 0 & 0 & 0 & 1 & 2 & 0 & 1 & 0 & 4 & 2 & 5 & 5 & 8 & 1 & 4 & $w_{323}$ & N & can. \\
1936 & 0 & 0 & 0 & 0 & 0 & 0 & 1 & 2 & 0 & 1 & 0 & 4 & 2 & 5 & 6 & 7 & 16 & 3 & $w_{342}$ & N & can. \\
1937 & 0 & 0 & 0 & 0 & 0 & 0 & 1 & 2 & 0 & 1 & 0 & 4 & 2 & 5 & 6 & 8 & 1 & 4 & $w_{350}$ & N & can. \\
1938 & 0 & 0 & 0 & 0 & 0 & 0 & 1 & 2 & 0 & 1 & 0 & 4 & 2 & 5 & 7 & 6 & 16 & 3 & $w_{88}$ & N & can. \\
1939 & 0 & 0 & 0 & 0 & 0 & 0 & 1 & 2 & 0 & 1 & 0 & 4 & 2 & 5 & 7 & 8 & 1 & 4 & $w_{240}$ & N & can. \\
1940 & 0 & 0 & 0 & 0 & 0 & 0 & 1 & 2 & 0 & 1 & 0 & 4 & 2 & 5 & 8 & 6 & 1 & 4 & $w_{240}$ & N & can. \\
1941 & 0 & 0 & 0 & 0 & 0 & 0 & 1 & 2 & 0 & 1 & 0 & 4 & 2 & 5 & 8 & 7 & 1 & 4 & $w_{350}$ & N & can. \\
1942 & 0 & 0 & 0 & 0 & 0 & 0 & 1 & 2 & 0 & 1 & 0 & 4 & 2 & 5 & 8 & 9 & 1 & 3 & $w_{371}$ & N & can. \\
1943 & 0 & 0 & 0 & 0 & 0 & 0 & 1 & 2 & 0 & 1 & 0 & 4 & 2 & 5 & 8 & 10 & 1 & 4 & $w_{344}$ & N & can. \\
1944 & 0 & 0 & 0 & 0 & 0 & 0 & 1 & 2 & 0 & 1 & 0 & 4 & 2 & 5 & 8 & 11 & 1 & 4 & $w_{262}$ & N & can. \\
1945 & 0 & 0 & 0 & 0 & 0 & 0 & 1 & 2 & 0 & 1 & 0 & 4 & 2 & 5 & 8 & 12 & 1 & 4 & $w_{341}$ & N & can. \\
1946 & 0 & 0 & 0 & 0 & 0 & 0 & 1 & 2 & 0 & 1 & 0 & 4 & 2 & 5 & 8 & 13 & 1 & 4 & $w_{258}$ & N & can. \\
1947 & 0 & 0 & 0 & 0 & 0 & 0 & 1 & 2 & 0 & 1 & 0 & 4 & 2 & 5 & 8 & 14 & 1 & 4 & $w_{260}$ & N & can. \\
1948 & 0 & 0 & 0 & 0 & 0 & 0 & 1 & 2 & 0 & 1 & 0 & 4 & 2 & 5 & 8 & 15 & 1 & 4 & $w_{372}$ & N & can. \\
1949 & 0 & 0 & 0 & 0 & 0 & 0 & 1 & 2 & 0 & 1 & 0 & 4 & 2 & 6 & 3 & 4 & 32 & 4 & $w_{320}$ & N & can. \\
1950 & 0 & 0 & 0 & 0 & 0 & 0 & 1 & 2 & 0 & 1 & 0 & 4 & 2 & 6 & 3 & 5 & 8 & 4 & $w_{255}$ & N & can. \\
1951 & 0 & 0 & 0 & 0 & 0 & 0 & 1 & 2 & 0 & 1 & 0 & 4 & 2 & 6 & 3 & 8 & 1 & 4 & $w_{323}$ & N & can. \\
1952 & 0 & 0 & 0 & 0 & 0 & 0 & 1 & 2 & 0 & 1 & 0 & 4 & 2 & 6 & 5 & 7 & 16 & 3 & $w_{342}$ & N & can. \\
1953 & 0 & 0 & 0 & 0 & 0 & 0 & 1 & 2 & 0 & 1 & 0 & 4 & 2 & 6 & 5 & 8 & 2 & 4 & $w_{349}$ & N & can. \\
1954 & 0 & 0 & 0 & 0 & 0 & 0 & 1 & 2 & 0 & 1 & 0 & 4 & 2 & 6 & 7 & 5 & 8 & 3 & $w_{88}$ & N & can. \\
1955 & 0 & 0 & 0 & 0 & 0 & 0 & 1 & 2 & 0 & 1 & 0 & 4 & 2 & 6 & 7 & 8 & 1 & 4 & $w_{236}$ & N & can. \\
1956 & 0 & 0 & 0 & 0 & 0 & 0 & 1 & 2 & 0 & 1 & 0 & 4 & 2 & 6 & 8 & 4 & 4 & 4 & $w_{325}$ & N & can. \\
1957 & 0 & 0 & 0 & 0 & 0 & 0 & 1 & 2 & 0 & 1 & 0 & 4 & 2 & 6 & 8 & 5 & 1 & 4 & $w_{240}$ & N & can. \\
1958 & 0 & 0 & 0 & 0 & 0 & 0 & 1 & 2 & 0 & 1 & 0 & 4 & 2 & 6 & 8 & 7 & 1 & 4 & $w_{349}$ & N & can. \\
1959 & 0 & 0 & 0 & 0 & 0 & 0 & 1 & 2 & 0 & 1 & 0 & 4 & 2 & 6 & 8 & 9 & 1 & 4 & $w_{338}$ & N & can. \\
1960 & 0 & 0 & 0 & 0 & 0 & 0 & 1 & 2 & 0 & 1 & 0 & 4 & 2 & 6 & 8 & 10 & 1 & 3 & $w_{371}$ & N & can. \\
1961 & 0 & 0 & 0 & 0 & 0 & 0 & 1 & 2 & 0 & 1 & 0 & 4 & 2 & 6 & 8 & 11 & 1 & 4 & $w_{225}$ & N & can. \\
1962 & 0 & 0 & 0 & 0 & 0 & 0 & 1 & 2 & 0 & 1 & 0 & 4 & 2 & 6 & 8 & 12 & 1 & 4 & $w_{365}$ & N & can. \\
1963 & 0 & 0 & 0 & 0 & 0 & 0 & 1 & 2 & 0 & 1 & 0 & 4 & 2 & 6 & 8 & 13 & 1 & 4 & $w_{241}$ & N & can. \\
1964 & 0 & 0 & 0 & 0 & 0 & 0 & 1 & 2 & 0 & 1 & 0 & 4 & 2 & 6 & 8 & 14 & 1 & 4 & $w_{262}$ & N & can. \\
1965 & 0 & 0 & 0 & 0 & 0 & 0 & 1 & 2 & 0 & 1 & 0 & 4 & 2 & 6 & 8 & 15 & 1 & 4 & $w_{372}$ & N & can. \\
1966 & 0 & 0 & 0 & 0 & 0 & 0 & 1 & 2 & 0 & 1 & 0 & 4 & 2 & 7 & 3 & 5 & 16 & 4 & $w_{255}$ & N & can. \\
1967 & 0 & 0 & 0 & 0 & 0 & 0 & 1 & 2 & 0 & 1 & 0 & 4 & 2 & 7 & 3 & 8 & 1 & 4 & $w_{382}$ & N & can. \\
1968 & 0 & 0 & 0 & 0 & 0 & 0 & 1 & 2 & 0 & 1 & 0 & 4 & 2 & 7 & 5 & 8 & 1 & 4 & $w_{257}$ & N & can. \\
1969 & 0 & 0 & 0 & 0 & 0 & 0 & 1 & 2 & 0 & 1 & 0 & 4 & 2 & 7 & 6 & 5 & 32 & 3 & $w_{88}$ & N & can. \\
1970 & 0 & 0 & 0 & 0 & 0 & 0 & 1 & 2 & 0 & 1 & 0 & 4 & 2 & 7 & 6 & 8 & 1 & 4 & $w_{240}$ & N & can. \\
1971 & 0 & 0 & 0 & 0 & 0 & 0 & 1 & 2 & 0 & 1 & 0 & 4 & 2 & 7 & 7 & 4 & 16 & 3 & $w_{88}$ & N & can. \\
1972 & 0 & 0 & 0 & 0 & 0 & 0 & 1 & 2 & 0 & 1 & 0 & 4 & 2 & 7 & 7 & 5 & 8 & 4 & $w_{255}$ & N & can. \\
1973 & 0 & 0 & 0 & 0 & 0 & 0 & 1 & 2 & 0 & 1 & 0 & 4 & 2 & 7 & 7 & 8 & 1 & 4 & $w_{240}$ & N & can. \\
1974 & 0 & 0 & 0 & 0 & 0 & 0 & 1 & 2 & 0 & 1 & 0 & 4 & 2 & 7 & 8 & 4 & 2 & 4 & $w_{257}$ & N & can. \\
1975 & 0 & 0 & 0 & 0 & 0 & 0 & 1 & 2 & 0 & 1 & 0 & 4 & 2 & 7 & 8 & 5 & 1 & 4 & $w_{257}$ & N & can. \\
1976 & 0 & 0 & 0 & 0 & 0 & 0 & 1 & 2 & 0 & 1 & 0 & 4 & 2 & 7 & 8 & 6 & 1 & 4 & $w_{240}$ & N & can. \\
1977 & 0 & 0 & 0 & 0 & 0 & 0 & 1 & 2 & 0 & 1 & 0 & 4 & 2 & 7 & 8 & 9 & 1 & 4 & $w_{258}$ & N & can. \\
1978 & 0 & 0 & 0 & 0 & 0 & 0 & 1 & 2 & 0 & 1 & 0 & 4 & 2 & 7 & 8 & 10 & 1 & 4 & $w_{351}$ & N & can. \\
1979 & 0 & 0 & 0 & 0 & 0 & 0 & 1 & 2 & 0 & 1 & 0 & 4 & 2 & 7 & 8 & 11 & 1 & 3 & $w_{259}$ & Y & can. \\
1980 & 0 & 0 & 0 & 0 & 0 & 0 & 1 & 2 & 0 & 1 & 0 & 4 & 2 & 7 & 8 & 12 & 1 & 4 & $w_{260}$ & N & can. \\
1981 & 0 & 0 & 0 & 0 & 0 & 0 & 1 & 2 & 0 & 1 & 0 & 4 & 2 & 7 & 8 & 13 & 1 & 4 & $w_{260}$ & N & can. \\
1982 & 0 & 0 & 0 & 0 & 0 & 0 & 1 & 2 & 0 & 1 & 0 & 4 & 2 & 7 & 8 & 14 & 1 & 4 & $w_{266}$ & N & can. \\
1983 & 0 & 0 & 0 & 0 & 0 & 0 & 1 & 2 & 0 & 1 & 0 & 4 & 2 & 7 & 8 & 15 & 1 & 4 & $w_{266}$ & N & can. \\
1984 & 0 & 0 & 0 & 0 & 0 & 0 & 1 & 2 & 0 & 1 & 0 & 4 & 2 & 8 & 3 & 9 & 2 & 4 & $w_{323}$ & N & can. \\
1985 & 0 & 0 & 0 & 0 & 0 & 0 & 1 & 2 & 0 & 1 & 0 & 4 & 2 & 8 & 3 & 10 & 2 & 4 & $w_{332}$ & N & can. \\
1986 & 0 & 0 & 0 & 0 & 0 & 0 & 1 & 2 & 0 & 1 & 0 & 4 & 2 & 8 & 3 & 11 & 2 & 4 & $w_{406}$ & N & can. \\
1987 & 0 & 0 & 0 & 0 & 0 & 0 & 1 & 2 & 0 & 1 & 0 & 4 & 2 & 8 & 3 & 12 & 2 & 4 & $w_{410}$ & N & can. \\
1988 & 0 & 0 & 0 & 0 & 0 & 0 & 1 & 2 & 0 & 1 & 0 & 4 & 2 & 8 & 3 & 13 & 2 & 4 & $w_{383}$ & N & can. \\
1989 & 0 & 0 & 0 & 0 & 0 & 0 & 1 & 2 & 0 & 1 & 0 & 4 & 2 & 8 & 3 & 14 & 2 & 4 & $w_{388}$ & N & can. \\
1990 & 0 & 0 & 0 & 0 & 0 & 0 & 1 & 2 & 0 & 1 & 0 & 4 & 2 & 8 & 3 & 15 & 2 & 3 & $w_{371}$ & N & can. \\
1991 & 0 & 0 & 0 & 0 & 0 & 0 & 1 & 2 & 0 & 1 & 0 & 4 & 2 & 8 & 5 & 6 & 1 & 4 & $w_{235}$ & N & can. \\
1992 & 0 & 0 & 0 & 0 & 0 & 0 & 1 & 2 & 0 & 1 & 0 & 4 & 2 & 8 & 5 & 7 & 1 & 4 & $w_{343}$ & N & can. \\
1993 & 0 & 0 & 0 & 0 & 0 & 0 & 1 & 2 & 0 & 1 & 0 & 4 & 2 & 8 & 5 & 9 & 1 & 3 & $w_{371}$ & N & can. \\
1994 & 0 & 0 & 0 & 0 & 0 & 0 & 1 & 2 & 0 & 1 & 0 & 4 & 2 & 8 & 5 & 10 & 1 & 4 & $w_{341}$ & N & can. \\
1995 & 0 & 0 & 0 & 0 & 0 & 0 & 1 & 2 & 0 & 1 & 0 & 4 & 2 & 8 & 5 & 11 & 1 & 4 & $w_{258}$ & N & can. \\
1996 & 0 & 0 & 0 & 0 & 0 & 0 & 1 & 2 & 0 & 1 & 0 & 4 & 2 & 8 & 5 & 12 & 1 & 4 & $w_{344}$ & N & can. \\
1997 & 0 & 0 & 0 & 0 & 0 & 0 & 1 & 2 & 0 & 1 & 0 & 4 & 2 & 8 & 5 & 13 & 1 & 4 & $w_{262}$ & N & can. \\
1998 & 0 & 0 & 0 & 0 & 0 & 0 & 1 & 2 & 0 & 1 & 0 & 4 & 2 & 8 & 5 & 14 & 1 & 4 & $w_{260}$ & N & can. \\
1999 & 0 & 0 & 0 & 0 & 0 & 0 & 1 & 2 & 0 & 1 & 0 & 4 & 2 & 8 & 5 & 15 & 1 & 4 & $w_{372}$ & N & can. \\
2000 & 0 & 0 & 0 & 0 & 0 & 0 & 1 & 2 & 0 & 1 & 0 & 4 & 2 & 8 & 6 & 9 & 1 & 4 & $w_{341}$ & N & can. \\
2001 & 0 & 0 & 0 & 0 & 0 & 0 & 1 & 2 & 0 & 1 & 0 & 4 & 2 & 8 & 6 & 10 & 2 & 3 & $w_{371}$ & N & can. \\
2002 & 0 & 0 & 0 & 0 & 0 & 0 & 1 & 2 & 0 & 1 & 0 & 4 & 2 & 8 & 6 & 11 & 1 & 4 & $w_{258}$ & N & can. \\
2003 & 0 & 0 & 0 & 0 & 0 & 0 & 1 & 2 & 0 & 1 & 0 & 4 & 2 & 8 & 6 & 13 & 2 & 4 & $w_{241}$ & N & can. \\
2004 & 0 & 0 & 0 & 0 & 0 & 0 & 1 & 2 & 0 & 1 & 0 & 4 & 2 & 8 & 6 & 15 & 2 & 4 & $w_{352}$ & N & can. \\
2005 & 0 & 0 & 0 & 0 & 0 & 0 & 1 & 2 & 0 & 1 & 0 & 4 & 2 & 8 & 7 & 4 & 2 & 4 & $w_{235}$ & N & can. \\
2006 & 0 & 0 & 0 & 0 & 0 & 0 & 1 & 2 & 0 & 1 & 0 & 4 & 2 & 8 & 7 & 5 & 1 & 4 & $w_{257}$ & N & can. \\
2007 & 0 & 0 & 0 & 0 & 0 & 0 & 1 & 2 & 0 & 1 & 0 & 4 & 2 & 8 & 7 & 6 & 1 & 4 & $w_{235}$ & N & can. \\
2008 & 0 & 0 & 0 & 0 & 0 & 0 & 1 & 2 & 0 & 1 & 0 & 4 & 2 & 8 & 7 & 9 & 1 & 4 & $w_{262}$ & N & can. \\
2009 & 0 & 0 & 0 & 0 & 0 & 0 & 1 & 2 & 0 & 1 & 0 & 4 & 2 & 8 & 7 & 10 & 1 & 4 & $w_{262}$ & N & can. \\
2010 & 0 & 0 & 0 & 0 & 0 & 0 & 1 & 2 & 0 & 1 & 0 & 4 & 2 & 8 & 7 & 11 & 1 & 3 & $w_{259}$ & Y & \#1979 \\
2011 & 0 & 0 & 0 & 0 & 0 & 0 & 1 & 2 & 0 & 1 & 0 & 4 & 2 & 8 & 7 & 12 & 1 & 4 & $w_{260}$ & N & can. \\
2012 & 0 & 0 & 0 & 0 & 0 & 0 & 1 & 2 & 0 & 1 & 0 & 4 & 2 & 8 & 7 & 13 & 1 & 4 & $w_{226}$ & N & can. \\
2013 & 0 & 0 & 0 & 0 & 0 & 0 & 1 & 2 & 0 & 1 & 0 & 4 & 2 & 8 & 7 & 14 & 1 & 4 & $w_{260}$ & N & can. \\
2014 & 0 & 0 & 0 & 0 & 0 & 0 & 1 & 2 & 0 & 1 & 0 & 4 & 2 & 8 & 7 & 15 & 1 & 4 & $w_{260}$ & N & can. \\
2015 & 0 & 0 & 0 & 0 & 0 & 0 & 1 & 2 & 0 & 1 & 0 & 4 & 2 & 8 & 8 & 4 & 2 & 3 & $w_{328}$ & N & can. \\
2016 & 0 & 0 & 0 & 0 & 0 & 0 & 1 & 2 & 0 & 1 & 0 & 4 & 2 & 8 & 8 & 5 & 1 & 4 & $w_{383}$ & N & can. \\
2017 & 0 & 0 & 0 & 0 & 0 & 0 & 1 & 2 & 0 & 1 & 0 & 4 & 2 & 8 & 8 & 6 & 1 & 4 & $w_{201}$ & N & can. \\
2018 & 0 & 0 & 0 & 0 & 0 & 0 & 1 & 2 & 0 & 1 & 0 & 4 & 2 & 8 & 8 & 7 & 1 & 4 & $w_{341}$ & N & can. \\
2019 & 0 & 0 & 0 & 0 & 0 & 0 & 1 & 2 & 0 & 1 & 0 & 4 & 2 & 8 & 8 & 9 & 2 & 4 & $w_{323}$ & N & can. \\
2020 & 0 & 0 & 0 & 0 & 0 & 0 & 1 & 2 & 0 & 1 & 0 & 4 & 2 & 8 & 8 & 10 & 1 & 4 & $w_{332}$ & N & can. \\
2021 & 0 & 0 & 0 & 0 & 0 & 0 & 1 & 2 & 0 & 1 & 0 & 4 & 2 & 8 & 8 & 11 & 1 & 4 & $w_{240}$ & N & \#1973 \\
2022 & 0 & 0 & 0 & 0 & 0 & 0 & 1 & 2 & 0 & 1 & 0 & 4 & 2 & 8 & 8 & 12 & 2 & 4 & $w_{335}$ & N & can. \\
2023 & 0 & 0 & 0 & 0 & 0 & 0 & 1 & 2 & 0 & 1 & 0 & 4 & 2 & 8 & 8 & 13 & 1 & 4 & $w_{225}$ & N & can. \\
2024 & 0 & 0 & 0 & 0 & 0 & 0 & 1 & 2 & 0 & 1 & 0 & 4 & 2 & 8 & 8 & 14 & 1 & 4 & $w_{262}$ & N & can. \\
2025 & 0 & 0 & 0 & 0 & 0 & 0 & 1 & 2 & 0 & 1 & 0 & 4 & 2 & 8 & 8 & 15 & 1 & 4 & $w_{372}$ & N & can. \\
2026 & 0 & 0 & 0 & 0 & 0 & 0 & 1 & 2 & 0 & 1 & 0 & 4 & 2 & 8 & 9 & 4 & 2 & 4 & $w_{201}$ & N & can. \\
2027 & 0 & 0 & 0 & 0 & 0 & 0 & 1 & 2 & 0 & 1 & 0 & 4 & 2 & 8 & 9 & 5 & 1 & 3 & $w_{237}$ & N & can. \\
2028 & 0 & 0 & 0 & 0 & 0 & 0 & 1 & 2 & 0 & 1 & 0 & 4 & 2 & 8 & 9 & 6 & 1 & 4 & $w_{202}$ & N & can. \\
2029 & 0 & 0 & 0 & 0 & 0 & 0 & 1 & 2 & 0 & 1 & 0 & 4 & 2 & 8 & 9 & 7 & 1 & 4 & $w_{262}$ & N & can. \\
2030 & 0 & 0 & 0 & 0 & 0 & 0 & 1 & 2 & 0 & 1 & 0 & 4 & 2 & 8 & 9 & 10 & 1 & 4 & $w_{235}$ & N & can. \\
2031 & 0 & 0 & 0 & 0 & 0 & 0 & 1 & 2 & 0 & 1 & 0 & 4 & 2 & 8 & 9 & 11 & 1 & 4 & $w_{257}$ & N & can. \\
2032 & 0 & 0 & 0 & 0 & 0 & 0 & 1 & 2 & 0 & 1 & 0 & 4 & 2 & 8 & 9 & 12 & 1 & 4 & $w_{262}$ & N & can. \\
2033 & 0 & 0 & 0 & 0 & 0 & 0 & 1 & 2 & 0 & 1 & 0 & 4 & 2 & 8 & 9 & 13 & 1 & 4 & $w_{202}$ & N & can. \\
2034 & 0 & 0 & 0 & 0 & 0 & 0 & 1 & 2 & 0 & 1 & 0 & 4 & 2 & 8 & 9 & 14 & 1 & 4 & $w_{241}$ & N & can. \\
2035 & 0 & 0 & 0 & 0 & 0 & 0 & 1 & 2 & 0 & 1 & 0 & 4 & 2 & 8 & 9 & 15 & 1 & 4 & $w_{241}$ & N & can. \\
2036 & 0 & 0 & 0 & 0 & 0 & 0 & 1 & 2 & 0 & 1 & 0 & 4 & 2 & 8 & 10 & 4 & 1 & 4 & $w_{383}$ & N & can. \\
2037 & 0 & 0 & 0 & 0 & 0 & 0 & 1 & 2 & 0 & 1 & 0 & 4 & 2 & 8 & 10 & 5 & 1 & 4 & $w_{262}$ & N & can. \\
2038 & 0 & 0 & 0 & 0 & 0 & 0 & 1 & 2 & 0 & 1 & 0 & 4 & 2 & 8 & 10 & 6 & 1 & 3 & $w_{237}$ & N & can. \\
2039 & 0 & 0 & 0 & 0 & 0 & 0 & 1 & 2 & 0 & 1 & 0 & 4 & 2 & 8 & 10 & 7 & 1 & 4 & $w_{258}$ & N & can. \\
2040 & 0 & 0 & 0 & 0 & 0 & 0 & 1 & 2 & 0 & 1 & 0 & 4 & 2 & 8 & 10 & 9 & 1 & 4 & $w_{240}$ & N & can. \\
2041 & 0 & 0 & 0 & 0 & 0 & 0 & 1 & 2 & 0 & 1 & 0 & 4 & 2 & 8 & 10 & 11 & 1 & 4 & $w_{257}$ & N & can. \\
2042 & 0 & 0 & 0 & 0 & 0 & 0 & 1 & 2 & 0 & 1 & 0 & 4 & 2 & 8 & 10 & 12 & 1 & 4 & $w_{258}$ & N & can. \\
2043 & 0 & 0 & 0 & 0 & 0 & 0 & 1 & 2 & 0 & 1 & 0 & 4 & 2 & 8 & 10 & 13 & 1 & 4 & $w_{241}$ & N & can. \\
2044 & 0 & 0 & 0 & 0 & 0 & 0 & 1 & 2 & 0 & 1 & 0 & 4 & 2 & 8 & 10 & 14 & 1 & 4 & $w_{262}$ & N & can. \\
2045 & 0 & 0 & 0 & 0 & 0 & 0 & 1 & 2 & 0 & 1 & 0 & 4 & 2 & 8 & 10 & 15 & 1 & 4 & $w_{260}$ & N & can. \\
2046 & 0 & 0 & 0 & 0 & 0 & 0 & 1 & 2 & 0 & 1 & 0 & 4 & 2 & 8 & 11 & 4 & 1 & 4 & $w_{341}$ & N & can. \\
2047 & 0 & 0 & 0 & 0 & 0 & 0 & 1 & 2 & 0 & 1 & 0 & 4 & 2 & 8 & 11 & 5 & 1 & 4 & $w_{258}$ & N & can. \\
2048 & 0 & 0 & 0 & 0 & 0 & 0 & 1 & 2 & 0 & 1 & 0 & 4 & 2 & 8 & 11 & 6 & 1 & 4 & $w_{262}$ & N & can. \\
2049 & 0 & 0 & 0 & 0 & 0 & 0 & 1 & 2 & 0 & 1 & 0 & 4 & 2 & 8 & 11 & 7 & 1 & 3 & $w_{345}$ & N & can. \\
2050 & 0 & 0 & 0 & 0 & 0 & 0 & 1 & 2 & 0 & 1 & 0 & 4 & 2 & 8 & 11 & 9 & 1 & 4 & $w_{343}$ & N & can. \\
2051 & 0 & 0 & 0 & 0 & 0 & 0 & 1 & 2 & 0 & 1 & 0 & 4 & 2 & 8 & 11 & 10 & 1 & 4 & $w_{350}$ & N & can. \\
2052 & 0 & 0 & 0 & 0 & 0 & 0 & 1 & 2 & 0 & 1 & 0 & 4 & 2 & 8 & 11 & 12 & 1 & 4 & $w_{353}$ & N & can. \\
2053 & 0 & 0 & 0 & 0 & 0 & 0 & 1 & 2 & 0 & 1 & 0 & 4 & 2 & 8 & 11 & 13 & 1 & 4 & $w_{260}$ & N & can. \\
2054 & 0 & 0 & 0 & 0 & 0 & 0 & 1 & 2 & 0 & 1 & 0 & 4 & 2 & 8 & 11 & 14 & 1 & 4 & $w_{266}$ & N & can. \\
2055 & 0 & 0 & 0 & 0 & 0 & 0 & 1 & 2 & 0 & 1 & 0 & 4 & 2 & 8 & 11 & 15 & 1 & 4 & $w_{352}$ & N & can. \\
2056 & 0 & 0 & 0 & 0 & 0 & 0 & 1 & 2 & 0 & 1 & 0 & 4 & 2 & 8 & 12 & 5 & 1 & 4 & $w_{262}$ & N & can. \\
2057 & 0 & 0 & 0 & 0 & 0 & 0 & 1 & 2 & 0 & 1 & 0 & 4 & 2 & 8 & 12 & 6 & 1 & 4 & $w_{202}$ & N & can. \\
2058 & 0 & 0 & 0 & 0 & 0 & 0 & 1 & 2 & 0 & 1 & 0 & 4 & 2 & 8 & 12 & 7 & 1 & 4 & $w_{260}$ & N & can. \\
2059 & 0 & 0 & 0 & 0 & 0 & 0 & 1 & 2 & 0 & 1 & 0 & 4 & 2 & 8 & 12 & 9 & 1 & 4 & $w_{262}$ & N & can. \\
2060 & 0 & 0 & 0 & 0 & 0 & 0 & 1 & 2 & 0 & 1 & 0 & 4 & 2 & 8 & 12 & 10 & 1 & 4 & $w_{262}$ & N & can. \\
2061 & 0 & 0 & 0 & 0 & 0 & 0 & 1 & 2 & 0 & 1 & 0 & 4 & 2 & 8 & 12 & 11 & 1 & 4 & $w_{260}$ & N & can. \\
2062 & 0 & 0 & 0 & 0 & 0 & 0 & 1 & 2 & 0 & 1 & 0 & 4 & 2 & 8 & 12 & 13 & 1 & 4 & $w_{225}$ & N & can. \\
2063 & 0 & 0 & 0 & 0 & 0 & 0 & 1 & 2 & 0 & 1 & 0 & 4 & 2 & 8 & 12 & 14 & 1 & 4 & $w_{258}$ & N & can. \\
2064 & 0 & 0 & 0 & 0 & 0 & 0 & 1 & 2 & 0 & 1 & 0 & 4 & 2 & 8 & 12 & 15 & 1 & 4 & $w_{241}$ & N & can. \\
2065 & 0 & 0 & 0 & 0 & 0 & 0 & 1 & 2 & 0 & 1 & 0 & 4 & 2 & 8 & 13 & 5 & 1 & 4 & $w_{258}$ & N & can. \\
2066 & 0 & 0 & 0 & 0 & 0 & 0 & 1 & 2 & 0 & 1 & 0 & 4 & 2 & 8 & 13 & 6 & 1 & 4 & $w_{241}$ & N & can. \\
2067 & 0 & 0 & 0 & 0 & 0 & 0 & 1 & 2 & 0 & 1 & 0 & 4 & 2 & 8 & 13 & 7 & 1 & 4 & $w_{352}$ & N & can. \\
2068 & 0 & 0 & 0 & 0 & 0 & 0 & 1 & 2 & 0 & 1 & 0 & 4 & 2 & 8 & 13 & 9 & 1 & 4 & $w_{341}$ & N & can. \\
2069 & 0 & 0 & 0 & 0 & 0 & 0 & 1 & 2 & 0 & 1 & 0 & 4 & 2 & 8 & 13 & 10 & 1 & 4 & $w_{352}$ & N & can. \\
2070 & 0 & 0 & 0 & 0 & 0 & 0 & 1 & 2 & 0 & 1 & 0 & 4 & 2 & 8 & 13 & 11 & 1 & 4 & $w_{266}$ & N & can. \\
2071 & 0 & 0 & 0 & 0 & 0 & 0 & 1 & 2 & 0 & 1 & 0 & 4 & 2 & 8 & 13 & 12 & 1 & 4 & $w_{344}$ & N & can. \\
2072 & 0 & 0 & 0 & 0 & 0 & 0 & 1 & 2 & 0 & 1 & 0 & 4 & 2 & 8 & 13 & 14 & 1 & 4 & $w_{260}$ & N & can. \\
2073 & 0 & 0 & 0 & 0 & 0 & 0 & 1 & 2 & 0 & 1 & 0 & 4 & 2 & 8 & 13 & 15 & 1 & 4 & $w_{353}$ & N & can. \\
2074 & 0 & 0 & 0 & 0 & 0 & 0 & 1 & 2 & 0 & 1 & 0 & 4 & 2 & 8 & 14 & 4 & 2 & 4 & $w_{338}$ & N & can. \\
2075 & 0 & 0 & 0 & 0 & 0 & 0 & 1 & 2 & 0 & 1 & 0 & 4 & 2 & 8 & 14 & 5 & 1 & 4 & $w_{260}$ & N & can. \\
2076 & 0 & 0 & 0 & 0 & 0 & 0 & 1 & 2 & 0 & 1 & 0 & 4 & 2 & 8 & 14 & 6 & 1 & 4 & $w_{262}$ & N & can. \\
2077 & 0 & 0 & 0 & 0 & 0 & 0 & 1 & 2 & 0 & 1 & 0 & 4 & 2 & 8 & 14 & 7 & 1 & 4 & $w_{353}$ & N & can. \\
2078 & 0 & 0 & 0 & 0 & 0 & 0 & 1 & 2 & 0 & 1 & 0 & 4 & 2 & 8 & 14 & 9 & 1 & 4 & $w_{352}$ & N & can. \\
2079 & 0 & 0 & 0 & 0 & 0 & 0 & 1 & 2 & 0 & 1 & 0 & 4 & 2 & 8 & 14 & 10 & 1 & 4 & $w_{341}$ & N & can. \\
2080 & 0 & 0 & 0 & 0 & 0 & 0 & 1 & 2 & 0 & 1 & 0 & 4 & 2 & 8 & 14 & 11 & 1 & 4 & $w_{266}$ & N & can. \\
2081 & 0 & 0 & 0 & 0 & 0 & 0 & 1 & 2 & 0 & 1 & 0 & 4 & 2 & 8 & 14 & 12 & 1 & 4 & $w_{409}$ & N & can. \\
2082 & 0 & 0 & 0 & 0 & 0 & 0 & 1 & 2 & 0 & 1 & 0 & 4 & 2 & 8 & 14 & 13 & 1 & 4 & $w_{241}$ & N & can. \\
2083 & 0 & 0 & 0 & 0 & 0 & 0 & 1 & 2 & 0 & 1 & 0 & 4 & 2 & 8 & 14 & 15 & 1 & 4 & $w_{352}$ & N & can. \\
2084 & 0 & 0 & 0 & 0 & 0 & 0 & 1 & 2 & 0 & 1 & 0 & 4 & 2 & 8 & 15 & 4 & 2 & 4 & $w_{241}$ & N & can. \\
2085 & 0 & 0 & 0 & 0 & 0 & 0 & 1 & 2 & 0 & 1 & 0 & 4 & 2 & 8 & 15 & 5 & 1 & 4 & $w_{241}$ & N & can. \\
2086 & 0 & 0 & 0 & 0 & 0 & 0 & 1 & 2 & 0 & 1 & 0 & 4 & 2 & 8 & 15 & 6 & 1 & 4 & $w_{241}$ & N & can. \\
2087 & 0 & 0 & 0 & 0 & 0 & 0 & 1 & 2 & 0 & 1 & 0 & 4 & 2 & 8 & 15 & 7 & 1 & 4 & $w_{266}$ & N & can. \\
2088 & 0 & 0 & 0 & 0 & 0 & 0 & 1 & 2 & 0 & 1 & 0 & 4 & 2 & 8 & 15 & 9 & 1 & 4 & $w_{260}$ & N & can. \\
2089 & 0 & 0 & 0 & 0 & 0 & 0 & 1 & 2 & 0 & 1 & 0 & 4 & 2 & 8 & 15 & 10 & 1 & 4 & $w_{260}$ & N & can. \\
2090 & 0 & 0 & 0 & 0 & 0 & 0 & 1 & 2 & 0 & 1 & 0 & 4 & 2 & 8 & 15 & 11 & 1 & 4 & $w_{260}$ & N & \#1980 \\
2091 & 0 & 0 & 0 & 0 & 0 & 0 & 1 & 2 & 0 & 1 & 0 & 4 & 2 & 8 & 15 & 12 & 1 & 4 & $w_{260}$ & N & \#2011 \\
2092 & 0 & 0 & 0 & 0 & 0 & 0 & 1 & 2 & 0 & 1 & 0 & 4 & 2 & 8 & 15 & 13 & 1 & 4 & $w_{260}$ & N & can. \\
2093 & 0 & 0 & 0 & 0 & 0 & 0 & 1 & 2 & 0 & 1 & 0 & 4 & 2 & 8 & 15 & 14 & 1 & 4 & $w_{260}$ & N & can. \\
2094 & 0 & 0 & 0 & 0 & 0 & 0 & 1 & 2 & 0 & 1 & 0 & 4 & 3 & 5 & 5 & 6 & 8 & 4 & $w_{255}$ & N & can. \\
2095 & 0 & 0 & 0 & 0 & 0 & 0 & 1 & 2 & 0 & 1 & 0 & 4 & 3 & 5 & 5 & 8 & 1 & 4 & $w_{382}$ & N & can. \\
2096 & 0 & 0 & 0 & 0 & 0 & 0 & 1 & 2 & 0 & 1 & 0 & 4 & 3 & 5 & 6 & 8 & 1 & 4 & $w_{257}$ & N & can. \\
2097 & 0 & 0 & 0 & 0 & 0 & 0 & 1 & 2 & 0 & 1 & 0 & 4 & 3 & 5 & 7 & 6 & 8 & 4 & $w_{256}$ & N & can. \\
2098 & 0 & 0 & 0 & 0 & 0 & 0 & 1 & 2 & 0 & 1 & 0 & 4 & 3 & 5 & 7 & 8 & 1 & 4 & $w_{411}$ & N & can. \\
2099 & 0 & 0 & 0 & 0 & 0 & 0 & 1 & 2 & 0 & 1 & 0 & 4 & 3 & 5 & 8 & 6 & 1 & 4 & $w_{411}$ & N & can. \\
2100 & 0 & 0 & 0 & 0 & 0 & 0 & 1 & 2 & 0 & 1 & 0 & 4 & 3 & 5 & 8 & 7 & 1 & 4 & $w_{257}$ & N & can. \\
2101 & 0 & 0 & 0 & 0 & 0 & 0 & 1 & 2 & 0 & 1 & 0 & 4 & 3 & 5 & 8 & 9 & 1 & 4 & $w_{408}$ & N & can. \\
2102 & 0 & 0 & 0 & 0 & 0 & 0 & 1 & 2 & 0 & 1 & 0 & 4 & 3 & 5 & 8 & 10 & 1 & 4 & $w_{351}$ & N & can. \\
2103 & 0 & 0 & 0 & 0 & 0 & 0 & 1 & 2 & 0 & 1 & 0 & 4 & 3 & 5 & 8 & 11 & 1 & 4 & $w_{258}$ & N & can. \\
2104 & 0 & 0 & 0 & 0 & 0 & 0 & 1 & 2 & 0 & 1 & 0 & 4 & 3 & 5 & 8 & 12 & 1 & 4 & $w_{258}$ & N & can. \\
2105 & 0 & 0 & 0 & 0 & 0 & 0 & 1 & 2 & 0 & 1 & 0 & 4 & 3 & 5 & 8 & 13 & 1 & 4 & $w_{351}$ & N & can. \\
2106 & 0 & 0 & 0 & 0 & 0 & 0 & 1 & 2 & 0 & 1 & 0 & 4 & 3 & 5 & 8 & 14 & 1 & 4 & $w_{260}$ & N & can. \\
2107 & 0 & 0 & 0 & 0 & 0 & 0 & 1 & 2 & 0 & 1 & 0 & 4 & 3 & 5 & 8 & 15 & 1 & 4 & $w_{266}$ & N & can. \\
2108 & 0 & 0 & 0 & 0 & 0 & 0 & 1 & 2 & 0 & 1 & 0 & 4 & 3 & 6 & 6 & 5 & 32 & 3 & $w_{88}$ & N & can. \\
2109 & 0 & 0 & 0 & 0 & 0 & 0 & 1 & 2 & 0 & 1 & 0 & 4 & 3 & 6 & 6 & 8 & 1 & 4 & $w_{235}$ & N & can. \\
2110 & 0 & 0 & 0 & 0 & 0 & 0 & 1 & 2 & 0 & 1 & 0 & 4 & 3 & 6 & 7 & 5 & 16 & 4 & $w_{256}$ & N & can. \\
2111 & 0 & 0 & 0 & 0 & 0 & 0 & 1 & 2 & 0 & 1 & 0 & 4 & 3 & 6 & 7 & 8 & 1 & 4 & $w_{257}$ & N & can. \\
2112 & 0 & 0 & 0 & 0 & 0 & 0 & 1 & 2 & 0 & 1 & 0 & 4 & 3 & 6 & 8 & 3 & 1 & 4 & $w_{326}$ & N & can. \\
2113 & 0 & 0 & 0 & 0 & 0 & 0 & 1 & 2 & 0 & 1 & 0 & 4 & 3 & 6 & 8 & 5 & 2 & 4 & $w_{411}$ & N & can. \\
2114 & 0 & 0 & 0 & 0 & 0 & 0 & 1 & 2 & 0 & 1 & 0 & 4 & 3 & 6 & 8 & 7 & 1 & 4 & $w_{257}$ & N & can. \\
2115 & 0 & 0 & 0 & 0 & 0 & 0 & 1 & 2 & 0 & 1 & 0 & 4 & 3 & 6 & 8 & 9 & 1 & 4 & $w_{258}$ & N & can. \\
2116 & 0 & 0 & 0 & 0 & 0 & 0 & 1 & 2 & 0 & 1 & 0 & 4 & 3 & 6 & 8 & 10 & 1 & 4 & $w_{351}$ & N & can. \\
2117 & 0 & 0 & 0 & 0 & 0 & 0 & 1 & 2 & 0 & 1 & 0 & 4 & 3 & 6 & 8 & 11 & 1 & 3 & $w_{259}$ & Y & can. \\
2118 & 0 & 0 & 0 & 0 & 0 & 0 & 1 & 2 & 0 & 1 & 0 & 4 & 3 & 6 & 8 & 12 & 1 & 4 & $w_{260}$ & N & can. \\
2119 & 0 & 0 & 0 & 0 & 0 & 0 & 1 & 2 & 0 & 1 & 0 & 4 & 3 & 6 & 8 & 13 & 1 & 4 & $w_{260}$ & N & can. \\
2120 & 0 & 0 & 0 & 0 & 0 & 0 & 1 & 2 & 0 & 1 & 0 & 4 & 3 & 6 & 8 & 14 & 1 & 4 & $w_{266}$ & N & can. \\
2121 & 0 & 0 & 0 & 0 & 0 & 0 & 1 & 2 & 0 & 1 & 0 & 4 & 3 & 6 & 8 & 15 & 1 & 4 & $w_{266}$ & N & can. \\
2122 & 0 & 0 & 0 & 0 & 0 & 0 & 1 & 2 & 0 & 1 & 0 & 4 & 3 & 7 & 7 & 5 & 48 & 3 & $w_{342}$ & N & can. \\
2123 & 0 & 0 & 0 & 0 & 0 & 0 & 1 & 2 & 0 & 1 & 0 & 4 & 3 & 7 & 7 & 8 & 1 & 4 & $w_{350}$ & N & can. \\
2124 & 0 & 0 & 0 & 0 & 0 & 0 & 1 & 2 & 0 & 1 & 0 & 4 & 3 & 7 & 8 & 5 & 2 & 4 & $w_{350}$ & N & can. \\
2125 & 0 & 0 & 0 & 0 & 0 & 0 & 1 & 2 & 0 & 1 & 0 & 4 & 3 & 7 & 8 & 6 & 1 & 4 & $w_{350}$ & N & can. \\
2126 & 0 & 0 & 0 & 0 & 0 & 0 & 1 & 2 & 0 & 1 & 0 & 4 & 3 & 7 & 8 & 9 & 1 & 4 & $w_{344}$ & N & can. \\
2127 & 0 & 0 & 0 & 0 & 0 & 0 & 1 & 2 & 0 & 1 & 0 & 4 & 3 & 7 & 8 & 10 & 1 & 3 & $w_{345}$ & N & can. \\
2128 & 0 & 0 & 0 & 0 & 0 & 0 & 1 & 2 & 0 & 1 & 0 & 4 & 3 & 7 & 8 & 11 & 1 & 4 & $w_{258}$ & N & can. \\
2129 & 0 & 0 & 0 & 0 & 0 & 0 & 1 & 2 & 0 & 1 & 0 & 4 & 3 & 7 & 8 & 12 & 1 & 4 & $w_{372}$ & N & can. \\
2130 & 0 & 0 & 0 & 0 & 0 & 0 & 1 & 2 & 0 & 1 & 0 & 4 & 3 & 7 & 8 & 13 & 1 & 4 & $w_{266}$ & N & can. \\
2131 & 0 & 0 & 0 & 0 & 0 & 0 & 1 & 2 & 0 & 1 & 0 & 4 & 3 & 7 & 8 & 14 & 1 & 4 & $w_{266}$ & N & can. \\
2132 & 0 & 0 & 0 & 0 & 0 & 0 & 1 & 2 & 0 & 1 & 0 & 4 & 3 & 7 & 8 & 15 & 1 & 4 & $w_{353}$ & N & can. \\
2133 & 0 & 0 & 0 & 0 & 0 & 0 & 1 & 2 & 0 & 1 & 0 & 4 & 3 & 8 & 6 & 5 & 4 & 4 & $w_{240}$ & N & can. \\
2134 & 0 & 0 & 0 & 0 & 0 & 0 & 1 & 2 & 0 & 1 & 0 & 4 & 3 & 8 & 6 & 7 & 1 & 4 & $w_{240}$ & N & can. \\
2135 & 0 & 0 & 0 & 0 & 0 & 0 & 1 & 2 & 0 & 1 & 0 & 4 & 3 & 8 & 6 & 9 & 1 & 4 & $w_{262}$ & N & can. \\
2136 & 0 & 0 & 0 & 0 & 0 & 0 & 1 & 2 & 0 & 1 & 0 & 4 & 3 & 8 & 6 & 10 & 1 & 4 & $w_{262}$ & N & can. \\
2137 & 0 & 0 & 0 & 0 & 0 & 0 & 1 & 2 & 0 & 1 & 0 & 4 & 3 & 8 & 6 & 11 & 1 & 3 & $w_{259}$ & Y & can. \\
2138 & 0 & 0 & 0 & 0 & 0 & 0 & 1 & 2 & 0 & 1 & 0 & 4 & 3 & 8 & 6 & 12 & 1 & 4 & $w_{260}$ & N & can. \\
2139 & 0 & 0 & 0 & 0 & 0 & 0 & 1 & 2 & 0 & 1 & 0 & 4 & 3 & 8 & 6 & 13 & 1 & 4 & $w_{226}$ & N & can. \\
2140 & 0 & 0 & 0 & 0 & 0 & 0 & 1 & 2 & 0 & 1 & 0 & 4 & 3 & 8 & 6 & 14 & 1 & 4 & $w_{260}$ & N & can. \\
2141 & 0 & 0 & 0 & 0 & 0 & 0 & 1 & 2 & 0 & 1 & 0 & 4 & 3 & 8 & 6 & 15 & 1 & 4 & $w_{260}$ & N & can. \\
2142 & 0 & 0 & 0 & 0 & 0 & 0 & 1 & 2 & 0 & 1 & 0 & 4 & 3 & 8 & 7 & 5 & 2 & 4 & $w_{412}$ & N & can. \\
2143 & 0 & 0 & 0 & 0 & 0 & 0 & 1 & 2 & 0 & 1 & 0 & 4 & 3 & 8 & 7 & 6 & 1 & 4 & $w_{343}$ & N & can. \\
2144 & 0 & 0 & 0 & 0 & 0 & 0 & 1 & 2 & 0 & 1 & 0 & 4 & 3 & 8 & 7 & 9 & 1 & 4 & $w_{409}$ & N & can. \\
2145 & 0 & 0 & 0 & 0 & 0 & 0 & 1 & 2 & 0 & 1 & 0 & 4 & 3 & 8 & 7 & 10 & 1 & 3 & $w_{345}$ & N & \#2127 \\
2146 & 0 & 0 & 0 & 0 & 0 & 0 & 1 & 2 & 0 & 1 & 0 & 4 & 3 & 8 & 7 & 11 & 1 & 4 & $w_{413}$ & N & can. \\
2147 & 0 & 0 & 0 & 0 & 0 & 0 & 1 & 2 & 0 & 1 & 0 & 4 & 3 & 8 & 7 & 12 & 1 & 4 & $w_{353}$ & N & can. \\
2148 & 0 & 0 & 0 & 0 & 0 & 0 & 1 & 2 & 0 & 1 & 0 & 4 & 3 & 8 & 7 & 13 & 1 & 4 & $w_{266}$ & N & can. \\
2149 & 0 & 0 & 0 & 0 & 0 & 0 & 1 & 2 & 0 & 1 & 0 & 4 & 3 & 8 & 7 & 14 & 1 & 4 & $w_{392}$ & N & can. \\
2150 & 0 & 0 & 0 & 0 & 0 & 0 & 1 & 2 & 0 & 1 & 0 & 4 & 3 & 8 & 7 & 15 & 1 & 4 & $w_{347}$ & N & can. \\
2151 & 0 & 0 & 0 & 0 & 0 & 0 & 1 & 2 & 0 & 1 & 0 & 4 & 3 & 8 & 8 & 5 & 2 & 3 & $w_{371}$ & N & can. \\
2152 & 0 & 0 & 0 & 0 & 0 & 0 & 1 & 2 & 0 & 1 & 0 & 4 & 3 & 8 & 8 & 6 & 1 & 4 & $w_{341}$ & N & can. \\
2153 & 0 & 0 & 0 & 0 & 0 & 0 & 1 & 2 & 0 & 1 & 0 & 4 & 3 & 8 & 8 & 7 & 1 & 4 & $w_{262}$ & N & can. \\
2154 & 0 & 0 & 0 & 0 & 0 & 0 & 1 & 2 & 0 & 1 & 0 & 4 & 3 & 8 & 8 & 9 & 2 & 4 & $w_{332}$ & N & can. \\
2155 & 0 & 0 & 0 & 0 & 0 & 0 & 1 & 2 & 0 & 1 & 0 & 4 & 3 & 8 & 8 & 11 & 1 & 4 & $w_{257}$ & N & can. \\
2156 & 0 & 0 & 0 & 0 & 0 & 0 & 1 & 2 & 0 & 1 & 0 & 4 & 3 & 8 & 8 & 12 & 1 & 4 & $w_{341}$ & N & can. \\
2157 & 0 & 0 & 0 & 0 & 0 & 0 & 1 & 2 & 0 & 1 & 0 & 4 & 3 & 8 & 8 & 13 & 1 & 4 & $w_{262}$ & N & can. \\
2158 & 0 & 0 & 0 & 0 & 0 & 0 & 1 & 2 & 0 & 1 & 0 & 4 & 3 & 8 & 8 & 14 & 1 & 4 & $w_{260}$ & N & can. \\
2159 & 0 & 0 & 0 & 0 & 0 & 0 & 1 & 2 & 0 & 1 & 0 & 4 & 3 & 8 & 8 & 15 & 1 & 4 & $w_{352}$ & N & can. \\
2160 & 0 & 0 & 0 & 0 & 0 & 0 & 1 & 2 & 0 & 1 & 0 & 4 & 3 & 8 & 9 & 3 & 2 & 4 & $w_{326}$ & N & can. \\
2161 & 0 & 0 & 0 & 0 & 0 & 0 & 1 & 2 & 0 & 1 & 0 & 4 & 3 & 8 & 9 & 5 & 2 & 4 & $w_{388}$ & N & can. \\
2162 & 0 & 0 & 0 & 0 & 0 & 0 & 1 & 2 & 0 & 1 & 0 & 4 & 3 & 8 & 9 & 6 & 1 & 4 & $w_{262}$ & N & can. \\
2163 & 0 & 0 & 0 & 0 & 0 & 0 & 1 & 2 & 0 & 1 & 0 & 4 & 3 & 8 & 9 & 7 & 1 & 4 & $w_{262}$ & N & can. \\
2164 & 0 & 0 & 0 & 0 & 0 & 0 & 1 & 2 & 0 & 1 & 0 & 4 & 3 & 8 & 9 & 11 & 1 & 4 & $w_{411}$ & N & can. \\
2165 & 0 & 0 & 0 & 0 & 0 & 0 & 1 & 2 & 0 & 1 & 0 & 4 & 3 & 8 & 9 & 12 & 1 & 4 & $w_{258}$ & N & can. \\
2166 & 0 & 0 & 0 & 0 & 0 & 0 & 1 & 2 & 0 & 1 & 0 & 4 & 3 & 8 & 9 & 13 & 1 & 4 & $w_{225}$ & N & can. \\
2167 & 0 & 0 & 0 & 0 & 0 & 0 & 1 & 2 & 0 & 1 & 0 & 4 & 3 & 8 & 9 & 14 & 1 & 4 & $w_{260}$ & N & can. \\
2168 & 0 & 0 & 0 & 0 & 0 & 0 & 1 & 2 & 0 & 1 & 0 & 4 & 3 & 8 & 9 & 15 & 1 & 4 & $w_{260}$ & N & can. \\
2169 & 0 & 0 & 0 & 0 & 0 & 0 & 1 & 2 & 0 & 1 & 0 & 4 & 3 & 8 & 10 & 3 & 2 & 4 & $w_{382}$ & N & can. \\
2170 & 0 & 0 & 0 & 0 & 0 & 0 & 1 & 2 & 0 & 1 & 0 & 4 & 3 & 8 & 10 & 5 & 1 & 4 & $w_{258}$ & N & can. \\
2171 & 0 & 0 & 0 & 0 & 0 & 0 & 1 & 2 & 0 & 1 & 0 & 4 & 3 & 8 & 10 & 6 & 1 & 4 & $w_{258}$ & N & can. \\
2172 & 0 & 0 & 0 & 0 & 0 & 0 & 1 & 2 & 0 & 1 & 0 & 4 & 3 & 8 & 10 & 7 & 1 & 3 & $w_{259}$ & Y & can. \\
2173 & 0 & 0 & 0 & 0 & 0 & 0 & 1 & 2 & 0 & 1 & 0 & 4 & 3 & 8 & 10 & 9 & 1 & 4 & $w_{257}$ & N & can. \\
2174 & 0 & 0 & 0 & 0 & 0 & 0 & 1 & 2 & 0 & 1 & 0 & 4 & 3 & 8 & 10 & 11 & 1 & 4 & $w_{411}$ & N & can. \\
2175 & 0 & 0 & 0 & 0 & 0 & 0 & 1 & 2 & 0 & 1 & 0 & 4 & 3 & 8 & 10 & 12 & 1 & 4 & $w_{266}$ & N & can. \\
2176 & 0 & 0 & 0 & 0 & 0 & 0 & 1 & 2 & 0 & 1 & 0 & 4 & 3 & 8 & 10 & 13 & 1 & 4 & $w_{260}$ & N & can. \\
2177 & 0 & 0 & 0 & 0 & 0 & 0 & 1 & 2 & 0 & 1 & 0 & 4 & 3 & 8 & 10 & 14 & 1 & 4 & $w_{260}$ & N & can. \\
2178 & 0 & 0 & 0 & 0 & 0 & 0 & 1 & 2 & 0 & 1 & 0 & 4 & 3 & 8 & 10 & 15 & 1 & 4 & $w_{266}$ & N & can. \\
2179 & 0 & 0 & 0 & 0 & 0 & 0 & 1 & 2 & 0 & 1 & 0 & 4 & 3 & 8 & 11 & 3 & 2 & 4 & $w_{405}$ & N & can. \\
2180 & 0 & 0 & 0 & 0 & 0 & 0 & 1 & 2 & 0 & 1 & 0 & 4 & 3 & 8 & 11 & 5 & 1 & 4 & $w_{409}$ & N & can. \\
2181 & 0 & 0 & 0 & 0 & 0 & 0 & 1 & 2 & 0 & 1 & 0 & 4 & 3 & 8 & 11 & 6 & 1 & 3 & $w_{345}$ & N & can. \\
2182 & 0 & 0 & 0 & 0 & 0 & 0 & 1 & 2 & 0 & 1 & 0 & 4 & 3 & 8 & 11 & 7 & 1 & 4 & $w_{351}$ & N & can. \\
2183 & 0 & 0 & 0 & 0 & 0 & 0 & 1 & 2 & 0 & 1 & 0 & 4 & 3 & 8 & 11 & 9 & 1 & 4 & $w_{412}$ & N & can. \\
2184 & 0 & 0 & 0 & 0 & 0 & 0 & 1 & 2 & 0 & 1 & 0 & 4 & 3 & 8 & 11 & 10 & 1 & 4 & $w_{343}$ & N & can. \\
2185 & 0 & 0 & 0 & 0 & 0 & 0 & 1 & 2 & 0 & 1 & 0 & 4 & 3 & 8 & 11 & 12 & 1 & 4 & $w_{347}$ & N & can. \\
2186 & 0 & 0 & 0 & 0 & 0 & 0 & 1 & 2 & 0 & 1 & 0 & 4 & 3 & 8 & 11 & 13 & 1 & 4 & $w_{266}$ & N & can. \\
2187 & 0 & 0 & 0 & 0 & 0 & 0 & 1 & 2 & 0 & 1 & 0 & 4 & 3 & 8 & 11 & 14 & 1 & 4 & $w_{392}$ & N & can. \\
2188 & 0 & 0 & 0 & 0 & 0 & 0 & 1 & 2 & 0 & 1 & 0 & 4 & 3 & 8 & 11 & 15 & 1 & 4 & $w_{353}$ & N & can. \\
2189 & 0 & 0 & 0 & 0 & 0 & 0 & 1 & 2 & 0 & 1 & 0 & 4 & 3 & 8 & 12 & 3 & 2 & 4 & $w_{383}$ & N & can. \\
2190 & 0 & 0 & 0 & 0 & 0 & 0 & 1 & 2 & 0 & 1 & 0 & 4 & 3 & 8 & 12 & 6 & 1 & 4 & $w_{260}$ & N & can. \\
2191 & 0 & 0 & 0 & 0 & 0 & 0 & 1 & 2 & 0 & 1 & 0 & 4 & 3 & 8 & 12 & 7 & 1 & 4 & $w_{241}$ & N & can. \\
2192 & 0 & 0 & 0 & 0 & 0 & 0 & 1 & 2 & 0 & 1 & 0 & 4 & 3 & 8 & 12 & 9 & 1 & 4 & $w_{258}$ & N & can. \\
2193 & 0 & 0 & 0 & 0 & 0 & 0 & 1 & 2 & 0 & 1 & 0 & 4 & 3 & 8 & 12 & 10 & 1 & 4 & $w_{260}$ & N & can. \\
2194 & 0 & 0 & 0 & 0 & 0 & 0 & 1 & 2 & 0 & 1 & 0 & 4 & 3 & 8 & 12 & 11 & 1 & 4 & $w_{266}$ & N & can. \\
2195 & 0 & 0 & 0 & 0 & 0 & 0 & 1 & 2 & 0 & 1 & 0 & 4 & 3 & 8 & 12 & 13 & 1 & 4 & $w_{262}$ & N & can. \\
2196 & 0 & 0 & 0 & 0 & 0 & 0 & 1 & 2 & 0 & 1 & 0 & 4 & 3 & 8 & 12 & 14 & 1 & 4 & $w_{266}$ & N & can. \\
2197 & 0 & 0 & 0 & 0 & 0 & 0 & 1 & 2 & 0 & 1 & 0 & 4 & 3 & 8 & 12 & 15 & 1 & 4 & $w_{260}$ & N & can. \\
2198 & 0 & 0 & 0 & 0 & 0 & 0 & 1 & 2 & 0 & 1 & 0 & 4 & 3 & 8 & 13 & 3 & 2 & 4 & $w_{410}$ & N & can. \\
2199 & 0 & 0 & 0 & 0 & 0 & 0 & 1 & 2 & 0 & 1 & 0 & 4 & 3 & 8 & 13 & 6 & 1 & 4 & $w_{352}$ & N & can. \\
2200 & 0 & 0 & 0 & 0 & 0 & 0 & 1 & 2 & 0 & 1 & 0 & 4 & 3 & 8 & 13 & 7 & 1 & 4 & $w_{266}$ & N & can. \\
2201 & 0 & 0 & 0 & 0 & 0 & 0 & 1 & 2 & 0 & 1 & 0 & 4 & 3 & 8 & 13 & 9 & 1 & 4 & $w_{344}$ & N & can. \\
2202 & 0 & 0 & 0 & 0 & 0 & 0 & 1 & 2 & 0 & 1 & 0 & 4 & 3 & 8 & 13 & 10 & 1 & 4 & $w_{353}$ & N & can. \\
2203 & 0 & 0 & 0 & 0 & 0 & 0 & 1 & 2 & 0 & 1 & 0 & 4 & 3 & 8 & 13 & 11 & 1 & 4 & $w_{392}$ & N & can. \\
2204 & 0 & 0 & 0 & 0 & 0 & 0 & 1 & 2 & 0 & 1 & 0 & 4 & 3 & 8 & 13 & 12 & 1 & 4 & $w_{409}$ & N & can. \\
2205 & 0 & 0 & 0 & 0 & 0 & 0 & 1 & 2 & 0 & 1 & 0 & 4 & 3 & 8 & 13 & 14 & 1 & 4 & $w_{266}$ & N & can. \\
2206 & 0 & 0 & 0 & 0 & 0 & 0 & 1 & 2 & 0 & 1 & 0 & 4 & 3 & 8 & 13 & 15 & 1 & 4 & $w_{347}$ & N & can. \\
2207 & 0 & 0 & 0 & 0 & 0 & 0 & 1 & 2 & 0 & 1 & 0 & 4 & 3 & 8 & 14 & 3 & 2 & 3 & $w_{371}$ & N & can. \\
2208 & 0 & 0 & 0 & 0 & 0 & 0 & 1 & 2 & 0 & 1 & 0 & 4 & 3 & 8 & 14 & 5 & 2 & 4 & $w_{352}$ & N & can. \\
2209 & 0 & 0 & 0 & 0 & 0 & 0 & 1 & 2 & 0 & 1 & 0 & 4 & 3 & 8 & 14 & 6 & 1 & 4 & $w_{353}$ & N & can. \\
2210 & 0 & 0 & 0 & 0 & 0 & 0 & 1 & 2 & 0 & 1 & 0 & 4 & 3 & 8 & 14 & 7 & 1 & 4 & $w_{266}$ & N & can. \\
2211 & 0 & 0 & 0 & 0 & 0 & 0 & 1 & 2 & 0 & 1 & 0 & 4 & 3 & 8 & 14 & 9 & 1 & 4 & $w_{353}$ & N & can. \\
2212 & 0 & 0 & 0 & 0 & 0 & 0 & 1 & 2 & 0 & 1 & 0 & 4 & 3 & 8 & 14 & 10 & 1 & 4 & $w_{352}$ & N & can. \\
2213 & 0 & 0 & 0 & 0 & 0 & 0 & 1 & 2 & 0 & 1 & 0 & 4 & 3 & 8 & 14 & 11 & 1 & 4 & $w_{392}$ & N & can. \\
2214 & 0 & 0 & 0 & 0 & 0 & 0 & 1 & 2 & 0 & 1 & 0 & 4 & 3 & 8 & 14 & 12 & 1 & 4 & $w_{347}$ & N & can. \\
2215 & 0 & 0 & 0 & 0 & 0 & 0 & 1 & 2 & 0 & 1 & 0 & 4 & 3 & 8 & 14 & 13 & 1 & 4 & $w_{260}$ & N & can. \\
2216 & 0 & 0 & 0 & 0 & 0 & 0 & 1 & 2 & 0 & 1 & 0 & 4 & 3 & 8 & 14 & 15 & 1 & 4 & $w_{353}$ & N & can. \\
2217 & 0 & 0 & 0 & 0 & 0 & 0 & 1 & 2 & 0 & 1 & 0 & 4 & 3 & 8 & 15 & 3 & 2 & 4 & $w_{388}$ & N & can. \\
2218 & 0 & 0 & 0 & 0 & 0 & 0 & 1 & 2 & 0 & 1 & 0 & 4 & 3 & 8 & 15 & 5 & 2 & 4 & $w_{266}$ & N & can. \\
2219 & 0 & 0 & 0 & 0 & 0 & 0 & 1 & 2 & 0 & 1 & 0 & 4 & 3 & 8 & 15 & 6 & 1 & 4 & $w_{266}$ & N & can. \\
2220 & 0 & 0 & 0 & 0 & 0 & 0 & 1 & 2 & 0 & 1 & 0 & 4 & 3 & 8 & 15 & 7 & 1 & 4 & $w_{266}$ & N & can. \\
2221 & 0 & 0 & 0 & 0 & 0 & 0 & 1 & 2 & 0 & 1 & 0 & 4 & 3 & 8 & 15 & 9 & 1 & 4 & $w_{266}$ & N & can. \\
2222 & 0 & 0 & 0 & 0 & 0 & 0 & 1 & 2 & 0 & 1 & 0 & 4 & 3 & 8 & 15 & 10 & 1 & 4 & $w_{266}$ & N & \#2130 \\
2223 & 0 & 0 & 0 & 0 & 0 & 0 & 1 & 2 & 0 & 1 & 0 & 4 & 3 & 8 & 15 & 11 & 1 & 4 & $w_{266}$ & N & can. \\
2224 & 0 & 0 & 0 & 0 & 0 & 0 & 1 & 2 & 0 & 1 & 0 & 4 & 3 & 8 & 15 & 12 & 1 & 4 & $w_{266}$ & N & can. \\
2225 & 0 & 0 & 0 & 0 & 0 & 0 & 1 & 2 & 0 & 1 & 0 & 4 & 3 & 8 & 15 & 13 & 1 & 4 & $w_{266}$ & N & \#2148 \\
2226 & 0 & 0 & 0 & 0 & 0 & 0 & 1 & 2 & 0 & 1 & 0 & 4 & 3 & 8 & 15 & 14 & 1 & 4 & $w_{266}$ & N & can. \\
2227 & 0 & 0 & 0 & 0 & 0 & 0 & 1 & 2 & 0 & 1 & 0 & 4 & 6 & 7 & 8 & 9 & 1 & 4 & $w_{341}$ & N & can. \\
2228 & 0 & 0 & 0 & 0 & 0 & 0 & 1 & 2 & 0 & 1 & 0 & 4 & 6 & 7 & 8 & 10 & 1 & 4 & $w_{353}$ & N & can. \\
2229 & 0 & 0 & 0 & 0 & 0 & 0 & 1 & 2 & 0 & 1 & 0 & 4 & 6 & 7 & 8 & 11 & 1 & 4 & $w_{260}$ & N & can. \\
2230 & 0 & 0 & 0 & 0 & 0 & 0 & 1 & 2 & 0 & 1 & 0 & 4 & 6 & 7 & 8 & 12 & 1 & 4 & $w_{352}$ & N & can. \\
2231 & 0 & 0 & 0 & 0 & 0 & 0 & 1 & 2 & 0 & 1 & 0 & 4 & 6 & 7 & 8 & 13 & 1 & 4 & $w_{266}$ & N & can. \\
2232 & 0 & 0 & 0 & 0 & 0 & 0 & 1 & 2 & 0 & 1 & 0 & 4 & 6 & 7 & 8 & 14 & 1 & 4 & $w_{351}$ & N & can. \\
2233 & 0 & 0 & 0 & 0 & 0 & 0 & 1 & 2 & 0 & 1 & 0 & 4 & 6 & 7 & 8 & 15 & 1 & 3 & $w_{345}$ & N & can. \\
2234 & 0 & 0 & 0 & 0 & 0 & 0 & 1 & 2 & 0 & 1 & 0 & 4 & 6 & 8 & 8 & 7 & 2 & 4 & $w_{225}$ & N & can. \\
2235 & 0 & 0 & 0 & 0 & 0 & 0 & 1 & 2 & 0 & 1 & 0 & 4 & 6 & 8 & 8 & 9 & 6 & 4 & $w_{335}$ & N & can. \\
2236 & 0 & 0 & 0 & 0 & 0 & 0 & 1 & 2 & 0 & 1 & 0 & 4 & 6 & 8 & 8 & 11 & 1 & 4 & $w_{241}$ & N & can. \\
2237 & 0 & 0 & 0 & 0 & 0 & 0 & 1 & 2 & 0 & 1 & 0 & 4 & 6 & 8 & 8 & 14 & 2 & 4 & $w_{258}$ & N & can. \\
2238 & 0 & 0 & 0 & 0 & 0 & 0 & 1 & 2 & 0 & 1 & 0 & 4 & 6 & 8 & 8 & 15 & 2 & 4 & $w_{352}$ & N & can. \\
2239 & 0 & 0 & 0 & 0 & 0 & 0 & 1 & 2 & 0 & 1 & 0 & 4 & 6 & 8 & 9 & 6 & 4 & 4 & $w_{202}$ & N & can. \\
2240 & 0 & 0 & 0 & 0 & 0 & 0 & 1 & 2 & 0 & 1 & 0 & 4 & 6 & 8 & 9 & 10 & 1 & 4 & $w_{226}$ & N & can. \\
2241 & 0 & 0 & 0 & 0 & 0 & 0 & 1 & 2 & 0 & 1 & 0 & 4 & 6 & 8 & 9 & 11 & 1 & 4 & $w_{260}$ & N & can. \\
2242 & 0 & 0 & 0 & 0 & 0 & 0 & 1 & 2 & 0 & 1 & 0 & 4 & 6 & 8 & 9 & 14 & 1 & 4 & $w_{260}$ & N & can. \\
2243 & 0 & 0 & 0 & 0 & 0 & 0 & 1 & 2 & 0 & 1 & 0 & 4 & 6 & 8 & 10 & 6 & 2 & 4 & $w_{225}$ & N & can. \\
2244 & 0 & 0 & 0 & 0 & 0 & 0 & 1 & 2 & 0 & 1 & 0 & 4 & 6 & 8 & 10 & 7 & 1 & 4 & $w_{241}$ & N & can. \\
2245 & 0 & 0 & 0 & 0 & 0 & 0 & 1 & 2 & 0 & 1 & 0 & 4 & 6 & 8 & 10 & 9 & 1 & 4 & $w_{241}$ & N & can. \\
2246 & 0 & 0 & 0 & 0 & 0 & 0 & 1 & 2 & 0 & 1 & 0 & 4 & 6 & 8 & 10 & 11 & 1 & 4 & $w_{260}$ & N & can. \\
2247 & 0 & 0 & 0 & 0 & 0 & 0 & 1 & 2 & 0 & 1 & 0 & 4 & 6 & 8 & 10 & 12 & 1 & 4 & $w_{266}$ & N & can. \\
2248 & 0 & 0 & 0 & 0 & 0 & 0 & 1 & 2 & 0 & 1 & 0 & 4 & 6 & 8 & 10 & 13 & 1 & 4 & $w_{268}$ & N & can. \\
2249 & 0 & 0 & 0 & 0 & 0 & 0 & 1 & 2 & 0 & 1 & 0 & 4 & 6 & 8 & 10 & 14 & 1 & 4 & $w_{260}$ & N & can. \\
2250 & 0 & 0 & 0 & 0 & 0 & 0 & 1 & 2 & 0 & 1 & 0 & 4 & 6 & 8 & 10 & 15 & 1 & 4 & $w_{267}$ & N & can. \\
2251 & 0 & 0 & 0 & 0 & 0 & 0 & 1 & 2 & 0 & 1 & 0 & 4 & 6 & 8 & 11 & 6 & 2 & 4 & $w_{372}$ & N & can. \\
2252 & 0 & 0 & 0 & 0 & 0 & 0 & 1 & 2 & 0 & 1 & 0 & 4 & 6 & 8 & 11 & 7 & 1 & 4 & $w_{260}$ & N & can. \\
2253 & 0 & 0 & 0 & 0 & 0 & 0 & 1 & 2 & 0 & 1 & 0 & 4 & 6 & 8 & 11 & 9 & 1 & 4 & $w_{353}$ & N & can. \\
2254 & 0 & 0 & 0 & 0 & 0 & 0 & 1 & 2 & 0 & 1 & 0 & 4 & 6 & 8 & 11 & 10 & 1 & 4 & $w_{352}$ & N & can. \\
2255 & 0 & 0 & 0 & 0 & 0 & 0 & 1 & 2 & 0 & 1 & 0 & 4 & 6 & 8 & 11 & 12 & 1 & 4 & $w_{354}$ & N & can. \\
2256 & 0 & 0 & 0 & 0 & 0 & 0 & 1 & 2 & 0 & 1 & 0 & 4 & 6 & 8 & 11 & 13 & 1 & 4 & $w_{267}$ & N & can. \\
2257 & 0 & 0 & 0 & 0 & 0 & 0 & 1 & 2 & 0 & 1 & 0 & 4 & 6 & 8 & 11 & 14 & 1 & 4 & $w_{281}$ & N & can. \\
2258 & 0 & 0 & 0 & 0 & 0 & 0 & 1 & 2 & 0 & 1 & 0 & 4 & 6 & 8 & 11 & 15 & 1 & 4 & $w_{373}$ & N & can. \\
2259 & 0 & 0 & 0 & 0 & 0 & 0 & 1 & 2 & 0 & 1 & 0 & 4 & 6 & 8 & 14 & 6 & 4 & 3 & $w_{371}$ & N & can. \\
2260 & 0 & 0 & 0 & 0 & 0 & 0 & 1 & 2 & 0 & 1 & 0 & 4 & 6 & 8 & 14 & 7 & 2 & 4 & $w_{258}$ & N & can. \\
2261 & 0 & 0 & 0 & 0 & 0 & 0 & 1 & 2 & 0 & 1 & 0 & 4 & 6 & 8 & 14 & 9 & 1 & 4 & $w_{353}$ & N & can. \\
2262 & 0 & 0 & 0 & 0 & 0 & 0 & 1 & 2 & 0 & 1 & 0 & 4 & 6 & 8 & 14 & 10 & 2 & 4 & $w_{352}$ & N & can. \\
2263 & 0 & 0 & 0 & 0 & 0 & 0 & 1 & 2 & 0 & 1 & 0 & 4 & 6 & 8 & 14 & 11 & 2 & 4 & $w_{281}$ & N & can. \\
2264 & 0 & 0 & 0 & 0 & 0 & 0 & 1 & 2 & 0 & 1 & 0 & 4 & 6 & 8 & 14 & 12 & 2 & 4 & $w_{347}$ & N & can. \\
2265 & 0 & 0 & 0 & 0 & 0 & 0 & 1 & 2 & 0 & 1 & 0 & 4 & 6 & 8 & 14 & 13 & 2 & 4 & $w_{268}$ & N & can. \\
2266 & 0 & 0 & 0 & 0 & 0 & 0 & 1 & 2 & 0 & 1 & 0 & 4 & 6 & 8 & 15 & 6 & 4 & 4 & $w_{262}$ & N & can. \\
2267 & 0 & 0 & 0 & 0 & 0 & 0 & 1 & 2 & 0 & 1 & 0 & 4 & 6 & 8 & 15 & 7 & 2 & 3 & $w_{259}$ & N & can. \\
2268 & 0 & 0 & 0 & 0 & 0 & 0 & 1 & 2 & 0 & 1 & 0 & 4 & 6 & 8 & 15 & 9 & 1 & 4 & $w_{266}$ & N & can. \\
2269 & 0 & 0 & 0 & 0 & 0 & 0 & 1 & 2 & 0 & 1 & 0 & 4 & 6 & 8 & 15 & 10 & 1 & 4 & $w_{267}$ & N & can. \\
2270 & 0 & 0 & 0 & 0 & 0 & 0 & 1 & 2 & 0 & 1 & 0 & 4 & 6 & 8 & 15 & 12 & 1 & 4 & $w_{267}$ & N & can. \\
2271 & 0 & 0 & 0 & 0 & 0 & 0 & 1 & 2 & 0 & 1 & 0 & 4 & 7 & 8 & 8 & 9 & 6 & 4 & $w_{410}$ & N & can. \\
2272 & 0 & 0 & 0 & 0 & 0 & 0 & 1 & 2 & 0 & 1 & 0 & 4 & 7 & 8 & 8 & 10 & 2 & 4 & $w_{352}$ & N & can. \\
2273 & 0 & 0 & 0 & 0 & 0 & 0 & 1 & 2 & 0 & 1 & 0 & 4 & 7 & 8 & 8 & 11 & 1 & 4 & $w_{260}$ & N & can. \\
2274 & 0 & 0 & 0 & 0 & 0 & 0 & 1 & 2 & 0 & 1 & 0 & 4 & 7 & 8 & 8 & 14 & 2 & 4 & $w_{266}$ & N & can. \\
2275 & 0 & 0 & 0 & 0 & 0 & 0 & 1 & 2 & 0 & 1 & 0 & 4 & 7 & 8 & 8 & 15 & 2 & 4 & $w_{353}$ & N & can. \\
2276 & 0 & 0 & 0 & 0 & 0 & 0 & 1 & 2 & 0 & 1 & 0 & 4 & 7 & 8 & 9 & 7 & 4 & 4 & $w_{262}$ & N & \#1829 \\
2277 & 0 & 0 & 0 & 0 & 0 & 0 & 1 & 2 & 0 & 1 & 0 & 4 & 7 & 8 & 9 & 10 & 1 & 4 & $w_{241}$ & N & \#1825 \\
2278 & 0 & 0 & 0 & 0 & 0 & 0 & 1 & 2 & 0 & 1 & 0 & 4 & 7 & 8 & 9 & 11 & 1 & 4 & $w_{266}$ & N & \#1822 \\
2279 & 0 & 0 & 0 & 0 & 0 & 0 & 1 & 2 & 0 & 1 & 0 & 4 & 7 & 8 & 9 & 14 & 1 & 4 & $w_{266}$ & N & can. \\
2280 & 0 & 0 & 0 & 0 & 0 & 0 & 1 & 2 & 0 & 1 & 0 & 4 & 7 & 8 & 10 & 7 & 2 & 4 & $w_{260}$ & N & can. \\
2281 & 0 & 0 & 0 & 0 & 0 & 0 & 1 & 2 & 0 & 1 & 0 & 4 & 7 & 8 & 10 & 9 & 1 & 4 & $w_{260}$ & N & can. \\
2282 & 0 & 0 & 0 & 0 & 0 & 0 & 1 & 2 & 0 & 1 & 0 & 4 & 7 & 8 & 10 & 11 & 1 & 4 & $w_{266}$ & N & can. \\
2283 & 0 & 0 & 0 & 0 & 0 & 0 & 1 & 2 & 0 & 1 & 0 & 4 & 7 & 8 & 10 & 12 & 1 & 4 & $w_{281}$ & N & can. \\
2284 & 0 & 0 & 0 & 0 & 0 & 0 & 1 & 2 & 0 & 1 & 0 & 4 & 7 & 8 & 10 & 13 & 1 & 4 & $w_{267}$ & N & can. \\
2285 & 0 & 0 & 0 & 0 & 0 & 0 & 1 & 2 & 0 & 1 & 0 & 4 & 7 & 8 & 10 & 14 & 1 & 4 & $w_{267}$ & N & can. \\
2286 & 0 & 0 & 0 & 0 & 0 & 0 & 1 & 2 & 0 & 1 & 0 & 4 & 7 & 8 & 10 & 15 & 1 & 4 & $w_{281}$ & N & can. \\
2287 & 0 & 0 & 0 & 0 & 0 & 0 & 1 & 2 & 0 & 1 & 0 & 4 & 7 & 8 & 11 & 7 & 2 & 4 & $w_{353}$ & N & can. \\
2288 & 0 & 0 & 0 & 0 & 0 & 0 & 1 & 2 & 0 & 1 & 0 & 4 & 7 & 8 & 11 & 9 & 1 & 4 & $w_{347}$ & N & can. \\
2289 & 0 & 0 & 0 & 0 & 0 & 0 & 1 & 2 & 0 & 1 & 0 & 4 & 7 & 8 & 11 & 10 & 1 & 4 & $w_{353}$ & N & can. \\
2290 & 0 & 0 & 0 & 0 & 0 & 0 & 1 & 2 & 0 & 1 & 0 & 4 & 7 & 8 & 11 & 12 & 1 & 4 & $w_{355}$ & N & can. \\
2291 & 0 & 0 & 0 & 0 & 0 & 0 & 1 & 2 & 0 & 1 & 0 & 4 & 7 & 8 & 11 & 13 & 1 & 4 & $w_{281}$ & N & can. \\
2292 & 0 & 0 & 0 & 0 & 0 & 0 & 1 & 2 & 0 & 1 & 0 & 4 & 7 & 8 & 11 & 14 & 1 & 4 & $w_{379}$ & N & can. \\
2293 & 0 & 0 & 0 & 0 & 0 & 0 & 1 & 2 & 0 & 1 & 0 & 4 & 7 & 8 & 11 & 15 & 1 & 4 & $w_{354}$ & N & can. \\
2294 & 0 & 0 & 0 & 0 & 0 & 0 & 1 & 2 & 0 & 1 & 0 & 4 & 7 & 8 & 14 & 7 & 4 & 3 & $w_{345}$ & N & can. \\
2295 & 0 & 0 & 0 & 0 & 0 & 0 & 1 & 2 & 0 & 1 & 0 & 4 & 7 & 8 & 14 & 9 & 1 & 4 & $w_{347}$ & N & can. \\
2296 & 0 & 0 & 0 & 0 & 0 & 0 & 1 & 2 & 0 & 1 & 0 & 4 & 7 & 8 & 14 & 10 & 2 & 4 & $w_{373}$ & N & can. \\
2297 & 0 & 0 & 0 & 0 & 0 & 0 & 1 & 2 & 0 & 1 & 0 & 4 & 7 & 8 & 14 & 11 & 2 & 4 & $w_{379}$ & N & can. \\
2298 & 0 & 0 & 0 & 0 & 0 & 0 & 1 & 2 & 0 & 1 & 0 & 4 & 7 & 8 & 14 & 12 & 2 & 4 & $w_{355}$ & N & can. \\
2299 & 0 & 0 & 0 & 0 & 0 & 0 & 1 & 2 & 0 & 1 & 0 & 4 & 7 & 8 & 14 & 13 & 2 & 4 & $w_{267}$ & N & can. \\
2300 & 0 & 0 & 0 & 0 & 0 & 0 & 1 & 2 & 0 & 1 & 0 & 4 & 7 & 8 & 15 & 7 & 4 & 4 & $w_{351}$ & N & can. \\
2301 & 0 & 0 & 0 & 0 & 0 & 0 & 1 & 2 & 0 & 1 & 0 & 4 & 7 & 8 & 15 & 9 & 1 & 4 & $w_{392}$ & N & can. \\
2302 & 0 & 0 & 0 & 0 & 0 & 0 & 1 & 2 & 0 & 1 & 0 & 4 & 7 & 8 & 15 & 10 & 1 & 4 & $w_{281}$ & N & can. \\
2303 & 0 & 0 & 0 & 0 & 0 & 0 & 1 & 2 & 0 & 1 & 0 & 4 & 7 & 8 & 15 & 12 & 1 & 4 & $w_{281}$ & N & can. \\
2304 & 0 & 0 & 0 & 0 & 0 & 0 & 1 & 2 & 0 & 1 & 0 & 4 & 8 & 9 & 9 & 14 & 6 & 3 & $w_{371}$ & N & can. \\
2305 & 0 & 0 & 0 & 0 & 0 & 0 & 1 & 2 & 0 & 1 & 0 & 4 & 8 & 9 & 9 & 15 & 6 & 4 & $w_{414}$ & N & can. \\
2306 & 0 & 0 & 0 & 0 & 0 & 0 & 1 & 2 & 0 & 1 & 0 & 4 & 8 & 9 & 10 & 11 & 8 & 4 & $w_{349}$ & N & can. \\
2307 & 0 & 0 & 0 & 0 & 0 & 0 & 1 & 2 & 0 & 1 & 0 & 4 & 8 & 9 & 10 & 12 & 1 & 4 & $w_{391}$ & N & can. \\
2308 & 0 & 0 & 0 & 0 & 0 & 0 & 1 & 2 & 0 & 1 & 0 & 4 & 8 & 9 & 10 & 13 & 1 & 3 & $w_{345}$ & N & can. \\
2309 & 0 & 0 & 0 & 0 & 0 & 0 & 1 & 2 & 0 & 1 & 0 & 4 & 8 & 9 & 10 & 14 & 1 & 4 & $w_{352}$ & N & can. \\
2310 & 0 & 0 & 0 & 0 & 0 & 0 & 1 & 2 & 0 & 1 & 0 & 4 & 8 & 9 & 10 & 15 & 1 & 4 & $w_{393}$ & N & can. \\
2311 & 0 & 0 & 0 & 0 & 0 & 0 & 1 & 2 & 0 & 1 & 0 & 4 & 8 & 9 & 11 & 10 & 4 & 4 & $w_{250}$ & N & can. \\
2312 & 0 & 0 & 0 & 0 & 0 & 0 & 1 & 2 & 0 & 1 & 0 & 4 & 8 & 9 & 11 & 12 & 1 & 3 & $w_{251}$ & N & can. \\
2313 & 0 & 0 & 0 & 0 & 0 & 0 & 1 & 2 & 0 & 1 & 0 & 4 & 8 & 9 & 11 & 13 & 1 & 4 & $w_{415}$ & N & can. \\
2314 & 0 & 0 & 0 & 0 & 0 & 0 & 1 & 2 & 0 & 1 & 0 & 4 & 8 & 9 & 11 & 14 & 1 & 4 & $w_{393}$ & N & can. \\
2315 & 0 & 0 & 0 & 0 & 0 & 0 & 1 & 2 & 0 & 1 & 0 & 4 & 8 & 9 & 11 & 15 & 1 & 4 & $w_{264}$ & N & can. \\
2316 & 0 & 0 & 0 & 0 & 0 & 0 & 1 & 2 & 0 & 1 & 0 & 4 & 8 & 9 & 12 & 13 & 4 & 4 & $w_{343}$ & N & can. \\
2317 & 0 & 0 & 0 & 0 & 0 & 0 & 1 & 2 & 0 & 1 & 0 & 4 & 8 & 9 & 12 & 14 & 1 & 4 & $w_{347}$ & N & can. \\
2318 & 0 & 0 & 0 & 0 & 0 & 0 & 1 & 2 & 0 & 1 & 0 & 4 & 8 & 9 & 12 & 15 & 1 & 4 & $w_{393}$ & N & can. \\
2319 & 0 & 0 & 0 & 0 & 0 & 0 & 1 & 2 & 0 & 1 & 0 & 4 & 8 & 9 & 13 & 14 & 1 & 4 & $w_{254}$ & N & can. \\
2320 & 0 & 0 & 0 & 0 & 0 & 0 & 1 & 2 & 0 & 1 & 0 & 4 & 8 & 9 & 13 & 15 & 1 & 4 & $w_{264}$ & N & can. \\
2321 & 0 & 0 & 0 & 0 & 0 & 0 & 1 & 2 & 0 & 1 & 0 & 4 & 8 & 9 & 14 & 15 & 4 & 4 & $w_{264}$ & N & can. \\
2322 & 0 & 0 & 0 & 0 & 0 & 0 & 1 & 2 & 0 & 1 & 0 & 4 & 8 & 9 & 15 & 14 & 4 & 4 & $w_{352}$ & N & can. \\
2323 & 0 & 0 & 0 & 0 & 0 & 0 & 1 & 2 & 0 & 1 & 0 & 4 & 8 & 10 & 10 & 12 & 1 & 4 & $w_{391}$ & N & can. \\
2324 & 0 & 0 & 0 & 0 & 0 & 0 & 1 & 2 & 0 & 1 & 0 & 4 & 8 & 10 & 10 & 13 & 2 & 4 & $w_{352}$ & N & can. \\
2325 & 0 & 0 & 0 & 0 & 0 & 0 & 1 & 2 & 0 & 1 & 0 & 4 & 8 & 10 & 10 & 14 & 6 & 3 & $w_{371}$ & N & can. \\
2326 & 0 & 0 & 0 & 0 & 0 & 0 & 1 & 2 & 0 & 1 & 0 & 4 & 8 & 10 & 11 & 12 & 1 & 4 & $w_{264}$ & N & can. \\
2327 & 0 & 0 & 0 & 0 & 0 & 0 & 1 & 2 & 0 & 1 & 0 & 4 & 8 & 10 & 11 & 13 & 1 & 4 & $w_{393}$ & N & can. \\
2328 & 0 & 0 & 0 & 0 & 0 & 0 & 1 & 2 & 0 & 1 & 0 & 4 & 8 & 10 & 11 & 14 & 2 & 4 & $w_{415}$ & N & can. \\
2329 & 0 & 0 & 0 & 0 & 0 & 0 & 1 & 2 & 0 & 1 & 0 & 4 & 8 & 10 & 11 & 15 & 1 & 3 & $w_{251}$ & Y & \#1820 \\
2330 & 0 & 0 & 0 & 0 & 0 & 0 & 1 & 2 & 0 & 1 & 0 & 4 & 8 & 10 & 12 & 14 & 8 & 4 & $w_{347}$ & N & can. \\
2331 & 0 & 0 & 0 & 0 & 0 & 0 & 1 & 2 & 0 & 1 & 0 & 4 & 8 & 10 & 12 & 15 & 2 & 4 & $w_{374}$ & N & can. \\
2332 & 0 & 0 & 0 & 0 & 0 & 0 & 1 & 2 & 0 & 1 & 0 & 4 & 8 & 10 & 13 & 15 & 4 & 4 & $w_{265}$ & N & can. \\
2333 & 0 & 0 & 0 & 0 & 0 & 0 & 1 & 2 & 0 & 1 & 0 & 4 & 8 & 10 & 14 & 12 & 4 & 4 & $w_{264}$ & N & can. \\
2334 & 0 & 0 & 0 & 0 & 0 & 0 & 1 & 2 & 0 & 1 & 0 & 4 & 8 & 10 & 14 & 13 & 1 & 4 & $w_{374}$ & N & can. \\
2335 & 0 & 0 & 0 & 0 & 0 & 0 & 1 & 2 & 0 & 1 & 0 & 4 & 8 & 10 & 15 & 13 & 4 & 4 & $w_{354}$ & N & can. \\
2336 & 0 & 0 & 0 & 0 & 0 & 0 & 1 & 2 & 0 & 1 & 0 & 4 & 8 & 11 & 10 & 12 & 1 & 4 & $w_{239}$ & N & can. \\
2337 & 0 & 0 & 0 & 0 & 0 & 0 & 1 & 2 & 0 & 1 & 0 & 4 & 8 & 11 & 10 & 13 & 1 & 4 & $w_{254}$ & N & can. \\
2338 & 0 & 0 & 0 & 0 & 0 & 0 & 1 & 2 & 0 & 1 & 0 & 4 & 8 & 11 & 10 & 14 & 2 & 4 & $w_{230}$ & N & can. \\
2339 & 0 & 0 & 0 & 0 & 0 & 0 & 1 & 2 & 0 & 1 & 0 & 4 & 8 & 11 & 11 & 12 & 1 & 4 & $w_{254}$ & N & can. \\
2340 & 0 & 0 & 0 & 0 & 0 & 0 & 1 & 2 & 0 & 1 & 0 & 4 & 8 & 11 & 11 & 13 & 1 & 4 & $w_{264}$ & N & can. \\
2341 & 0 & 0 & 0 & 0 & 0 & 0 & 1 & 2 & 0 & 1 & 0 & 4 & 8 & 11 & 11 & 14 & 2 & 3 & $w_{251}$ & Y & can. \\
2342 & 0 & 0 & 0 & 0 & 0 & 0 & 1 & 2 & 0 & 1 & 0 & 4 & 8 & 11 & 13 & 14 & 16 & 4 & $w_{243}$ & N & can. \\
2343 & 0 & 0 & 0 & 0 & 0 & 0 & 1 & 2 & 0 & 1 & 0 & 4 & 8 & 11 & 14 & 13 & 4 & 4 & $w_{265}$ & N & can. \\
2344 & 0 & 0 & 0 & 0 & 0 & 0 & 1 & 2 & 0 & 1 & 0 & 4 & 8 & 11 & 15 & 12 & 4 & 4 & $w_{243}$ & N & can. \\
2345 & 0 & 0 & 0 & 0 & 0 & 0 & 1 & 2 & 0 & 1 & 0 & 4 & 8 & 12 & 10 & 14 & 16 & 4 & $w_{370}$ & N & can. \\
2346 & 0 & 0 & 0 & 0 & 0 & 0 & 1 & 2 & 0 & 1 & 0 & 4 & 8 & 12 & 10 & 15 & 2 & 4 & $w_{374}$ & N & can. \\
2347 & 0 & 0 & 0 & 0 & 0 & 0 & 1 & 2 & 0 & 1 & 0 & 4 & 8 & 12 & 11 & 15 & 4 & 4 & $w_{265}$ & N & can. \\
2348 & 0 & 0 & 0 & 0 & 0 & 0 & 1 & 2 & 0 & 1 & 0 & 4 & 8 & 12 & 14 & 10 & 4 & 4 & $w_{239}$ & N & can. \\
2349 & 0 & 0 & 0 & 0 & 0 & 0 & 1 & 2 & 0 & 1 & 0 & 4 & 8 & 12 & 14 & 11 & 1 & 4 & $w_{374}$ & N & can. \\
2350 & 0 & 0 & 0 & 0 & 0 & 0 & 1 & 2 & 0 & 1 & 0 & 4 & 8 & 12 & 15 & 11 & 4 & 4 & $w_{356}$ & N & can. \\
2351 & 0 & 0 & 0 & 0 & 0 & 0 & 1 & 2 & 0 & 1 & 0 & 4 & 8 & 13 & 11 & 14 & 8 & 4 & $w_{278}$ & N & can. \\
2352 & 0 & 0 & 0 & 0 & 0 & 0 & 1 & 2 & 0 & 1 & 0 & 4 & 8 & 13 & 14 & 11 & 4 & 4 & $w_{278}$ & N & can. \\
2353 & 0 & 0 & 0 & 0 & 0 & 0 & 1 & 2 & 0 & 1 & 0 & 4 & 8 & 13 & 15 & 10 & 4 & 4 & $w_{265}$ & N & can. \\
2354 & 0 & 0 & 0 & 0 & 0 & 0 & 1 & 2 & 0 & 1 & 0 & 4 & 8 & 14 & 14 & 9 & 2 & 4 & $w_{254}$ & N & can. \\
2355 & 0 & 0 & 0 & 0 & 0 & 0 & 1 & 2 & 0 & 1 & 2 & 3 & 4 & 5 & 8 & 10 & 8 & 3 & $w_{345}$ & N & can. \\
2356 & 0 & 0 & 0 & 0 & 0 & 0 & 1 & 2 & 0 & 1 & 2 & 3 & 4 & 5 & 8 & 11 & 8 & 4 & $w_{263}$ & N & can. \\
2357 & 0 & 0 & 0 & 0 & 0 & 0 & 1 & 2 & 0 & 1 & 2 & 3 & 4 & 5 & 8 & 12 & 2 & 4 & $w_{353}$ & N & can. \\
2358 & 0 & 0 & 0 & 0 & 0 & 0 & 1 & 2 & 0 & 1 & 2 & 3 & 4 & 6 & 5 & 8 & 4 & 4 & $w_{276}$ & N & can. \\
2359 & 0 & 0 & 0 & 0 & 0 & 0 & 1 & 2 & 0 & 1 & 2 & 3 & 4 & 6 & 7 & 5 & 768 & 4 & $w_{416}$ & N & can. \\
2360 & 0 & 0 & 0 & 0 & 0 & 0 & 1 & 2 & 0 & 1 & 2 & 3 & 4 & 6 & 7 & 8 & 8 & 4 & $w_{417}$ & N & can. \\
2361 & 0 & 0 & 0 & 0 & 0 & 0 & 1 & 2 & 0 & 1 & 2 & 3 & 4 & 6 & 8 & 10 & 16 & 4 & $w_{418}$ & N & can. \\
2362 & 0 & 0 & 0 & 0 & 0 & 0 & 1 & 2 & 0 & 1 & 2 & 3 & 4 & 6 & 8 & 11 & 8 & 4 & $w_{346}$ & N & can. \\
2363 & 0 & 0 & 0 & 0 & 0 & 0 & 1 & 2 & 0 & 1 & 2 & 3 & 4 & 6 & 8 & 12 & 2 & 4 & $w_{419}$ & N & can. \\
2364 & 0 & 0 & 0 & 0 & 0 & 0 & 1 & 2 & 0 & 1 & 2 & 3 & 4 & 7 & 5 & 8 & 8 & 4 & $w_{276}$ & N & can. \\
2365 & 0 & 0 & 0 & 0 & 0 & 0 & 1 & 2 & 0 & 1 & 2 & 3 & 4 & 7 & 8 & 11 & 16 & 4 & $w_{263}$ & N & can. \\
2366 & 0 & 0 & 0 & 0 & 0 & 0 & 1 & 2 & 0 & 1 & 2 & 3 & 4 & 7 & 8 & 12 & 2 & 4 & $w_{392}$ & N & can. \\
2367 & 0 & 0 & 0 & 0 & 0 & 0 & 1 & 2 & 0 & 1 & 2 & 3 & 4 & 8 & 8 & 5 & 8 & 4 & $w_{263}$ & N & can. \\
2368 & 0 & 0 & 0 & 0 & 0 & 0 & 1 & 2 & 0 & 1 & 2 & 3 & 4 & 8 & 8 & 6 & 8 & 4 & $w_{263}$ & N & can. \\
2369 & 0 & 0 & 0 & 0 & 0 & 0 & 1 & 2 & 0 & 1 & 2 & 3 & 4 & 8 & 8 & 7 & 16 & 3 & $w_{251}$ & N & can. \\
2370 & 0 & 0 & 0 & 0 & 0 & 0 & 1 & 2 & 0 & 1 & 2 & 3 & 4 & 8 & 8 & 12 & 2 & 4 & $w_{266}$ & N & can. \\
2371 & 0 & 0 & 0 & 0 & 0 & 0 & 1 & 2 & 0 & 1 & 2 & 3 & 4 & 8 & 9 & 5 & 32 & 4 & $w_{344}$ & N & can. \\
2372 & 0 & 0 & 0 & 0 & 0 & 0 & 1 & 2 & 0 & 1 & 2 & 3 & 4 & 8 & 9 & 6 & 16 & 3 & $w_{345}$ & N & can. \\
2373 & 0 & 0 & 0 & 0 & 0 & 0 & 1 & 2 & 0 & 1 & 2 & 3 & 4 & 8 & 9 & 12 & 4 & 4 & $w_{347}$ & N & can. \\
2374 & 0 & 0 & 0 & 0 & 0 & 0 & 1 & 2 & 0 & 1 & 2 & 3 & 4 & 8 & 10 & 6 & 32 & 4 & $w_{344}$ & N & can. \\
2375 & 0 & 0 & 0 & 0 & 0 & 0 & 1 & 2 & 0 & 1 & 2 & 3 & 4 & 8 & 10 & 12 & 4 & 4 & $w_{347}$ & N & can. \\
2376 & 0 & 0 & 0 & 0 & 0 & 0 & 1 & 2 & 0 & 1 & 2 & 4 & 2 & 5 & 3 & 6 & 8 & 4 & $w_{255}$ & N & can. \\
2377 & 0 & 0 & 0 & 0 & 0 & 0 & 1 & 2 & 0 & 1 & 2 & 4 & 2 & 5 & 3 & 8 & 2 & 4 & $w_{382}$ & N & can. \\
2378 & 0 & 0 & 0 & 0 & 0 & 0 & 1 & 2 & 0 & 1 & 2 & 4 & 2 & 5 & 4 & 7 & 56 & 3 & $w_{88}$ & Y & can. \\
2379 & 0 & 0 & 0 & 0 & 0 & 0 & 1 & 2 & 0 & 1 & 2 & 4 & 2 & 5 & 4 & 8 & 1 & 4 & $w_{240}$ & N & can. \\
2380 & 0 & 0 & 0 & 0 & 0 & 0 & 1 & 2 & 0 & 1 & 2 & 4 & 2 & 5 & 8 & 4 & 1 & 4 & $w_{240}$ & N & can. \\
2381 & 0 & 0 & 0 & 0 & 0 & 0 & 1 & 2 & 0 & 1 & 2 & 4 & 2 & 5 & 8 & 6 & 1 & 4 & $w_{257}$ & N & can. \\
2382 & 0 & 0 & 0 & 0 & 0 & 0 & 1 & 2 & 0 & 1 & 2 & 4 & 2 & 5 & 8 & 9 & 2 & 4 & $w_{258}$ & N & can. \\
2383 & 0 & 0 & 0 & 0 & 0 & 0 & 1 & 2 & 0 & 1 & 2 & 4 & 2 & 5 & 8 & 10 & 2 & 4 & $w_{351}$ & N & can. \\
2384 & 0 & 0 & 0 & 0 & 0 & 0 & 1 & 2 & 0 & 1 & 2 & 4 & 2 & 5 & 8 & 11 & 2 & 3 & $w_{259}$ & Y & can. \\
2385 & 0 & 0 & 0 & 0 & 0 & 0 & 1 & 2 & 0 & 1 & 2 & 4 & 2 & 5 & 8 & 12 & 1 & 4 & $w_{266}$ & N & can. \\
2386 & 0 & 0 & 0 & 0 & 0 & 0 & 1 & 2 & 0 & 1 & 2 & 4 & 2 & 5 & 8 & 14 & 1 & 4 & $w_{260}$ & N & can. \\
2387 & 0 & 0 & 0 & 0 & 0 & 0 & 1 & 2 & 0 & 1 & 2 & 4 & 2 & 6 & 5 & 7 & 16 & 4 & $w_{256}$ & N & can. \\
2388 & 0 & 0 & 0 & 0 & 0 & 0 & 1 & 2 & 0 & 1 & 2 & 4 & 2 & 6 & 5 & 8 & 1 & 4 & $w_{257}$ & N & can. \\
2389 & 0 & 0 & 0 & 0 & 0 & 0 & 1 & 2 & 0 & 1 & 2 & 4 & 2 & 6 & 8 & 4 & 2 & 4 & $w_{326}$ & N & can. \\
2390 & 0 & 0 & 0 & 0 & 0 & 0 & 1 & 2 & 0 & 1 & 2 & 4 & 2 & 6 & 8 & 5 & 1 & 4 & $w_{411}$ & N & can. \\
2391 & 0 & 0 & 0 & 0 & 0 & 0 & 1 & 2 & 0 & 1 & 2 & 4 & 2 & 6 & 8 & 9 & 2 & 4 & $w_{258}$ & N & can. \\
2392 & 0 & 0 & 0 & 0 & 0 & 0 & 1 & 2 & 0 & 1 & 2 & 4 & 2 & 6 & 8 & 10 & 4 & 4 & $w_{420}$ & N & can. \\
2393 & 0 & 0 & 0 & 0 & 0 & 0 & 1 & 2 & 0 & 1 & 2 & 4 & 2 & 6 & 8 & 11 & 2 & 4 & $w_{258}$ & N & can. \\
2394 & 0 & 0 & 0 & 0 & 0 & 0 & 1 & 2 & 0 & 1 & 2 & 4 & 2 & 6 & 8 & 12 & 1 & 4 & $w_{258}$ & N & can. \\
2395 & 0 & 0 & 0 & 0 & 0 & 0 & 1 & 2 & 0 & 1 & 2 & 4 & 2 & 6 & 8 & 13 & 1 & 4 & $w_{266}$ & N & can. \\
2396 & 0 & 0 & 0 & 0 & 0 & 0 & 1 & 2 & 0 & 1 & 2 & 4 & 2 & 7 & 4 & 8 & 1 & 4 & $w_{257}$ & N & can. \\
2397 & 0 & 0 & 0 & 0 & 0 & 0 & 1 & 2 & 0 & 1 & 2 & 4 & 2 & 7 & 8 & 4 & 2 & 4 & $w_{343}$ & N & can. \\
2398 & 0 & 0 & 0 & 0 & 0 & 0 & 1 & 2 & 0 & 1 & 2 & 4 & 2 & 7 & 8 & 5 & 1 & 4 & $w_{257}$ & N & can. \\
2399 & 0 & 0 & 0 & 0 & 0 & 0 & 1 & 2 & 0 & 1 & 2 & 4 & 2 & 7 & 8 & 9 & 2 & 3 & $w_{345}$ & N & can. \\
2400 & 0 & 0 & 0 & 0 & 0 & 0 & 1 & 2 & 0 & 1 & 2 & 4 & 2 & 7 & 8 & 10 & 2 & 4 & $w_{418}$ & N & can. \\
2401 & 0 & 0 & 0 & 0 & 0 & 0 & 1 & 2 & 0 & 1 & 2 & 4 & 2 & 7 & 8 & 11 & 2 & 4 & $w_{351}$ & N & can. \\
2402 & 0 & 0 & 0 & 0 & 0 & 0 & 1 & 2 & 0 & 1 & 2 & 4 & 2 & 7 & 8 & 12 & 1 & 4 & $w_{392}$ & N & can. \\
2403 & 0 & 0 & 0 & 0 & 0 & 0 & 1 & 2 & 0 & 1 & 2 & 4 & 2 & 7 & 8 & 13 & 1 & 4 & $w_{353}$ & N & can. \\
2404 & 0 & 0 & 0 & 0 & 0 & 0 & 1 & 2 & 0 & 1 & 2 & 4 & 2 & 8 & 3 & 4 & 1 & 4 & $w_{326}$ & N & can. \\
2405 & 0 & 0 & 0 & 0 & 0 & 0 & 1 & 2 & 0 & 1 & 2 & 4 & 2 & 8 & 3 & 7 & 3 & 4 & $w_{406}$ & N & can. \\
2406 & 0 & 0 & 0 & 0 & 0 & 0 & 1 & 2 & 0 & 1 & 2 & 4 & 2 & 8 & 3 & 12 & 2 & 4 & $w_{388}$ & N & can. \\
2407 & 0 & 0 & 0 & 0 & 0 & 0 & 1 & 2 & 0 & 1 & 2 & 4 & 2 & 8 & 3 & 13 & 6 & 3 & $w_{237}$ & Y & can. \\
2408 & 0 & 0 & 0 & 0 & 0 & 0 & 1 & 2 & 0 & 1 & 2 & 4 & 2 & 8 & 4 & 8 & 1 & 4 & $w_{383}$ & N & can. \\
2409 & 0 & 0 & 0 & 0 & 0 & 0 & 1 & 2 & 0 & 1 & 2 & 4 & 2 & 8 & 4 & 9 & 1 & 4 & $w_{262}$ & N & can. \\
2410 & 0 & 0 & 0 & 0 & 0 & 0 & 1 & 2 & 0 & 1 & 2 & 4 & 2 & 8 & 4 & 10 & 1 & 3 & $w_{237}$ & Y & can. \\
2411 & 0 & 0 & 0 & 0 & 0 & 0 & 1 & 2 & 0 & 1 & 2 & 4 & 2 & 8 & 4 & 11 & 1 & 4 & $w_{258}$ & N & can. \\
2412 & 0 & 0 & 0 & 0 & 0 & 0 & 1 & 2 & 0 & 1 & 2 & 4 & 2 & 8 & 4 & 12 & 1 & 4 & $w_{258}$ & N & can. \\
2413 & 0 & 0 & 0 & 0 & 0 & 0 & 1 & 2 & 0 & 1 & 2 & 4 & 2 & 8 & 4 & 13 & 1 & 4 & $w_{241}$ & N & can. \\
2414 & 0 & 0 & 0 & 0 & 0 & 0 & 1 & 2 & 0 & 1 & 2 & 4 & 2 & 8 & 4 & 14 & 1 & 4 & $w_{262}$ & N & can. \\
2415 & 0 & 0 & 0 & 0 & 0 & 0 & 1 & 2 & 0 & 1 & 2 & 4 & 2 & 8 & 4 & 15 & 1 & 4 & $w_{260}$ & N & can. \\
2416 & 0 & 0 & 0 & 0 & 0 & 0 & 1 & 2 & 0 & 1 & 2 & 4 & 2 & 8 & 5 & 7 & 2 & 4 & $w_{412}$ & N & can. \\
2417 & 0 & 0 & 0 & 0 & 0 & 0 & 1 & 2 & 0 & 1 & 2 & 4 & 2 & 8 & 5 & 8 & 1 & 4 & $w_{341}$ & N & can. \\
2418 & 0 & 0 & 0 & 0 & 0 & 0 & 1 & 2 & 0 & 1 & 2 & 4 & 2 & 8 & 5 & 9 & 1 & 4 & $w_{258}$ & N & can. \\
2419 & 0 & 0 & 0 & 0 & 0 & 0 & 1 & 2 & 0 & 1 & 2 & 4 & 2 & 8 & 5 & 10 & 1 & 4 & $w_{262}$ & N & can. \\
2420 & 0 & 0 & 0 & 0 & 0 & 0 & 1 & 2 & 0 & 1 & 2 & 4 & 2 & 8 & 5 & 11 & 1 & 3 & $w_{345}$ & N & can. \\
2421 & 0 & 0 & 0 & 0 & 0 & 0 & 1 & 2 & 0 & 1 & 2 & 4 & 2 & 8 & 5 & 12 & 1 & 4 & $w_{353}$ & N & can. \\
2422 & 0 & 0 & 0 & 0 & 0 & 0 & 1 & 2 & 0 & 1 & 2 & 4 & 2 & 8 & 5 & 13 & 1 & 4 & $w_{260}$ & N & can. \\
2423 & 0 & 0 & 0 & 0 & 0 & 0 & 1 & 2 & 0 & 1 & 2 & 4 & 2 & 8 & 5 & 14 & 1 & 4 & $w_{266}$ & N & can. \\
2424 & 0 & 0 & 0 & 0 & 0 & 0 & 1 & 2 & 0 & 1 & 2 & 4 & 2 & 8 & 5 & 15 & 1 & 4 & $w_{352}$ & N & can. \\
2425 & 0 & 0 & 0 & 0 & 0 & 0 & 1 & 2 & 0 & 1 & 2 & 4 & 2 & 8 & 7 & 4 & 1 & 4 & $w_{240}$ & N & can. \\
2426 & 0 & 0 & 0 & 0 & 0 & 0 & 1 & 2 & 0 & 1 & 2 & 4 & 2 & 8 & 7 & 5 & 2 & 4 & $w_{411}$ & N & can. \\
2427 & 0 & 0 & 0 & 0 & 0 & 0 & 1 & 2 & 0 & 1 & 2 & 4 & 2 & 8 & 7 & 8 & 1 & 4 & $w_{262}$ & N & can. \\
2428 & 0 & 0 & 0 & 0 & 0 & 0 & 1 & 2 & 0 & 1 & 2 & 4 & 2 & 8 & 7 & 9 & 1 & 3 & $w_{259}$ & Y & can. \\
2429 & 0 & 0 & 0 & 0 & 0 & 0 & 1 & 2 & 0 & 1 & 2 & 4 & 2 & 8 & 7 & 10 & 1 & 4 & $w_{262}$ & N & can. \\
2430 & 0 & 0 & 0 & 0 & 0 & 0 & 1 & 2 & 0 & 1 & 2 & 4 & 2 & 8 & 7 & 11 & 1 & 4 & $w_{351}$ & N & can. \\
2431 & 0 & 0 & 0 & 0 & 0 & 0 & 1 & 2 & 0 & 1 & 2 & 4 & 2 & 8 & 7 & 12 & 1 & 4 & $w_{266}$ & N & can. \\
2432 & 0 & 0 & 0 & 0 & 0 & 0 & 1 & 2 & 0 & 1 & 2 & 4 & 2 & 8 & 7 & 13 & 1 & 4 & $w_{241}$ & N & can. \\
2433 & 0 & 0 & 0 & 0 & 0 & 0 & 1 & 2 & 0 & 1 & 2 & 4 & 2 & 8 & 7 & 14 & 1 & 4 & $w_{266}$ & N & can. \\
2434 & 0 & 0 & 0 & 0 & 0 & 0 & 1 & 2 & 0 & 1 & 2 & 4 & 2 & 8 & 7 & 15 & 1 & 4 & $w_{266}$ & N & can. \\
2435 & 0 & 0 & 0 & 0 & 0 & 0 & 1 & 2 & 0 & 1 & 2 & 4 & 2 & 8 & 12 & 4 & 1 & 4 & $w_{262}$ & N & can. \\
2436 & 0 & 0 & 0 & 0 & 0 & 0 & 1 & 2 & 0 & 1 & 2 & 4 & 2 & 8 & 12 & 5 & 1 & 4 & $w_{260}$ & N & can. \\
2437 & 0 & 0 & 0 & 0 & 0 & 0 & 1 & 2 & 0 & 1 & 2 & 4 & 2 & 8 & 12 & 6 & 1 & 4 & $w_{225}$ & N & can. \\
2438 & 0 & 0 & 0 & 0 & 0 & 0 & 1 & 2 & 0 & 1 & 2 & 4 & 2 & 8 & 12 & 7 & 1 & 4 & $w_{266}$ & N & can. \\
2439 & 0 & 0 & 0 & 0 & 0 & 0 & 1 & 2 & 0 & 1 & 2 & 4 & 2 & 8 & 12 & 13 & 2 & 4 & $w_{241}$ & N & can. \\
2440 & 0 & 0 & 0 & 0 & 0 & 0 & 1 & 2 & 0 & 1 & 2 & 4 & 2 & 8 & 12 & 15 & 2 & 4 & $w_{260}$ & N & can. \\
2441 & 0 & 0 & 0 & 0 & 0 & 0 & 1 & 2 & 0 & 1 & 2 & 4 & 2 & 8 & 13 & 4 & 1 & 4 & $w_{352}$ & N & can. \\
2442 & 0 & 0 & 0 & 0 & 0 & 0 & 1 & 2 & 0 & 1 & 2 & 4 & 2 & 8 & 13 & 5 & 1 & 4 & $w_{266}$ & N & \#2434 \\
2443 & 0 & 0 & 0 & 0 & 0 & 0 & 1 & 2 & 0 & 1 & 2 & 4 & 2 & 8 & 13 & 6 & 1 & 4 & $w_{260}$ & N & \#2415 \\
2444 & 0 & 0 & 0 & 0 & 0 & 0 & 1 & 2 & 0 & 1 & 2 & 4 & 2 & 8 & 13 & 7 & 1 & 4 & $w_{353}$ & N & can. \\
2445 & 0 & 0 & 0 & 0 & 0 & 0 & 1 & 2 & 0 & 1 & 2 & 4 & 2 & 8 & 13 & 12 & 2 & 4 & $w_{353}$ & N & can. \\
2446 & 0 & 0 & 0 & 0 & 0 & 0 & 1 & 2 & 0 & 1 & 2 & 4 & 2 & 8 & 13 & 14 & 2 & 4 & $w_{266}$ & N & can. \\
2447 & 0 & 0 & 0 & 0 & 0 & 0 & 1 & 2 & 0 & 1 & 2 & 4 & 2 & 8 & 13 & 15 & 4 & 4 & $w_{347}$ & N & can. \\
2448 & 0 & 0 & 0 & 0 & 0 & 0 & 1 & 2 & 0 & 1 & 2 & 4 & 2 & 8 & 14 & 4 & 1 & 4 & $w_{341}$ & N & can. \\
2449 & 0 & 0 & 0 & 0 & 0 & 0 & 1 & 2 & 0 & 1 & 2 & 4 & 2 & 8 & 14 & 5 & 1 & 4 & $w_{266}$ & N & can. \\
2450 & 0 & 0 & 0 & 0 & 0 & 0 & 1 & 2 & 0 & 1 & 2 & 4 & 2 & 8 & 14 & 6 & 1 & 4 & $w_{258}$ & N & can. \\
2451 & 0 & 0 & 0 & 0 & 0 & 0 & 1 & 2 & 0 & 1 & 2 & 4 & 2 & 8 & 14 & 7 & 1 & 4 & $w_{347}$ & N & can. \\
2452 & 0 & 0 & 0 & 0 & 0 & 0 & 1 & 2 & 0 & 1 & 2 & 4 & 2 & 8 & 14 & 13 & 2 & 4 & $w_{260}$ & N & can. \\
2453 & 0 & 0 & 0 & 0 & 0 & 0 & 1 & 2 & 0 & 1 & 2 & 4 & 2 & 8 & 14 & 15 & 2 & 4 & $w_{353}$ & N & can. \\
2454 & 0 & 0 & 0 & 0 & 0 & 0 & 1 & 2 & 0 & 1 & 2 & 4 & 2 & 8 & 15 & 4 & 1 & 4 & $w_{260}$ & N & can. \\
2455 & 0 & 0 & 0 & 0 & 0 & 0 & 1 & 2 & 0 & 1 & 2 & 4 & 2 & 8 & 15 & 5 & 1 & 4 & $w_{260}$ & N & can. \\
2456 & 0 & 0 & 0 & 0 & 0 & 0 & 1 & 2 & 0 & 1 & 2 & 4 & 2 & 8 & 15 & 6 & 1 & 4 & $w_{260}$ & N & can. \\
2457 & 0 & 0 & 0 & 0 & 0 & 0 & 1 & 2 & 0 & 1 & 2 & 4 & 2 & 8 & 15 & 7 & 1 & 4 & $w_{392}$ & N & can. \\
2458 & 0 & 0 & 0 & 0 & 0 & 0 & 1 & 2 & 0 & 1 & 2 & 4 & 2 & 8 & 15 & 12 & 2 & 4 & $w_{266}$ & N & can. \\
2459 & 0 & 0 & 0 & 0 & 0 & 0 & 1 & 2 & 0 & 1 & 2 & 4 & 2 & 8 & 15 & 13 & 4 & 4 & $w_{266}$ & N & can. \\
2460 & 0 & 0 & 0 & 0 & 0 & 0 & 1 & 2 & 0 & 1 & 2 & 4 & 2 & 8 & 15 & 14 & 2 & 4 & $w_{266}$ & N & can. \\
2461 & 0 & 0 & 0 & 0 & 0 & 0 & 1 & 2 & 0 & 1 & 2 & 4 & 3 & 4 & 4 & 7 & 32 & 3 & $w_{88}$ & N & can. \\
2462 & 0 & 0 & 0 & 0 & 0 & 0 & 1 & 2 & 0 & 1 & 2 & 4 & 3 & 4 & 4 & 8 & 1 & 4 & $w_{235}$ & N & can. \\
2463 & 0 & 0 & 0 & 0 & 0 & 0 & 1 & 2 & 0 & 1 & 2 & 4 & 3 & 4 & 7 & 8 & 1 & 4 & $w_{411}$ & N & can. \\
2464 & 0 & 0 & 0 & 0 & 0 & 0 & 1 & 2 & 0 & 1 & 2 & 4 & 3 & 4 & 8 & 9 & 2 & 4 & $w_{258}$ & N & can. \\
2465 & 0 & 0 & 0 & 0 & 0 & 0 & 1 & 2 & 0 & 1 & 2 & 4 & 3 & 4 & 8 & 10 & 2 & 4 & $w_{351}$ & N & can. \\
2466 & 0 & 0 & 0 & 0 & 0 & 0 & 1 & 2 & 0 & 1 & 2 & 4 & 3 & 4 & 8 & 11 & 2 & 3 & $w_{259}$ & Y & can. \\
2467 & 0 & 0 & 0 & 0 & 0 & 0 & 1 & 2 & 0 & 1 & 2 & 4 & 3 & 4 & 8 & 12 & 1 & 4 & $w_{266}$ & N & can. \\
2468 & 0 & 0 & 0 & 0 & 0 & 0 & 1 & 2 & 0 & 1 & 2 & 4 & 3 & 4 & 8 & 14 & 1 & 4 & $w_{260}$ & N & can. \\
2469 & 0 & 0 & 0 & 0 & 0 & 0 & 1 & 2 & 0 & 1 & 2 & 4 & 3 & 5 & 7 & 6 & 64 & 4 & $w_{402}$ & N & can. \\
2470 & 0 & 0 & 0 & 0 & 0 & 0 & 1 & 2 & 0 & 1 & 2 & 4 & 3 & 5 & 7 & 8 & 2 & 4 & $w_{343}$ & N & can. \\
2471 & 0 & 0 & 0 & 0 & 0 & 0 & 1 & 2 & 0 & 1 & 2 & 4 & 3 & 5 & 8 & 9 & 2 & 4 & $w_{344}$ & N & can. \\
2472 & 0 & 0 & 0 & 0 & 0 & 0 & 1 & 2 & 0 & 1 & 2 & 4 & 3 & 5 & 8 & 10 & 2 & 3 & $w_{345}$ & N & can. \\
2473 & 0 & 0 & 0 & 0 & 0 & 0 & 1 & 2 & 0 & 1 & 2 & 4 & 3 & 5 & 8 & 11 & 2 & 4 & $w_{258}$ & N & can. \\
2474 & 0 & 0 & 0 & 0 & 0 & 0 & 1 & 2 & 0 & 1 & 2 & 4 & 3 & 5 & 8 & 12 & 2 & 4 & $w_{266}$ & N & can. \\
2475 & 0 & 0 & 0 & 0 & 0 & 0 & 1 & 2 & 0 & 1 & 2 & 4 & 3 & 5 & 8 & 13 & 2 & 4 & $w_{353}$ & N & can. \\
2476 & 0 & 0 & 0 & 0 & 0 & 0 & 1 & 2 & 0 & 1 & 2 & 4 & 3 & 5 & 8 & 14 & 2 & 4 & $w_{372}$ & N & can. \\
2477 & 0 & 0 & 0 & 0 & 0 & 0 & 1 & 2 & 0 & 1 & 2 & 4 & 3 & 5 & 8 & 15 & 2 & 4 & $w_{266}$ & N & can. \\
2478 & 0 & 0 & 0 & 0 & 0 & 0 & 1 & 2 & 0 & 1 & 2 & 4 & 3 & 6 & 5 & 7 & 32 & 4 & $w_{256}$ & N & can. \\
2479 & 0 & 0 & 0 & 0 & 0 & 0 & 1 & 2 & 0 & 1 & 2 & 4 & 3 & 6 & 5 & 8 & 1 & 4 & $w_{411}$ & N & can. \\
2480 & 0 & 0 & 0 & 0 & 0 & 0 & 1 & 2 & 0 & 1 & 2 & 4 & 3 & 6 & 6 & 8 & 1 & 4 & $w_{240}$ & N & can. \\
2481 & 0 & 0 & 0 & 0 & 0 & 0 & 1 & 2 & 0 & 1 & 2 & 4 & 3 & 6 & 7 & 5 & 32 & 4 & $w_{416}$ & N & can. \\
2482 & 0 & 0 & 0 & 0 & 0 & 0 & 1 & 2 & 0 & 1 & 2 & 4 & 3 & 6 & 7 & 8 & 1 & 4 & $w_{412}$ & N & can. \\
2483 & 0 & 0 & 0 & 0 & 0 & 0 & 1 & 2 & 0 & 1 & 2 & 4 & 3 & 6 & 8 & 9 & 2 & 3 & $w_{345}$ & N & can. \\
2484 & 0 & 0 & 0 & 0 & 0 & 0 & 1 & 2 & 0 & 1 & 2 & 4 & 3 & 6 & 8 & 10 & 2 & 4 & $w_{418}$ & N & can. \\
2485 & 0 & 0 & 0 & 0 & 0 & 0 & 1 & 2 & 0 & 1 & 2 & 4 & 3 & 6 & 8 & 11 & 2 & 4 & $w_{351}$ & N & can. \\
2486 & 0 & 0 & 0 & 0 & 0 & 0 & 1 & 2 & 0 & 1 & 2 & 4 & 3 & 6 & 8 & 12 & 1 & 4 & $w_{392}$ & N & can. \\
2487 & 0 & 0 & 0 & 0 & 0 & 0 & 1 & 2 & 0 & 1 & 2 & 4 & 3 & 6 & 8 & 13 & 1 & 4 & $w_{353}$ & N & can. \\
2488 & 0 & 0 & 0 & 0 & 0 & 0 & 1 & 2 & 0 & 1 & 2 & 4 & 3 & 7 & 4 & 7 & 16 & 4 & $w_{255}$ & N & can. \\
2489 & 0 & 0 & 0 & 0 & 0 & 0 & 1 & 2 & 0 & 1 & 2 & 4 & 3 & 7 & 4 & 8 & 1 & 4 & $w_{411}$ & N & can. \\
2490 & 0 & 0 & 0 & 0 & 0 & 0 & 1 & 2 & 0 & 1 & 2 & 4 & 3 & 7 & 5 & 8 & 1 & 4 & $w_{421}$ & N & can. \\
2491 & 0 & 0 & 0 & 0 & 0 & 0 & 1 & 2 & 0 & 1 & 2 & 4 & 3 & 7 & 6 & 8 & 1 & 4 & $w_{411}$ & N & can. \\
2492 & 0 & 0 & 0 & 0 & 0 & 0 & 1 & 2 & 0 & 1 & 2 & 4 & 3 & 7 & 7 & 8 & 1 & 4 & $w_{421}$ & N & can. \\
2493 & 0 & 0 & 0 & 0 & 0 & 0 & 1 & 2 & 0 & 1 & 2 & 4 & 3 & 7 & 8 & 9 & 2 & 4 & $w_{413}$ & N & can. \\
2494 & 0 & 0 & 0 & 0 & 0 & 0 & 1 & 2 & 0 & 1 & 2 & 4 & 3 & 7 & 8 & 10 & 2 & 4 & $w_{422}$ & N & can. \\
2495 & 0 & 0 & 0 & 0 & 0 & 0 & 1 & 2 & 0 & 1 & 2 & 4 & 3 & 7 & 8 & 11 & 2 & 4 & $w_{413}$ & N & can. \\
2496 & 0 & 0 & 0 & 0 & 0 & 0 & 1 & 2 & 0 & 1 & 2 & 4 & 3 & 7 & 8 & 12 & 1 & 4 & $w_{392}$ & N & can. \\
2497 & 0 & 0 & 0 & 0 & 0 & 0 & 1 & 2 & 0 & 1 & 2 & 4 & 3 & 7 & 8 & 13 & 1 & 4 & $w_{423}$ & N & can. \\
2498 & 0 & 0 & 0 & 0 & 0 & 0 & 1 & 2 & 0 & 1 & 2 & 4 & 3 & 8 & 4 & 7 & 1 & 4 & $w_{257}$ & N & can. \\
2499 & 0 & 0 & 0 & 0 & 0 & 0 & 1 & 2 & 0 & 1 & 2 & 4 & 3 & 8 & 4 & 8 & 2 & 4 & $w_{341}$ & N & can. \\
2500 & 0 & 0 & 0 & 0 & 0 & 0 & 1 & 2 & 0 & 1 & 2 & 4 & 3 & 8 & 4 & 9 & 1 & 4 & $w_{258}$ & N & can. \\
2501 & 0 & 0 & 0 & 0 & 0 & 0 & 1 & 2 & 0 & 1 & 2 & 4 & 3 & 8 & 4 & 10 & 1 & 4 & $w_{262}$ & N & can. \\
2502 & 0 & 0 & 0 & 0 & 0 & 0 & 1 & 2 & 0 & 1 & 2 & 4 & 3 & 8 & 4 & 11 & 1 & 3 & $w_{345}$ & N & can. \\
2503 & 0 & 0 & 0 & 0 & 0 & 0 & 1 & 2 & 0 & 1 & 2 & 4 & 3 & 8 & 4 & 12 & 1 & 4 & $w_{353}$ & N & can. \\
2504 & 0 & 0 & 0 & 0 & 0 & 0 & 1 & 2 & 0 & 1 & 2 & 4 & 3 & 8 & 4 & 13 & 1 & 4 & $w_{260}$ & N & can. \\
2505 & 0 & 0 & 0 & 0 & 0 & 0 & 1 & 2 & 0 & 1 & 2 & 4 & 3 & 8 & 4 & 14 & 1 & 4 & $w_{266}$ & N & can. \\
2506 & 0 & 0 & 0 & 0 & 0 & 0 & 1 & 2 & 0 & 1 & 2 & 4 & 3 & 8 & 4 & 15 & 1 & 4 & $w_{352}$ & N & can. \\
2507 & 0 & 0 & 0 & 0 & 0 & 0 & 1 & 2 & 0 & 1 & 2 & 4 & 3 & 8 & 5 & 7 & 1 & 4 & $w_{421}$ & N & can. \\
2508 & 0 & 0 & 0 & 0 & 0 & 0 & 1 & 2 & 0 & 1 & 2 & 4 & 3 & 8 & 5 & 9 & 2 & 4 & $w_{351}$ & N & can. \\
2509 & 0 & 0 & 0 & 0 & 0 & 0 & 1 & 2 & 0 & 1 & 2 & 4 & 3 & 8 & 5 & 10 & 1 & 3 & $w_{259}$ & N & can. \\
2510 & 0 & 0 & 0 & 0 & 0 & 0 & 1 & 2 & 0 & 1 & 2 & 4 & 3 & 8 & 5 & 11 & 1 & 4 & $w_{413}$ & N & can. \\
2511 & 0 & 0 & 0 & 0 & 0 & 0 & 1 & 2 & 0 & 1 & 2 & 4 & 3 & 8 & 5 & 12 & 1 & 4 & $w_{392}$ & N & can. \\
2512 & 0 & 0 & 0 & 0 & 0 & 0 & 1 & 2 & 0 & 1 & 2 & 4 & 3 & 8 & 5 & 13 & 1 & 4 & $w_{266}$ & N & can. \\
2513 & 0 & 0 & 0 & 0 & 0 & 0 & 1 & 2 & 0 & 1 & 2 & 4 & 3 & 8 & 5 & 14 & 1 & 4 & $w_{266}$ & N & can. \\
2514 & 0 & 0 & 0 & 0 & 0 & 0 & 1 & 2 & 0 & 1 & 2 & 4 & 3 & 8 & 5 & 15 & 1 & 4 & $w_{392}$ & N & can. \\
2515 & 0 & 0 & 0 & 0 & 0 & 0 & 1 & 2 & 0 & 1 & 2 & 4 & 3 & 8 & 6 & 10 & 2 & 4 & $w_{262}$ & N & can. \\
2516 & 0 & 0 & 0 & 0 & 0 & 0 & 1 & 2 & 0 & 1 & 2 & 4 & 3 & 8 & 6 & 11 & 1 & 4 & $w_{351}$ & N & can. \\
2517 & 0 & 0 & 0 & 0 & 0 & 0 & 1 & 2 & 0 & 1 & 2 & 4 & 3 & 8 & 6 & 12 & 1 & 4 & $w_{266}$ & N & can. \\
2518 & 0 & 0 & 0 & 0 & 0 & 0 & 1 & 2 & 0 & 1 & 2 & 4 & 3 & 8 & 6 & 13 & 1 & 4 & $w_{241}$ & N & can. \\
2519 & 0 & 0 & 0 & 0 & 0 & 0 & 1 & 2 & 0 & 1 & 2 & 4 & 3 & 8 & 6 & 14 & 1 & 4 & $w_{266}$ & N & can. \\
2520 & 0 & 0 & 0 & 0 & 0 & 0 & 1 & 2 & 0 & 1 & 2 & 4 & 3 & 8 & 6 & 15 & 1 & 4 & $w_{266}$ & N & can. \\
2521 & 0 & 0 & 0 & 0 & 0 & 0 & 1 & 2 & 0 & 1 & 2 & 4 & 3 & 8 & 7 & 5 & 1 & 4 & $w_{417}$ & N & can. \\
2522 & 0 & 0 & 0 & 0 & 0 & 0 & 1 & 2 & 0 & 1 & 2 & 4 & 3 & 8 & 7 & 6 & 1 & 4 & $w_{412}$ & N & can. \\
2523 & 0 & 0 & 0 & 0 & 0 & 0 & 1 & 2 & 0 & 1 & 2 & 4 & 3 & 8 & 7 & 11 & 2 & 4 & $w_{424}$ & N & can. \\
2524 & 0 & 0 & 0 & 0 & 0 & 0 & 1 & 2 & 0 & 1 & 2 & 4 & 3 & 8 & 7 & 12 & 1 & 4 & $w_{347}$ & N & can. \\
2525 & 0 & 0 & 0 & 0 & 0 & 0 & 1 & 2 & 0 & 1 & 2 & 4 & 3 & 8 & 7 & 13 & 1 & 4 & $w_{392}$ & N & can. \\
2526 & 0 & 0 & 0 & 0 & 0 & 0 & 1 & 2 & 0 & 1 & 2 & 4 & 3 & 8 & 7 & 14 & 1 & 4 & $w_{423}$ & N & can. \\
2527 & 0 & 0 & 0 & 0 & 0 & 0 & 1 & 2 & 0 & 1 & 2 & 4 & 3 & 8 & 7 & 15 & 1 & 4 & $w_{419}$ & N & can. \\
2528 & 0 & 0 & 0 & 0 & 0 & 0 & 1 & 2 & 0 & 1 & 2 & 4 & 3 & 8 & 8 & 4 & 2 & 4 & $w_{262}$ & N & can. \\
2529 & 0 & 0 & 0 & 0 & 0 & 0 & 1 & 2 & 0 & 1 & 2 & 4 & 3 & 8 & 8 & 5 & 1 & 4 & $w_{258}$ & N & can. \\
2530 & 0 & 0 & 0 & 0 & 0 & 0 & 1 & 2 & 0 & 1 & 2 & 4 & 3 & 8 & 8 & 6 & 1 & 4 & $w_{258}$ & N & can. \\
2531 & 0 & 0 & 0 & 0 & 0 & 0 & 1 & 2 & 0 & 1 & 2 & 4 & 3 & 8 & 8 & 7 & 1 & 3 & $w_{259}$ & Y & \#2466 \\
2532 & 0 & 0 & 0 & 0 & 0 & 0 & 1 & 2 & 0 & 1 & 2 & 4 & 3 & 8 & 8 & 12 & 1 & 4 & $w_{266}$ & N & can. \\
2533 & 0 & 0 & 0 & 0 & 0 & 0 & 1 & 2 & 0 & 1 & 2 & 4 & 3 & 8 & 8 & 13 & 1 & 4 & $w_{260}$ & N & can. \\
2534 & 0 & 0 & 0 & 0 & 0 & 0 & 1 & 2 & 0 & 1 & 2 & 4 & 3 & 8 & 8 & 14 & 1 & 4 & $w_{260}$ & N & \#2468 \\
2535 & 0 & 0 & 0 & 0 & 0 & 0 & 1 & 2 & 0 & 1 & 2 & 4 & 3 & 8 & 8 & 15 & 1 & 4 & $w_{266}$ & N & can. \\
2536 & 0 & 0 & 0 & 0 & 0 & 0 & 1 & 2 & 0 & 1 & 2 & 4 & 3 & 8 & 9 & 5 & 2 & 4 & $w_{409}$ & N & can. \\
2537 & 0 & 0 & 0 & 0 & 0 & 0 & 1 & 2 & 0 & 1 & 2 & 4 & 3 & 8 & 9 & 6 & 1 & 3 & $w_{345}$ & N & can. \\
2538 & 0 & 0 & 0 & 0 & 0 & 0 & 1 & 2 & 0 & 1 & 2 & 4 & 3 & 8 & 9 & 7 & 1 & 4 & $w_{351}$ & N & can. \\
2539 & 0 & 0 & 0 & 0 & 0 & 0 & 1 & 2 & 0 & 1 & 2 & 4 & 3 & 8 & 9 & 12 & 1 & 4 & $w_{347}$ & N & can. \\
2540 & 0 & 0 & 0 & 0 & 0 & 0 & 1 & 2 & 0 & 1 & 2 & 4 & 3 & 8 & 9 & 13 & 1 & 4 & $w_{266}$ & N & can. \\
2541 & 0 & 0 & 0 & 0 & 0 & 0 & 1 & 2 & 0 & 1 & 2 & 4 & 3 & 8 & 9 & 14 & 1 & 4 & $w_{392}$ & N & can. \\
2542 & 0 & 0 & 0 & 0 & 0 & 0 & 1 & 2 & 0 & 1 & 2 & 4 & 3 & 8 & 9 & 15 & 1 & 4 & $w_{353}$ & N & can. \\
2543 & 0 & 0 & 0 & 0 & 0 & 0 & 1 & 2 & 0 & 1 & 2 & 4 & 3 & 8 & 10 & 6 & 2 & 4 & $w_{409}$ & N & can. \\
2544 & 0 & 0 & 0 & 0 & 0 & 0 & 1 & 2 & 0 & 1 & 2 & 4 & 3 & 8 & 10 & 7 & 1 & 4 & $w_{351}$ & N & can. \\
2545 & 0 & 0 & 0 & 0 & 0 & 0 & 1 & 2 & 0 & 1 & 2 & 4 & 3 & 8 & 10 & 12 & 1 & 4 & $w_{353}$ & N & can. \\
2546 & 0 & 0 & 0 & 0 & 0 & 0 & 1 & 2 & 0 & 1 & 2 & 4 & 3 & 8 & 10 & 13 & 1 & 4 & $w_{266}$ & N & can. \\
2547 & 0 & 0 & 0 & 0 & 0 & 0 & 1 & 2 & 0 & 1 & 2 & 4 & 3 & 8 & 10 & 14 & 1 & 4 & $w_{392}$ & N & can. \\
2548 & 0 & 0 & 0 & 0 & 0 & 0 & 1 & 2 & 0 & 1 & 2 & 4 & 3 & 8 & 10 & 15 & 1 & 4 & $w_{347}$ & N & can. \\
2549 & 0 & 0 & 0 & 0 & 0 & 0 & 1 & 2 & 0 & 1 & 2 & 4 & 3 & 8 & 11 & 7 & 2 & 4 & $w_{351}$ & N & can. \\
2550 & 0 & 0 & 0 & 0 & 0 & 0 & 1 & 2 & 0 & 1 & 2 & 4 & 3 & 8 & 11 & 12 & 1 & 4 & $w_{392}$ & N & can. \\
2551 & 0 & 0 & 0 & 0 & 0 & 0 & 1 & 2 & 0 & 1 & 2 & 4 & 3 & 8 & 11 & 13 & 1 & 4 & $w_{260}$ & N & can. \\
2552 & 0 & 0 & 0 & 0 & 0 & 0 & 1 & 2 & 0 & 1 & 2 & 4 & 3 & 8 & 11 & 14 & 1 & 4 & $w_{392}$ & N & can. \\
2553 & 0 & 0 & 0 & 0 & 0 & 0 & 1 & 2 & 0 & 1 & 2 & 4 & 3 & 8 & 11 & 15 & 1 & 4 & $w_{392}$ & N & can. \\
2554 & 0 & 0 & 0 & 0 & 0 & 0 & 1 & 2 & 0 & 1 & 2 & 4 & 3 & 8 & 12 & 13 & 2 & 4 & $w_{260}$ & N & can. \\
2555 & 0 & 0 & 0 & 0 & 0 & 0 & 1 & 2 & 0 & 1 & 2 & 4 & 3 & 8 & 12 & 14 & 1 & 4 & $w_{392}$ & N & can. \\
2556 & 0 & 0 & 0 & 0 & 0 & 0 & 1 & 2 & 0 & 1 & 2 & 4 & 3 & 8 & 12 & 15 & 1 & 4 & $w_{266}$ & N & can. \\
2557 & 0 & 0 & 0 & 0 & 0 & 0 & 1 & 2 & 0 & 1 & 2 & 4 & 3 & 8 & 13 & 12 & 2 & 4 & $w_{347}$ & N & can. \\
2558 & 0 & 0 & 0 & 0 & 0 & 0 & 1 & 2 & 0 & 1 & 2 & 4 & 3 & 8 & 13 & 14 & 1 & 4 & $w_{392}$ & N & \#2550 \\
2559 & 0 & 0 & 0 & 0 & 0 & 0 & 1 & 2 & 0 & 1 & 2 & 4 & 3 & 8 & 13 & 15 & 1 & 4 & $w_{419}$ & N & can. \\
2560 & 0 & 0 & 0 & 0 & 0 & 0 & 1 & 2 & 0 & 1 & 2 & 4 & 3 & 8 & 14 & 15 & 2 & 4 & $w_{347}$ & N & can. \\
2561 & 0 & 0 & 0 & 0 & 0 & 0 & 1 & 2 & 0 & 1 & 2 & 4 & 3 & 8 & 15 & 14 & 2 & 4 & $w_{392}$ & N & can. \\
2562 & 0 & 0 & 0 & 0 & 0 & 0 & 1 & 2 & 0 & 1 & 2 & 4 & 4 & 7 & 6 & 8 & 1 & 4 & $w_{240}$ & N & can. \\
2563 & 0 & 0 & 0 & 0 & 0 & 0 & 1 & 2 & 0 & 1 & 2 & 4 & 4 & 7 & 8 & 4 & 2 & 4 & $w_{240}$ & N & can. \\
2564 & 0 & 0 & 0 & 0 & 0 & 0 & 1 & 2 & 0 & 1 & 2 & 4 & 4 & 7 & 8 & 7 & 1 & 4 & $w_{257}$ & N & can. \\
2565 & 0 & 0 & 0 & 0 & 0 & 0 & 1 & 2 & 0 & 1 & 2 & 4 & 4 & 7 & 8 & 9 & 1 & 4 & $w_{266}$ & N & can. \\
2566 & 0 & 0 & 0 & 0 & 0 & 0 & 1 & 2 & 0 & 1 & 2 & 4 & 4 & 7 & 8 & 10 & 1 & 4 & $w_{392}$ & N & can. \\
2567 & 0 & 0 & 0 & 0 & 0 & 0 & 1 & 2 & 0 & 1 & 2 & 4 & 4 & 7 & 8 & 11 & 1 & 4 & $w_{260}$ & N & can. \\
2568 & 0 & 0 & 0 & 0 & 0 & 0 & 1 & 2 & 0 & 1 & 2 & 4 & 4 & 7 & 8 & 12 & 1 & 4 & $w_{392}$ & N & can. \\
2569 & 0 & 0 & 0 & 0 & 0 & 0 & 1 & 2 & 0 & 1 & 2 & 4 & 4 & 7 & 8 & 13 & 1 & 4 & $w_{392}$ & N & can. \\
2570 & 0 & 0 & 0 & 0 & 0 & 0 & 1 & 2 & 0 & 1 & 2 & 4 & 4 & 7 & 8 & 14 & 1 & 4 & $w_{266}$ & N & can. \\
2571 & 0 & 0 & 0 & 0 & 0 & 0 & 1 & 2 & 0 & 1 & 2 & 4 & 4 & 7 & 8 & 15 & 1 & 3 & $w_{280}$ & Y & can. \\
2572 & 0 & 0 & 0 & 0 & 0 & 0 & 1 & 2 & 0 & 1 & 2 & 4 & 4 & 8 & 4 & 9 & 2 & 4 & $w_{225}$ & N & can. \\
2573 & 0 & 0 & 0 & 0 & 0 & 0 & 1 & 2 & 0 & 1 & 2 & 4 & 4 & 8 & 4 & 10 & 4 & 4 & $w_{202}$ & N & can. \\
2574 & 0 & 0 & 0 & 0 & 0 & 0 & 1 & 2 & 0 & 1 & 2 & 4 & 4 & 8 & 4 & 11 & 2 & 4 & $w_{372}$ & N & can. \\
2575 & 0 & 0 & 0 & 0 & 0 & 0 & 1 & 2 & 0 & 1 & 2 & 4 & 4 & 8 & 4 & 12 & 8 & 3 & $w_{371}$ & N & can. \\
2576 & 0 & 0 & 0 & 0 & 0 & 0 & 1 & 2 & 0 & 1 & 2 & 4 & 4 & 8 & 4 & 14 & 4 & 4 & $w_{262}$ & N & can. \\
2577 & 0 & 0 & 0 & 0 & 0 & 0 & 1 & 2 & 0 & 1 & 2 & 4 & 4 & 8 & 6 & 8 & 1 & 4 & $w_{225}$ & N & can. \\
2578 & 0 & 0 & 0 & 0 & 0 & 0 & 1 & 2 & 0 & 1 & 2 & 4 & 4 & 8 & 6 & 9 & 1 & 4 & $w_{241}$ & N & can. \\
2579 & 0 & 0 & 0 & 0 & 0 & 0 & 1 & 2 & 0 & 1 & 2 & 4 & 4 & 8 & 6 & 11 & 1 & 4 & $w_{260}$ & N & can. \\
2580 & 0 & 0 & 0 & 0 & 0 & 0 & 1 & 2 & 0 & 1 & 2 & 4 & 4 & 8 & 6 & 13 & 1 & 4 & $w_{241}$ & N & can. \\
2581 & 0 & 0 & 0 & 0 & 0 & 0 & 1 & 2 & 0 & 1 & 2 & 4 & 4 & 8 & 6 & 14 & 1 & 3 & $w_{259}$ & Y & can. \\
2582 & 0 & 0 & 0 & 0 & 0 & 0 & 1 & 2 & 0 & 1 & 2 & 4 & 4 & 8 & 6 & 15 & 1 & 4 & $w_{260}$ & N & can. \\
2583 & 0 & 0 & 0 & 0 & 0 & 0 & 1 & 2 & 0 & 1 & 2 & 4 & 4 & 8 & 7 & 8 & 1 & 4 & $w_{353}$ & N & can. \\
2584 & 0 & 0 & 0 & 0 & 0 & 0 & 1 & 2 & 0 & 1 & 2 & 4 & 4 & 8 & 7 & 9 & 1 & 4 & $w_{392}$ & N & can. \\
2585 & 0 & 0 & 0 & 0 & 0 & 0 & 1 & 2 & 0 & 1 & 2 & 4 & 4 & 8 & 7 & 10 & 1 & 4 & $w_{266}$ & N & can. \\
2586 & 0 & 0 & 0 & 0 & 0 & 0 & 1 & 2 & 0 & 1 & 2 & 4 & 4 & 8 & 7 & 11 & 1 & 4 & $w_{347}$ & N & can. \\
2587 & 0 & 0 & 0 & 0 & 0 & 0 & 1 & 2 & 0 & 1 & 2 & 4 & 4 & 8 & 7 & 12 & 1 & 4 & $w_{419}$ & N & can. \\
2588 & 0 & 0 & 0 & 0 & 0 & 0 & 1 & 2 & 0 & 1 & 2 & 4 & 4 & 8 & 7 & 13 & 1 & 4 & $w_{392}$ & N & can. \\
2589 & 0 & 0 & 0 & 0 & 0 & 0 & 1 & 2 & 0 & 1 & 2 & 4 & 4 & 8 & 7 & 14 & 1 & 4 & $w_{423}$ & N & can. \\
2590 & 0 & 0 & 0 & 0 & 0 & 0 & 1 & 2 & 0 & 1 & 2 & 4 & 4 & 8 & 7 & 15 & 1 & 3 & $w_{378}$ & N & can. \\
2591 & 0 & 0 & 0 & 0 & 0 & 0 & 1 & 2 & 0 & 1 & 2 & 4 & 4 & 8 & 8 & 7 & 1 & 4 & $w_{260}$ & N & can. \\
2592 & 0 & 0 & 0 & 0 & 0 & 0 & 1 & 2 & 0 & 1 & 2 & 4 & 4 & 8 & 8 & 11 & 1 & 4 & $w_{260}$ & N & can. \\
2593 & 0 & 0 & 0 & 0 & 0 & 0 & 1 & 2 & 0 & 1 & 2 & 4 & 4 & 8 & 8 & 13 & 1 & 4 & $w_{260}$ & N & can. \\
2594 & 0 & 0 & 0 & 0 & 0 & 0 & 1 & 2 & 0 & 1 & 2 & 4 & 4 & 8 & 8 & 14 & 1 & 4 & $w_{260}$ & N & can. \\
2595 & 0 & 0 & 0 & 0 & 0 & 0 & 1 & 2 & 0 & 1 & 2 & 4 & 4 & 8 & 8 & 15 & 1 & 4 & $w_{267}$ & N & can. \\
2596 & 0 & 0 & 0 & 0 & 0 & 0 & 1 & 2 & 0 & 1 & 2 & 4 & 4 & 8 & 9 & 4 & 2 & 4 & $w_{225}$ & N & can. \\
2597 & 0 & 0 & 0 & 0 & 0 & 0 & 1 & 2 & 0 & 1 & 2 & 4 & 4 & 8 & 9 & 7 & 1 & 4 & $w_{266}$ & N & can. \\
2598 & 0 & 0 & 0 & 0 & 0 & 0 & 1 & 2 & 0 & 1 & 2 & 4 & 4 & 8 & 9 & 10 & 1 & 4 & $w_{241}$ & N & can. \\
2599 & 0 & 0 & 0 & 0 & 0 & 0 & 1 & 2 & 0 & 1 & 2 & 4 & 4 & 8 & 9 & 11 & 2 & 4 & $w_{347}$ & N & can. \\
2600 & 0 & 0 & 0 & 0 & 0 & 0 & 1 & 2 & 0 & 1 & 2 & 4 & 4 & 8 & 9 & 12 & 2 & 4 & $w_{347}$ & N & can. \\
2601 & 0 & 0 & 0 & 0 & 0 & 0 & 1 & 2 & 0 & 1 & 2 & 4 & 4 & 8 & 9 & 13 & 1 & 4 & $w_{266}$ & N & \#2597 \\
2602 & 0 & 0 & 0 & 0 & 0 & 0 & 1 & 2 & 0 & 1 & 2 & 4 & 4 & 8 & 9 & 14 & 1 & 4 & $w_{281}$ & N & can. \\
2603 & 0 & 0 & 0 & 0 & 0 & 0 & 1 & 2 & 0 & 1 & 2 & 4 & 4 & 8 & 9 & 15 & 1 & 4 & $w_{373}$ & N & can. \\
2604 & 0 & 0 & 0 & 0 & 0 & 0 & 1 & 2 & 0 & 1 & 2 & 4 & 4 & 8 & 10 & 4 & 2 & 4 & $w_{225}$ & N & can. \\
2605 & 0 & 0 & 0 & 0 & 0 & 0 & 1 & 2 & 0 & 1 & 2 & 4 & 4 & 8 & 10 & 7 & 1 & 4 & $w_{266}$ & N & can. \\
2606 & 0 & 0 & 0 & 0 & 0 & 0 & 1 & 2 & 0 & 1 & 2 & 4 & 4 & 8 & 10 & 9 & 1 & 4 & $w_{260}$ & N & can. \\
2607 & 0 & 0 & 0 & 0 & 0 & 0 & 1 & 2 & 0 & 1 & 2 & 4 & 4 & 8 & 10 & 11 & 2 & 4 & $w_{353}$ & N & can. \\
2608 & 0 & 0 & 0 & 0 & 0 & 0 & 1 & 2 & 0 & 1 & 2 & 4 & 4 & 8 & 10 & 13 & 1 & 4 & $w_{267}$ & N & can. \\
2609 & 0 & 0 & 0 & 0 & 0 & 0 & 1 & 2 & 0 & 1 & 2 & 4 & 4 & 8 & 10 & 14 & 1 & 4 & $w_{392}$ & N & can. \\
2610 & 0 & 0 & 0 & 0 & 0 & 0 & 1 & 2 & 0 & 1 & 2 & 4 & 4 & 8 & 10 & 15 & 1 & 4 & $w_{354}$ & N & can. \\
2611 & 0 & 0 & 0 & 0 & 0 & 0 & 1 & 2 & 0 & 1 & 2 & 4 & 4 & 8 & 11 & 4 & 2 & 4 & $w_{226}$ & N & can. \\
2612 & 0 & 0 & 0 & 0 & 0 & 0 & 1 & 2 & 0 & 1 & 2 & 4 & 4 & 8 & 11 & 7 & 1 & 4 & $w_{260}$ & N & \#2567 \\
2613 & 0 & 0 & 0 & 0 & 0 & 0 & 1 & 2 & 0 & 1 & 2 & 4 & 4 & 8 & 11 & 8 & 1 & 4 & $w_{241}$ & N & can. \\
2614 & 0 & 0 & 0 & 0 & 0 & 0 & 1 & 2 & 0 & 1 & 2 & 4 & 4 & 8 & 11 & 9 & 2 & 4 & $w_{266}$ & N & can. \\
2615 & 0 & 0 & 0 & 0 & 0 & 0 & 1 & 2 & 0 & 1 & 2 & 4 & 4 & 8 & 11 & 10 & 2 & 4 & $w_{241}$ & N & can. \\
2616 & 0 & 0 & 0 & 0 & 0 & 0 & 1 & 2 & 0 & 1 & 2 & 4 & 4 & 8 & 11 & 12 & 1 & 4 & $w_{281}$ & N & can. \\
2617 & 0 & 0 & 0 & 0 & 0 & 0 & 1 & 2 & 0 & 1 & 2 & 4 & 4 & 8 & 11 & 13 & 1 & 4 & $w_{268}$ & N & can. \\
2618 & 0 & 0 & 0 & 0 & 0 & 0 & 1 & 2 & 0 & 1 & 2 & 4 & 4 & 8 & 11 & 14 & 1 & 4 & $w_{281}$ & N & can. \\
2619 & 0 & 0 & 0 & 0 & 0 & 0 & 1 & 2 & 0 & 1 & 2 & 4 & 4 & 8 & 11 & 15 & 1 & 4 & $w_{281}$ & N & can. \\
2620 & 0 & 0 & 0 & 0 & 0 & 0 & 1 & 2 & 0 & 1 & 2 & 4 & 4 & 8 & 12 & 4 & 2 & 3 & $w_{237}$ & Y & can. \\
2621 & 0 & 0 & 0 & 0 & 0 & 0 & 1 & 2 & 0 & 1 & 2 & 4 & 4 & 8 & 12 & 7 & 1 & 4 & $w_{266}$ & N & can. \\
2622 & 0 & 0 & 0 & 0 & 0 & 0 & 1 & 2 & 0 & 1 & 2 & 4 & 4 & 8 & 12 & 8 & 1 & 4 & $w_{258}$ & N & can. \\
2623 & 0 & 0 & 0 & 0 & 0 & 0 & 1 & 2 & 0 & 1 & 2 & 4 & 4 & 8 & 12 & 9 & 1 & 4 & $w_{266}$ & N & can. \\
2624 & 0 & 0 & 0 & 0 & 0 & 0 & 1 & 2 & 0 & 1 & 2 & 4 & 4 & 8 & 12 & 11 & 1 & 4 & $w_{281}$ & N & can. \\
2625 & 0 & 0 & 0 & 0 & 0 & 0 & 1 & 2 & 0 & 1 & 2 & 4 & 4 & 8 & 12 & 13 & 1 & 4 & $w_{260}$ & N & can. \\
2626 & 0 & 0 & 0 & 0 & 0 & 0 & 1 & 2 & 0 & 1 & 2 & 4 & 4 & 8 & 12 & 14 & 1 & 4 & $w_{392}$ & N & \#2566 \\
2627 & 0 & 0 & 0 & 0 & 0 & 0 & 1 & 2 & 0 & 1 & 2 & 4 & 4 & 8 & 12 & 15 & 1 & 4 & $w_{267}$ & N & can. \\
2628 & 0 & 0 & 0 & 0 & 0 & 0 & 1 & 2 & 0 & 1 & 2 & 4 & 4 & 8 & 13 & 4 & 2 & 4 & $w_{262}$ & N & can. \\
2629 & 0 & 0 & 0 & 0 & 0 & 0 & 1 & 2 & 0 & 1 & 2 & 4 & 4 & 8 & 13 & 7 & 1 & 4 & $w_{392}$ & N & can. \\
2630 & 0 & 0 & 0 & 0 & 0 & 0 & 1 & 2 & 0 & 1 & 2 & 4 & 4 & 8 & 13 & 9 & 1 & 4 & $w_{392}$ & N & \#2584 \\
2631 & 0 & 0 & 0 & 0 & 0 & 0 & 1 & 2 & 0 & 1 & 2 & 4 & 4 & 8 & 13 & 10 & 1 & 4 & $w_{267}$ & N & can. \\
2632 & 0 & 0 & 0 & 0 & 0 & 0 & 1 & 2 & 0 & 1 & 2 & 4 & 4 & 8 & 13 & 11 & 1 & 4 & $w_{354}$ & N & can. \\
2633 & 0 & 0 & 0 & 0 & 0 & 0 & 1 & 2 & 0 & 1 & 2 & 4 & 4 & 8 & 13 & 12 & 2 & 4 & $w_{347}$ & N & can. \\
2634 & 0 & 0 & 0 & 0 & 0 & 0 & 1 & 2 & 0 & 1 & 2 & 4 & 4 & 8 & 13 & 14 & 1 & 4 & $w_{281}$ & N & can. \\
2635 & 0 & 0 & 0 & 0 & 0 & 0 & 1 & 2 & 0 & 1 & 2 & 4 & 4 & 8 & 13 & 15 & 1 & 4 & $w_{355}$ & N & can. \\
2636 & 0 & 0 & 0 & 0 & 0 & 0 & 1 & 2 & 0 & 1 & 2 & 4 & 4 & 8 & 14 & 4 & 2 & 4 & $w_{262}$ & N & can. \\
2637 & 0 & 0 & 0 & 0 & 0 & 0 & 1 & 2 & 0 & 1 & 2 & 4 & 4 & 8 & 14 & 7 & 1 & 4 & $w_{392}$ & N & can. \\
2638 & 0 & 0 & 0 & 0 & 0 & 0 & 1 & 2 & 0 & 1 & 2 & 4 & 4 & 8 & 14 & 8 & 1 & 4 & $w_{352}$ & N & can. \\
2639 & 0 & 0 & 0 & 0 & 0 & 0 & 1 & 2 & 0 & 1 & 2 & 4 & 4 & 8 & 14 & 9 & 1 & 4 & $w_{281}$ & N & can. \\
2640 & 0 & 0 & 0 & 0 & 0 & 0 & 1 & 2 & 0 & 1 & 2 & 4 & 4 & 8 & 14 & 10 & 1 & 4 & $w_{266}$ & N & can. \\
2641 & 0 & 0 & 0 & 0 & 0 & 0 & 1 & 2 & 0 & 1 & 2 & 4 & 4 & 8 & 14 & 11 & 1 & 4 & $w_{355}$ & N & can. \\
2642 & 0 & 0 & 0 & 0 & 0 & 0 & 1 & 2 & 0 & 1 & 2 & 4 & 4 & 8 & 14 & 13 & 1 & 4 & $w_{267}$ & N & can. \\
2643 & 0 & 0 & 0 & 0 & 0 & 0 & 1 & 2 & 0 & 1 & 2 & 4 & 4 & 8 & 14 & 15 & 1 & 4 & $w_{354}$ & N & can. \\
2644 & 0 & 0 & 0 & 0 & 0 & 0 & 1 & 2 & 0 & 1 & 2 & 4 & 4 & 8 & 15 & 4 & 2 & 4 & $w_{241}$ & N & can. \\
2645 & 0 & 0 & 0 & 0 & 0 & 0 & 1 & 2 & 0 & 1 & 2 & 4 & 4 & 8 & 15 & 7 & 1 & 3 & $w_{280}$ & Y & \#2571 \\
2646 & 0 & 0 & 0 & 0 & 0 & 0 & 1 & 2 & 0 & 1 & 2 & 4 & 4 & 8 & 15 & 8 & 1 & 4 & $w_{267}$ & N & can. \\
2647 & 0 & 0 & 0 & 0 & 0 & 0 & 1 & 2 & 0 & 1 & 2 & 4 & 4 & 8 & 15 & 9 & 1 & 4 & $w_{267}$ & N & can. \\
2648 & 0 & 0 & 0 & 0 & 0 & 0 & 1 & 2 & 0 & 1 & 2 & 4 & 4 & 8 & 15 & 10 & 1 & 4 & $w_{267}$ & N & can. \\
2649 & 0 & 0 & 0 & 0 & 0 & 0 & 1 & 2 & 0 & 1 & 2 & 4 & 4 & 8 & 15 & 11 & 1 & 4 & $w_{379}$ & N & can. \\
2650 & 0 & 0 & 0 & 0 & 0 & 0 & 1 & 2 & 0 & 1 & 2 & 4 & 4 & 8 & 15 & 13 & 1 & 4 & $w_{281}$ & N & can. \\
2651 & 0 & 0 & 0 & 0 & 0 & 0 & 1 & 2 & 0 & 1 & 2 & 4 & 4 & 8 & 15 & 14 & 1 & 4 & $w_{281}$ & N & can. \\
2652 & 0 & 0 & 0 & 0 & 0 & 0 & 1 & 2 & 0 & 1 & 2 & 4 & 5 & 6 & 8 & 9 & 1 & 4 & $w_{266}$ & N & can. \\
2653 & 0 & 0 & 0 & 0 & 0 & 0 & 1 & 2 & 0 & 1 & 2 & 4 & 5 & 6 & 8 & 10 & 1 & 4 & $w_{392}$ & N & can. \\
2654 & 0 & 0 & 0 & 0 & 0 & 0 & 1 & 2 & 0 & 1 & 2 & 4 & 5 & 6 & 8 & 11 & 1 & 4 & $w_{260}$ & N & can. \\
2655 & 0 & 0 & 0 & 0 & 0 & 0 & 1 & 2 & 0 & 1 & 2 & 4 & 5 & 6 & 8 & 12 & 1 & 4 & $w_{392}$ & N & can. \\
2656 & 0 & 0 & 0 & 0 & 0 & 0 & 1 & 2 & 0 & 1 & 2 & 4 & 5 & 6 & 8 & 13 & 1 & 4 & $w_{392}$ & N & can. \\
2657 & 0 & 0 & 0 & 0 & 0 & 0 & 1 & 2 & 0 & 1 & 2 & 4 & 5 & 6 & 8 & 14 & 1 & 4 & $w_{266}$ & N & can. \\
2658 & 0 & 0 & 0 & 0 & 0 & 0 & 1 & 2 & 0 & 1 & 2 & 4 & 5 & 6 & 8 & 15 & 1 & 3 & $w_{280}$ & Y & can. \\
2659 & 0 & 0 & 0 & 0 & 0 & 0 & 1 & 2 & 0 & 1 & 2 & 4 & 5 & 7 & 8 & 9 & 1 & 4 & $w_{419}$ & N & can. \\
2660 & 0 & 0 & 0 & 0 & 0 & 0 & 1 & 2 & 0 & 1 & 2 & 4 & 5 & 7 & 8 & 11 & 2 & 4 & $w_{423}$ & N & can. \\
2661 & 0 & 0 & 0 & 0 & 0 & 0 & 1 & 2 & 0 & 1 & 2 & 4 & 5 & 7 & 8 & 12 & 1 & 4 & $w_{425}$ & N & can. \\
2662 & 0 & 0 & 0 & 0 & 0 & 0 & 1 & 2 & 0 & 1 & 2 & 4 & 5 & 7 & 8 & 13 & 2 & 4 & $w_{426}$ & N & can. \\
2663 & 0 & 0 & 0 & 0 & 0 & 0 & 1 & 2 & 0 & 1 & 2 & 4 & 5 & 7 & 8 & 14 & 2 & 3 & $w_{378}$ & N & can. \\
2664 & 0 & 0 & 0 & 0 & 0 & 0 & 1 & 2 & 0 & 1 & 2 & 4 & 5 & 8 & 6 & 8 & 1 & 4 & $w_{353}$ & N & can. \\
2665 & 0 & 0 & 0 & 0 & 0 & 0 & 1 & 2 & 0 & 1 & 2 & 4 & 5 & 8 & 6 & 9 & 1 & 4 & $w_{392}$ & N & can. \\
2666 & 0 & 0 & 0 & 0 & 0 & 0 & 1 & 2 & 0 & 1 & 2 & 4 & 5 & 8 & 6 & 10 & 1 & 4 & $w_{266}$ & N & can. \\
2667 & 0 & 0 & 0 & 0 & 0 & 0 & 1 & 2 & 0 & 1 & 2 & 4 & 5 & 8 & 6 & 11 & 1 & 4 & $w_{347}$ & N & can. \\
2668 & 0 & 0 & 0 & 0 & 0 & 0 & 1 & 2 & 0 & 1 & 2 & 4 & 5 & 8 & 6 & 12 & 1 & 4 & $w_{419}$ & N & can. \\
2669 & 0 & 0 & 0 & 0 & 0 & 0 & 1 & 2 & 0 & 1 & 2 & 4 & 5 & 8 & 6 & 13 & 1 & 4 & $w_{392}$ & N & can. \\
2670 & 0 & 0 & 0 & 0 & 0 & 0 & 1 & 2 & 0 & 1 & 2 & 4 & 5 & 8 & 6 & 14 & 1 & 4 & $w_{423}$ & N & can. \\
2671 & 0 & 0 & 0 & 0 & 0 & 0 & 1 & 2 & 0 & 1 & 2 & 4 & 5 & 8 & 6 & 15 & 1 & 3 & $w_{378}$ & N & can. \\
2672 & 0 & 0 & 0 & 0 & 0 & 0 & 1 & 2 & 0 & 1 & 2 & 4 & 5 & 8 & 8 & 6 & 1 & 4 & $w_{260}$ & N & can. \\
2673 & 0 & 0 & 0 & 0 & 0 & 0 & 1 & 2 & 0 & 1 & 2 & 4 & 5 & 8 & 8 & 7 & 1 & 4 & $w_{266}$ & N & can. \\
2674 & 0 & 0 & 0 & 0 & 0 & 0 & 1 & 2 & 0 & 1 & 2 & 4 & 5 & 8 & 8 & 11 & 1 & 4 & $w_{266}$ & N & can. \\
2675 & 0 & 0 & 0 & 0 & 0 & 0 & 1 & 2 & 0 & 1 & 2 & 4 & 5 & 8 & 8 & 13 & 1 & 4 & $w_{266}$ & N & can. \\
2676 & 0 & 0 & 0 & 0 & 0 & 0 & 1 & 2 & 0 & 1 & 2 & 4 & 5 & 8 & 8 & 14 & 1 & 4 & $w_{267}$ & N & can. \\
2677 & 0 & 0 & 0 & 0 & 0 & 0 & 1 & 2 & 0 & 1 & 2 & 4 & 5 & 8 & 8 & 15 & 1 & 4 & $w_{281}$ & N & can. \\
2678 & 0 & 0 & 0 & 0 & 0 & 0 & 1 & 2 & 0 & 1 & 2 & 4 & 5 & 8 & 9 & 6 & 1 & 4 & $w_{266}$ & N & can. \\
2679 & 0 & 0 & 0 & 0 & 0 & 0 & 1 & 2 & 0 & 1 & 2 & 4 & 5 & 8 & 9 & 7 & 1 & 4 & $w_{347}$ & N & can. \\
2680 & 0 & 0 & 0 & 0 & 0 & 0 & 1 & 2 & 0 & 1 & 2 & 4 & 5 & 8 & 9 & 10 & 1 & 4 & $w_{260}$ & N & can. \\
2681 & 0 & 0 & 0 & 0 & 0 & 0 & 1 & 2 & 0 & 1 & 2 & 4 & 5 & 8 & 9 & 11 & 1 & 4 & $w_{419}$ & N & can. \\
2682 & 0 & 0 & 0 & 0 & 0 & 0 & 1 & 2 & 0 & 1 & 2 & 4 & 5 & 8 & 9 & 12 & 1 & 4 & $w_{419}$ & N & can. \\
2683 & 0 & 0 & 0 & 0 & 0 & 0 & 1 & 2 & 0 & 1 & 2 & 4 & 5 & 8 & 9 & 13 & 1 & 4 & $w_{392}$ & N & can. \\
2684 & 0 & 0 & 0 & 0 & 0 & 0 & 1 & 2 & 0 & 1 & 2 & 4 & 5 & 8 & 9 & 14 & 1 & 4 & $w_{379}$ & N & can. \\
2685 & 0 & 0 & 0 & 0 & 0 & 0 & 1 & 2 & 0 & 1 & 2 & 4 & 5 & 8 & 9 & 15 & 1 & 4 & $w_{354}$ & N & can. \\
2686 & 0 & 0 & 0 & 0 & 0 & 0 & 1 & 2 & 0 & 1 & 2 & 4 & 5 & 8 & 10 & 6 & 1 & 4 & $w_{266}$ & N & can. \\
2687 & 0 & 0 & 0 & 0 & 0 & 0 & 1 & 2 & 0 & 1 & 2 & 4 & 5 & 8 & 10 & 7 & 1 & 4 & $w_{353}$ & N & can. \\
2688 & 0 & 0 & 0 & 0 & 0 & 0 & 1 & 2 & 0 & 1 & 2 & 4 & 5 & 8 & 10 & 9 & 1 & 4 & $w_{266}$ & N & can. \\
2689 & 0 & 0 & 0 & 0 & 0 & 0 & 1 & 2 & 0 & 1 & 2 & 4 & 5 & 8 & 10 & 11 & 2 & 4 & $w_{347}$ & N & can. \\
2690 & 0 & 0 & 0 & 0 & 0 & 0 & 1 & 2 & 0 & 1 & 2 & 4 & 5 & 8 & 10 & 12 & 1 & 4 & $w_{354}$ & N & can. \\
2691 & 0 & 0 & 0 & 0 & 0 & 0 & 1 & 2 & 0 & 1 & 2 & 4 & 5 & 8 & 10 & 13 & 1 & 4 & $w_{281}$ & N & can. \\
2692 & 0 & 0 & 0 & 0 & 0 & 0 & 1 & 2 & 0 & 1 & 2 & 4 & 5 & 8 & 10 & 14 & 1 & 4 & $w_{379}$ & N & can. \\
2693 & 0 & 0 & 0 & 0 & 0 & 0 & 1 & 2 & 0 & 1 & 2 & 4 & 5 & 8 & 10 & 15 & 1 & 4 & $w_{355}$ & N & can. \\
2694 & 0 & 0 & 0 & 0 & 0 & 0 & 1 & 2 & 0 & 1 & 2 & 4 & 5 & 8 & 11 & 6 & 1 & 4 & $w_{260}$ & N & can. \\
2695 & 0 & 0 & 0 & 0 & 0 & 0 & 1 & 2 & 0 & 1 & 2 & 4 & 5 & 8 & 11 & 7 & 1 & 4 & $w_{392}$ & N & can. \\
2696 & 0 & 0 & 0 & 0 & 0 & 0 & 1 & 2 & 0 & 1 & 2 & 4 & 5 & 8 & 11 & 8 & 1 & 4 & $w_{260}$ & N & can. \\
2697 & 0 & 0 & 0 & 0 & 0 & 0 & 1 & 2 & 0 & 1 & 2 & 4 & 5 & 8 & 11 & 9 & 1 & 4 & $w_{392}$ & N & can. \\
2698 & 0 & 0 & 0 & 0 & 0 & 0 & 1 & 2 & 0 & 1 & 2 & 4 & 5 & 8 & 11 & 10 & 2 & 4 & $w_{260}$ & N & can. \\
2699 & 0 & 0 & 0 & 0 & 0 & 0 & 1 & 2 & 0 & 1 & 2 & 4 & 5 & 8 & 11 & 12 & 1 & 4 & $w_{379}$ & N & can. \\
2700 & 0 & 0 & 0 & 0 & 0 & 0 & 1 & 2 & 0 & 1 & 2 & 4 & 5 & 8 & 11 & 13 & 2 & 4 & $w_{267}$ & N & can. \\
2701 & 0 & 0 & 0 & 0 & 0 & 0 & 1 & 2 & 0 & 1 & 2 & 4 & 5 & 8 & 11 & 14 & 1 & 4 & $w_{379}$ & N & can. \\
2702 & 0 & 0 & 0 & 0 & 0 & 0 & 1 & 2 & 0 & 1 & 2 & 4 & 5 & 8 & 11 & 15 & 1 & 4 & $w_{379}$ & N & can. \\
2703 & 0 & 0 & 0 & 0 & 0 & 0 & 1 & 2 & 0 & 1 & 2 & 4 & 5 & 8 & 12 & 6 & 1 & 4 & $w_{266}$ & N & can. \\
2704 & 0 & 0 & 0 & 0 & 0 & 0 & 1 & 2 & 0 & 1 & 2 & 4 & 5 & 8 & 12 & 7 & 1 & 4 & $w_{392}$ & N & can. \\
2705 & 0 & 0 & 0 & 0 & 0 & 0 & 1 & 2 & 0 & 1 & 2 & 4 & 5 & 8 & 12 & 8 & 1 & 4 & $w_{266}$ & N & can. \\
2706 & 0 & 0 & 0 & 0 & 0 & 0 & 1 & 2 & 0 & 1 & 2 & 4 & 5 & 8 & 12 & 9 & 1 & 4 & $w_{392}$ & N & can. \\
2707 & 0 & 0 & 0 & 0 & 0 & 0 & 1 & 2 & 0 & 1 & 2 & 4 & 5 & 8 & 12 & 10 & 1 & 4 & $w_{268}$ & N & can. \\
2708 & 0 & 0 & 0 & 0 & 0 & 0 & 1 & 2 & 0 & 1 & 2 & 4 & 5 & 8 & 12 & 11 & 1 & 4 & $w_{379}$ & N & can. \\
2709 & 0 & 0 & 0 & 0 & 0 & 0 & 1 & 2 & 0 & 1 & 2 & 4 & 5 & 8 & 12 & 13 & 2 & 4 & $w_{266}$ & N & can. \\
2710 & 0 & 0 & 0 & 0 & 0 & 0 & 1 & 2 & 0 & 1 & 2 & 4 & 5 & 8 & 12 & 14 & 1 & 4 & $w_{379}$ & N & can. \\
2711 & 0 & 0 & 0 & 0 & 0 & 0 & 1 & 2 & 0 & 1 & 2 & 4 & 5 & 8 & 12 & 15 & 1 & 4 & $w_{281}$ & N & can. \\
2712 & 0 & 0 & 0 & 0 & 0 & 0 & 1 & 2 & 0 & 1 & 2 & 4 & 5 & 8 & 13 & 6 & 1 & 4 & $w_{392}$ & N & can. \\
2713 & 0 & 0 & 0 & 0 & 0 & 0 & 1 & 2 & 0 & 1 & 2 & 4 & 5 & 8 & 13 & 7 & 1 & 4 & $w_{419}$ & N & can. \\
2714 & 0 & 0 & 0 & 0 & 0 & 0 & 1 & 2 & 0 & 1 & 2 & 4 & 5 & 8 & 13 & 9 & 1 & 4 & $w_{423}$ & N & can. \\
2715 & 0 & 0 & 0 & 0 & 0 & 0 & 1 & 2 & 0 & 1 & 2 & 4 & 5 & 8 & 13 & 10 & 1 & 4 & $w_{281}$ & N & can. \\
2716 & 0 & 0 & 0 & 0 & 0 & 0 & 1 & 2 & 0 & 1 & 2 & 4 & 5 & 8 & 13 & 11 & 1 & 4 & $w_{355}$ & N & can. \\
2717 & 0 & 0 & 0 & 0 & 0 & 0 & 1 & 2 & 0 & 1 & 2 & 4 & 5 & 8 & 13 & 12 & 1 & 4 & $w_{419}$ & N & can. \\
2718 & 0 & 0 & 0 & 0 & 0 & 0 & 1 & 2 & 0 & 1 & 2 & 4 & 5 & 8 & 13 & 14 & 1 & 4 & $w_{379}$ & N & \#2708 \\
2719 & 0 & 0 & 0 & 0 & 0 & 0 & 1 & 2 & 0 & 1 & 2 & 4 & 5 & 8 & 13 & 15 & 1 & 4 & $w_{427}$ & N & can. \\
2720 & 0 & 0 & 0 & 0 & 0 & 0 & 1 & 2 & 0 & 1 & 2 & 4 & 5 & 8 & 14 & 6 & 1 & 4 & $w_{392}$ & N & can. \\
2721 & 0 & 0 & 0 & 0 & 0 & 0 & 1 & 2 & 0 & 1 & 2 & 4 & 5 & 8 & 14 & 7 & 1 & 3 & $w_{378}$ & N & can. \\
2722 & 0 & 0 & 0 & 0 & 0 & 0 & 1 & 2 & 0 & 1 & 2 & 4 & 5 & 8 & 14 & 8 & 1 & 4 & $w_{373}$ & N & can. \\
2723 & 0 & 0 & 0 & 0 & 0 & 0 & 1 & 2 & 0 & 1 & 2 & 4 & 5 & 8 & 14 & 9 & 1 & 4 & $w_{379}$ & N & can. \\
2724 & 0 & 0 & 0 & 0 & 0 & 0 & 1 & 2 & 0 & 1 & 2 & 4 & 5 & 8 & 14 & 10 & 1 & 4 & $w_{281}$ & N & can. \\
2725 & 0 & 0 & 0 & 0 & 0 & 0 & 1 & 2 & 0 & 1 & 2 & 4 & 5 & 8 & 14 & 11 & 1 & 4 & $w_{427}$ & N & can. \\
2726 & 0 & 0 & 0 & 0 & 0 & 0 & 1 & 2 & 0 & 1 & 2 & 4 & 5 & 8 & 14 & 12 & 1 & 4 & $w_{427}$ & N & can. \\
2727 & 0 & 0 & 0 & 0 & 0 & 0 & 1 & 2 & 0 & 1 & 2 & 4 & 5 & 8 & 14 & 13 & 1 & 4 & $w_{281}$ & N & can. \\
2728 & 0 & 0 & 0 & 0 & 0 & 0 & 1 & 2 & 0 & 1 & 2 & 4 & 5 & 8 & 14 & 15 & 1 & 4 & $w_{355}$ & N & can. \\
2729 & 0 & 0 & 0 & 0 & 0 & 0 & 1 & 2 & 0 & 1 & 2 & 4 & 5 & 8 & 15 & 6 & 1 & 3 & $w_{280}$ & Y & can. \\
2730 & 0 & 0 & 0 & 0 & 0 & 0 & 1 & 2 & 0 & 1 & 2 & 4 & 5 & 8 & 15 & 7 & 1 & 4 & $w_{423}$ & N & can. \\
2731 & 0 & 0 & 0 & 0 & 0 & 0 & 1 & 2 & 0 & 1 & 2 & 4 & 5 & 8 & 15 & 8 & 1 & 4 & $w_{281}$ & N & can. \\
2732 & 0 & 0 & 0 & 0 & 0 & 0 & 1 & 2 & 0 & 1 & 2 & 4 & 5 & 8 & 15 & 9 & 1 & 4 & $w_{281}$ & N & can. \\
2733 & 0 & 0 & 0 & 0 & 0 & 0 & 1 & 2 & 0 & 1 & 2 & 4 & 5 & 8 & 15 & 10 & 1 & 4 & $w_{281}$ & N & can. \\
2734 & 0 & 0 & 0 & 0 & 0 & 0 & 1 & 2 & 0 & 1 & 2 & 4 & 5 & 8 & 15 & 11 & 1 & 4 & $w_{428}$ & N & can. \\
2735 & 0 & 0 & 0 & 0 & 0 & 0 & 1 & 2 & 0 & 1 & 2 & 4 & 5 & 8 & 15 & 12 & 1 & 4 & $w_{379}$ & N & can. \\
2736 & 0 & 0 & 0 & 0 & 0 & 0 & 1 & 2 & 0 & 1 & 2 & 4 & 5 & 8 & 15 & 13 & 1 & 4 & $w_{379}$ & N & can. \\
2737 & 0 & 0 & 0 & 0 & 0 & 0 & 1 & 2 & 0 & 1 & 2 & 4 & 5 & 8 & 15 & 14 & 1 & 4 & $w_{379}$ & N & can. \\
2738 & 0 & 0 & 0 & 0 & 0 & 0 & 1 & 2 & 0 & 1 & 2 & 4 & 6 & 7 & 8 & 9 & 2 & 4 & $w_{353}$ & N & can. \\
2739 & 0 & 0 & 0 & 0 & 0 & 0 & 1 & 2 & 0 & 1 & 2 & 4 & 6 & 7 & 8 & 10 & 2 & 4 & $w_{419}$ & N & can. \\
2740 & 0 & 0 & 0 & 0 & 0 & 0 & 1 & 2 & 0 & 1 & 2 & 4 & 6 & 7 & 8 & 11 & 2 & 4 & $w_{392}$ & N & can. \\
2741 & 0 & 0 & 0 & 0 & 0 & 0 & 1 & 2 & 0 & 1 & 2 & 4 & 6 & 7 & 8 & 12 & 2 & 4 & $w_{423}$ & N & can. \\
2742 & 0 & 0 & 0 & 0 & 0 & 0 & 1 & 2 & 0 & 1 & 2 & 4 & 6 & 7 & 8 & 13 & 2 & 3 & $w_{378}$ & N & can. \\
2743 & 0 & 0 & 0 & 0 & 0 & 0 & 1 & 2 & 0 & 1 & 2 & 4 & 6 & 7 & 8 & 14 & 2 & 4 & $w_{347}$ & N & can. \\
2744 & 0 & 0 & 0 & 0 & 0 & 0 & 1 & 2 & 0 & 1 & 2 & 4 & 6 & 7 & 8 & 15 & 2 & 4 & $w_{423}$ & N & can. \\
2745 & 0 & 0 & 0 & 0 & 0 & 0 & 1 & 2 & 0 & 1 & 2 & 4 & 6 & 8 & 6 & 9 & 2 & 4 & $w_{260}$ & N & can. \\
2746 & 0 & 0 & 0 & 0 & 0 & 0 & 1 & 2 & 0 & 1 & 2 & 4 & 6 & 8 & 6 & 10 & 4 & 4 & $w_{262}$ & N & can. \\
2747 & 0 & 0 & 0 & 0 & 0 & 0 & 1 & 2 & 0 & 1 & 2 & 4 & 6 & 8 & 6 & 11 & 2 & 4 & $w_{353}$ & N & can. \\
2748 & 0 & 0 & 0 & 0 & 0 & 0 & 1 & 2 & 0 & 1 & 2 & 4 & 6 & 8 & 6 & 12 & 4 & 3 & $w_{345}$ & N & can. \\
2749 & 0 & 0 & 0 & 0 & 0 & 0 & 1 & 2 & 0 & 1 & 2 & 4 & 6 & 8 & 6 & 14 & 4 & 4 & $w_{351}$ & N & can. \\
2750 & 0 & 0 & 0 & 0 & 0 & 0 & 1 & 2 & 0 & 1 & 2 & 4 & 6 & 8 & 7 & 8 & 1 & 4 & $w_{266}$ & N & can. \\
2751 & 0 & 0 & 0 & 0 & 0 & 0 & 1 & 2 & 0 & 1 & 2 & 4 & 6 & 8 & 7 & 9 & 1 & 4 & $w_{266}$ & N & can. \\
2752 & 0 & 0 & 0 & 0 & 0 & 0 & 1 & 2 & 0 & 1 & 2 & 4 & 6 & 8 & 7 & 10 & 1 & 4 & $w_{266}$ & N & can. \\
2753 & 0 & 0 & 0 & 0 & 0 & 0 & 1 & 2 & 0 & 1 & 2 & 4 & 6 & 8 & 7 & 11 & 1 & 4 & $w_{423}$ & N & can. \\
2754 & 0 & 0 & 0 & 0 & 0 & 0 & 1 & 2 & 0 & 1 & 2 & 4 & 6 & 8 & 7 & 12 & 1 & 4 & $w_{423}$ & N & can. \\
2755 & 0 & 0 & 0 & 0 & 0 & 0 & 1 & 2 & 0 & 1 & 2 & 4 & 6 & 8 & 7 & 13 & 1 & 3 & $w_{280}$ & Y & can. \\
2756 & 0 & 0 & 0 & 0 & 0 & 0 & 1 & 2 & 0 & 1 & 2 & 4 & 6 & 8 & 7 & 14 & 1 & 4 & $w_{423}$ & N & can. \\
2757 & 0 & 0 & 0 & 0 & 0 & 0 & 1 & 2 & 0 & 1 & 2 & 4 & 6 & 8 & 7 & 15 & 1 & 4 & $w_{423}$ & N & can. \\
2758 & 0 & 0 & 0 & 0 & 0 & 0 & 1 & 2 & 0 & 1 & 2 & 4 & 6 & 8 & 8 & 7 & 1 & 4 & $w_{266}$ & N & can. \\
2759 & 0 & 0 & 0 & 0 & 0 & 0 & 1 & 2 & 0 & 1 & 2 & 4 & 6 & 8 & 8 & 11 & 1 & 4 & $w_{266}$ & N & can. \\
2760 & 0 & 0 & 0 & 0 & 0 & 0 & 1 & 2 & 0 & 1 & 2 & 4 & 6 & 8 & 8 & 13 & 1 & 4 & $w_{267}$ & N & can. \\
2761 & 0 & 0 & 0 & 0 & 0 & 0 & 1 & 2 & 0 & 1 & 2 & 4 & 6 & 8 & 8 & 14 & 1 & 4 & $w_{266}$ & N & can. \\
2762 & 0 & 0 & 0 & 0 & 0 & 0 & 1 & 2 & 0 & 1 & 2 & 4 & 6 & 8 & 8 & 15 & 1 & 4 & $w_{281}$ & N & can. \\
2763 & 0 & 0 & 0 & 0 & 0 & 0 & 1 & 2 & 0 & 1 & 2 & 4 & 6 & 8 & 9 & 6 & 2 & 4 & $w_{260}$ & N & can. \\
2764 & 0 & 0 & 0 & 0 & 0 & 0 & 1 & 2 & 0 & 1 & 2 & 4 & 6 & 8 & 9 & 7 & 1 & 4 & $w_{353}$ & N & can. \\
2765 & 0 & 0 & 0 & 0 & 0 & 0 & 1 & 2 & 0 & 1 & 2 & 4 & 6 & 8 & 9 & 10 & 1 & 4 & $w_{260}$ & N & can. \\
2766 & 0 & 0 & 0 & 0 & 0 & 0 & 1 & 2 & 0 & 1 & 2 & 4 & 6 & 8 & 9 & 12 & 1 & 4 & $w_{355}$ & N & can. \\
2767 & 0 & 0 & 0 & 0 & 0 & 0 & 1 & 2 & 0 & 1 & 2 & 4 & 6 & 8 & 9 & 13 & 1 & 4 & $w_{281}$ & N & can. \\
2768 & 0 & 0 & 0 & 0 & 0 & 0 & 1 & 2 & 0 & 1 & 2 & 4 & 6 & 8 & 9 & 14 & 1 & 4 & $w_{379}$ & N & can. \\
2769 & 0 & 0 & 0 & 0 & 0 & 0 & 1 & 2 & 0 & 1 & 2 & 4 & 6 & 8 & 9 & 15 & 1 & 4 & $w_{354}$ & N & can. \\
2770 & 0 & 0 & 0 & 0 & 0 & 0 & 1 & 2 & 0 & 1 & 2 & 4 & 6 & 8 & 10 & 6 & 2 & 4 & $w_{258}$ & N & can. \\
2771 & 0 & 0 & 0 & 0 & 0 & 0 & 1 & 2 & 0 & 1 & 2 & 4 & 6 & 8 & 10 & 7 & 1 & 4 & $w_{347}$ & N & can. \\
2772 & 0 & 0 & 0 & 0 & 0 & 0 & 1 & 2 & 0 & 1 & 2 & 4 & 6 & 8 & 10 & 9 & 1 & 4 & $w_{266}$ & N & can. \\
2773 & 0 & 0 & 0 & 0 & 0 & 0 & 1 & 2 & 0 & 1 & 2 & 4 & 6 & 8 & 10 & 11 & 2 & 4 & $w_{347}$ & N & can. \\
2774 & 0 & 0 & 0 & 0 & 0 & 0 & 1 & 2 & 0 & 1 & 2 & 4 & 6 & 8 & 10 & 12 & 1 & 4 & $w_{347}$ & N & can. \\
2775 & 0 & 0 & 0 & 0 & 0 & 0 & 1 & 2 & 0 & 1 & 2 & 4 & 6 & 8 & 10 & 13 & 1 & 4 & $w_{281}$ & N & can. \\
2776 & 0 & 0 & 0 & 0 & 0 & 0 & 1 & 2 & 0 & 1 & 2 & 4 & 6 & 8 & 10 & 14 & 1 & 4 & $w_{423}$ & N & can. \\
2777 & 0 & 0 & 0 & 0 & 0 & 0 & 1 & 2 & 0 & 1 & 2 & 4 & 6 & 8 & 10 & 15 & 1 & 4 & $w_{355}$ & N & can. \\
2778 & 0 & 0 & 0 & 0 & 0 & 0 & 1 & 2 & 0 & 1 & 2 & 4 & 6 & 8 & 11 & 6 & 2 & 4 & $w_{260}$ & N & can. \\
2779 & 0 & 0 & 0 & 0 & 0 & 0 & 1 & 2 & 0 & 1 & 2 & 4 & 6 & 8 & 11 & 7 & 1 & 4 & $w_{392}$ & N & can. \\
2780 & 0 & 0 & 0 & 0 & 0 & 0 & 1 & 2 & 0 & 1 & 2 & 4 & 6 & 8 & 11 & 8 & 1 & 4 & $w_{260}$ & N & can. \\
2781 & 0 & 0 & 0 & 0 & 0 & 0 & 1 & 2 & 0 & 1 & 2 & 4 & 6 & 8 & 11 & 10 & 2 & 4 & $w_{260}$ & N & can. \\
2782 & 0 & 0 & 0 & 0 & 0 & 0 & 1 & 2 & 0 & 1 & 2 & 4 & 6 & 8 & 11 & 12 & 1 & 4 & $w_{379}$ & N & can. \\
2783 & 0 & 0 & 0 & 0 & 0 & 0 & 1 & 2 & 0 & 1 & 2 & 4 & 6 & 8 & 11 & 13 & 1 & 4 & $w_{267}$ & N & can. \\
2784 & 0 & 0 & 0 & 0 & 0 & 0 & 1 & 2 & 0 & 1 & 2 & 4 & 6 & 8 & 11 & 14 & 1 & 4 & $w_{379}$ & N & can. \\
2785 & 0 & 0 & 0 & 0 & 0 & 0 & 1 & 2 & 0 & 1 & 2 & 4 & 6 & 8 & 11 & 15 & 1 & 4 & $w_{379}$ & N & can. \\
2786 & 0 & 0 & 0 & 0 & 0 & 0 & 1 & 2 & 0 & 1 & 2 & 4 & 6 & 8 & 12 & 6 & 2 & 3 & $w_{259}$ & N & can. \\
2787 & 0 & 0 & 0 & 0 & 0 & 0 & 1 & 2 & 0 & 1 & 2 & 4 & 6 & 8 & 12 & 7 & 1 & 4 & $w_{392}$ & N & can. \\
2788 & 0 & 0 & 0 & 0 & 0 & 0 & 1 & 2 & 0 & 1 & 2 & 4 & 6 & 8 & 12 & 8 & 1 & 4 & $w_{266}$ & N & can. \\
2789 & 0 & 0 & 0 & 0 & 0 & 0 & 1 & 2 & 0 & 1 & 2 & 4 & 6 & 8 & 12 & 9 & 1 & 4 & $w_{281}$ & N & can. \\
2790 & 0 & 0 & 0 & 0 & 0 & 0 & 1 & 2 & 0 & 1 & 2 & 4 & 6 & 8 & 12 & 11 & 1 & 4 & $w_{379}$ & N & can. \\
2791 & 0 & 0 & 0 & 0 & 0 & 0 & 1 & 2 & 0 & 1 & 2 & 4 & 6 & 8 & 12 & 13 & 1 & 4 & $w_{267}$ & N & can. \\
2792 & 0 & 0 & 0 & 0 & 0 & 0 & 1 & 2 & 0 & 1 & 2 & 4 & 6 & 8 & 12 & 14 & 1 & 4 & $w_{423}$ & N & can. \\
2793 & 0 & 0 & 0 & 0 & 0 & 0 & 1 & 2 & 0 & 1 & 2 & 4 & 6 & 8 & 12 & 15 & 1 & 4 & $w_{281}$ & N & can. \\
2794 & 0 & 0 & 0 & 0 & 0 & 0 & 1 & 2 & 0 & 1 & 2 & 4 & 6 & 8 & 13 & 6 & 2 & 4 & $w_{266}$ & N & can. \\
2795 & 0 & 0 & 0 & 0 & 0 & 0 & 1 & 2 & 0 & 1 & 2 & 4 & 6 & 8 & 13 & 7 & 1 & 3 & $w_{378}$ & N & can. \\
2796 & 0 & 0 & 0 & 0 & 0 & 0 & 1 & 2 & 0 & 1 & 2 & 4 & 6 & 8 & 13 & 9 & 1 & 4 & $w_{379}$ & N & can. \\
2797 & 0 & 0 & 0 & 0 & 0 & 0 & 1 & 2 & 0 & 1 & 2 & 4 & 6 & 8 & 13 & 10 & 1 & 4 & $w_{281}$ & N & can. \\
2798 & 0 & 0 & 0 & 0 & 0 & 0 & 1 & 2 & 0 & 1 & 2 & 4 & 6 & 8 & 13 & 11 & 1 & 4 & $w_{355}$ & N & can. \\
2799 & 0 & 0 & 0 & 0 & 0 & 0 & 1 & 2 & 0 & 1 & 2 & 4 & 6 & 8 & 13 & 12 & 1 & 4 & $w_{355}$ & N & can. \\
2800 & 0 & 0 & 0 & 0 & 0 & 0 & 1 & 2 & 0 & 1 & 2 & 4 & 6 & 8 & 13 & 14 & 1 & 4 & $w_{379}$ & N & can. \\
2801 & 0 & 0 & 0 & 0 & 0 & 0 & 1 & 2 & 0 & 1 & 2 & 4 & 6 & 8 & 13 & 15 & 1 & 4 & $w_{427}$ & N & can. \\
2802 & 0 & 0 & 0 & 0 & 0 & 0 & 1 & 2 & 0 & 1 & 2 & 4 & 6 & 8 & 14 & 6 & 2 & 4 & $w_{351}$ & N & can. \\
2803 & 0 & 0 & 0 & 0 & 0 & 0 & 1 & 2 & 0 & 1 & 2 & 4 & 6 & 8 & 14 & 7 & 1 & 4 & $w_{419}$ & N & can. \\
2804 & 0 & 0 & 0 & 0 & 0 & 0 & 1 & 2 & 0 & 1 & 2 & 4 & 6 & 8 & 14 & 8 & 1 & 4 & $w_{353}$ & N & can. \\
2805 & 0 & 0 & 0 & 0 & 0 & 0 & 1 & 2 & 0 & 1 & 2 & 4 & 6 & 8 & 14 & 9 & 1 & 4 & $w_{379}$ & N & can. \\
2806 & 0 & 0 & 0 & 0 & 0 & 0 & 1 & 2 & 0 & 1 & 2 & 4 & 6 & 8 & 14 & 10 & 1 & 4 & $w_{392}$ & N & can. \\
2807 & 0 & 0 & 0 & 0 & 0 & 0 & 1 & 2 & 0 & 1 & 2 & 4 & 6 & 8 & 14 & 11 & 1 & 4 & $w_{427}$ & N & can. \\
2808 & 0 & 0 & 0 & 0 & 0 & 0 & 1 & 2 & 0 & 1 & 2 & 4 & 6 & 8 & 14 & 12 & 1 & 4 & $w_{426}$ & N & can. \\
2809 & 0 & 0 & 0 & 0 & 0 & 0 & 1 & 2 & 0 & 1 & 2 & 4 & 6 & 8 & 14 & 13 & 1 & 4 & $w_{281}$ & N & can. \\
2810 & 0 & 0 & 0 & 0 & 0 & 0 & 1 & 2 & 0 & 1 & 2 & 4 & 6 & 8 & 14 & 15 & 1 & 4 & $w_{355}$ & N & can. \\
2811 & 0 & 0 & 0 & 0 & 0 & 0 & 1 & 2 & 0 & 1 & 2 & 4 & 6 & 8 & 15 & 6 & 2 & 4 & $w_{266}$ & N & can. \\
2812 & 0 & 0 & 0 & 0 & 0 & 0 & 1 & 2 & 0 & 1 & 2 & 4 & 6 & 8 & 15 & 7 & 1 & 4 & $w_{423}$ & N & can. \\
2813 & 0 & 0 & 0 & 0 & 0 & 0 & 1 & 2 & 0 & 1 & 2 & 4 & 6 & 8 & 15 & 8 & 1 & 4 & $w_{281}$ & N & can. \\
2814 & 0 & 0 & 0 & 0 & 0 & 0 & 1 & 2 & 0 & 1 & 2 & 4 & 6 & 8 & 15 & 9 & 1 & 4 & $w_{281}$ & N & can. \\
2815 & 0 & 0 & 0 & 0 & 0 & 0 & 1 & 2 & 0 & 1 & 2 & 4 & 6 & 8 & 15 & 10 & 1 & 4 & $w_{281}$ & N & can. \\
2816 & 0 & 0 & 0 & 0 & 0 & 0 & 1 & 2 & 0 & 1 & 2 & 4 & 6 & 8 & 15 & 11 & 1 & 4 & $w_{428}$ & N & can. \\
2817 & 0 & 0 & 0 & 0 & 0 & 0 & 1 & 2 & 0 & 1 & 2 & 4 & 6 & 8 & 15 & 12 & 1 & 4 & $w_{379}$ & N & can. \\
2818 & 0 & 0 & 0 & 0 & 0 & 0 & 1 & 2 & 0 & 1 & 2 & 4 & 6 & 8 & 15 & 13 & 1 & 4 & $w_{379}$ & N & can. \\
2819 & 0 & 0 & 0 & 0 & 0 & 0 & 1 & 2 & 0 & 1 & 2 & 4 & 6 & 8 & 15 & 14 & 1 & 4 & $w_{379}$ & N & can. \\
2820 & 0 & 0 & 0 & 0 & 0 & 0 & 1 & 2 & 0 & 1 & 2 & 4 & 7 & 8 & 7 & 9 & 2 & 4 & $w_{392}$ & N & can. \\
2821 & 0 & 0 & 0 & 0 & 0 & 0 & 1 & 2 & 0 & 1 & 2 & 4 & 7 & 8 & 7 & 10 & 2 & 4 & $w_{266}$ & N & can. \\
2822 & 0 & 0 & 0 & 0 & 0 & 0 & 1 & 2 & 0 & 1 & 2 & 4 & 7 & 8 & 7 & 11 & 2 & 4 & $w_{419}$ & N & can. \\
2823 & 0 & 0 & 0 & 0 & 0 & 0 & 1 & 2 & 0 & 1 & 2 & 4 & 7 & 8 & 7 & 12 & 2 & 3 & $w_{378}$ & N & can. \\
2824 & 0 & 0 & 0 & 0 & 0 & 0 & 1 & 2 & 0 & 1 & 2 & 4 & 7 & 8 & 7 & 13 & 2 & 4 & $w_{392}$ & N & can. \\
2825 & 0 & 0 & 0 & 0 & 0 & 0 & 1 & 2 & 0 & 1 & 2 & 4 & 7 & 8 & 7 & 14 & 2 & 4 & $w_{423}$ & N & can. \\
2826 & 0 & 0 & 0 & 0 & 0 & 0 & 1 & 2 & 0 & 1 & 2 & 4 & 7 & 8 & 7 & 15 & 2 & 4 & $w_{419}$ & N & can. \\
2827 & 0 & 0 & 0 & 0 & 0 & 0 & 1 & 2 & 0 & 1 & 2 & 4 & 7 & 8 & 8 & 11 & 1 & 4 & $w_{392}$ & N & can. \\
2828 & 0 & 0 & 0 & 0 & 0 & 0 & 1 & 2 & 0 & 1 & 2 & 4 & 7 & 8 & 8 & 12 & 1 & 4 & $w_{379}$ & N & can. \\
2829 & 0 & 0 & 0 & 0 & 0 & 0 & 1 & 2 & 0 & 1 & 2 & 4 & 7 & 8 & 8 & 13 & 1 & 4 & $w_{281}$ & N & can. \\
2830 & 0 & 0 & 0 & 0 & 0 & 0 & 1 & 2 & 0 & 1 & 2 & 4 & 7 & 8 & 8 & 14 & 1 & 4 & $w_{281}$ & N & can. \\
2831 & 0 & 0 & 0 & 0 & 0 & 0 & 1 & 2 & 0 & 1 & 2 & 4 & 7 & 8 & 8 & 15 & 1 & 4 & $w_{379}$ & N & can. \\
2832 & 0 & 0 & 0 & 0 & 0 & 0 & 1 & 2 & 0 & 1 & 2 & 4 & 7 & 8 & 9 & 7 & 2 & 4 & $w_{392}$ & N & can. \\
2833 & 0 & 0 & 0 & 0 & 0 & 0 & 1 & 2 & 0 & 1 & 2 & 4 & 7 & 8 & 9 & 11 & 2 & 4 & $w_{426}$ & N & can. \\
2834 & 0 & 0 & 0 & 0 & 0 & 0 & 1 & 2 & 0 & 1 & 2 & 4 & 7 & 8 & 9 & 12 & 1 & 4 & $w_{427}$ & N & can. \\
2835 & 0 & 0 & 0 & 0 & 0 & 0 & 1 & 2 & 0 & 1 & 2 & 4 & 7 & 8 & 9 & 13 & 1 & 4 & $w_{379}$ & N & can. \\
2836 & 0 & 0 & 0 & 0 & 0 & 0 & 1 & 2 & 0 & 1 & 2 & 4 & 7 & 8 & 9 & 14 & 1 & 4 & $w_{428}$ & N & can. \\
2837 & 0 & 0 & 0 & 0 & 0 & 0 & 1 & 2 & 0 & 1 & 2 & 4 & 7 & 8 & 9 & 15 & 1 & 4 & $w_{355}$ & N & can. \\
2838 & 0 & 0 & 0 & 0 & 0 & 0 & 1 & 2 & 0 & 1 & 2 & 4 & 7 & 8 & 10 & 7 & 2 & 4 & $w_{392}$ & N & can. \\
2839 & 0 & 0 & 0 & 0 & 0 & 0 & 1 & 2 & 0 & 1 & 2 & 4 & 7 & 8 & 10 & 11 & 2 & 4 & $w_{419}$ & N & can. \\
2840 & 0 & 0 & 0 & 0 & 0 & 0 & 1 & 2 & 0 & 1 & 2 & 4 & 7 & 8 & 10 & 12 & 1 & 4 & $w_{355}$ & N & can. \\
2841 & 0 & 0 & 0 & 0 & 0 & 0 & 1 & 2 & 0 & 1 & 2 & 4 & 7 & 8 & 10 & 13 & 1 & 4 & $w_{379}$ & N & can. \\
2842 & 0 & 0 & 0 & 0 & 0 & 0 & 1 & 2 & 0 & 1 & 2 & 4 & 7 & 8 & 10 & 14 & 1 & 4 & $w_{428}$ & N & can. \\
2843 & 0 & 0 & 0 & 0 & 0 & 0 & 1 & 2 & 0 & 1 & 2 & 4 & 7 & 8 & 10 & 15 & 1 & 4 & $w_{427}$ & N & can. \\
2844 & 0 & 0 & 0 & 0 & 0 & 0 & 1 & 2 & 0 & 1 & 2 & 4 & 7 & 8 & 11 & 7 & 2 & 4 & $w_{392}$ & N & can. \\
2845 & 0 & 0 & 0 & 0 & 0 & 0 & 1 & 2 & 0 & 1 & 2 & 4 & 7 & 8 & 11 & 8 & 1 & 4 & $w_{266}$ & N & can. \\
2846 & 0 & 0 & 0 & 0 & 0 & 0 & 1 & 2 & 0 & 1 & 2 & 4 & 7 & 8 & 11 & 9 & 2 & 4 & $w_{423}$ & N & can. \\
2847 & 0 & 0 & 0 & 0 & 0 & 0 & 1 & 2 & 0 & 1 & 2 & 4 & 7 & 8 & 11 & 10 & 2 & 4 & $w_{266}$ & N & can. \\
2848 & 0 & 0 & 0 & 0 & 0 & 0 & 1 & 2 & 0 & 1 & 2 & 4 & 7 & 8 & 11 & 12 & 1 & 4 & $w_{428}$ & N & can. \\
2849 & 0 & 0 & 0 & 0 & 0 & 0 & 1 & 2 & 0 & 1 & 2 & 4 & 7 & 8 & 11 & 13 & 1 & 4 & $w_{281}$ & N & can. \\
2850 & 0 & 0 & 0 & 0 & 0 & 0 & 1 & 2 & 0 & 1 & 2 & 4 & 7 & 8 & 11 & 14 & 1 & 4 & $w_{428}$ & N & can. \\
2851 & 0 & 0 & 0 & 0 & 0 & 0 & 1 & 2 & 0 & 1 & 2 & 4 & 7 & 8 & 11 & 15 & 1 & 4 & $w_{428}$ & N & can. \\
2852 & 0 & 0 & 0 & 0 & 0 & 0 & 1 & 2 & 0 & 1 & 2 & 4 & 7 & 8 & 12 & 7 & 2 & 3 & $w_{280}$ & N & can. \\
2853 & 0 & 0 & 0 & 0 & 0 & 0 & 1 & 2 & 0 & 1 & 2 & 4 & 7 & 8 & 12 & 8 & 1 & 4 & $w_{281}$ & N & can. \\
2854 & 0 & 0 & 0 & 0 & 0 & 0 & 1 & 2 & 0 & 1 & 2 & 4 & 7 & 8 & 12 & 9 & 1 & 4 & $w_{379}$ & N & can. \\
2855 & 0 & 0 & 0 & 0 & 0 & 0 & 1 & 2 & 0 & 1 & 2 & 4 & 7 & 8 & 12 & 10 & 1 & 4 & $w_{267}$ & N & can. \\
2856 & 0 & 0 & 0 & 0 & 0 & 0 & 1 & 2 & 0 & 1 & 2 & 4 & 7 & 8 & 12 & 11 & 1 & 4 & $w_{428}$ & N & can. \\
2857 & 0 & 0 & 0 & 0 & 0 & 0 & 1 & 2 & 0 & 1 & 2 & 4 & 7 & 8 & 12 & 14 & 1 & 4 & $w_{428}$ & N & can. \\
2858 & 0 & 0 & 0 & 0 & 0 & 0 & 1 & 2 & 0 & 1 & 2 & 4 & 7 & 8 & 12 & 15 & 1 & 4 & $w_{379}$ & N & can. \\
2859 & 0 & 0 & 0 & 0 & 0 & 0 & 1 & 2 & 0 & 1 & 2 & 4 & 7 & 8 & 13 & 7 & 2 & 4 & $w_{423}$ & N & can. \\
2860 & 0 & 0 & 0 & 0 & 0 & 0 & 1 & 2 & 0 & 1 & 2 & 4 & 7 & 8 & 13 & 8 & 1 & 4 & $w_{355}$ & N & can. \\
2861 & 0 & 0 & 0 & 0 & 0 & 0 & 1 & 2 & 0 & 1 & 2 & 4 & 7 & 8 & 13 & 9 & 1 & 4 & $w_{428}$ & N & can. \\
2862 & 0 & 0 & 0 & 0 & 0 & 0 & 1 & 2 & 0 & 1 & 2 & 4 & 7 & 8 & 13 & 10 & 1 & 4 & $w_{379}$ & N & can. \\
2863 & 0 & 0 & 0 & 0 & 0 & 0 & 1 & 2 & 0 & 1 & 2 & 4 & 7 & 8 & 13 & 11 & 1 & 4 & $w_{427}$ & N & can. \\
2864 & 0 & 0 & 0 & 0 & 0 & 0 & 1 & 2 & 0 & 1 & 2 & 4 & 7 & 8 & 13 & 12 & 1 & 4 & $w_{427}$ & N & can. \\
2865 & 0 & 0 & 0 & 0 & 0 & 0 & 1 & 2 & 0 & 1 & 2 & 4 & 7 & 8 & 13 & 14 & 1 & 4 & $w_{428}$ & N & can. \\
2866 & 0 & 0 & 0 & 0 & 0 & 0 & 1 & 2 & 0 & 1 & 2 & 4 & 7 & 8 & 13 & 15 & 1 & 4 & $w_{429}$ & N & can. \\
2867 & 0 & 0 & 0 & 0 & 0 & 0 & 1 & 2 & 0 & 1 & 2 & 4 & 7 & 8 & 14 & 7 & 2 & 4 & $w_{423}$ & N & can. \\
2868 & 0 & 0 & 0 & 0 & 0 & 0 & 1 & 2 & 0 & 1 & 2 & 4 & 7 & 8 & 14 & 8 & 1 & 4 & $w_{354}$ & N & can. \\
2869 & 0 & 0 & 0 & 0 & 0 & 0 & 1 & 2 & 0 & 1 & 2 & 4 & 7 & 8 & 14 & 9 & 1 & 4 & $w_{428}$ & N & can. \\
2870 & 0 & 0 & 0 & 0 & 0 & 0 & 1 & 2 & 0 & 1 & 2 & 4 & 7 & 8 & 14 & 10 & 1 & 4 & $w_{379}$ & N & can. \\
2871 & 0 & 0 & 0 & 0 & 0 & 0 & 1 & 2 & 0 & 1 & 2 & 4 & 7 & 8 & 14 & 11 & 1 & 4 & $w_{429}$ & N & can. \\
2872 & 0 & 0 & 0 & 0 & 0 & 0 & 1 & 2 & 0 & 1 & 2 & 4 & 7 & 8 & 14 & 12 & 1 & 4 & $w_{429}$ & N & can. \\
2873 & 0 & 0 & 0 & 0 & 0 & 0 & 1 & 2 & 0 & 1 & 2 & 4 & 7 & 8 & 14 & 13 & 1 & 4 & $w_{379}$ & N & can. \\
2874 & 0 & 0 & 0 & 0 & 0 & 0 & 1 & 2 & 0 & 1 & 2 & 4 & 7 & 8 & 14 & 15 & 1 & 4 & $w_{427}$ & N & can. \\
2875 & 0 & 0 & 0 & 0 & 0 & 0 & 1 & 2 & 0 & 1 & 2 & 4 & 7 & 8 & 15 & 7 & 2 & 4 & $w_{423}$ & N & can. \\
2876 & 0 & 0 & 0 & 0 & 0 & 0 & 1 & 2 & 0 & 1 & 2 & 4 & 7 & 8 & 15 & 8 & 1 & 4 & $w_{379}$ & N & can. \\
2877 & 0 & 0 & 0 & 0 & 0 & 0 & 1 & 2 & 0 & 1 & 2 & 4 & 7 & 8 & 15 & 9 & 1 & 4 & $w_{379}$ & N & can. \\
2878 & 0 & 0 & 0 & 0 & 0 & 0 & 1 & 2 & 0 & 1 & 2 & 4 & 7 & 8 & 15 & 10 & 1 & 4 & $w_{379}$ & N & can. \\
2879 & 0 & 0 & 0 & 0 & 0 & 0 & 1 & 2 & 0 & 1 & 2 & 4 & 7 & 8 & 15 & 11 & 1 & 4 & $w_{430}$ & N & can. \\
2880 & 0 & 0 & 0 & 0 & 0 & 0 & 1 & 2 & 0 & 1 & 2 & 4 & 7 & 8 & 15 & 12 & 1 & 4 & $w_{428}$ & N & \#2848 \\
2881 & 0 & 0 & 0 & 0 & 0 & 0 & 1 & 2 & 0 & 1 & 2 & 4 & 7 & 8 & 15 & 13 & 1 & 4 & $w_{428}$ & N & can. \\
2882 & 0 & 0 & 0 & 0 & 0 & 0 & 1 & 2 & 0 & 1 & 2 & 4 & 7 & 8 & 15 & 14 & 1 & 4 & $w_{428}$ & N & can. \\
2883 & 0 & 0 & 0 & 0 & 0 & 0 & 1 & 2 & 0 & 1 & 2 & 4 & 8 & 9 & 10 & 13 & 2 & 4 & $w_{393}$ & N & can. \\
2884 & 0 & 0 & 0 & 0 & 0 & 0 & 1 & 2 & 0 & 1 & 2 & 4 & 8 & 9 & 10 & 14 & 2 & 4 & $w_{393}$ & N & can. \\
2885 & 0 & 0 & 0 & 0 & 0 & 0 & 1 & 2 & 0 & 1 & 2 & 4 & 8 & 9 & 10 & 15 & 2 & 3 & $w_{277}$ & N & can. \\
2886 & 0 & 0 & 0 & 0 & 0 & 0 & 1 & 2 & 0 & 1 & 2 & 4 & 8 & 9 & 11 & 12 & 2 & 4 & $w_{393}$ & N & can. \\
2887 & 0 & 0 & 0 & 0 & 0 & 0 & 1 & 2 & 0 & 1 & 2 & 4 & 8 & 9 & 11 & 13 & 2 & 4 & $w_{347}$ & N & can. \\
2888 & 0 & 0 & 0 & 0 & 0 & 0 & 1 & 2 & 0 & 1 & 2 & 4 & 8 & 9 & 11 & 14 & 2 & 3 & $w_{378}$ & N & can. \\
2889 & 0 & 0 & 0 & 0 & 0 & 0 & 1 & 2 & 0 & 1 & 2 & 4 & 8 & 9 & 11 & 15 & 2 & 4 & $w_{431}$ & N & can. \\
2890 & 0 & 0 & 0 & 0 & 0 & 0 & 1 & 2 & 0 & 1 & 2 & 4 & 8 & 9 & 12 & 13 & 8 & 4 & $w_{347}$ & N & can. \\
2891 & 0 & 0 & 0 & 0 & 0 & 0 & 1 & 2 & 0 & 1 & 2 & 4 & 8 & 9 & 12 & 14 & 1 & 4 & $w_{427}$ & N & can. \\
2892 & 0 & 0 & 0 & 0 & 0 & 0 & 1 & 2 & 0 & 1 & 2 & 4 & 8 & 9 & 12 & 15 & 1 & 4 & $w_{432}$ & N & can. \\
2893 & 0 & 0 & 0 & 0 & 0 & 0 & 1 & 2 & 0 & 1 & 2 & 4 & 8 & 9 & 13 & 14 & 1 & 4 & $w_{374}$ & N & can. \\
2894 & 0 & 0 & 0 & 0 & 0 & 0 & 1 & 2 & 0 & 1 & 2 & 4 & 8 & 9 & 13 & 15 & 1 & 4 & $w_{278}$ & N & can. \\
2895 & 0 & 0 & 0 & 0 & 0 & 0 & 1 & 2 & 0 & 1 & 2 & 4 & 8 & 9 & 14 & 15 & 4 & 4 & $w_{278}$ & N & can. \\
2896 & 0 & 0 & 0 & 0 & 0 & 0 & 1 & 2 & 0 & 1 & 2 & 4 & 8 & 9 & 15 & 14 & 4 & 4 & $w_{354}$ & N & can. \\
2897 & 0 & 0 & 0 & 0 & 0 & 0 & 1 & 2 & 0 & 1 & 2 & 4 & 8 & 10 & 9 & 12 & 2 & 4 & $w_{433}$ & N & can. \\
2898 & 0 & 0 & 0 & 0 & 0 & 0 & 1 & 2 & 0 & 1 & 2 & 4 & 8 & 10 & 9 & 13 & 1 & 4 & $w_{431}$ & N & can. \\
2899 & 0 & 0 & 0 & 0 & 0 & 0 & 1 & 2 & 0 & 1 & 2 & 4 & 8 & 10 & 9 & 15 & 2 & 3 & $w_{277}$ & Y & can. \\
2900 & 0 & 0 & 0 & 0 & 0 & 0 & 1 & 2 & 0 & 1 & 2 & 4 & 8 & 10 & 11 & 12 & 1 & 4 & $w_{431}$ & N & can. \\
2901 & 0 & 0 & 0 & 0 & 0 & 0 & 1 & 2 & 0 & 1 & 2 & 4 & 8 & 10 & 11 & 13 & 2 & 3 & $w_{378}$ & N & can. \\
2902 & 0 & 0 & 0 & 0 & 0 & 0 & 1 & 2 & 0 & 1 & 2 & 4 & 8 & 10 & 11 & 14 & 2 & 4 & $w_{426}$ & N & can. \\
2903 & 0 & 0 & 0 & 0 & 0 & 0 & 1 & 2 & 0 & 1 & 2 & 4 & 8 & 10 & 12 & 14 & 4 & 4 & $w_{426}$ & N & can. \\
2904 & 0 & 0 & 0 & 0 & 0 & 0 & 1 & 2 & 0 & 1 & 2 & 4 & 8 & 10 & 12 & 15 & 1 & 4 & $w_{432}$ & N & can. \\
2905 & 0 & 0 & 0 & 0 & 0 & 0 & 1 & 2 & 0 & 1 & 2 & 4 & 8 & 10 & 13 & 15 & 4 & 4 & $w_{278}$ & N & can. \\
2906 & 0 & 0 & 0 & 0 & 0 & 0 & 1 & 2 & 0 & 1 & 2 & 4 & 8 & 10 & 14 & 12 & 4 & 4 & $w_{433}$ & N & can. \\
2907 & 0 & 0 & 0 & 0 & 0 & 0 & 1 & 2 & 0 & 1 & 2 & 4 & 8 & 10 & 14 & 13 & 1 & 4 & $w_{432}$ & N & can. \\
2908 & 0 & 0 & 0 & 0 & 0 & 0 & 1 & 2 & 0 & 1 & 2 & 4 & 8 & 10 & 15 & 13 & 4 & 4 & $w_{427}$ & N & can. \\
2909 & 0 & 0 & 0 & 0 & 0 & 0 & 1 & 2 & 0 & 1 & 2 & 4 & 8 & 11 & 8 & 12 & 1 & 4 & $w_{264}$ & N & can. \\
2910 & 0 & 0 & 0 & 0 & 0 & 0 & 1 & 2 & 0 & 1 & 2 & 4 & 8 & 11 & 8 & 13 & 1 & 4 & $w_{393}$ & N & can. \\
2911 & 0 & 0 & 0 & 0 & 0 & 0 & 1 & 2 & 0 & 1 & 2 & 4 & 8 & 11 & 8 & 14 & 1 & 4 & $w_{254}$ & N & can. \\
2912 & 0 & 0 & 0 & 0 & 0 & 0 & 1 & 2 & 0 & 1 & 2 & 4 & 8 & 11 & 8 & 15 & 1 & 3 & $w_{277}$ & Y & can. \\
2913 & 0 & 0 & 0 & 0 & 0 & 0 & 1 & 2 & 0 & 1 & 2 & 4 & 8 & 11 & 9 & 12 & 1 & 4 & $w_{393}$ & N & can. \\
2914 & 0 & 0 & 0 & 0 & 0 & 0 & 1 & 2 & 0 & 1 & 2 & 4 & 8 & 11 & 9 & 13 & 1 & 4 & $w_{433}$ & N & can. \\
2915 & 0 & 0 & 0 & 0 & 0 & 0 & 1 & 2 & 0 & 1 & 2 & 4 & 8 & 11 & 9 & 14 & 1 & 3 & $w_{277}$ & N & can. \\
2916 & 0 & 0 & 0 & 0 & 0 & 0 & 1 & 2 & 0 & 1 & 2 & 4 & 8 & 11 & 9 & 15 & 1 & 4 & $w_{393}$ & N & can. \\
2917 & 0 & 0 & 0 & 0 & 0 & 0 & 1 & 2 & 0 & 1 & 2 & 4 & 8 & 11 & 10 & 12 & 1 & 4 & $w_{254}$ & N & can. \\
2918 & 0 & 0 & 0 & 0 & 0 & 0 & 1 & 2 & 0 & 1 & 2 & 4 & 8 & 11 & 10 & 13 & 1 & 3 & $w_{277}$ & Y & can. \\
2919 & 0 & 0 & 0 & 0 & 0 & 0 & 1 & 2 & 0 & 1 & 2 & 4 & 8 & 11 & 10 & 14 & 1 & 4 & $w_{264}$ & N & can. \\
2920 & 0 & 0 & 0 & 0 & 0 & 0 & 1 & 2 & 0 & 1 & 2 & 4 & 8 & 11 & 10 & 15 & 1 & 4 & $w_{393}$ & N & can. \\
2921 & 0 & 0 & 0 & 0 & 0 & 0 & 1 & 2 & 0 & 1 & 2 & 4 & 8 & 11 & 11 & 12 & 1 & 3 & $w_{277}$ & Y & can. \\
2922 & 0 & 0 & 0 & 0 & 0 & 0 & 1 & 2 & 0 & 1 & 2 & 4 & 8 & 11 & 11 & 13 & 1 & 4 & $w_{393}$ & N & can. \\
2923 & 0 & 0 & 0 & 0 & 0 & 0 & 1 & 2 & 0 & 1 & 2 & 4 & 8 & 11 & 11 & 14 & 1 & 4 & $w_{393}$ & N & can. \\
2924 & 0 & 0 & 0 & 0 & 0 & 0 & 1 & 2 & 0 & 1 & 2 & 4 & 8 & 11 & 11 & 15 & 1 & 4 & $w_{433}$ & N & can. \\
2925 & 0 & 0 & 0 & 0 & 0 & 0 & 1 & 2 & 0 & 1 & 2 & 4 & 8 & 11 & 12 & 15 & 8 & 4 & $w_{278}$ & N & can. \\
2926 & 0 & 0 & 0 & 0 & 0 & 0 & 1 & 2 & 0 & 1 & 2 & 4 & 8 & 11 & 13 & 14 & 4 & 4 & $w_{265}$ & N & can. \\
2927 & 0 & 0 & 0 & 0 & 0 & 0 & 1 & 2 & 0 & 1 & 2 & 4 & 8 & 11 & 14 & 13 & 4 & 4 & $w_{278}$ & N & can. \\
2928 & 0 & 0 & 0 & 0 & 0 & 0 & 1 & 2 & 0 & 1 & 2 & 4 & 8 & 12 & 8 & 14 & 1 & 4 & $w_{254}$ & N & can. \\
2929 & 0 & 0 & 0 & 0 & 0 & 0 & 1 & 2 & 0 & 1 & 2 & 4 & 8 & 12 & 8 & 15 & 1 & 4 & $w_{278}$ & N & can. \\
2930 & 0 & 0 & 0 & 0 & 0 & 0 & 1 & 2 & 0 & 1 & 2 & 4 & 8 & 12 & 9 & 13 & 4 & 4 & $w_{433}$ & N & can. \\
2931 & 0 & 0 & 0 & 0 & 0 & 0 & 1 & 2 & 0 & 1 & 2 & 4 & 8 & 12 & 9 & 14 & 1 & 4 & $w_{278}$ & N & can. \\
2932 & 0 & 0 & 0 & 0 & 0 & 0 & 1 & 2 & 0 & 1 & 2 & 4 & 8 & 12 & 9 & 15 & 1 & 4 & $w_{432}$ & N & can. \\
2933 & 0 & 0 & 0 & 0 & 0 & 0 & 1 & 2 & 0 & 1 & 2 & 4 & 8 & 12 & 10 & 15 & 1 & 4 & $w_{432}$ & N & can. \\
2934 & 0 & 0 & 0 & 0 & 0 & 0 & 1 & 2 & 0 & 1 & 2 & 4 & 8 & 12 & 11 & 15 & 4 & 4 & $w_{434}$ & N & can. \\
2935 & 0 & 0 & 0 & 0 & 0 & 0 & 1 & 2 & 0 & 1 & 2 & 4 & 8 & 12 & 12 & 11 & 1 & 4 & $w_{278}$ & N & can. \\
2936 & 0 & 0 & 0 & 0 & 0 & 0 & 1 & 2 & 0 & 1 & 2 & 4 & 8 & 12 & 13 & 9 & 4 & 4 & $w_{264}$ & N & can. \\
2937 & 0 & 0 & 0 & 0 & 0 & 0 & 1 & 2 & 0 & 1 & 2 & 4 & 8 & 12 & 13 & 10 & 1 & 4 & $w_{278}$ & N & can. \\
2938 & 0 & 0 & 0 & 0 & 0 & 0 & 1 & 2 & 0 & 1 & 2 & 4 & 8 & 12 & 13 & 11 & 1 & 4 & $w_{374}$ & N & can. \\
2939 & 0 & 0 & 0 & 0 & 0 & 0 & 1 & 2 & 0 & 1 & 2 & 4 & 8 & 12 & 14 & 10 & 4 & 4 & $w_{433}$ & N & can. \\
2940 & 0 & 0 & 0 & 0 & 0 & 0 & 1 & 2 & 0 & 1 & 2 & 4 & 8 & 12 & 14 & 11 & 1 & 4 & $w_{432}$ & N & can. \\
2941 & 0 & 0 & 0 & 0 & 0 & 0 & 1 & 2 & 0 & 1 & 2 & 4 & 8 & 12 & 15 & 11 & 4 & 4 & $w_{278}$ & N & can. \\
2942 & 0 & 0 & 0 & 0 & 0 & 0 & 1 & 2 & 0 & 1 & 2 & 4 & 8 & 13 & 8 & 14 & 2 & 4 & $w_{354}$ & N & can. \\
2943 & 0 & 0 & 0 & 0 & 0 & 0 & 1 & 2 & 0 & 1 & 2 & 4 & 8 & 13 & 8 & 15 & 1 & 4 & $w_{374}$ & N & can. \\
2944 & 0 & 0 & 0 & 0 & 0 & 0 & 1 & 2 & 0 & 1 & 2 & 4 & 8 & 13 & 9 & 12 & 4 & 4 & $w_{264}$ & N & can. \\
2945 & 0 & 0 & 0 & 0 & 0 & 0 & 1 & 2 & 0 & 1 & 2 & 4 & 8 & 13 & 9 & 14 & 1 & 4 & $w_{432}$ & N & can. \\
2946 & 0 & 0 & 0 & 0 & 0 & 0 & 1 & 2 & 0 & 1 & 2 & 4 & 8 & 13 & 9 & 15 & 1 & 4 & $w_{265}$ & N & can. \\
2947 & 0 & 0 & 0 & 0 & 0 & 0 & 1 & 2 & 0 & 1 & 2 & 4 & 8 & 13 & 11 & 14 & 4 & 4 & $w_{427}$ & N & can. \\
2948 & 0 & 0 & 0 & 0 & 0 & 0 & 1 & 2 & 0 & 1 & 2 & 4 & 8 & 13 & 12 & 9 & 8 & 4 & $w_{347}$ & N & can. \\
2949 & 0 & 0 & 0 & 0 & 0 & 0 & 1 & 2 & 0 & 1 & 2 & 4 & 8 & 13 & 12 & 10 & 1 & 4 & $w_{354}$ & N & can. \\
2950 & 0 & 0 & 0 & 0 & 0 & 0 & 1 & 2 & 0 & 1 & 2 & 4 & 8 & 13 & 12 & 11 & 1 & 4 & $w_{432}$ & N & can. \\
2951 & 0 & 0 & 0 & 0 & 0 & 0 & 1 & 2 & 0 & 1 & 2 & 4 & 8 & 13 & 13 & 10 & 1 & 4 & $w_{374}$ & N & can. \\
2952 & 0 & 0 & 0 & 0 & 0 & 0 & 1 & 2 & 0 & 1 & 2 & 4 & 8 & 13 & 13 & 11 & 1 & 4 & $w_{265}$ & N & can. \\
2953 & 0 & 0 & 0 & 0 & 0 & 0 & 1 & 2 & 0 & 1 & 2 & 4 & 8 & 13 & 14 & 11 & 4 & 4 & $w_{278}$ & N & can. \\
2954 & 0 & 0 & 0 & 0 & 0 & 0 & 1 & 2 & 0 & 1 & 2 & 4 & 8 & 13 & 15 & 10 & 8 & 4 & $w_{354}$ & N & can. \\
2955 & 0 & 0 & 0 & 0 & 0 & 0 & 1 & 2 & 0 & 1 & 2 & 4 & 8 & 14 & 8 & 15 & 1 & 4 & $w_{374}$ & N & can. \\
2956 & 0 & 0 & 0 & 0 & 0 & 0 & 1 & 2 & 0 & 1 & 2 & 4 & 8 & 14 & 9 & 15 & 4 & 4 & $w_{265}$ & N & can. \\
2957 & 0 & 0 & 0 & 0 & 0 & 0 & 1 & 2 & 0 & 1 & 2 & 4 & 8 & 14 & 10 & 13 & 1 & 4 & $w_{374}$ & N & can. \\
2958 & 0 & 0 & 0 & 0 & 0 & 0 & 1 & 2 & 0 & 1 & 2 & 4 & 8 & 14 & 12 & 11 & 1 & 4 & $w_{374}$ & N & can. \\
2959 & 0 & 0 & 0 & 0 & 0 & 0 & 1 & 2 & 0 & 1 & 2 & 4 & 8 & 14 & 14 & 9 & 1 & 4 & $w_{374}$ & N & can. \\
2960 & 0 & 0 & 0 & 0 & 0 & 0 & 1 & 2 & 0 & 1 & 2 & 4 & 8 & 15 & 9 & 14 & 4 & 4 & $w_{278}$ & N & can. \\
2961 & 0 & 0 & 0 & 0 & 0 & 0 & 1 & 2 & 0 & 1 & 2 & 4 & 8 & 15 & 12 & 11 & 8 & 4 & $w_{278}$ & N & \#2925 \\
2962 & 0 & 0 & 0 & 0 & 0 & 0 & 1 & 2 & 0 & 1 & 2 & 4 & 8 & 15 & 13 & 10 & 4 & 4 & $w_{265}$ & N & can. \\
2963 & 0 & 0 & 0 & 0 & 0 & 0 & 1 & 2 & 0 & 1 & 2 & 4 & 8 & 15 & 14 & 9 & 4 & 4 & $w_{278}$ & N & can. \\
2964 & 0 & 0 & 0 & 0 & 0 & 0 & 1 & 2 & 0 & 1 & 4 & 5 & 8 & 9 & 13 & 14 & 48 & 4 & $w_{435}$ & N & can. \\
2965 & 0 & 0 & 0 & 0 & 0 & 0 & 1 & 2 & 0 & 1 & 4 & 5 & 8 & 9 & 13 & 15 & 48 & 3 & $w_{378}$ & N & can. \\
2966 & 0 & 0 & 0 & 0 & 0 & 0 & 1 & 2 & 0 & 1 & 4 & 5 & 8 & 10 & 12 & 15 & 4 & 4 & $w_{436}$ & N & can. \\
2967 & 0 & 0 & 0 & 0 & 0 & 0 & 1 & 2 & 0 & 1 & 4 & 5 & 8 & 10 & 13 & 15 & 32 & 4 & $w_{427}$ & N & can. \\
2968 & 0 & 0 & 0 & 0 & 0 & 0 & 1 & 2 & 0 & 1 & 4 & 5 & 8 & 11 & 8 & 12 & 4 & 4 & $w_{433}$ & N & can. \\
2969 & 0 & 0 & 0 & 0 & 0 & 0 & 1 & 2 & 0 & 1 & 4 & 5 & 8 & 11 & 8 & 14 & 4 & 4 & $w_{278}$ & N & can. \\
2970 & 0 & 0 & 0 & 0 & 0 & 0 & 1 & 2 & 0 & 1 & 4 & 5 & 8 & 11 & 9 & 12 & 4 & 4 & $w_{433}$ & N & can. \\
2971 & 0 & 0 & 0 & 0 & 0 & 0 & 1 & 2 & 0 & 1 & 4 & 5 & 8 & 11 & 9 & 13 & 8 & 4 & $w_{264}$ & N & can. \\
2972 & 0 & 0 & 0 & 0 & 0 & 0 & 1 & 2 & 0 & 1 & 4 & 5 & 8 & 11 & 9 & 14 & 4 & 4 & $w_{278}$ & N & can. \\
2973 & 0 & 0 & 0 & 0 & 0 & 0 & 1 & 2 & 0 & 1 & 4 & 5 & 8 & 11 & 9 & 15 & 4 & 4 & $w_{265}$ & N & can. \\
2974 & 0 & 0 & 0 & 0 & 0 & 0 & 1 & 2 & 0 & 1 & 4 & 5 & 8 & 11 & 13 & 14 & 16 & 4 & $w_{436}$ & N & can. \\
2975 & 0 & 0 & 0 & 0 & 0 & 0 & 1 & 2 & 0 & 1 & 4 & 5 & 8 & 12 & 8 & 15 & 8 & 3 & $w_{378}$ & N & can. \\
2976 & 0 & 0 & 0 & 0 & 0 & 0 & 1 & 2 & 0 & 1 & 4 & 5 & 8 & 12 & 9 & 14 & 8 & 3 & $w_{277}$ & N & can. \\
2977 & 0 & 0 & 0 & 0 & 0 & 0 & 1 & 2 & 0 & 1 & 4 & 5 & 8 & 12 & 10 & 14 & 16 & 4 & $w_{278}$ & N & can. \\
2978 & 0 & 0 & 0 & 0 & 0 & 0 & 1 & 2 & 0 & 1 & 4 & 5 & 8 & 12 & 10 & 15 & 4 & 4 & $w_{434}$ & N & can. \\
2979 & 0 & 0 & 0 & 0 & 0 & 0 & 1 & 2 & 0 & 1 & 4 & 5 & 8 & 12 & 11 & 15 & 16 & 4 & $w_{427}$ & N & can. \\
2980 & 0 & 0 & 0 & 0 & 0 & 0 & 1 & 2 & 0 & 1 & 4 & 5 & 8 & 12 & 12 & 11 & 4 & 3 & $w_{277}$ & N & can. \\
2981 & 0 & 0 & 0 & 0 & 0 & 0 & 1 & 2 & 0 & 1 & 4 & 5 & 8 & 12 & 13 & 9 & 32 & 4 & $w_{417}$ & N & can. \\
2982 & 0 & 0 & 0 & 0 & 0 & 0 & 1 & 2 & 0 & 1 & 4 & 5 & 8 & 12 & 13 & 10 & 4 & 3 & $w_{378}$ & N & can. \\
2983 & 0 & 0 & 0 & 0 & 0 & 0 & 1 & 2 & 0 & 1 & 4 & 5 & 8 & 12 & 13 & 11 & 4 & 4 & $w_{437}$ & N & can. \\
2984 & 0 & 0 & 0 & 0 & 0 & 0 & 1 & 2 & 0 & 1 & 4 & 5 & 8 & 12 & 14 & 10 & 32 & 4 & $w_{354}$ & N & can. \\
2985 & 0 & 0 & 0 & 0 & 0 & 0 & 1 & 2 & 0 & 1 & 4 & 5 & 8 & 12 & 15 & 11 & 16 & 4 & $w_{434}$ & N & can. \\
2986 & 0 & 0 & 0 & 0 & 0 & 0 & 1 & 2 & 0 & 1 & 4 & 5 & 8 & 14 & 8 & 15 & 4 & 4 & $w_{278}$ & N & can. \\
2987 & 0 & 0 & 0 & 0 & 0 & 0 & 1 & 2 & 0 & 1 & 4 & 5 & 8 & 14 & 9 & 15 & 16 & 4 & $w_{354}$ & N & can. \\
2988 & 0 & 0 & 0 & 0 & 0 & 0 & 1 & 2 & 0 & 1 & 4 & 5 & 8 & 14 & 13 & 11 & 16 & 4 & $w_{427}$ & N & can. \\
2989 & 0 & 0 & 0 & 0 & 0 & 0 & 1 & 2 & 0 & 1 & 4 & 5 & 8 & 14 & 15 & 9 & 16 & 4 & $w_{278}$ & N & can. \\
2990 & 0 & 0 & 0 & 0 & 0 & 0 & 1 & 2 & 0 & 1 & 4 & 6 & 8 & 11 & 8 & 12 & 2 & 4 & $w_{278}$ & N & can. \\
2991 & 0 & 0 & 0 & 0 & 0 & 0 & 1 & 2 & 0 & 1 & 4 & 6 & 8 & 11 & 8 & 13 & 2 & 4 & $w_{278}$ & N & can. \\
2992 & 0 & 0 & 0 & 0 & 0 & 0 & 1 & 2 & 0 & 1 & 4 & 6 & 8 & 11 & 9 & 12 & 1 & 4 & $w_{278}$ & N & can. \\
2993 & 0 & 0 & 0 & 0 & 0 & 0 & 1 & 2 & 0 & 1 & 4 & 6 & 8 & 11 & 12 & 15 & 8 & 3 & $w_{376}$ & Y & can. \\
2994 & 0 & 0 & 0 & 0 & 0 & 0 & 1 & 2 & 0 & 1 & 4 & 6 & 8 & 12 & 8 & 15 & 8 & 4 & $w_{427}$ & N & can. \\
2995 & 0 & 0 & 0 & 0 & 0 & 0 & 1 & 2 & 0 & 1 & 4 & 6 & 8 & 12 & 9 & 14 & 4 & 4 & $w_{278}$ & N & can. \\
2996 & 0 & 0 & 0 & 0 & 0 & 0 & 1 & 2 & 0 & 1 & 4 & 6 & 8 & 12 & 9 & 15 & 1 & 4 & $w_{434}$ & N & can. \\
2997 & 0 & 0 & 0 & 0 & 0 & 0 & 1 & 2 & 0 & 1 & 4 & 6 & 8 & 12 & 11 & 15 & 8 & 3 & $w_{438}$ & N & can. \\
2998 & 0 & 0 & 0 & 0 & 0 & 0 & 1 & 2 & 0 & 1 & 4 & 6 & 8 & 12 & 12 & 11 & 4 & 4 & $w_{278}$ & N & can. \\
2999 & 0 & 0 & 0 & 0 & 0 & 0 & 1 & 2 & 0 & 1 & 4 & 6 & 8 & 12 & 13 & 9 & 8 & 3 & $w_{378}$ & N & can. \\
3000 & 0 & 0 & 0 & 0 & 0 & 0 & 1 & 2 & 0 & 1 & 4 & 6 & 8 & 12 & 13 & 10 & 4 & 4 & $w_{427}$ & N & can. \\
3001 & 0 & 0 & 0 & 0 & 0 & 0 & 1 & 2 & 0 & 1 & 4 & 6 & 8 & 12 & 15 & 11 & 16 & 3 & $w_{377}$ & N & can. \\
3002 & 0 & 0 & 0 & 0 & 0 & 0 & 1 & 2 & 0 & 1 & 4 & 7 & 4 & 8 & 6 & 8 & 1 & 4 & $w_{225}$ & N & can. \\
3003 & 0 & 0 & 0 & 0 & 0 & 0 & 1 & 2 & 0 & 1 & 4 & 7 & 4 & 8 & 6 & 9 & 1 & 4 & $w_{262}$ & N & can. \\
3004 & 0 & 0 & 0 & 0 & 0 & 0 & 1 & 2 & 0 & 1 & 4 & 7 & 4 & 8 & 6 & 10 & 1 & 4 & $w_{225}$ & N & can. \\
3005 & 0 & 0 & 0 & 0 & 0 & 0 & 1 & 2 & 0 & 1 & 4 & 7 & 4 & 8 & 6 & 12 & 1 & 4 & $w_{241}$ & N & can. \\
3006 & 0 & 0 & 0 & 0 & 0 & 0 & 1 & 2 & 0 & 1 & 4 & 7 & 4 & 8 & 6 & 13 & 1 & 4 & $w_{260}$ & N & can. \\
3007 & 0 & 0 & 0 & 0 & 0 & 0 & 1 & 2 & 0 & 1 & 4 & 7 & 4 & 8 & 6 & 14 & 1 & 4 & $w_{241}$ & N & can. \\
3008 & 0 & 0 & 0 & 0 & 0 & 0 & 1 & 2 & 0 & 1 & 4 & 7 & 4 & 8 & 6 & 15 & 1 & 4 & $w_{260}$ & N & \#2579 \\
3009 & 0 & 0 & 0 & 0 & 0 & 0 & 1 & 2 & 0 & 1 & 4 & 7 & 4 & 8 & 7 & 8 & 2 & 4 & $w_{262}$ & N & can. \\
3010 & 0 & 0 & 0 & 0 & 0 & 0 & 1 & 2 & 0 & 1 & 4 & 7 & 4 & 8 & 7 & 9 & 2 & 4 & $w_{258}$ & N & can. \\
3011 & 0 & 0 & 0 & 0 & 0 & 0 & 1 & 2 & 0 & 1 & 4 & 7 & 4 & 8 & 7 & 10 & 2 & 3 & $w_{259}$ & Y & can. \\
3012 & 0 & 0 & 0 & 0 & 0 & 0 & 1 & 2 & 0 & 1 & 4 & 7 & 4 & 8 & 7 & 11 & 2 & 4 & $w_{258}$ & N & can. \\
3013 & 0 & 0 & 0 & 0 & 0 & 0 & 1 & 2 & 0 & 1 & 4 & 7 & 4 & 8 & 7 & 12 & 1 & 4 & $w_{260}$ & N & can. \\
3014 & 0 & 0 & 0 & 0 & 0 & 0 & 1 & 2 & 0 & 1 & 4 & 7 & 4 & 8 & 7 & 13 & 1 & 4 & $w_{266}$ & N & can. \\
3015 & 0 & 0 & 0 & 0 & 0 & 0 & 1 & 2 & 0 & 1 & 4 & 7 & 4 & 8 & 8 & 11 & 2 & 4 & $w_{241}$ & N & can. \\
3016 & 0 & 0 & 0 & 0 & 0 & 0 & 1 & 2 & 0 & 1 & 4 & 7 & 4 & 8 & 8 & 12 & 1 & 4 & $w_{260}$ & N & \#3013 \\
3017 & 0 & 0 & 0 & 0 & 0 & 0 & 1 & 2 & 0 & 1 & 4 & 7 & 4 & 8 & 8 & 14 & 1 & 4 & $w_{268}$ & N & can. \\
3018 & 0 & 0 & 0 & 0 & 0 & 0 & 1 & 2 & 0 & 1 & 4 & 7 & 4 & 8 & 8 & 15 & 1 & 4 & $w_{267}$ & N & can. \\
3019 & 0 & 0 & 0 & 0 & 0 & 0 & 1 & 2 & 0 & 1 & 4 & 7 & 4 & 8 & 9 & 12 & 1 & 4 & $w_{266}$ & N & can. \\
3020 & 0 & 0 & 0 & 0 & 0 & 0 & 1 & 2 & 0 & 1 & 4 & 7 & 4 & 8 & 9 & 13 & 1 & 4 & $w_{392}$ & N & can. \\
3021 & 0 & 0 & 0 & 0 & 0 & 0 & 1 & 2 & 0 & 1 & 4 & 7 & 4 & 8 & 9 & 14 & 1 & 4 & $w_{267}$ & N & \#2627 \\
3022 & 0 & 0 & 0 & 0 & 0 & 0 & 1 & 2 & 0 & 1 & 4 & 7 & 4 & 8 & 9 & 15 & 1 & 4 & $w_{281}$ & N & \#2624 \\
3023 & 0 & 0 & 0 & 0 & 0 & 0 & 1 & 2 & 0 & 1 & 4 & 7 & 4 & 8 & 10 & 12 & 1 & 4 & $w_{267}$ & N & can. \\
3024 & 0 & 0 & 0 & 0 & 0 & 0 & 1 & 2 & 0 & 1 & 4 & 7 & 4 & 8 & 10 & 13 & 1 & 4 & $w_{281}$ & N & can. \\
3025 & 0 & 0 & 0 & 0 & 0 & 0 & 1 & 2 & 0 & 1 & 4 & 7 & 4 & 8 & 10 & 14 & 1 & 4 & $w_{267}$ & N & can. \\
3026 & 0 & 0 & 0 & 0 & 0 & 0 & 1 & 2 & 0 & 1 & 4 & 7 & 4 & 8 & 10 & 15 & 1 & 4 & $w_{281}$ & N & can. \\
3027 & 0 & 0 & 0 & 0 & 0 & 0 & 1 & 2 & 0 & 1 & 4 & 7 & 4 & 8 & 11 & 12 & 1 & 4 & $w_{281}$ & N & can. \\
3028 & 0 & 0 & 0 & 0 & 0 & 0 & 1 & 2 & 0 & 1 & 4 & 7 & 4 & 8 & 11 & 13 & 1 & 4 & $w_{379}$ & N & can. \\
3029 & 0 & 0 & 0 & 0 & 0 & 0 & 1 & 2 & 0 & 1 & 4 & 7 & 4 & 8 & 11 & 14 & 1 & 4 & $w_{281}$ & N & can. \\
3030 & 0 & 0 & 0 & 0 & 0 & 0 & 1 & 2 & 0 & 1 & 4 & 7 & 4 & 8 & 11 & 15 & 1 & 4 & $w_{379}$ & N & can. \\
3031 & 0 & 0 & 0 & 0 & 0 & 0 & 1 & 2 & 0 & 1 & 4 & 7 & 4 & 8 & 13 & 14 & 1 & 4 & $w_{286}$ & N & can. \\
3032 & 0 & 0 & 0 & 0 & 0 & 0 & 1 & 2 & 0 & 1 & 4 & 7 & 5 & 8 & 8 & 12 & 1 & 4 & $w_{260}$ & N & can. \\
3033 & 0 & 0 & 0 & 0 & 0 & 0 & 1 & 2 & 0 & 1 & 4 & 7 & 5 & 8 & 8 & 14 & 1 & 4 & $w_{268}$ & N & can. \\
3034 & 0 & 0 & 0 & 0 & 0 & 0 & 1 & 2 & 0 & 1 & 4 & 7 & 5 & 8 & 9 & 10 & 2 & 4 & $w_{241}$ & N & can. \\
3035 & 0 & 0 & 0 & 0 & 0 & 0 & 1 & 2 & 0 & 1 & 4 & 7 & 5 & 8 & 9 & 12 & 1 & 4 & $w_{266}$ & N & can. \\
3036 & 0 & 0 & 0 & 0 & 0 & 0 & 1 & 2 & 0 & 1 & 4 & 7 & 5 & 8 & 9 & 13 & 1 & 4 & $w_{392}$ & N & can. \\
3037 & 0 & 0 & 0 & 0 & 0 & 0 & 1 & 2 & 0 & 1 & 4 & 7 & 5 & 8 & 9 & 14 & 1 & 4 & $w_{267}$ & N & can. \\
3038 & 0 & 0 & 0 & 0 & 0 & 0 & 1 & 2 & 0 & 1 & 4 & 7 & 5 & 8 & 9 & 15 & 1 & 4 & $w_{281}$ & N & can. \\
3039 & 0 & 0 & 0 & 0 & 0 & 0 & 1 & 2 & 0 & 1 & 4 & 7 & 5 & 8 & 10 & 12 & 1 & 4 & $w_{267}$ & N & can. \\
3040 & 0 & 0 & 0 & 0 & 0 & 0 & 1 & 2 & 0 & 1 & 4 & 7 & 5 & 8 & 10 & 13 & 1 & 4 & $w_{281}$ & N & can. \\
3041 & 0 & 0 & 0 & 0 & 0 & 0 & 1 & 2 & 0 & 1 & 4 & 7 & 5 & 8 & 10 & 14 & 1 & 4 & $w_{267}$ & N & can. \\
3042 & 0 & 0 & 0 & 0 & 0 & 0 & 1 & 2 & 0 & 1 & 4 & 7 & 5 & 8 & 10 & 15 & 1 & 4 & $w_{281}$ & N & can. \\
3043 & 0 & 0 & 0 & 0 & 0 & 0 & 1 & 2 & 0 & 1 & 4 & 7 & 5 & 8 & 12 & 15 & 1 & 4 & $w_{374}$ & N & \#2957 \\
3044 & 0 & 0 & 0 & 0 & 0 & 0 & 1 & 2 & 0 & 1 & 4 & 7 & 5 & 8 & 13 & 14 & 1 & 4 & $w_{286}$ & N & can. \\
3045 & 0 & 0 & 0 & 0 & 0 & 0 & 1 & 2 & 0 & 1 & 4 & 7 & 8 & 12 & 9 & 14 & 1 & 4 & $w_{278}$ & N & can. \\
3046 & 0 & 0 & 0 & 0 & 0 & 0 & 1 & 2 & 0 & 1 & 4 & 7 & 8 & 12 & 9 & 15 & 2 & 4 & $w_{278}$ & N & can. \\
3047 & 0 & 0 & 0 & 0 & 0 & 0 & 1 & 2 & 0 & 1 & 4 & 7 & 8 & 12 & 10 & 15 & 1 & 3 & $w_{377}$ & N & can. \\
3048 & 0 & 0 & 0 & 0 & 0 & 0 & 1 & 2 & 0 & 1 & 4 & 7 & 8 & 12 & 12 & 10 & 2 & 4 & $w_{265}$ & N & can. \\
3049 & 0 & 0 & 0 & 0 & 0 & 0 & 1 & 2 & 0 & 1 & 4 & 7 & 8 & 12 & 13 & 11 & 2 & 4 & $w_{434}$ & N & can. \\
3050 & 0 & 0 & 0 & 0 & 0 & 0 & 1 & 2 & 0 & 1 & 4 & 7 & 8 & 12 & 15 & 11 & 4 & 4 & $w_{434}$ & N & can. \\
3051 & 0 & 0 & 0 & 0 & 0 & 0 & 1 & 2 & 0 & 1 & 4 & 8 & 4 & 9 & 4 & 10 & 1 & 4 & $w_{225}$ & N & can. \\
3052 & 0 & 0 & 0 & 0 & 0 & 0 & 1 & 2 & 0 & 1 & 4 & 8 & 4 & 9 & 4 & 14 & 1 & 4 & $w_{241}$ & N & can. \\
3053 & 0 & 0 & 0 & 0 & 0 & 0 & 1 & 2 & 0 & 1 & 4 & 8 & 4 & 9 & 6 & 10 & 1 & 4 & $w_{241}$ & N & can. \\
3054 & 0 & 0 & 0 & 0 & 0 & 0 & 1 & 2 & 0 & 1 & 4 & 8 & 4 & 9 & 6 & 14 & 1 & 4 & $w_{267}$ & N & can. \\
3055 & 0 & 0 & 0 & 0 & 0 & 0 & 1 & 2 & 0 & 1 & 4 & 8 & 4 & 9 & 7 & 10 & 1 & 4 & $w_{266}$ & N & \#2761 \\
3056 & 0 & 0 & 0 & 0 & 0 & 0 & 1 & 2 & 0 & 1 & 4 & 8 & 4 & 9 & 7 & 14 & 1 & 4 & $w_{379}$ & N & \#2831 \\
3057 & 0 & 0 & 0 & 0 & 0 & 0 & 1 & 2 & 0 & 1 & 4 & 8 & 4 & 9 & 10 & 14 & 1 & 4 & $w_{281}$ & N & \#2762 \\
3058 & 0 & 0 & 0 & 0 & 0 & 0 & 1 & 2 & 0 & 1 & 4 & 8 & 4 & 9 & 10 & 15 & 1 & 4 & $w_{281}$ & N & can. \\
3059 & 0 & 0 & 0 & 0 & 0 & 0 & 1 & 2 & 0 & 1 & 4 & 8 & 4 & 10 & 6 & 8 & 2 & 4 & $w_{201}$ & N & can. \\
3060 & 0 & 0 & 0 & 0 & 0 & 0 & 1 & 2 & 0 & 1 & 4 & 8 & 4 & 10 & 6 & 9 & 1 & 4 & $w_{260}$ & N & can. \\
3061 & 0 & 0 & 0 & 0 & 0 & 0 & 1 & 2 & 0 & 1 & 4 & 8 & 4 & 10 & 6 & 12 & 1 & 4 & $w_{260}$ & N & can. \\
3062 & 0 & 0 & 0 & 0 & 0 & 0 & 1 & 2 & 0 & 1 & 4 & 8 & 4 & 10 & 7 & 8 & 1 & 4 & $w_{260}$ & N & can. \\
3063 & 0 & 0 & 0 & 0 & 0 & 0 & 1 & 2 & 0 & 1 & 4 & 8 & 4 & 10 & 7 & 9 & 1 & 4 & $w_{266}$ & N & can. \\
3064 & 0 & 0 & 0 & 0 & 0 & 0 & 1 & 2 & 0 & 1 & 4 & 8 & 4 & 10 & 7 & 12 & 1 & 4 & $w_{281}$ & N & can. \\
3065 & 0 & 0 & 0 & 0 & 0 & 0 & 1 & 2 & 0 & 1 & 4 & 8 & 4 & 10 & 8 & 12 & 1 & 4 & $w_{262}$ & N & can. \\
3066 & 0 & 0 & 0 & 0 & 0 & 0 & 1 & 2 & 0 & 1 & 4 & 8 & 4 & 10 & 8 & 14 & 1 & 4 & $w_{241}$ & N & can. \\
3067 & 0 & 0 & 0 & 0 & 0 & 0 & 1 & 2 & 0 & 1 & 4 & 8 & 4 & 10 & 9 & 12 & 1 & 4 & $w_{266}$ & N & can. \\
3068 & 0 & 0 & 0 & 0 & 0 & 0 & 1 & 2 & 0 & 1 & 4 & 8 & 4 & 10 & 9 & 14 & 1 & 4 & $w_{267}$ & N & can. \\
3069 & 0 & 0 & 0 & 0 & 0 & 0 & 1 & 2 & 0 & 1 & 4 & 8 & 4 & 11 & 4 & 13 & 4 & 4 & $w_{352}$ & N & can. \\
3070 & 0 & 0 & 0 & 0 & 0 & 0 & 1 & 2 & 0 & 1 & 4 & 8 & 4 & 11 & 6 & 9 & 1 & 4 & $w_{372}$ & N & can. \\
3071 & 0 & 0 & 0 & 0 & 0 & 0 & 1 & 2 & 0 & 1 & 4 & 8 & 4 & 11 & 6 & 12 & 1 & 4 & $w_{267}$ & N & can. \\
3072 & 0 & 0 & 0 & 0 & 0 & 0 & 1 & 2 & 0 & 1 & 4 & 8 & 4 & 11 & 6 & 13 & 1 & 4 & $w_{354}$ & N & can. \\
3073 & 0 & 0 & 0 & 0 & 0 & 0 & 1 & 2 & 0 & 1 & 4 & 8 & 4 & 11 & 7 & 8 & 2 & 4 & $w_{353}$ & N & can. \\
3074 & 0 & 0 & 0 & 0 & 0 & 0 & 1 & 2 & 0 & 1 & 4 & 8 & 4 & 11 & 7 & 9 & 1 & 4 & $w_{392}$ & N & can. \\
3075 & 0 & 0 & 0 & 0 & 0 & 0 & 1 & 2 & 0 & 1 & 4 & 8 & 4 & 11 & 7 & 12 & 1 & 4 & $w_{379}$ & N & can. \\
3076 & 0 & 0 & 0 & 0 & 0 & 0 & 1 & 2 & 0 & 1 & 4 & 8 & 4 & 11 & 7 & 13 & 2 & 4 & $w_{427}$ & N & can. \\
3077 & 0 & 0 & 0 & 0 & 0 & 0 & 1 & 2 & 0 & 1 & 4 & 8 & 4 & 11 & 8 & 12 & 1 & 4 & $w_{266}$ & N & can. \\
3078 & 0 & 0 & 0 & 0 & 0 & 0 & 1 & 2 & 0 & 1 & 4 & 8 & 4 & 11 & 8 & 15 & 1 & 4 & $w_{267}$ & N & can. \\
3079 & 0 & 0 & 0 & 0 & 0 & 0 & 1 & 2 & 0 & 1 & 4 & 8 & 4 & 11 & 9 & 12 & 1 & 4 & $w_{266}$ & N & can. \\
3080 & 0 & 0 & 0 & 0 & 0 & 0 & 1 & 2 & 0 & 1 & 4 & 8 & 4 & 11 & 9 & 13 & 2 & 4 & $w_{347}$ & N & can. \\
3081 & 0 & 0 & 0 & 0 & 0 & 0 & 1 & 2 & 0 & 1 & 4 & 8 & 4 & 11 & 9 & 14 & 2 & 4 & $w_{354}$ & N & can. \\
3082 & 0 & 0 & 0 & 0 & 0 & 0 & 1 & 2 & 0 & 1 & 4 & 8 & 4 & 11 & 9 & 15 & 1 & 4 & $w_{267}$ & N & can. \\
3083 & 0 & 0 & 0 & 0 & 0 & 0 & 1 & 2 & 0 & 1 & 4 & 8 & 4 & 12 & 6 & 10 & 2 & 4 & $w_{352}$ & N & can. \\
3084 & 0 & 0 & 0 & 0 & 0 & 0 & 1 & 2 & 0 & 1 & 4 & 8 & 4 & 12 & 6 & 11 & 1 & 4 & $w_{281}$ & N & can. \\
3085 & 0 & 0 & 0 & 0 & 0 & 0 & 1 & 2 & 0 & 1 & 4 & 8 & 4 & 12 & 7 & 10 & 1 & 4 & $w_{379}$ & N & can. \\
3086 & 0 & 0 & 0 & 0 & 0 & 0 & 1 & 2 & 0 & 1 & 4 & 8 & 4 & 12 & 7 & 11 & 1 & 4 & $w_{379}$ & N & can. \\
3087 & 0 & 0 & 0 & 0 & 0 & 0 & 1 & 2 & 0 & 1 & 4 & 8 & 4 & 12 & 9 & 6 & 1 & 4 & $w_{260}$ & N & can. \\
3088 & 0 & 0 & 0 & 0 & 0 & 0 & 1 & 2 & 0 & 1 & 4 & 8 & 4 & 12 & 9 & 7 & 1 & 4 & $w_{392}$ & N & can. \\
3089 & 0 & 0 & 0 & 0 & 0 & 0 & 1 & 2 & 0 & 1 & 4 & 8 & 4 & 12 & 9 & 12 & 2 & 4 & $w_{257}$ & N & can. \\
3090 & 0 & 0 & 0 & 0 & 0 & 0 & 1 & 2 & 0 & 1 & 4 & 8 & 4 & 12 & 9 & 14 & 1 & 4 & $w_{266}$ & N & can. \\
3091 & 0 & 0 & 0 & 0 & 0 & 0 & 1 & 2 & 0 & 1 & 4 & 8 & 4 & 12 & 9 & 15 & 1 & 3 & $w_{280}$ & Y & \#2571 \\
3092 & 0 & 0 & 0 & 0 & 0 & 0 & 1 & 2 & 0 & 1 & 4 & 8 & 4 & 12 & 10 & 6 & 1 & 4 & $w_{260}$ & N & can. \\
3093 & 0 & 0 & 0 & 0 & 0 & 0 & 1 & 2 & 0 & 1 & 4 & 8 & 4 & 12 & 10 & 7 & 1 & 4 & $w_{379}$ & N & can. \\
3094 & 0 & 0 & 0 & 0 & 0 & 0 & 1 & 2 & 0 & 1 & 4 & 8 & 4 & 12 & 10 & 12 & 1 & 3 & $w_{259}$ & Y & \#3011 \\
3095 & 0 & 0 & 0 & 0 & 0 & 0 & 1 & 2 & 0 & 1 & 4 & 8 & 4 & 12 & 10 & 14 & 1 & 4 & $w_{392}$ & N & can. \\
3096 & 0 & 0 & 0 & 0 & 0 & 0 & 1 & 2 & 0 & 1 & 4 & 8 & 4 & 12 & 10 & 15 & 1 & 4 & $w_{379}$ & N & can. \\
3097 & 0 & 0 & 0 & 0 & 0 & 0 & 1 & 2 & 0 & 1 & 4 & 8 & 4 & 12 & 11 & 6 & 1 & 4 & $w_{281}$ & N & can. \\
3098 & 0 & 0 & 0 & 0 & 0 & 0 & 1 & 2 & 0 & 1 & 4 & 8 & 4 & 12 & 11 & 7 & 1 & 4 & $w_{355}$ & N & can. \\
3099 & 0 & 0 & 0 & 0 & 0 & 0 & 1 & 2 & 0 & 1 & 4 & 8 & 4 & 12 & 11 & 12 & 1 & 4 & $w_{266}$ & N & can. \\
3100 & 0 & 0 & 0 & 0 & 0 & 0 & 1 & 2 & 0 & 1 & 4 & 8 & 4 & 12 & 11 & 14 & 1 & 4 & $w_{355}$ & N & can. \\
3101 & 0 & 0 & 0 & 0 & 0 & 0 & 1 & 2 & 0 & 1 & 4 & 8 & 4 & 12 & 11 & 15 & 1 & 4 & $w_{428}$ & N & can. \\
3102 & 0 & 0 & 0 & 0 & 0 & 0 & 1 & 2 & 0 & 1 & 4 & 8 & 4 & 14 & 6 & 9 & 1 & 4 & $w_{268}$ & N & can. \\
3103 & 0 & 0 & 0 & 0 & 0 & 0 & 1 & 2 & 0 & 1 & 4 & 8 & 4 & 14 & 6 & 10 & 1 & 4 & $w_{226}$ & N & can. \\
3104 & 0 & 0 & 0 & 0 & 0 & 0 & 1 & 2 & 0 & 1 & 4 & 8 & 4 & 14 & 7 & 9 & 1 & 4 & $w_{379}$ & N & \#2649 \\
3105 & 0 & 0 & 0 & 0 & 0 & 0 & 1 & 2 & 0 & 1 & 4 & 8 & 4 & 14 & 7 & 10 & 1 & 4 & $w_{281}$ & N & \#3026 \\
3106 & 0 & 0 & 0 & 0 & 0 & 0 & 1 & 2 & 0 & 1 & 4 & 8 & 4 & 14 & 8 & 14 & 3 & 4 & $w_{260}$ & N & can. \\
3107 & 0 & 0 & 0 & 0 & 0 & 0 & 1 & 2 & 0 & 1 & 4 & 8 & 4 & 14 & 8 & 15 & 2 & 4 & $w_{267}$ & N & can. \\
3108 & 0 & 0 & 0 & 0 & 0 & 0 & 1 & 2 & 0 & 1 & 4 & 8 & 4 & 14 & 9 & 6 & 1 & 4 & $w_{373}$ & N & can. \\
3109 & 0 & 0 & 0 & 0 & 0 & 0 & 1 & 2 & 0 & 1 & 4 & 8 & 4 & 14 & 9 & 7 & 1 & 4 & $w_{281}$ & N & \#2619 \\
3110 & 0 & 0 & 0 & 0 & 0 & 0 & 1 & 2 & 0 & 1 & 4 & 8 & 4 & 14 & 9 & 12 & 1 & 4 & $w_{266}$ & N & \#3090 \\
3111 & 0 & 0 & 0 & 0 & 0 & 0 & 1 & 2 & 0 & 1 & 4 & 8 & 4 & 14 & 9 & 13 & 2 & 3 & $w_{378}$ & N & \#2590 \\
3112 & 0 & 0 & 0 & 0 & 0 & 0 & 1 & 2 & 0 & 1 & 4 & 8 & 4 & 14 & 9 & 14 & 2 & 4 & $w_{373}$ & N & can. \\
3113 & 0 & 0 & 0 & 0 & 0 & 0 & 1 & 2 & 0 & 1 & 4 & 8 & 4 & 14 & 9 & 15 & 1 & 4 & $w_{281}$ & N & can. \\
3114 & 0 & 0 & 0 & 0 & 0 & 0 & 1 & 2 & 0 & 1 & 4 & 8 & 4 & 14 & 11 & 6 & 1 & 4 & $w_{267}$ & N & can. \\
3115 & 0 & 0 & 0 & 0 & 0 & 0 & 1 & 2 & 0 & 1 & 4 & 8 & 4 & 14 & 11 & 12 & 1 & 4 & $w_{267}$ & N & can. \\
3116 & 0 & 0 & 0 & 0 & 0 & 0 & 1 & 2 & 0 & 1 & 4 & 8 & 4 & 14 & 11 & 15 & 1 & 3 & $w_{280}$ & Y & \#2571 \\
3117 & 0 & 0 & 0 & 0 & 0 & 0 & 1 & 2 & 0 & 1 & 4 & 8 & 5 & 9 & 6 & 10 & 2 & 4 & $w_{372}$ & N & can. \\
3118 & 0 & 0 & 0 & 0 & 0 & 0 & 1 & 2 & 0 & 1 & 4 & 8 & 5 & 9 & 6 & 11 & 2 & 4 & $w_{266}$ & N & can. \\
3119 & 0 & 0 & 0 & 0 & 0 & 0 & 1 & 2 & 0 & 1 & 4 & 8 & 5 & 9 & 6 & 12 & 2 & 4 & $w_{347}$ & N & can. \\
3120 & 0 & 0 & 0 & 0 & 0 & 0 & 1 & 2 & 0 & 1 & 4 & 8 & 5 & 9 & 6 & 14 & 1 & 4 & $w_{281}$ & N & can. \\
3121 & 0 & 0 & 0 & 0 & 0 & 0 & 1 & 2 & 0 & 1 & 4 & 8 & 5 & 9 & 6 & 15 & 1 & 4 & $w_{354}$ & N & can. \\
3122 & 0 & 0 & 0 & 0 & 0 & 0 & 1 & 2 & 0 & 1 & 4 & 8 & 5 & 10 & 6 & 9 & 2 & 4 & $w_{353}$ & N & can. \\
3123 & 0 & 0 & 0 & 0 & 0 & 0 & 1 & 2 & 0 & 1 & 4 & 8 & 5 & 10 & 6 & 11 & 1 & 4 & $w_{392}$ & N & can. \\
3124 & 0 & 0 & 0 & 0 & 0 & 0 & 1 & 2 & 0 & 1 & 4 & 8 & 5 & 10 & 6 & 12 & 1 & 4 & $w_{355}$ & N & can. \\
3125 & 0 & 0 & 0 & 0 & 0 & 0 & 1 & 2 & 0 & 1 & 4 & 8 & 5 & 10 & 6 & 13 & 1 & 4 & $w_{281}$ & N & can. \\
3126 & 0 & 0 & 0 & 0 & 0 & 0 & 1 & 2 & 0 & 1 & 4 & 8 & 5 & 10 & 6 & 14 & 1 & 4 & $w_{281}$ & N & can. \\
3127 & 0 & 0 & 0 & 0 & 0 & 0 & 1 & 2 & 0 & 1 & 4 & 8 & 5 & 10 & 6 & 15 & 1 & 4 & $w_{355}$ & N & can. \\
3128 & 0 & 0 & 0 & 0 & 0 & 0 & 1 & 2 & 0 & 1 & 4 & 8 & 5 & 10 & 9 & 12 & 1 & 4 & $w_{419}$ & N & can. \\
3129 & 0 & 0 & 0 & 0 & 0 & 0 & 1 & 2 & 0 & 1 & 4 & 8 & 5 & 10 & 9 & 13 & 1 & 4 & $w_{392}$ & N & can. \\
3130 & 0 & 0 & 0 & 0 & 0 & 0 & 1 & 2 & 0 & 1 & 4 & 8 & 5 & 10 & 9 & 14 & 1 & 4 & $w_{281}$ & N & can. \\
3131 & 0 & 0 & 0 & 0 & 0 & 0 & 1 & 2 & 0 & 1 & 4 & 8 & 5 & 10 & 9 & 15 & 1 & 4 & $w_{355}$ & N & can. \\
3132 & 0 & 0 & 0 & 0 & 0 & 0 & 1 & 2 & 0 & 1 & 4 & 8 & 5 & 11 & 6 & 10 & 2 & 4 & $w_{266}$ & N & can. \\
3133 & 0 & 0 & 0 & 0 & 0 & 0 & 1 & 2 & 0 & 1 & 4 & 8 & 5 & 11 & 6 & 12 & 1 & 4 & $w_{428}$ & N & \#2816 \\
3134 & 0 & 0 & 0 & 0 & 0 & 0 & 1 & 2 & 0 & 1 & 4 & 8 & 5 & 11 & 6 & 13 & 1 & 4 & $w_{379}$ & N & \#2785 \\
3135 & 0 & 0 & 0 & 0 & 0 & 0 & 1 & 2 & 0 & 1 & 4 & 8 & 5 & 11 & 6 & 14 & 1 & 4 & $w_{428}$ & N & can. \\
3136 & 0 & 0 & 0 & 0 & 0 & 0 & 1 & 2 & 0 & 1 & 4 & 8 & 5 & 11 & 6 & 15 & 1 & 4 & $w_{379}$ & N & can. \\
3137 & 0 & 0 & 0 & 0 & 0 & 0 & 1 & 2 & 0 & 1 & 4 & 8 & 5 & 11 & 9 & 12 & 1 & 4 & $w_{423}$ & N & can. \\
3138 & 0 & 0 & 0 & 0 & 0 & 0 & 1 & 2 & 0 & 1 & 4 & 8 & 5 & 11 & 9 & 13 & 1 & 4 & $w_{392}$ & N & can. \\
3139 & 0 & 0 & 0 & 0 & 0 & 0 & 1 & 2 & 0 & 1 & 4 & 8 & 5 & 11 & 9 & 14 & 1 & 4 & $w_{379}$ & N & can. \\
3140 & 0 & 0 & 0 & 0 & 0 & 0 & 1 & 2 & 0 & 1 & 4 & 8 & 5 & 11 & 9 & 15 & 1 & 4 & $w_{281}$ & N & can. \\
3141 & 0 & 0 & 0 & 0 & 0 & 0 & 1 & 2 & 0 & 1 & 4 & 8 & 5 & 12 & 6 & 10 & 2 & 4 & $w_{354}$ & N & can. \\
3142 & 0 & 0 & 0 & 0 & 0 & 0 & 1 & 2 & 0 & 1 & 4 & 8 & 5 & 12 & 6 & 11 & 1 & 4 & $w_{428}$ & N & can. \\
3143 & 0 & 0 & 0 & 0 & 0 & 0 & 1 & 2 & 0 & 1 & 4 & 8 & 5 & 12 & 6 & 14 & 1 & 4 & $w_{379}$ & N & can. \\
3144 & 0 & 0 & 0 & 0 & 0 & 0 & 1 & 2 & 0 & 1 & 4 & 8 & 5 & 12 & 6 & 15 & 1 & 4 & $w_{355}$ & N & can. \\
3145 & 0 & 0 & 0 & 0 & 0 & 0 & 1 & 2 & 0 & 1 & 4 & 8 & 5 & 12 & 7 & 10 & 1 & 4 & $w_{379}$ & N & can. \\
3146 & 0 & 0 & 0 & 0 & 0 & 0 & 1 & 2 & 0 & 1 & 4 & 8 & 5 & 12 & 7 & 11 & 1 & 4 & $w_{379}$ & N & can. \\
3147 & 0 & 0 & 0 & 0 & 0 & 0 & 1 & 2 & 0 & 1 & 4 & 8 & 5 & 12 & 7 & 14 & 1 & 4 & $w_{281}$ & N & can. \\
3148 & 0 & 0 & 0 & 0 & 0 & 0 & 1 & 2 & 0 & 1 & 4 & 8 & 5 & 12 & 7 & 15 & 1 & 4 & $w_{428}$ & N & can. \\
3149 & 0 & 0 & 0 & 0 & 0 & 0 & 1 & 2 & 0 & 1 & 4 & 8 & 5 & 12 & 9 & 13 & 6 & 4 & $w_{421}$ & N & can. \\
3150 & 0 & 0 & 0 & 0 & 0 & 0 & 1 & 2 & 0 & 1 & 4 & 8 & 5 & 12 & 9 & 14 & 1 & 3 & $w_{280}$ & N & can. \\
3151 & 0 & 0 & 0 & 0 & 0 & 0 & 1 & 2 & 0 & 1 & 4 & 8 & 5 & 12 & 9 & 15 & 1 & 4 & $w_{423}$ & N & can. \\
3152 & 0 & 0 & 0 & 0 & 0 & 0 & 1 & 2 & 0 & 1 & 4 & 8 & 5 & 12 & 10 & 14 & 2 & 4 & $w_{379}$ & N & can. \\
3153 & 0 & 0 & 0 & 0 & 0 & 0 & 1 & 2 & 0 & 1 & 4 & 8 & 5 & 12 & 10 & 15 & 1 & 4 & $w_{428}$ & N & can. \\
3154 & 0 & 0 & 0 & 0 & 0 & 0 & 1 & 2 & 0 & 1 & 4 & 8 & 5 & 12 & 11 & 15 & 2 & 4 & $w_{429}$ & N & can. \\
3155 & 0 & 0 & 0 & 0 & 0 & 0 & 1 & 2 & 0 & 1 & 4 & 8 & 5 & 12 & 13 & 9 & 6 & 4 & $w_{439}$ & N & can. \\
3156 & 0 & 0 & 0 & 0 & 0 & 0 & 1 & 2 & 0 & 1 & 4 & 8 & 5 & 12 & 13 & 10 & 1 & 3 & $w_{378}$ & N & can. \\
3157 & 0 & 0 & 0 & 0 & 0 & 0 & 1 & 2 & 0 & 1 & 4 & 8 & 5 & 12 & 13 & 11 & 1 & 4 & $w_{425}$ & N & can. \\
3158 & 0 & 0 & 0 & 0 & 0 & 0 & 1 & 2 & 0 & 1 & 4 & 8 & 5 & 12 & 14 & 10 & 2 & 4 & $w_{355}$ & N & can. \\
3159 & 0 & 0 & 0 & 0 & 0 & 0 & 1 & 2 & 0 & 1 & 4 & 8 & 5 & 12 & 14 & 11 & 1 & 4 & $w_{428}$ & N & \#3153 \\
3160 & 0 & 0 & 0 & 0 & 0 & 0 & 1 & 2 & 0 & 1 & 4 & 8 & 5 & 12 & 15 & 11 & 2 & 4 & $w_{430}$ & N & can. \\
3161 & 0 & 0 & 0 & 0 & 0 & 0 & 1 & 2 & 0 & 1 & 4 & 8 & 5 & 14 & 6 & 10 & 1 & 4 & $w_{268}$ & N & can. \\
3162 & 0 & 0 & 0 & 0 & 0 & 0 & 1 & 2 & 0 & 1 & 4 & 8 & 5 & 14 & 6 & 11 & 1 & 4 & $w_{379}$ & N & can. \\
3163 & 0 & 0 & 0 & 0 & 0 & 0 & 1 & 2 & 0 & 1 & 4 & 8 & 5 & 14 & 6 & 12 & 1 & 4 & $w_{281}$ & N & can. \\
3164 & 0 & 0 & 0 & 0 & 0 & 0 & 1 & 2 & 0 & 1 & 4 & 8 & 5 & 14 & 7 & 10 & 1 & 4 & $w_{281}$ & N & can. \\
3165 & 0 & 0 & 0 & 0 & 0 & 0 & 1 & 2 & 0 & 1 & 4 & 8 & 5 & 14 & 7 & 11 & 1 & 4 & $w_{355}$ & N & \#2798 \\
3166 & 0 & 0 & 0 & 0 & 0 & 0 & 1 & 2 & 0 & 1 & 4 & 8 & 5 & 14 & 7 & 13 & 1 & 4 & $w_{379}$ & N & \#2800 \\
3167 & 0 & 0 & 0 & 0 & 0 & 0 & 1 & 2 & 0 & 1 & 4 & 8 & 5 & 14 & 9 & 15 & 6 & 4 & $w_{355}$ & N & can. \\
3168 & 0 & 0 & 0 & 0 & 0 & 0 & 1 & 2 & 0 & 1 & 4 & 8 & 5 & 14 & 10 & 12 & 2 & 4 & $w_{355}$ & N & can. \\
3169 & 0 & 0 & 0 & 0 & 0 & 0 & 1 & 2 & 0 & 1 & 4 & 8 & 5 & 14 & 10 & 13 & 1 & 4 & $w_{379}$ & N & can. \\
3170 & 0 & 0 & 0 & 0 & 0 & 0 & 1 & 2 & 0 & 1 & 4 & 8 & 5 & 14 & 11 & 13 & 2 & 4 & $w_{379}$ & N & can. \\
3171 & 0 & 0 & 0 & 0 & 0 & 0 & 1 & 2 & 0 & 1 & 4 & 8 & 5 & 14 & 12 & 10 & 2 & 4 & $w_{267}$ & N & can. \\
3172 & 0 & 0 & 0 & 0 & 0 & 0 & 1 & 2 & 0 & 1 & 4 & 8 & 5 & 14 & 12 & 11 & 1 & 4 & $w_{379}$ & N & \#3169 \\
3173 & 0 & 0 & 0 & 0 & 0 & 0 & 1 & 2 & 0 & 1 & 4 & 8 & 5 & 14 & 13 & 11 & 2 & 4 & $w_{429}$ & N & can. \\
3174 & 0 & 0 & 0 & 0 & 0 & 0 & 1 & 2 & 0 & 1 & 4 & 8 & 5 & 14 & 15 & 9 & 6 & 4 & $w_{379}$ & N & can. \\
3175 & 0 & 0 & 0 & 0 & 0 & 0 & 1 & 2 & 0 & 1 & 4 & 8 & 6 & 12 & 10 & 14 & 3 & 3 & $w_{280}$ & N & can. \\
3176 & 0 & 0 & 0 & 0 & 0 & 0 & 1 & 2 & 0 & 1 & 4 & 8 & 6 & 12 & 10 & 15 & 1 & 4 & $w_{428}$ & N & can. \\
3177 & 0 & 0 & 0 & 0 & 0 & 0 & 1 & 2 & 0 & 1 & 4 & 8 & 6 & 12 & 11 & 14 & 1 & 4 & $w_{379}$ & N & can. \\
3178 & 0 & 0 & 0 & 0 & 0 & 0 & 1 & 2 & 0 & 1 & 4 & 8 & 6 & 12 & 11 & 15 & 1 & 3 & $w_{438}$ & N & can. \\
3179 & 0 & 0 & 0 & 0 & 0 & 0 & 1 & 2 & 0 & 1 & 4 & 8 & 6 & 12 & 14 & 10 & 3 & 3 & $w_{378}$ & N & can. \\
3180 & 0 & 0 & 0 & 0 & 0 & 0 & 1 & 2 & 0 & 1 & 4 & 8 & 6 & 12 & 15 & 11 & 1 & 3 & $w_{440}$ & N & can. \\
3181 & 0 & 0 & 0 & 0 & 0 & 0 & 1 & 2 & 0 & 1 & 4 & 8 & 7 & 12 & 11 & 15 & 3 & 4 & $w_{430}$ & N & can. \\
3182 & 0 & 0 & 0 & 0 & 0 & 0 & 1 & 2 & 0 & 1 & 4 & 8 & 7 & 12 & 15 & 11 & 3 & 4 & $w_{441}$ & N & can. \\
3183 & 0 & 0 & 0 & 0 & 0 & 0 & 1 & 2 & 0 & 3 & 0 & 4 & 0 & 5 & 6 & 0 & 192 & 4 & $w_{68}$ & N & can. \\
3184 & 0 & 0 & 0 & 0 & 0 & 0 & 1 & 2 & 0 & 3 & 0 & 4 & 0 & 5 & 6 & 8 & 4 & 4 & $w_{236}$ & N & can. \\
3185 & 0 & 0 & 0 & 0 & 0 & 0 & 1 & 2 & 0 & 3 & 0 & 4 & 0 & 5 & 8 & 0 & 8 & 4 & $w_{135}$ & N & can. \\
3186 & 0 & 0 & 0 & 0 & 0 & 0 & 1 & 2 & 0 & 3 & 0 & 4 & 0 & 5 & 8 & 11 & 2 & 4 & $w_{203}$ & N & can. \\
3187 & 0 & 0 & 0 & 0 & 0 & 0 & 1 & 2 & 0 & 3 & 0 & 4 & 0 & 5 & 8 & 12 & 1 & 4 & $w_{202}$ & N & can. \\
3188 & 0 & 0 & 0 & 0 & 0 & 0 & 1 & 2 & 0 & 3 & 0 & 4 & 0 & 5 & 8 & 14 & 1 & 4 & $w_{226}$ & N & can. \\
3189 & 0 & 0 & 0 & 0 & 0 & 0 & 1 & 2 & 0 & 3 & 0 & 4 & 0 & 7 & 5 & 8 & 6 & 4 & $w_{368}$ & N & can. \\
3190 & 0 & 0 & 0 & 0 & 0 & 0 & 1 & 2 & 0 & 3 & 0 & 4 & 0 & 7 & 8 & 5 & 2 & 4 & $w_{368}$ & N & \#3189 \\
3191 & 0 & 0 & 0 & 0 & 0 & 0 & 1 & 2 & 0 & 3 & 0 & 4 & 0 & 7 & 8 & 11 & 4 & 3 & $w_{223}$ & Y & can. \\
3192 & 0 & 0 & 0 & 0 & 0 & 0 & 1 & 2 & 0 & 3 & 0 & 4 & 0 & 7 & 8 & 12 & 2 & 4 & $w_{224}$ & N & can. \\
3193 & 0 & 0 & 0 & 0 & 0 & 0 & 1 & 2 & 0 & 3 & 0 & 4 & 0 & 7 & 8 & 13 & 2 & 4 & $w_{241}$ & N & can. \\
3194 & 0 & 0 & 0 & 0 & 0 & 0 & 1 & 2 & 0 & 3 & 0 & 4 & 0 & 8 & 5 & 9 & 2 & 3 & $w_{223}$ & Y & can. \\
3195 & 0 & 0 & 0 & 0 & 0 & 0 & 1 & 2 & 0 & 3 & 0 & 4 & 0 & 8 & 5 & 10 & 1 & 4 & $w_{230}$ & N & can. \\
3196 & 0 & 0 & 0 & 0 & 0 & 0 & 1 & 2 & 0 & 3 & 0 & 4 & 0 & 8 & 5 & 14 & 2 & 4 & $w_{239}$ & N & can. \\
3197 & 0 & 0 & 0 & 0 & 0 & 0 & 1 & 2 & 0 & 3 & 0 & 4 & 0 & 8 & 5 & 15 & 2 & 4 & $w_{238}$ & N & can. \\
3198 & 0 & 0 & 0 & 0 & 0 & 0 & 1 & 2 & 0 & 3 & 0 & 4 & 0 & 8 & 12 & 0 & 72 & 3 & $w_{136}$ & Y & can. \\
3199 & 0 & 0 & 0 & 0 & 0 & 0 & 1 & 2 & 0 & 3 & 0 & 4 & 0 & 8 & 12 & 5 & 2 & 4 & $w_{202}$ & N & can. \\
3200 & 0 & 0 & 0 & 0 & 0 & 0 & 1 & 2 & 0 & 3 & 0 & 4 & 0 & 8 & 12 & 15 & 8 & 4 & $w_{91}$ & N & can. \\
3201 & 0 & 0 & 0 & 0 & 0 & 0 & 1 & 2 & 0 & 3 & 0 & 4 & 0 & 8 & 13 & 0 & 12 & 4 & $w_{148}$ & N & can. \\
3202 & 0 & 0 & 0 & 0 & 0 & 0 & 1 & 2 & 0 & 3 & 0 & 4 & 0 & 8 & 13 & 5 & 1 & 4 & $w_{230}$ & N & \#3195 \\
3203 & 0 & 0 & 0 & 0 & 0 & 0 & 1 & 2 & 0 & 3 & 0 & 4 & 0 & 8 & 13 & 6 & 1 & 4 & $w_{239}$ & N & \#3196 \\
3204 & 0 & 0 & 0 & 0 & 0 & 0 & 1 & 2 & 0 & 3 & 0 & 4 & 0 & 8 & 13 & 14 & 2 & 4 & $w_{231}$ & N & can. \\
3205 & 0 & 0 & 0 & 0 & 0 & 0 & 1 & 2 & 0 & 3 & 0 & 4 & 0 & 8 & 15 & 5 & 1 & 4 & $w_{226}$ & N & can. \\
3206 & 0 & 0 & 0 & 0 & 0 & 0 & 1 & 2 & 0 & 3 & 0 & 4 & 1 & 4 & 3 & 5 & 48 & 4 & $w_{168}$ & N & can. \\
3207 & 0 & 0 & 0 & 0 & 0 & 0 & 1 & 2 & 0 & 3 & 0 & 4 & 1 & 4 & 3 & 8 & 1 & 4 & $w_{177}$ & N & can. \\
3208 & 0 & 0 & 0 & 0 & 0 & 0 & 1 & 2 & 0 & 3 & 0 & 4 & 1 & 4 & 6 & 8 & 7 & 4 & $w_{89}$ & N & can. \\
3209 & 0 & 0 & 0 & 0 & 0 & 0 & 1 & 2 & 0 & 3 & 0 & 4 & 1 & 4 & 8 & 11 & 1 & 4 & $w_{203}$ & N & can. \\
3210 & 0 & 0 & 0 & 0 & 0 & 0 & 1 & 2 & 0 & 3 & 0 & 4 & 1 & 4 & 8 & 12 & 1 & 4 & $w_{202}$ & N & can. \\
3211 & 0 & 0 & 0 & 0 & 0 & 0 & 1 & 2 & 0 & 3 & 0 & 4 & 1 & 4 & 8 & 13 & 1 & 4 & $w_{202}$ & N & can. \\
3212 & 0 & 0 & 0 & 0 & 0 & 0 & 1 & 2 & 0 & 3 & 0 & 4 & 1 & 4 & 8 & 14 & 1 & 4 & $w_{226}$ & N & can. \\
3213 & 0 & 0 & 0 & 0 & 0 & 0 & 1 & 2 & 0 & 3 & 0 & 4 & 1 & 4 & 8 & 15 & 1 & 4 & $w_{226}$ & N & can. \\
3214 & 0 & 0 & 0 & 0 & 0 & 0 & 1 & 2 & 0 & 3 & 0 & 4 & 1 & 5 & 3 & 4 & 16 & 4 & $w_{193}$ & N & can. \\
3215 & 0 & 0 & 0 & 0 & 0 & 0 & 1 & 2 & 0 & 3 & 0 & 4 & 1 & 5 & 3 & 7 & 16 & 4 & $w_{193}$ & N & \#3214 \\
3216 & 0 & 0 & 0 & 0 & 0 & 0 & 1 & 2 & 0 & 3 & 0 & 4 & 1 & 5 & 3 & 8 & 1 & 4 & $w_{194}$ & N & can. \\
3217 & 0 & 0 & 0 & 0 & 0 & 0 & 1 & 2 & 0 & 3 & 0 & 4 & 1 & 5 & 7 & 8 & 1 & 4 & $w_{368}$ & N & can. \\
3218 & 0 & 0 & 0 & 0 & 0 & 0 & 1 & 2 & 0 & 3 & 0 & 4 & 1 & 5 & 8 & 3 & 2 & 4 & $w_{194}$ & N & \#3216 \\
3219 & 0 & 0 & 0 & 0 & 0 & 0 & 1 & 2 & 0 & 3 & 0 & 4 & 1 & 5 & 8 & 4 & 2 & 4 & $w_{194}$ & N & \#3216 \\
3220 & 0 & 0 & 0 & 0 & 0 & 0 & 1 & 2 & 0 & 3 & 0 & 4 & 1 & 5 & 8 & 7 & 1 & 4 & $w_{368}$ & N & \#3217 \\
3221 & 0 & 0 & 0 & 0 & 0 & 0 & 1 & 2 & 0 & 3 & 0 & 4 & 1 & 5 & 8 & 11 & 2 & 4 & $w_{199}$ & N & can. \\
3222 & 0 & 0 & 0 & 0 & 0 & 0 & 1 & 2 & 0 & 3 & 0 & 4 & 1 & 5 & 8 & 12 & 2 & 4 & $w_{199}$ & N & \#3221 \\
3223 & 0 & 0 & 0 & 0 & 0 & 0 & 1 & 2 & 0 & 3 & 0 & 4 & 1 & 5 & 8 & 13 & 1 & 4 & $w_{225}$ & N & can. \\
3224 & 0 & 0 & 0 & 0 & 0 & 0 & 1 & 2 & 0 & 3 & 0 & 4 & 1 & 5 & 8 & 14 & 1 & 4 & $w_{226}$ & N & can. \\
3225 & 0 & 0 & 0 & 0 & 0 & 0 & 1 & 2 & 0 & 3 & 0 & 4 & 1 & 5 & 8 & 15 & 1 & 4 & $w_{224}$ & N & can. \\
3226 & 0 & 0 & 0 & 0 & 0 & 0 & 1 & 2 & 0 & 3 & 0 & 4 & 1 & 6 & 3 & 7 & 16 & 4 & $w_{384}$ & N & can. \\
3227 & 0 & 0 & 0 & 0 & 0 & 0 & 1 & 2 & 0 & 3 & 0 & 4 & 1 & 6 & 3 & 8 & 1 & 4 & $w_{333}$ & N & can. \\
3228 & 0 & 0 & 0 & 0 & 0 & 0 & 1 & 2 & 0 & 3 & 0 & 4 & 1 & 6 & 4 & 7 & 64 & 3 & $w_{84}$ & N & can. \\
3229 & 0 & 0 & 0 & 0 & 0 & 0 & 1 & 2 & 0 & 3 & 0 & 4 & 1 & 6 & 4 & 8 & 1 & 4 & $w_{222}$ & N & can. \\
3230 & 0 & 0 & 0 & 0 & 0 & 0 & 1 & 2 & 0 & 3 & 0 & 4 & 1 & 6 & 7 & 4 & 32 & 3 & $w_{84}$ & N & \#3228 \\
3231 & 0 & 0 & 0 & 0 & 0 & 0 & 1 & 2 & 0 & 3 & 0 & 4 & 1 & 6 & 7 & 8 & 1 & 4 & $w_{250}$ & N & can. \\
3232 & 0 & 0 & 0 & 0 & 0 & 0 & 1 & 2 & 0 & 3 & 0 & 4 & 1 & 6 & 8 & 3 & 2 & 4 & $w_{333}$ & N & \#3227 \\
3233 & 0 & 0 & 0 & 0 & 0 & 0 & 1 & 2 & 0 & 3 & 0 & 4 & 1 & 6 & 8 & 4 & 1 & 4 & $w_{222}$ & N & \#3229 \\
3234 & 0 & 0 & 0 & 0 & 0 & 0 & 1 & 2 & 0 & 3 & 0 & 4 & 1 & 6 & 8 & 7 & 1 & 4 & $w_{250}$ & N & \#3231 \\
3235 & 0 & 0 & 0 & 0 & 0 & 0 & 1 & 2 & 0 & 3 & 0 & 4 & 1 & 6 & 8 & 11 & 2 & 3 & $w_{223}$ & Y & can. \\
3236 & 0 & 0 & 0 & 0 & 0 & 0 & 1 & 2 & 0 & 3 & 0 & 4 & 1 & 6 & 8 & 12 & 1 & 4 & $w_{224}$ & N & can. \\
3237 & 0 & 0 & 0 & 0 & 0 & 0 & 1 & 2 & 0 & 3 & 0 & 4 & 1 & 6 & 8 & 13 & 1 & 4 & $w_{241}$ & N & can. \\
3238 & 0 & 0 & 0 & 0 & 0 & 0 & 1 & 2 & 0 & 3 & 0 & 4 & 1 & 6 & 8 & 14 & 1 & 4 & $w_{241}$ & N & can. \\
3239 & 0 & 0 & 0 & 0 & 0 & 0 & 1 & 2 & 0 & 3 & 0 & 4 & 1 & 6 & 8 & 15 & 1 & 4 & $w_{224}$ & N & can. \\
3240 & 0 & 0 & 0 & 0 & 0 & 0 & 1 & 2 & 0 & 3 & 0 & 4 & 1 & 8 & 3 & 13 & 2 & 4 & $w_{390}$ & N & can. \\
3241 & 0 & 0 & 0 & 0 & 0 & 0 & 1 & 2 & 0 & 3 & 0 & 4 & 1 & 8 & 3 & 14 & 2 & 3 & $w_{223}$ & Y & can. \\
3242 & 0 & 0 & 0 & 0 & 0 & 0 & 1 & 2 & 0 & 3 & 0 & 4 & 1 & 8 & 3 & 15 & 2 & 4 & $w_{182}$ & N & can. \\
3243 & 0 & 0 & 0 & 0 & 0 & 0 & 1 & 2 & 0 & 3 & 0 & 4 & 1 & 8 & 4 & 9 & 2 & 3 & $w_{223}$ & Y & \#3235 \\
3244 & 0 & 0 & 0 & 0 & 0 & 0 & 1 & 2 & 0 & 3 & 0 & 4 & 1 & 8 & 4 & 10 & 1 & 4 & $w_{230}$ & N & can. \\
3245 & 0 & 0 & 0 & 0 & 0 & 0 & 1 & 2 & 0 & 3 & 0 & 4 & 1 & 8 & 4 & 14 & 2 & 4 & $w_{239}$ & N & can. \\
3246 & 0 & 0 & 0 & 0 & 0 & 0 & 1 & 2 & 0 & 3 & 0 & 4 & 1 & 8 & 4 & 15 & 2 & 4 & $w_{238}$ & N & can. \\
3247 & 0 & 0 & 0 & 0 & 0 & 0 & 1 & 2 & 0 & 3 & 0 & 4 & 1 & 8 & 7 & 4 & 1 & 4 & $w_{222}$ & N & \#3229 \\
3248 & 0 & 0 & 0 & 0 & 0 & 0 & 1 & 2 & 0 & 3 & 0 & 4 & 1 & 8 & 7 & 9 & 1 & 4 & $w_{391}$ & N & can. \\
3249 & 0 & 0 & 0 & 0 & 0 & 0 & 1 & 2 & 0 & 3 & 0 & 4 & 1 & 8 & 7 & 10 & 1 & 3 & $w_{251}$ & Y & can. \\
3250 & 0 & 0 & 0 & 0 & 0 & 0 & 1 & 2 & 0 & 3 & 0 & 4 & 1 & 8 & 7 & 13 & 1 & 4 & $w_{264}$ & N & can. \\
3251 & 0 & 0 & 0 & 0 & 0 & 0 & 1 & 2 & 0 & 3 & 0 & 4 & 1 & 8 & 7 & 14 & 1 & 4 & $w_{254}$ & N & can. \\
3252 & 0 & 0 & 0 & 0 & 0 & 0 & 1 & 2 & 0 & 3 & 0 & 4 & 1 & 8 & 7 & 15 & 1 & 4 & $w_{241}$ & N & can. \\
3253 & 0 & 0 & 0 & 0 & 0 & 0 & 1 & 2 & 0 & 3 & 0 & 4 & 1 & 8 & 8 & 4 & 1 & 4 & $w_{182}$ & N & can. \\
3254 & 0 & 0 & 0 & 0 & 0 & 0 & 1 & 2 & 0 & 3 & 0 & 4 & 1 & 8 & 8 & 5 & 4 & 3 & $w_{181}$ & Y & can. \\
3255 & 0 & 0 & 0 & 0 & 0 & 0 & 1 & 2 & 0 & 3 & 0 & 4 & 1 & 8 & 8 & 6 & 1 & 4 & $w_{201}$ & N & can. \\
3256 & 0 & 0 & 0 & 0 & 0 & 0 & 1 & 2 & 0 & 3 & 0 & 4 & 1 & 8 & 8 & 7 & 1 & 4 & $w_{225}$ & N & can. \\
3257 & 0 & 0 & 0 & 0 & 0 & 0 & 1 & 2 & 0 & 3 & 0 & 4 & 1 & 8 & 8 & 13 & 1 & 4 & $w_{225}$ & N & can. \\
3258 & 0 & 0 & 0 & 0 & 0 & 0 & 1 & 2 & 0 & 3 & 0 & 4 & 1 & 8 & 8 & 14 & 1 & 4 & $w_{226}$ & N & can. \\
3259 & 0 & 0 & 0 & 0 & 0 & 0 & 1 & 2 & 0 & 3 & 0 & 4 & 1 & 8 & 8 & 15 & 1 & 4 & $w_{202}$ & N & can. \\
3260 & 0 & 0 & 0 & 0 & 0 & 0 & 1 & 2 & 0 & 3 & 0 & 4 & 1 & 8 & 9 & 4 & 1 & 3 & $w_{223}$ & Y & \#3235 \\
3261 & 0 & 0 & 0 & 0 & 0 & 0 & 1 & 2 & 0 & 3 & 0 & 4 & 1 & 8 & 9 & 6 & 1 & 4 & $w_{225}$ & N & can. \\
3262 & 0 & 0 & 0 & 0 & 0 & 0 & 1 & 2 & 0 & 3 & 0 & 4 & 1 & 8 & 9 & 7 & 1 & 4 & $w_{391}$ & N & \#3248 \\
3263 & 0 & 0 & 0 & 0 & 0 & 0 & 1 & 2 & 0 & 3 & 0 & 4 & 1 & 8 & 9 & 13 & 1 & 4 & $w_{224}$ & N & \#3236 \\
3264 & 0 & 0 & 0 & 0 & 0 & 0 & 1 & 2 & 0 & 3 & 0 & 4 & 1 & 8 & 9 & 14 & 1 & 4 & $w_{254}$ & N & can. \\
3265 & 0 & 0 & 0 & 0 & 0 & 0 & 1 & 2 & 0 & 3 & 0 & 4 & 1 & 8 & 9 & 15 & 1 & 4 & $w_{226}$ & N & can. \\
3266 & 0 & 0 & 0 & 0 & 0 & 0 & 1 & 2 & 0 & 3 & 0 & 4 & 1 & 8 & 10 & 4 & 1 & 4 & $w_{230}$ & N & \#3244 \\
3267 & 0 & 0 & 0 & 0 & 0 & 0 & 1 & 2 & 0 & 3 & 0 & 4 & 1 & 8 & 10 & 6 & 1 & 4 & $w_{225}$ & N & can. \\
3268 & 0 & 0 & 0 & 0 & 0 & 0 & 1 & 2 & 0 & 3 & 0 & 4 & 1 & 8 & 10 & 7 & 1 & 3 & $w_{251}$ & Y & \#3249 \\
3269 & 0 & 0 & 0 & 0 & 0 & 0 & 1 & 2 & 0 & 3 & 0 & 4 & 1 & 8 & 10 & 13 & 1 & 4 & $w_{254}$ & N & can. \\
3270 & 0 & 0 & 0 & 0 & 0 & 0 & 1 & 2 & 0 & 3 & 0 & 4 & 1 & 8 & 10 & 14 & 1 & 4 & $w_{239}$ & N & can. \\
3271 & 0 & 0 & 0 & 0 & 0 & 0 & 1 & 2 & 0 & 3 & 0 & 4 & 1 & 8 & 10 & 15 & 1 & 4 & $w_{226}$ & N & can. \\
3272 & 0 & 0 & 0 & 0 & 0 & 0 & 1 & 2 & 0 & 3 & 0 & 4 & 1 & 8 & 12 & 4 & 1 & 4 & $w_{202}$ & N & can. \\
3273 & 0 & 0 & 0 & 0 & 0 & 0 & 1 & 2 & 0 & 3 & 0 & 4 & 1 & 8 & 12 & 6 & 1 & 4 & $w_{225}$ & N & can. \\
3274 & 0 & 0 & 0 & 0 & 0 & 0 & 1 & 2 & 0 & 3 & 0 & 4 & 1 & 8 & 12 & 7 & 1 & 4 & $w_{241}$ & N & can. \\
3275 & 0 & 0 & 0 & 0 & 0 & 0 & 1 & 2 & 0 & 3 & 0 & 4 & 1 & 8 & 12 & 9 & 1 & 4 & $w_{225}$ & N & can. \\
3276 & 0 & 0 & 0 & 0 & 0 & 0 & 1 & 2 & 0 & 3 & 0 & 4 & 1 & 8 & 12 & 10 & 1 & 4 & $w_{226}$ & N & can. \\
3277 & 0 & 0 & 0 & 0 & 0 & 0 & 1 & 2 & 0 & 3 & 0 & 4 & 1 & 8 & 13 & 4 & 1 & 4 & $w_{230}$ & N & \#3244 \\
3278 & 0 & 0 & 0 & 0 & 0 & 0 & 1 & 2 & 0 & 3 & 0 & 4 & 1 & 8 & 13 & 6 & 1 & 4 & $w_{241}$ & N & can. \\
3279 & 0 & 0 & 0 & 0 & 0 & 0 & 1 & 2 & 0 & 3 & 0 & 4 & 1 & 8 & 13 & 7 & 1 & 4 & $w_{264}$ & N & \#3250 \\
3280 & 0 & 0 & 0 & 0 & 0 & 0 & 1 & 2 & 0 & 3 & 0 & 4 & 1 & 8 & 13 & 9 & 1 & 4 & $w_{224}$ & N & \#3236 \\
3281 & 0 & 0 & 0 & 0 & 0 & 0 & 1 & 2 & 0 & 3 & 0 & 4 & 1 & 8 & 13 & 10 & 1 & 4 & $w_{254}$ & N & \#3269 \\
3282 & 0 & 0 & 0 & 0 & 0 & 0 & 1 & 2 & 0 & 3 & 0 & 4 & 1 & 8 & 13 & 14 & 2 & 4 & $w_{239}$ & N & \#3245 \\
3283 & 0 & 0 & 0 & 0 & 0 & 0 & 1 & 2 & 0 & 3 & 0 & 4 & 1 & 8 & 14 & 4 & 1 & 4 & $w_{239}$ & N & \#3245 \\
3284 & 0 & 0 & 0 & 0 & 0 & 0 & 1 & 2 & 0 & 3 & 0 & 4 & 1 & 8 & 14 & 6 & 1 & 4 & $w_{241}$ & N & can. \\
3285 & 0 & 0 & 0 & 0 & 0 & 0 & 1 & 2 & 0 & 3 & 0 & 4 & 1 & 8 & 14 & 7 & 1 & 4 & $w_{254}$ & N & \#3251 \\
3286 & 0 & 0 & 0 & 0 & 0 & 0 & 1 & 2 & 0 & 3 & 0 & 4 & 1 & 8 & 14 & 9 & 1 & 4 & $w_{254}$ & N & \#3264 \\
3287 & 0 & 0 & 0 & 0 & 0 & 0 & 1 & 2 & 0 & 3 & 0 & 4 & 1 & 8 & 14 & 10 & 1 & 4 & $w_{239}$ & N & \#3270 \\
3288 & 0 & 0 & 0 & 0 & 0 & 0 & 1 & 2 & 0 & 3 & 0 & 4 & 1 & 8 & 14 & 13 & 2 & 4 & $w_{239}$ & N & \#3245 \\
3289 & 0 & 0 & 0 & 0 & 0 & 0 & 1 & 2 & 0 & 3 & 0 & 4 & 1 & 8 & 15 & 6 & 1 & 4 & $w_{225}$ & N & can. \\
3290 & 0 & 0 & 0 & 0 & 0 & 0 & 1 & 2 & 0 & 3 & 0 & 4 & 1 & 8 & 15 & 7 & 1 & 4 & $w_{260}$ & N & can. \\
3291 & 0 & 0 & 0 & 0 & 0 & 0 & 1 & 2 & 0 & 3 & 0 & 4 & 1 & 8 & 15 & 9 & 1 & 4 & $w_{241}$ & N & \#3237 \\
3292 & 0 & 0 & 0 & 0 & 0 & 0 & 1 & 2 & 0 & 3 & 0 & 4 & 1 & 8 & 15 & 10 & 1 & 4 & $w_{241}$ & N & can. \\
3293 & 0 & 0 & 0 & 0 & 0 & 0 & 1 & 2 & 0 & 3 & 0 & 4 & 3 & 4 & 8 & 11 & 4 & 3 & $w_{223}$ & N & can. \\
3294 & 0 & 0 & 0 & 0 & 0 & 0 & 1 & 2 & 0 & 3 & 0 & 4 & 3 & 4 & 8 & 12 & 2 & 4 & $w_{224}$ & N & can. \\
3295 & 0 & 0 & 0 & 0 & 0 & 0 & 1 & 2 & 0 & 3 & 0 & 4 & 3 & 4 & 8 & 13 & 2 & 4 & $w_{241}$ & N & can. \\
3296 & 0 & 0 & 0 & 0 & 0 & 0 & 1 & 2 & 0 & 3 & 0 & 4 & 3 & 5 & 5 & 8 & 1 & 4 & $w_{235}$ & N & can. \\
3297 & 0 & 0 & 0 & 0 & 0 & 0 & 1 & 2 & 0 & 3 & 0 & 4 & 3 & 5 & 6 & 8 & 1 & 4 & $w_{240}$ & N & can. \\
3298 & 0 & 0 & 0 & 0 & 0 & 0 & 1 & 2 & 0 & 3 & 0 & 4 & 3 & 5 & 8 & 11 & 2 & 4 & $w_{225}$ & N & can. \\
3299 & 0 & 0 & 0 & 0 & 0 & 0 & 1 & 2 & 0 & 3 & 0 & 4 & 3 & 5 & 8 & 12 & 1 & 4 & $w_{241}$ & N & can. \\
3300 & 0 & 0 & 0 & 0 & 0 & 0 & 1 & 2 & 0 & 3 & 0 & 4 & 3 & 5 & 8 & 13 & 1 & 4 & $w_{260}$ & N & can. \\
3301 & 0 & 0 & 0 & 0 & 0 & 0 & 1 & 2 & 0 & 3 & 0 & 4 & 3 & 7 & 5 & 8 & 1 & 4 & $w_{250}$ & N & can. \\
3302 & 0 & 0 & 0 & 0 & 0 & 0 & 1 & 2 & 0 & 3 & 0 & 4 & 3 & 7 & 8 & 11 & 4 & 4 & $w_{389}$ & N & can. \\
3303 & 0 & 0 & 0 & 0 & 0 & 0 & 1 & 2 & 0 & 3 & 0 & 4 & 3 & 7 & 8 & 12 & 4 & 4 & $w_{224}$ & N & can. \\
3304 & 0 & 0 & 0 & 0 & 0 & 0 & 1 & 2 & 0 & 3 & 0 & 4 & 3 & 7 & 8 & 13 & 2 & 4 & $w_{260}$ & N & can. \\
3305 & 0 & 0 & 0 & 0 & 0 & 0 & 1 & 2 & 0 & 3 & 0 & 4 & 3 & 7 & 8 & 15 & 4 & 4 & $w_{263}$ & N & can. \\
3306 & 0 & 0 & 0 & 0 & 0 & 0 & 1 & 2 & 0 & 3 & 0 & 4 & 3 & 8 & 5 & 9 & 2 & 4 & $w_{391}$ & N & can. \\
3307 & 0 & 0 & 0 & 0 & 0 & 0 & 1 & 2 & 0 & 3 & 0 & 4 & 3 & 8 & 5 & 10 & 2 & 3 & $w_{251}$ & Y & can. \\
3308 & 0 & 0 & 0 & 0 & 0 & 0 & 1 & 2 & 0 & 3 & 0 & 4 & 3 & 8 & 5 & 13 & 1 & 4 & $w_{264}$ & N & can. \\
3309 & 0 & 0 & 0 & 0 & 0 & 0 & 1 & 2 & 0 & 3 & 0 & 4 & 3 & 8 & 5 & 14 & 1 & 4 & $w_{254}$ & N & can. \\
3310 & 0 & 0 & 0 & 0 & 0 & 0 & 1 & 2 & 0 & 3 & 0 & 4 & 3 & 8 & 5 & 15 & 1 & 4 & $w_{241}$ & N & can. \\
3311 & 0 & 0 & 0 & 0 & 0 & 0 & 1 & 2 & 0 & 3 & 0 & 4 & 3 & 8 & 8 & 4 & 4 & 4 & $w_{201}$ & N & can. \\
3312 & 0 & 0 & 0 & 0 & 0 & 0 & 1 & 2 & 0 & 3 & 0 & 4 & 3 & 8 & 8 & 5 & 1 & 4 & $w_{225}$ & N & can. \\
3313 & 0 & 0 & 0 & 0 & 0 & 0 & 1 & 2 & 0 & 3 & 0 & 4 & 3 & 8 & 8 & 7 & 2 & 3 & $w_{237}$ & Y & can. \\
3314 & 0 & 0 & 0 & 0 & 0 & 0 & 1 & 2 & 0 & 3 & 0 & 4 & 3 & 8 & 8 & 13 & 1 & 4 & $w_{241}$ & N & can. \\
3315 & 0 & 0 & 0 & 0 & 0 & 0 & 1 & 2 & 0 & 3 & 0 & 4 & 3 & 8 & 8 & 15 & 2 & 4 & $w_{225}$ & N & can. \\
3316 & 0 & 0 & 0 & 0 & 0 & 0 & 1 & 2 & 0 & 3 & 0 & 4 & 3 & 8 & 9 & 5 & 2 & 4 & $w_{391}$ & N & \#3306 \\
3317 & 0 & 0 & 0 & 0 & 0 & 0 & 1 & 2 & 0 & 3 & 0 & 4 & 3 & 8 & 9 & 6 & 2 & 3 & $w_{251}$ & Y & \#3307 \\
3318 & 0 & 0 & 0 & 0 & 0 & 0 & 1 & 2 & 0 & 3 & 0 & 4 & 3 & 8 & 9 & 7 & 1 & 4 & $w_{262}$ & N & can. \\
3319 & 0 & 0 & 0 & 0 & 0 & 0 & 1 & 2 & 0 & 3 & 0 & 4 & 3 & 8 & 9 & 13 & 1 & 4 & $w_{254}$ & N & \#3309 \\
3320 & 0 & 0 & 0 & 0 & 0 & 0 & 1 & 2 & 0 & 3 & 0 & 4 & 3 & 8 & 9 & 14 & 1 & 4 & $w_{264}$ & N & \#3308 \\
3321 & 0 & 0 & 0 & 0 & 0 & 0 & 1 & 2 & 0 & 3 & 0 & 4 & 3 & 8 & 9 & 15 & 1 & 4 & $w_{241}$ & N & can. \\
3322 & 0 & 0 & 0 & 0 & 0 & 0 & 1 & 2 & 0 & 3 & 0 & 4 & 3 & 8 & 11 & 7 & 4 & 4 & $w_{388}$ & N & can. \\
3323 & 0 & 0 & 0 & 0 & 0 & 0 & 1 & 2 & 0 & 3 & 0 & 4 & 3 & 8 & 11 & 13 & 1 & 4 & $w_{260}$ & N & can. \\
3324 & 0 & 0 & 0 & 0 & 0 & 0 & 1 & 2 & 0 & 3 & 0 & 4 & 3 & 8 & 11 & 15 & 2 & 4 & $w_{225}$ & N & can. \\
3325 & 0 & 0 & 0 & 0 & 0 & 0 & 1 & 2 & 0 & 3 & 0 & 4 & 3 & 8 & 12 & 15 & 4 & 4 & $w_{202}$ & N & can. \\
3326 & 0 & 0 & 0 & 0 & 0 & 0 & 1 & 2 & 0 & 3 & 0 & 4 & 3 & 8 & 13 & 14 & 2 & 4 & $w_{254}$ & N & can. \\
3327 & 0 & 0 & 0 & 0 & 0 & 0 & 1 & 2 & 0 & 3 & 0 & 4 & 4 & 7 & 8 & 12 & 4 & 4 & $w_{224}$ & N & \#3303 \\
3328 & 0 & 0 & 0 & 0 & 0 & 0 & 1 & 2 & 0 & 3 & 0 & 4 & 4 & 7 & 8 & 15 & 4 & 3 & $w_{251}$ & N & can. \\
3329 & 0 & 0 & 0 & 0 & 0 & 0 & 1 & 2 & 0 & 3 & 0 & 4 & 4 & 8 & 5 & 9 & 1 & 4 & $w_{230}$ & N & \#3244 \\
3330 & 0 & 0 & 0 & 0 & 0 & 0 & 1 & 2 & 0 & 3 & 0 & 4 & 4 & 8 & 5 & 10 & 1 & 4 & $w_{239}$ & N & \#3270 \\
3331 & 0 & 0 & 0 & 0 & 0 & 0 & 1 & 2 & 0 & 3 & 0 & 4 & 4 & 8 & 5 & 13 & 1 & 3 & $w_{251}$ & Y & \#3249 \\
3332 & 0 & 0 & 0 & 0 & 0 & 0 & 1 & 2 & 0 & 3 & 0 & 4 & 4 & 8 & 5 & 14 & 1 & 4 & $w_{254}$ & N & \#3269 \\
3333 & 0 & 0 & 0 & 0 & 0 & 0 & 1 & 2 & 0 & 3 & 0 & 4 & 4 & 8 & 5 & 15 & 1 & 4 & $w_{226}$ & N & can. \\
3334 & 0 & 0 & 0 & 0 & 0 & 0 & 1 & 2 & 0 & 3 & 0 & 4 & 4 & 8 & 8 & 7 & 2 & 4 & $w_{225}$ & N & can. \\
3335 & 0 & 0 & 0 & 0 & 0 & 0 & 1 & 2 & 0 & 3 & 0 & 4 & 4 & 8 & 8 & 13 & 1 & 4 & $w_{241}$ & N & can. \\
3336 & 0 & 0 & 0 & 0 & 0 & 0 & 1 & 2 & 0 & 3 & 0 & 4 & 4 & 8 & 8 & 15 & 2 & 4 & $w_{226}$ & N & can. \\
3337 & 0 & 0 & 0 & 0 & 0 & 0 & 1 & 2 & 0 & 3 & 0 & 4 & 4 & 8 & 9 & 7 & 1 & 4 & $w_{241}$ & N & can. \\
3338 & 0 & 0 & 0 & 0 & 0 & 0 & 1 & 2 & 0 & 3 & 0 & 4 & 4 & 8 & 9 & 10 & 1 & 4 & $w_{231}$ & N & can. \\
3339 & 0 & 0 & 0 & 0 & 0 & 0 & 1 & 2 & 0 & 3 & 0 & 4 & 4 & 8 & 9 & 13 & 1 & 4 & $w_{254}$ & N & \#3309 \\
3340 & 0 & 0 & 0 & 0 & 0 & 0 & 1 & 2 & 0 & 3 & 0 & 4 & 4 & 8 & 9 & 14 & 1 & 4 & $w_{265}$ & N & can. \\
3341 & 0 & 0 & 0 & 0 & 0 & 0 & 1 & 2 & 0 & 3 & 0 & 4 & 4 & 8 & 9 & 15 & 1 & 4 & $w_{244}$ & N & can. \\
3342 & 0 & 0 & 0 & 0 & 0 & 0 & 1 & 2 & 0 & 3 & 0 & 4 & 4 & 8 & 11 & 7 & 2 & 4 & $w_{225}$ & N & can. \\
3343 & 0 & 0 & 0 & 0 & 0 & 0 & 1 & 2 & 0 & 3 & 0 & 4 & 4 & 8 & 11 & 13 & 1 & 4 & $w_{268}$ & N & can. \\
3344 & 0 & 0 & 0 & 0 & 0 & 0 & 1 & 2 & 0 & 3 & 0 & 4 & 4 & 8 & 11 & 15 & 2 & 4 & $w_{226}$ & N & can. \\
3345 & 0 & 0 & 0 & 0 & 0 & 0 & 1 & 2 & 0 & 3 & 0 & 4 & 4 & 8 & 12 & 7 & 2 & 4 & $w_{262}$ & N & can. \\
3346 & 0 & 0 & 0 & 0 & 0 & 0 & 1 & 2 & 0 & 3 & 0 & 4 & 4 & 8 & 12 & 9 & 1 & 4 & $w_{241}$ & N & \#3310 \\
3347 & 0 & 0 & 0 & 0 & 0 & 0 & 1 & 2 & 0 & 3 & 0 & 4 & 4 & 8 & 12 & 15 & 2 & 4 & $w_{238}$ & N & can. \\
3348 & 0 & 0 & 0 & 0 & 0 & 0 & 1 & 2 & 0 & 3 & 0 & 4 & 4 & 8 & 13 & 7 & 1 & 4 & $w_{260}$ & N & can. \\
3349 & 0 & 0 & 0 & 0 & 0 & 0 & 1 & 2 & 0 & 3 & 0 & 4 & 4 & 8 & 13 & 9 & 1 & 4 & $w_{254}$ & N & \#3309 \\
3350 & 0 & 0 & 0 & 0 & 0 & 0 & 1 & 2 & 0 & 3 & 0 & 4 & 4 & 8 & 13 & 10 & 1 & 4 & $w_{265}$ & N & \#3340 \\
3351 & 0 & 0 & 0 & 0 & 0 & 0 & 1 & 2 & 0 & 3 & 0 & 4 & 5 & 8 & 7 & 9 & 1 & 4 & $w_{264}$ & N & can. \\
3352 & 0 & 0 & 0 & 0 & 0 & 0 & 1 & 2 & 0 & 3 & 0 & 4 & 5 & 8 & 7 & 10 & 1 & 4 & $w_{254}$ & N & can. \\
3353 & 0 & 0 & 0 & 0 & 0 & 0 & 1 & 2 & 0 & 3 & 0 & 4 & 5 & 8 & 7 & 13 & 1 & 4 & $w_{393}$ & N & can. \\
3354 & 0 & 0 & 0 & 0 & 0 & 0 & 1 & 2 & 0 & 3 & 0 & 4 & 5 & 8 & 7 & 14 & 1 & 3 & $w_{277}$ & Y & can. \\
3355 & 0 & 0 & 0 & 0 & 0 & 0 & 1 & 2 & 0 & 3 & 0 & 4 & 5 & 8 & 7 & 15 & 1 & 4 & $w_{260}$ & N & can. \\
3356 & 0 & 0 & 0 & 0 & 0 & 0 & 1 & 2 & 0 & 3 & 0 & 4 & 5 & 8 & 8 & 6 & 4 & 4 & $w_{225}$ & N & can. \\
3357 & 0 & 0 & 0 & 0 & 0 & 0 & 1 & 2 & 0 & 3 & 0 & 4 & 5 & 8 & 8 & 7 & 1 & 4 & $w_{241}$ & N & can. \\
3358 & 0 & 0 & 0 & 0 & 0 & 0 & 1 & 2 & 0 & 3 & 0 & 4 & 5 & 8 & 8 & 13 & 1 & 4 & $w_{260}$ & N & can. \\
3359 & 0 & 0 & 0 & 0 & 0 & 0 & 1 & 2 & 0 & 3 & 0 & 4 & 5 & 8 & 8 & 14 & 1 & 4 & $w_{268}$ & N & can. \\
3360 & 0 & 0 & 0 & 0 & 0 & 0 & 1 & 2 & 0 & 3 & 0 & 4 & 5 & 8 & 9 & 6 & 1 & 4 & $w_{241}$ & N & can. \\
3361 & 0 & 0 & 0 & 0 & 0 & 0 & 1 & 2 & 0 & 3 & 0 & 4 & 5 & 8 & 9 & 7 & 1 & 4 & $w_{264}$ & N & \#3351 \\
3362 & 0 & 0 & 0 & 0 & 0 & 0 & 1 & 2 & 0 & 3 & 0 & 4 & 5 & 8 & 9 & 10 & 2 & 4 & $w_{239}$ & N & can. \\
3363 & 0 & 0 & 0 & 0 & 0 & 0 & 1 & 2 & 0 & 3 & 0 & 4 & 5 & 8 & 9 & 13 & 1 & 4 & $w_{264}$ & N & \#3351 \\
3364 & 0 & 0 & 0 & 0 & 0 & 0 & 1 & 2 & 0 & 3 & 0 & 4 & 5 & 8 & 9 & 14 & 1 & 4 & $w_{374}$ & N & can. \\
3365 & 0 & 0 & 0 & 0 & 0 & 0 & 1 & 2 & 0 & 3 & 0 & 4 & 5 & 8 & 9 & 15 & 1 & 4 & $w_{268}$ & N & can. \\
3366 & 0 & 0 & 0 & 0 & 0 & 0 & 1 & 2 & 0 & 3 & 0 & 4 & 5 & 8 & 10 & 7 & 1 & 4 & $w_{254}$ & N & \#3352 \\
3367 & 0 & 0 & 0 & 0 & 0 & 0 & 1 & 2 & 0 & 3 & 0 & 4 & 5 & 8 & 10 & 9 & 2 & 4 & $w_{239}$ & N & \#3362 \\
3368 & 0 & 0 & 0 & 0 & 0 & 0 & 1 & 2 & 0 & 3 & 0 & 4 & 5 & 8 & 10 & 13 & 1 & 4 & $w_{374}$ & N & can. \\
3369 & 0 & 0 & 0 & 0 & 0 & 0 & 1 & 2 & 0 & 3 & 0 & 4 & 5 & 8 & 10 & 14 & 1 & 4 & $w_{265}$ & N & can. \\
3370 & 0 & 0 & 0 & 0 & 0 & 0 & 1 & 2 & 0 & 3 & 0 & 4 & 5 & 8 & 10 & 15 & 1 & 4 & $w_{268}$ & N & can. \\
3371 & 0 & 0 & 0 & 0 & 0 & 0 & 1 & 2 & 0 & 3 & 0 & 4 & 5 & 8 & 11 & 6 & 2 & 4 & $w_{225}$ & N & can. \\
3372 & 0 & 0 & 0 & 0 & 0 & 0 & 1 & 2 & 0 & 3 & 0 & 4 & 5 & 8 & 11 & 7 & 1 & 4 & $w_{260}$ & N & can. \\
3373 & 0 & 0 & 0 & 0 & 0 & 0 & 1 & 2 & 0 & 3 & 0 & 4 & 5 & 8 & 11 & 13 & 1 & 4 & $w_{267}$ & N & can. \\
3374 & 0 & 0 & 0 & 0 & 0 & 0 & 1 & 2 & 0 & 3 & 0 & 4 & 5 & 8 & 11 & 14 & 1 & 4 & $w_{267}$ & N & can. \\
3375 & 0 & 0 & 0 & 0 & 0 & 0 & 1 & 2 & 0 & 3 & 0 & 4 & 5 & 8 & 12 & 6 & 2 & 4 & $w_{262}$ & N & can. \\
3376 & 0 & 0 & 0 & 0 & 0 & 0 & 1 & 2 & 0 & 3 & 0 & 4 & 5 & 8 & 12 & 7 & 1 & 4 & $w_{260}$ & N & can. \\
3377 & 0 & 0 & 0 & 0 & 0 & 0 & 1 & 2 & 0 & 3 & 0 & 4 & 5 & 8 & 12 & 9 & 1 & 4 & $w_{260}$ & N & can. \\
3378 & 0 & 0 & 0 & 0 & 0 & 0 & 1 & 2 & 0 & 3 & 0 & 4 & 5 & 8 & 12 & 10 & 1 & 4 & $w_{268}$ & N & can. \\
3379 & 0 & 0 & 0 & 0 & 0 & 0 & 1 & 2 & 0 & 3 & 0 & 4 & 5 & 8 & 13 & 5 & 2 & 4 & $w_{262}$ & N & can. \\
3380 & 0 & 0 & 0 & 0 & 0 & 0 & 1 & 2 & 0 & 3 & 0 & 4 & 5 & 8 & 13 & 6 & 1 & 4 & $w_{260}$ & N & can. \\
3381 & 0 & 0 & 0 & 0 & 0 & 0 & 1 & 2 & 0 & 3 & 0 & 4 & 5 & 8 & 13 & 7 & 1 & 4 & $w_{393}$ & N & \#3353 \\
3382 & 0 & 0 & 0 & 0 & 0 & 0 & 1 & 2 & 0 & 3 & 0 & 4 & 5 & 8 & 13 & 9 & 1 & 4 & $w_{264}$ & N & \#3351 \\
3383 & 0 & 0 & 0 & 0 & 0 & 0 & 1 & 2 & 0 & 3 & 0 & 4 & 5 & 8 & 13 & 10 & 1 & 4 & $w_{374}$ & N & \#3368 \\
3384 & 0 & 0 & 0 & 0 & 0 & 0 & 1 & 2 & 0 & 3 & 0 & 4 & 5 & 8 & 13 & 14 & 1 & 4 & $w_{265}$ & N & can. \\
3385 & 0 & 0 & 0 & 0 & 0 & 0 & 1 & 2 & 0 & 3 & 0 & 4 & 5 & 8 & 14 & 5 & 2 & 4 & $w_{241}$ & N & can. \\
3386 & 0 & 0 & 0 & 0 & 0 & 0 & 1 & 2 & 0 & 3 & 0 & 4 & 5 & 8 & 14 & 7 & 1 & 3 & $w_{277}$ & Y & \#3354 \\
3387 & 0 & 0 & 0 & 0 & 0 & 0 & 1 & 2 & 0 & 3 & 0 & 4 & 5 & 8 & 14 & 9 & 1 & 4 & $w_{374}$ & N & \#3364 \\
3388 & 0 & 0 & 0 & 0 & 0 & 0 & 1 & 2 & 0 & 3 & 0 & 4 & 5 & 8 & 14 & 10 & 1 & 4 & $w_{265}$ & N & \#3369 \\
3389 & 0 & 0 & 0 & 0 & 0 & 0 & 1 & 2 & 0 & 3 & 0 & 4 & 5 & 8 & 14 & 13 & 1 & 4 & $w_{265}$ & N & \#3384 \\
3390 & 0 & 0 & 0 & 0 & 0 & 0 & 1 & 2 & 0 & 3 & 0 & 4 & 5 & 8 & 15 & 5 & 2 & 4 & $w_{262}$ & N & can. \\
3391 & 0 & 0 & 0 & 0 & 0 & 0 & 1 & 2 & 0 & 3 & 0 & 4 & 5 & 8 & 15 & 6 & 2 & 3 & $w_{259}$ & N & can. \\
3392 & 0 & 0 & 0 & 0 & 0 & 0 & 1 & 2 & 0 & 3 & 0 & 4 & 5 & 8 & 15 & 7 & 1 & 4 & $w_{266}$ & N & can. \\
3393 & 0 & 0 & 0 & 0 & 0 & 0 & 1 & 2 & 0 & 3 & 0 & 4 & 5 & 8 & 15 & 9 & 1 & 4 & $w_{267}$ & N & can. \\
3394 & 0 & 0 & 0 & 0 & 0 & 0 & 1 & 2 & 0 & 3 & 0 & 4 & 5 & 8 & 15 & 10 & 1 & 4 & $w_{267}$ & N & can. \\
3395 & 0 & 0 & 0 & 0 & 0 & 0 & 1 & 2 & 0 & 3 & 0 & 4 & 7 & 8 & 8 & 13 & 1 & 4 & $w_{267}$ & N & can. \\
3396 & 0 & 0 & 0 & 0 & 0 & 0 & 1 & 2 & 0 & 3 & 0 & 4 & 7 & 8 & 8 & 15 & 2 & 4 & $w_{260}$ & N & can. \\
3397 & 0 & 0 & 0 & 0 & 0 & 0 & 1 & 2 & 0 & 3 & 0 & 4 & 7 & 8 & 9 & 13 & 1 & 4 & $w_{374}$ & N & can. \\
3398 & 0 & 0 & 0 & 0 & 0 & 0 & 1 & 2 & 0 & 3 & 0 & 4 & 7 & 8 & 9 & 14 & 1 & 4 & $w_{278}$ & N & can. \\
3399 & 0 & 0 & 0 & 0 & 0 & 0 & 1 & 2 & 0 & 3 & 0 & 4 & 7 & 8 & 9 & 15 & 1 & 4 & $w_{267}$ & N & can. \\
3400 & 0 & 0 & 0 & 0 & 0 & 0 & 1 & 2 & 0 & 3 & 0 & 4 & 7 & 8 & 11 & 7 & 8 & 4 & $w_{258}$ & N & can. \\
3401 & 0 & 0 & 0 & 0 & 0 & 0 & 1 & 2 & 0 & 3 & 0 & 4 & 7 & 8 & 11 & 13 & 1 & 4 & $w_{281}$ & N & can. \\
3402 & 0 & 0 & 0 & 0 & 0 & 0 & 1 & 2 & 0 & 3 & 0 & 4 & 7 & 8 & 12 & 7 & 8 & 3 & $w_{259}$ & N & can. \\
3403 & 0 & 0 & 0 & 0 & 0 & 0 & 1 & 2 & 0 & 3 & 0 & 4 & 7 & 8 & 12 & 9 & 1 & 4 & $w_{267}$ & N & can. \\
3404 & 0 & 0 & 0 & 0 & 0 & 0 & 1 & 2 & 0 & 3 & 0 & 4 & 7 & 8 & 13 & 7 & 2 & 4 & $w_{266}$ & N & can. \\
3405 & 0 & 0 & 0 & 0 & 0 & 0 & 1 & 2 & 0 & 3 & 0 & 4 & 7 & 8 & 13 & 9 & 1 & 4 & $w_{374}$ & N & \#3397 \\
3406 & 0 & 0 & 0 & 0 & 0 & 0 & 1 & 2 & 0 & 3 & 0 & 4 & 7 & 8 & 13 & 10 & 1 & 4 & $w_{278}$ & N & \#3398 \\
3407 & 0 & 0 & 0 & 0 & 0 & 0 & 1 & 2 & 0 & 3 & 0 & 4 & 7 & 8 & 15 & 7 & 4 & 4 & $w_{351}$ & N & can. \\
3408 & 0 & 0 & 0 & 0 & 0 & 0 & 1 & 2 & 0 & 3 & 0 & 4 & 7 & 8 & 15 & 9 & 1 & 4 & $w_{281}$ & N & can. \\
3409 & 0 & 0 & 0 & 0 & 0 & 0 & 1 & 2 & 0 & 3 & 0 & 4 & 8 & 11 & 9 & 13 & 2 & 4 & $w_{271}$ & N & can. \\
3410 & 0 & 0 & 0 & 0 & 0 & 0 & 1 & 2 & 0 & 3 & 0 & 4 & 8 & 11 & 9 & 14 & 2 & 3 & $w_{274}$ & Y & can. \\
3411 & 0 & 0 & 0 & 0 & 0 & 0 & 1 & 2 & 0 & 3 & 0 & 4 & 8 & 12 & 9 & 13 & 8 & 4 & $w_{271}$ & N & \#3409 \\
3412 & 0 & 0 & 0 & 0 & 0 & 0 & 1 & 2 & 0 & 3 & 0 & 4 & 8 & 12 & 9 & 14 & 2 & 4 & $w_{275}$ & N & can. \\
3413 & 0 & 0 & 0 & 0 & 0 & 0 & 1 & 2 & 0 & 3 & 0 & 4 & 8 & 12 & 13 & 9 & 8 & 4 & $w_{271}$ & N & \#3409 \\
3414 & 0 & 0 & 0 & 0 & 0 & 0 & 1 & 2 & 0 & 3 & 0 & 4 & 8 & 12 & 13 & 10 & 2 & 4 & $w_{275}$ & N & \#3412 \\
3415 & 0 & 0 & 0 & 0 & 0 & 0 & 1 & 2 & 0 & 3 & 0 & 4 & 8 & 12 & 13 & 11 & 1 & 4 & $w_{286}$ & N & can. \\
3416 & 0 & 0 & 0 & 0 & 0 & 0 & 1 & 2 & 0 & 3 & 0 & 4 & 8 & 13 & 9 & 15 & 2 & 4 & $w_{265}$ & N & can. \\
3417 & 0 & 0 & 0 & 0 & 0 & 0 & 1 & 2 & 0 & 3 & 0 & 4 & 8 & 13 & 13 & 10 & 3 & 4 & $w_{286}$ & N & can. \\
3418 & 0 & 0 & 0 & 0 & 0 & 0 & 1 & 2 & 0 & 3 & 0 & 4 & 8 & 13 & 13 & 11 & 4 & 4 & $w_{265}$ & N & can. \\
3419 & 0 & 0 & 0 & 0 & 0 & 0 & 1 & 2 & 0 & 3 & 0 & 4 & 8 & 15 & 9 & 14 & 4 & 4 & $w_{275}$ & N & can. \\
3420 & 0 & 0 & 0 & 0 & 0 & 0 & 1 & 2 & 0 & 3 & 0 & 4 & 8 & 15 & 13 & 10 & 4 & 4 & $w_{275}$ & N & \#3419 \\
3421 & 0 & 0 & 0 & 0 & 0 & 0 & 1 & 2 & 0 & 3 & 1 & 4 & 4 & 8 & 8 & 13 & 1 & 4 & $w_{224}$ & N & can. \\
3422 & 0 & 0 & 0 & 0 & 0 & 0 & 1 & 2 & 0 & 3 & 1 & 4 & 4 & 8 & 8 & 14 & 1 & 4 & $w_{286}$ & N & can. \\
3423 & 0 & 0 & 0 & 0 & 0 & 0 & 1 & 2 & 0 & 3 & 1 & 4 & 4 & 8 & 8 & 15 & 1 & 4 & $w_{226}$ & N & can. \\
3424 & 0 & 0 & 0 & 0 & 0 & 0 & 1 & 2 & 0 & 3 & 1 & 4 & 4 & 8 & 9 & 13 & 1 & 4 & $w_{260}$ & N & can. \\
3425 & 0 & 0 & 0 & 0 & 0 & 0 & 1 & 2 & 0 & 3 & 1 & 4 & 4 & 8 & 9 & 14 & 1 & 4 & $w_{268}$ & N & can. \\
3426 & 0 & 0 & 0 & 0 & 0 & 0 & 1 & 2 & 0 & 3 & 1 & 4 & 4 & 8 & 9 & 15 & 1 & 4 & $w_{241}$ & N & can. \\
3427 & 0 & 0 & 0 & 0 & 0 & 0 & 1 & 2 & 0 & 3 & 1 & 4 & 4 & 8 & 10 & 14 & 4 & 4 & $w_{244}$ & N & can. \\
3428 & 0 & 0 & 0 & 0 & 0 & 0 & 1 & 2 & 0 & 3 & 1 & 4 & 4 & 8 & 11 & 13 & 1 & 4 & $w_{374}$ & N & can. \\
3429 & 0 & 0 & 0 & 0 & 0 & 0 & 1 & 2 & 0 & 3 & 1 & 4 & 4 & 8 & 11 & 14 & 1 & 4 & $w_{265}$ & N & can. \\
3430 & 0 & 0 & 0 & 0 & 0 & 0 & 1 & 2 & 0 & 3 & 1 & 4 & 4 & 8 & 11 & 15 & 1 & 4 & $w_{268}$ & N & can. \\
3431 & 0 & 0 & 0 & 0 & 0 & 0 & 1 & 2 & 0 & 3 & 1 & 4 & 5 & 8 & 5 & 11 & 2 & 4 & $w_{254}$ & N & can. \\
3432 & 0 & 0 & 0 & 0 & 0 & 0 & 1 & 2 & 0 & 3 & 1 & 4 & 5 & 8 & 5 & 12 & 4 & 4 & $w_{262}$ & N & can. \\
3433 & 0 & 0 & 0 & 0 & 0 & 0 & 1 & 2 & 0 & 3 & 1 & 4 & 5 & 8 & 5 & 13 & 4 & 3 & $w_{251}$ & N & \#3328 \\
3434 & 0 & 0 & 0 & 0 & 0 & 0 & 1 & 2 & 0 & 3 & 1 & 4 & 5 & 8 & 6 & 8 & 1 & 4 & $w_{254}$ & N & can. \\
3435 & 0 & 0 & 0 & 0 & 0 & 0 & 1 & 2 & 0 & 3 & 1 & 4 & 5 & 8 & 6 & 10 & 1 & 4 & $w_{260}$ & N & can. \\
3436 & 0 & 0 & 0 & 0 & 0 & 0 & 1 & 2 & 0 & 3 & 1 & 4 & 5 & 8 & 6 & 11 & 1 & 4 & $w_{264}$ & N & can. \\
3437 & 0 & 0 & 0 & 0 & 0 & 0 & 1 & 2 & 0 & 3 & 1 & 4 & 5 & 8 & 6 & 12 & 1 & 4 & $w_{266}$ & N & can. \\
3438 & 0 & 0 & 0 & 0 & 0 & 0 & 1 & 2 & 0 & 3 & 1 & 4 & 5 & 8 & 6 & 13 & 1 & 4 & $w_{393}$ & N & can. \\
3439 & 0 & 0 & 0 & 0 & 0 & 0 & 1 & 2 & 0 & 3 & 1 & 4 & 5 & 8 & 6 & 14 & 1 & 3 & $w_{277}$ & N & can. \\
3440 & 0 & 0 & 0 & 0 & 0 & 0 & 1 & 2 & 0 & 3 & 1 & 4 & 5 & 8 & 6 & 15 & 1 & 4 & $w_{260}$ & N & can. \\
3441 & 0 & 0 & 0 & 0 & 0 & 0 & 1 & 2 & 0 & 3 & 1 & 4 & 5 & 8 & 8 & 13 & 1 & 4 & $w_{254}$ & N & \#3352 \\
3442 & 0 & 0 & 0 & 0 & 0 & 0 & 1 & 2 & 0 & 3 & 1 & 4 & 5 & 8 & 8 & 14 & 1 & 4 & $w_{265}$ & N & \#3384 \\
3443 & 0 & 0 & 0 & 0 & 0 & 0 & 1 & 2 & 0 & 3 & 1 & 4 & 5 & 8 & 8 & 15 & 1 & 4 & $w_{244}$ & N & can. \\
3444 & 0 & 0 & 0 & 0 & 0 & 0 & 1 & 2 & 0 & 3 & 1 & 4 & 5 & 8 & 9 & 12 & 1 & 4 & $w_{258}$ & N & can. \\
3445 & 0 & 0 & 0 & 0 & 0 & 0 & 1 & 2 & 0 & 3 & 1 & 4 & 5 & 8 & 9 & 13 & 1 & 4 & $w_{266}$ & N & can. \\
3446 & 0 & 0 & 0 & 0 & 0 & 0 & 1 & 2 & 0 & 3 & 1 & 4 & 5 & 8 & 9 & 14 & 1 & 4 & $w_{267}$ & N & can. \\
3447 & 0 & 0 & 0 & 0 & 0 & 0 & 1 & 2 & 0 & 3 & 1 & 4 & 5 & 8 & 9 & 15 & 1 & 4 & $w_{260}$ & N & can. \\
3448 & 0 & 0 & 0 & 0 & 0 & 0 & 1 & 2 & 0 & 3 & 1 & 4 & 5 & 8 & 11 & 5 & 2 & 4 & $w_{254}$ & N & \#3431 \\
3449 & 0 & 0 & 0 & 0 & 0 & 0 & 1 & 2 & 0 & 3 & 1 & 4 & 5 & 8 & 11 & 12 & 1 & 4 & $w_{281}$ & N & can. \\
3450 & 0 & 0 & 0 & 0 & 0 & 0 & 1 & 2 & 0 & 3 & 1 & 4 & 5 & 8 & 11 & 13 & 1 & 4 & $w_{278}$ & N & can. \\
3451 & 0 & 0 & 0 & 0 & 0 & 0 & 1 & 2 & 0 & 3 & 1 & 4 & 5 & 8 & 11 & 14 & 1 & 4 & $w_{374}$ & N & can. \\
3452 & 0 & 0 & 0 & 0 & 0 & 0 & 1 & 2 & 0 & 3 & 1 & 4 & 5 & 8 & 11 & 15 & 1 & 4 & $w_{267}$ & N & can. \\
3453 & 0 & 0 & 0 & 0 & 0 & 0 & 1 & 2 & 0 & 3 & 1 & 4 & 5 & 8 & 12 & 8 & 1 & 4 & $w_{241}$ & N & can. \\
3454 & 0 & 0 & 0 & 0 & 0 & 0 & 1 & 2 & 0 & 3 & 1 & 4 & 5 & 8 & 12 & 10 & 1 & 4 & $w_{260}$ & N & can. \\
3455 & 0 & 0 & 0 & 0 & 0 & 0 & 1 & 2 & 0 & 3 & 1 & 4 & 5 & 8 & 12 & 11 & 1 & 4 & $w_{267}$ & N & can. \\
3456 & 0 & 0 & 0 & 0 & 0 & 0 & 1 & 2 & 0 & 3 & 1 & 4 & 5 & 8 & 13 & 5 & 2 & 3 & $w_{251}$ & N & \#3328 \\
3457 & 0 & 0 & 0 & 0 & 0 & 0 & 1 & 2 & 0 & 3 & 1 & 4 & 5 & 8 & 13 & 8 & 1 & 4 & $w_{254}$ & N & \#3352 \\
3458 & 0 & 0 & 0 & 0 & 0 & 0 & 1 & 2 & 0 & 3 & 1 & 4 & 5 & 8 & 13 & 10 & 1 & 4 & $w_{267}$ & N & can. \\
3459 & 0 & 0 & 0 & 0 & 0 & 0 & 1 & 2 & 0 & 3 & 1 & 4 & 5 & 8 & 13 & 11 & 1 & 4 & $w_{278}$ & N & \#3450 \\
3460 & 0 & 0 & 0 & 0 & 0 & 0 & 1 & 2 & 0 & 3 & 1 & 4 & 5 & 8 & 14 & 5 & 2 & 4 & $w_{254}$ & N & \#3431 \\
3461 & 0 & 0 & 0 & 0 & 0 & 0 & 1 & 2 & 0 & 3 & 1 & 4 & 5 & 8 & 14 & 8 & 1 & 4 & $w_{265}$ & N & \#3384 \\
3462 & 0 & 0 & 0 & 0 & 0 & 0 & 1 & 2 & 0 & 3 & 1 & 4 & 5 & 8 & 14 & 10 & 1 & 4 & $w_{267}$ & N & can. \\
3463 & 0 & 0 & 0 & 0 & 0 & 0 & 1 & 2 & 0 & 3 & 1 & 4 & 5 & 8 & 14 & 11 & 1 & 4 & $w_{374}$ & N & \#3451 \\
3464 & 0 & 0 & 0 & 0 & 0 & 0 & 1 & 2 & 0 & 3 & 1 & 4 & 5 & 8 & 15 & 5 & 2 & 4 & $w_{241}$ & N & can. \\
3465 & 0 & 0 & 0 & 0 & 0 & 0 & 1 & 2 & 0 & 3 & 1 & 4 & 5 & 8 & 15 & 10 & 1 & 4 & $w_{260}$ & N & can. \\
3466 & 0 & 0 & 0 & 0 & 0 & 0 & 1 & 2 & 0 & 3 & 1 & 4 & 5 & 8 & 15 & 11 & 1 & 4 & $w_{281}$ & N & can. \\
3467 & 0 & 0 & 0 & 0 & 0 & 0 & 1 & 2 & 0 & 3 & 1 & 4 & 8 & 12 & 10 & 14 & 8 & 4 & $w_{265}$ & N & can. \\
3468 & 0 & 0 & 0 & 0 & 0 & 0 & 1 & 2 & 0 & 3 & 1 & 4 & 8 & 12 & 10 & 15 & 1 & 4 & $w_{374}$ & N & can. \\
3469 & 0 & 0 & 0 & 0 & 0 & 0 & 1 & 2 & 0 & 3 & 1 & 4 & 8 & 12 & 11 & 15 & 4 & 4 & $w_{278}$ & N & can. \\
3470 & 0 & 0 & 0 & 0 & 0 & 0 & 1 & 2 & 0 & 3 & 1 & 4 & 8 & 12 & 13 & 9 & 8 & 4 & $w_{264}$ & N & can. \\
3471 & 0 & 0 & 0 & 0 & 0 & 0 & 1 & 2 & 0 & 3 & 1 & 4 & 8 & 12 & 13 & 11 & 1 & 4 & $w_{374}$ & N & can. \\
3472 & 0 & 0 & 0 & 0 & 0 & 0 & 1 & 2 & 0 & 3 & 1 & 4 & 8 & 13 & 8 & 14 & 1 & 4 & $w_{275}$ & N & \#3419 \\
3473 & 0 & 0 & 0 & 0 & 0 & 0 & 1 & 2 & 0 & 3 & 1 & 4 & 8 & 13 & 11 & 14 & 8 & 4 & $w_{275}$ & N & can. \\
3474 & 0 & 0 & 0 & 0 & 0 & 0 & 1 & 2 & 0 & 3 & 1 & 4 & 8 & 14 & 11 & 13 & 4 & 4 & $w_{436}$ & N & can. \\
3475 & 0 & 0 & 0 & 0 & 0 & 0 & 1 & 2 & 0 & 3 & 4 & 7 & 8 & 12 & 9 & 13 & 16 & 4 & $w_{275}$ & N & \#3473 \\
3476 & 0 & 0 & 0 & 0 & 0 & 0 & 1 & 2 & 0 & 3 & 4 & 7 & 8 & 12 & 9 & 14 & 16 & 3 & $w_{376}$ & N & can. \\
3477 & 0 & 0 & 0 & 0 & 0 & 0 & 1 & 2 & 0 & 3 & 4 & 7 & 8 & 12 & 15 & 11 & 64 & 4 & $w_{346}$ & N & can. \\
3478 & 0 & 0 & 0 & 0 & 0 & 0 & 1 & 2 & 0 & 3 & 4 & 8 & 4 & 11 & 4 & 13 & 6 & 4 & $w_{286}$ & N & can. \\
3479 & 0 & 0 & 0 & 0 & 0 & 0 & 1 & 2 & 0 & 3 & 4 & 8 & 4 & 11 & 5 & 9 & 1 & 4 & $w_{224}$ & N & \#3239 \\
3480 & 0 & 0 & 0 & 0 & 0 & 0 & 1 & 2 & 0 & 3 & 4 & 8 & 4 & 11 & 5 & 13 & 1 & 4 & $w_{374}$ & N & can. \\
3481 & 0 & 0 & 0 & 0 & 0 & 0 & 1 & 2 & 0 & 3 & 4 & 8 & 4 & 11 & 7 & 8 & 4 & 4 & $w_{263}$ & N & can. \\
3482 & 0 & 0 & 0 & 0 & 0 & 0 & 1 & 2 & 0 & 3 & 4 & 8 & 4 & 11 & 7 & 9 & 2 & 4 & $w_{266}$ & N & can. \\
3483 & 0 & 0 & 0 & 0 & 0 & 0 & 1 & 2 & 0 & 3 & 4 & 8 & 4 & 11 & 7 & 13 & 2 & 4 & $w_{432}$ & N & can. \\
3484 & 0 & 0 & 0 & 0 & 0 & 0 & 1 & 2 & 0 & 3 & 4 & 8 & 4 & 11 & 8 & 13 & 1 & 4 & $w_{374}$ & N & can. \\
3485 & 0 & 0 & 0 & 0 & 0 & 0 & 1 & 2 & 0 & 3 & 4 & 8 & 4 & 11 & 9 & 13 & 2 & 4 & $w_{374}$ & N & can. \\
3486 & 0 & 0 & 0 & 0 & 0 & 0 & 1 & 2 & 0 & 3 & 4 & 8 & 4 & 11 & 9 & 14 & 2 & 4 & $w_{374}$ & N & \#3485 \\
3487 & 0 & 0 & 0 & 0 & 0 & 0 & 1 & 2 & 0 & 3 & 4 & 8 & 4 & 12 & 5 & 9 & 1 & 4 & $w_{254}$ & N & \#3251 \\
3488 & 0 & 0 & 0 & 0 & 0 & 0 & 1 & 2 & 0 & 3 & 4 & 8 & 4 & 12 & 5 & 10 & 1 & 4 & $w_{265}$ & N & \#3340 \\
3489 & 0 & 0 & 0 & 0 & 0 & 0 & 1 & 2 & 0 & 3 & 4 & 8 & 4 & 12 & 7 & 9 & 1 & 4 & $w_{281}$ & N & can. \\
3490 & 0 & 0 & 0 & 0 & 0 & 0 & 1 & 2 & 0 & 3 & 4 & 8 & 4 & 12 & 9 & 5 & 1 & 4 & $w_{254}$ & N & \#3251 \\
3491 & 0 & 0 & 0 & 0 & 0 & 0 & 1 & 2 & 0 & 3 & 4 & 8 & 4 & 12 & 9 & 6 & 1 & 4 & $w_{265}$ & N & \#3340 \\
3492 & 0 & 0 & 0 & 0 & 0 & 0 & 1 & 2 & 0 & 3 & 4 & 8 & 4 & 12 & 9 & 7 & 2 & 4 & $w_{267}$ & N & can. \\
3493 & 0 & 0 & 0 & 0 & 0 & 0 & 1 & 2 & 0 & 3 & 4 & 8 & 4 & 12 & 9 & 12 & 1 & 4 & $w_{241}$ & N & can. \\
3494 & 0 & 0 & 0 & 0 & 0 & 0 & 1 & 2 & 0 & 3 & 4 & 8 & 4 & 12 & 9 & 13 & 1 & 3 & $w_{277}$ & Y & can. \\
3495 & 0 & 0 & 0 & 0 & 0 & 0 & 1 & 2 & 0 & 3 & 4 & 8 & 4 & 12 & 9 & 14 & 1 & 4 & $w_{374}$ & N & can. \\
3496 & 0 & 0 & 0 & 0 & 0 & 0 & 1 & 2 & 0 & 3 & 4 & 8 & 4 & 12 & 11 & 5 & 1 & 4 & $w_{268}$ & N & can. \\
3497 & 0 & 0 & 0 & 0 & 0 & 0 & 1 & 2 & 0 & 3 & 4 & 8 & 4 & 12 & 11 & 12 & 4 & 4 & $w_{241}$ & N & can. \\
3498 & 0 & 0 & 0 & 0 & 0 & 0 & 1 & 2 & 0 & 3 & 4 & 8 & 4 & 12 & 11 & 13 & 1 & 4 & $w_{267}$ & N & can. \\
3499 & 0 & 0 & 0 & 0 & 0 & 0 & 1 & 2 & 0 & 3 & 4 & 8 & 4 & 13 & 6 & 9 & 1 & 4 & $w_{241}$ & N & can. \\
3500 & 0 & 0 & 0 & 0 & 0 & 0 & 1 & 2 & 0 & 3 & 4 & 8 & 4 & 13 & 6 & 11 & 1 & 4 & $w_{267}$ & N & can. \\
3501 & 0 & 0 & 0 & 0 & 0 & 0 & 1 & 2 & 0 & 3 & 4 & 8 & 4 & 13 & 7 & 9 & 1 & 4 & $w_{267}$ & N & can. \\
3502 & 0 & 0 & 0 & 0 & 0 & 0 & 1 & 2 & 0 & 3 & 4 & 8 & 4 & 13 & 7 & 11 & 1 & 4 & $w_{278}$ & N & can. \\
3503 & 0 & 0 & 0 & 0 & 0 & 0 & 1 & 2 & 0 & 3 & 4 & 8 & 4 & 13 & 8 & 13 & 6 & 3 & $w_{251}$ & Y & can. \\
3504 & 0 & 0 & 0 & 0 & 0 & 0 & 1 & 2 & 0 & 3 & 4 & 8 & 4 & 13 & 8 & 14 & 2 & 4 & $w_{374}$ & N & can. \\
3505 & 0 & 0 & 0 & 0 & 0 & 0 & 1 & 2 & 0 & 3 & 4 & 8 & 4 & 13 & 10 & 12 & 2 & 4 & $w_{241}$ & N & can. \\
3506 & 0 & 0 & 0 & 0 & 0 & 0 & 1 & 2 & 0 & 3 & 4 & 8 & 4 & 13 & 10 & 14 & 1 & 4 & $w_{267}$ & N & \#3498 \\
3507 & 0 & 0 & 0 & 0 & 0 & 0 & 1 & 2 & 0 & 3 & 4 & 8 & 4 & 13 & 11 & 14 & 2 & 3 & $w_{377}$ & N & can. \\
3508 & 0 & 0 & 0 & 0 & 0 & 0 & 1 & 2 & 0 & 3 & 4 & 8 & 5 & 9 & 6 & 11 & 2 & 4 & $w_{260}$ & N & can. \\
3509 & 0 & 0 & 0 & 0 & 0 & 0 & 1 & 2 & 0 & 3 & 4 & 8 & 5 & 9 & 6 & 12 & 1 & 4 & $w_{374}$ & N & \#3451 \\
3510 & 0 & 0 & 0 & 0 & 0 & 0 & 1 & 2 & 0 & 3 & 4 & 8 & 5 & 9 & 6 & 15 & 1 & 4 & $w_{374}$ & N & \#3451 \\
3511 & 0 & 0 & 0 & 0 & 0 & 0 & 1 & 2 & 0 & 3 & 4 & 8 & 5 & 10 & 6 & 9 & 4 & 4 & $w_{263}$ & N & can. \\
3512 & 0 & 0 & 0 & 0 & 0 & 0 & 1 & 2 & 0 & 3 & 4 & 8 & 5 & 10 & 6 & 11 & 2 & 4 & $w_{266}$ & N & can. \\
3513 & 0 & 0 & 0 & 0 & 0 & 0 & 1 & 2 & 0 & 3 & 4 & 8 & 5 & 10 & 6 & 12 & 1 & 4 & $w_{278}$ & N & can. \\
3514 & 0 & 0 & 0 & 0 & 0 & 0 & 1 & 2 & 0 & 3 & 4 & 8 & 5 & 10 & 6 & 13 & 1 & 4 & $w_{266}$ & N & can. \\
3515 & 0 & 0 & 0 & 0 & 0 & 0 & 1 & 2 & 0 & 3 & 4 & 8 & 5 & 10 & 6 & 14 & 1 & 4 & $w_{266}$ & N & can. \\
3516 & 0 & 0 & 0 & 0 & 0 & 0 & 1 & 2 & 0 & 3 & 4 & 8 & 5 & 10 & 6 & 15 & 1 & 4 & $w_{278}$ & N & \#3513 \\
3517 & 0 & 0 & 0 & 0 & 0 & 0 & 1 & 2 & 0 & 3 & 4 & 8 & 5 & 11 & 6 & 12 & 1 & 4 & $w_{379}$ & N & can. \\
3518 & 0 & 0 & 0 & 0 & 0 & 0 & 1 & 2 & 0 & 3 & 4 & 8 & 5 & 11 & 6 & 13 & 1 & 4 & $w_{281}$ & N & can. \\
3519 & 0 & 0 & 0 & 0 & 0 & 0 & 1 & 2 & 0 & 3 & 4 & 8 & 5 & 11 & 6 & 14 & 1 & 4 & $w_{379}$ & N & can. \\
3520 & 0 & 0 & 0 & 0 & 0 & 0 & 1 & 2 & 0 & 3 & 4 & 8 & 5 & 11 & 6 & 15 & 1 & 4 & $w_{281}$ & N & can. \\
3521 & 0 & 0 & 0 & 0 & 0 & 0 & 1 & 2 & 0 & 3 & 4 & 8 & 5 & 11 & 9 & 12 & 1 & 4 & $w_{281}$ & N & can. \\
3522 & 0 & 0 & 0 & 0 & 0 & 0 & 1 & 2 & 0 & 3 & 4 & 8 & 5 & 11 & 9 & 13 & 1 & 4 & $w_{267}$ & N & can. \\
3523 & 0 & 0 & 0 & 0 & 0 & 0 & 1 & 2 & 0 & 3 & 4 & 8 & 5 & 11 & 9 & 14 & 1 & 4 & $w_{281}$ & N & can. \\
3524 & 0 & 0 & 0 & 0 & 0 & 0 & 1 & 2 & 0 & 3 & 4 & 8 & 5 & 11 & 9 & 15 & 1 & 4 & $w_{267}$ & N & can. \\
3525 & 0 & 0 & 0 & 0 & 0 & 0 & 1 & 2 & 0 & 3 & 4 & 8 & 5 & 12 & 6 & 15 & 1 & 4 & $w_{278}$ & N & can. \\
3526 & 0 & 0 & 0 & 0 & 0 & 0 & 1 & 2 & 0 & 3 & 4 & 8 & 5 & 12 & 7 & 11 & 2 & 4 & $w_{392}$ & N & can. \\
3527 & 0 & 0 & 0 & 0 & 0 & 0 & 1 & 2 & 0 & 3 & 4 & 8 & 5 & 12 & 9 & 13 & 3 & 4 & $w_{266}$ & N & can. \\
3528 & 0 & 0 & 0 & 0 & 0 & 0 & 1 & 2 & 0 & 3 & 4 & 8 & 5 & 12 & 9 & 14 & 1 & 4 & $w_{281}$ & N & can. \\
3529 & 0 & 0 & 0 & 0 & 0 & 0 & 1 & 2 & 0 & 3 & 4 & 8 & 5 & 12 & 9 & 15 & 2 & 4 & $w_{432}$ & N & can. \\
3530 & 0 & 0 & 0 & 0 & 0 & 0 & 1 & 2 & 0 & 3 & 4 & 8 & 5 & 12 & 10 & 13 & 1 & 4 & $w_{267}$ & N & can. \\
3531 & 0 & 0 & 0 & 0 & 0 & 0 & 1 & 2 & 0 & 3 & 4 & 8 & 5 & 12 & 10 & 15 & 2 & 3 & $w_{377}$ & N & can. \\
3532 & 0 & 0 & 0 & 0 & 0 & 0 & 1 & 2 & 0 & 3 & 4 & 8 & 5 & 12 & 11 & 13 & 2 & 4 & $w_{266}$ & N & can. \\
3533 & 0 & 0 & 0 & 0 & 0 & 0 & 1 & 2 & 0 & 3 & 4 & 8 & 5 & 12 & 11 & 14 & 1 & 3 & $w_{280}$ & Y & can. \\
3534 & 0 & 0 & 0 & 0 & 0 & 0 & 1 & 2 & 0 & 3 & 4 & 8 & 5 & 12 & 11 & 15 & 1 & 4 & $w_{379}$ & N & can. \\
3535 & 0 & 0 & 0 & 0 & 0 & 0 & 1 & 2 & 0 & 3 & 4 & 8 & 5 & 12 & 13 & 9 & 3 & 4 & $w_{392}$ & N & can. \\
3536 & 0 & 0 & 0 & 0 & 0 & 0 & 1 & 2 & 0 & 3 & 4 & 8 & 5 & 12 & 13 & 11 & 2 & 4 & $w_{423}$ & N & can. \\
3537 & 0 & 0 & 0 & 0 & 0 & 0 & 1 & 2 & 0 & 3 & 4 & 8 & 5 & 12 & 15 & 9 & 2 & 4 & $w_{432}$ & N & \#3529 \\
3538 & 0 & 0 & 0 & 0 & 0 & 0 & 1 & 2 & 0 & 3 & 4 & 8 & 5 & 12 & 15 & 10 & 2 & 3 & $w_{377}$ & Y & \#3531 \\
3539 & 0 & 0 & 0 & 0 & 0 & 0 & 1 & 2 & 0 & 3 & 4 & 8 & 5 & 12 & 15 & 11 & 1 & 4 & $w_{428}$ & N & can. \\
3540 & 0 & 0 & 0 & 0 & 0 & 0 & 1 & 2 & 0 & 3 & 4 & 8 & 7 & 12 & 11 & 15 & 12 & 4 & $w_{423}$ & N & can. \\
3541 & 0 & 0 & 0 & 0 & 0 & 0 & 1 & 2 & 0 & 3 & 4 & 8 & 7 & 12 & 15 & 11 & 12 & 4 & $w_{442}$ & N & can. \\
3542 & 0 & 0 & 0 & 0 & 0 & 0 & 1 & 2 & 0 & 3 & 4 & 8 & 7 & 13 & 11 & 14 & 6 & 4 & $w_{443}$ & N & can. \\
3543 & 0 & 0 & 0 & 0 & 0 & 0 & 1 & 2 & 0 & 4 & 0 & 7 & 0 & 8 & 11 & 0 & 48 & 4 & $w_{358}$ & N & can. \\
3544 & 0 & 0 & 0 & 0 & 0 & 0 & 1 & 2 & 0 & 4 & 0 & 7 & 0 & 8 & 11 & 5 & 4 & 4 & $w_{370}$ & N & can. \\
3545 & 0 & 0 & 0 & 0 & 0 & 0 & 1 & 2 & 0 & 4 & 0 & 7 & 0 & 8 & 11 & 13 & 4 & 4 & $w_{268}$ & N & can. \\
3546 & 0 & 0 & 0 & 0 & 0 & 0 & 1 & 2 & 0 & 4 & 0 & 7 & 0 & 8 & 13 & 5 & 12 & 3 & $w_{223}$ & N & \#3194 \\
3547 & 0 & 0 & 0 & 0 & 0 & 0 & 1 & 2 & 0 & 4 & 0 & 7 & 0 & 8 & 13 & 6 & 4 & 4 & $w_{239}$ & N & \#3196 \\
3548 & 0 & 0 & 0 & 0 & 0 & 0 & 1 & 2 & 0 & 4 & 0 & 7 & 1 & 5 & 3 & 4 & 96 & 4 & $w_{384}$ & N & \#3226 \\
3549 & 0 & 0 & 0 & 0 & 0 & 0 & 1 & 2 & 0 & 4 & 0 & 7 & 1 & 5 & 3 & 7 & 128 & 3 & $w_{84}$ & N & \#3228 \\
3550 & 0 & 0 & 0 & 0 & 0 & 0 & 1 & 2 & 0 & 4 & 0 & 7 & 1 & 5 & 3 & 8 & 2 & 4 & $w_{250}$ & N & \#3231 \\
3551 & 0 & 0 & 0 & 0 & 0 & 0 & 1 & 2 & 0 & 4 & 0 & 7 & 1 & 5 & 8 & 3 & 1 & 4 & $w_{250}$ & N & \#3231 \\
3552 & 0 & 0 & 0 & 0 & 0 & 0 & 1 & 2 & 0 & 4 & 0 & 7 & 1 & 5 & 8 & 4 & 4 & 4 & $w_{333}$ & N & \#3227 \\
3553 & 0 & 0 & 0 & 0 & 0 & 0 & 1 & 2 & 0 & 4 & 0 & 7 & 1 & 5 & 8 & 7 & 2 & 4 & $w_{222}$ & N & \#3229 \\
3554 & 0 & 0 & 0 & 0 & 0 & 0 & 1 & 2 & 0 & 4 & 0 & 7 & 1 & 5 & 8 & 12 & 2 & 3 & $w_{223}$ & N & \#3235 \\
3555 & 0 & 0 & 0 & 0 & 0 & 0 & 1 & 2 & 0 & 4 & 0 & 7 & 1 & 5 & 8 & 13 & 1 & 4 & $w_{391}$ & N & \#3248 \\
3556 & 0 & 0 & 0 & 0 & 0 & 0 & 1 & 2 & 0 & 4 & 0 & 7 & 1 & 5 & 8 & 14 & 1 & 4 & $w_{254}$ & N & \#3264 \\
3557 & 0 & 0 & 0 & 0 & 0 & 0 & 1 & 2 & 0 & 4 & 0 & 7 & 1 & 5 & 8 & 15 & 1 & 4 & $w_{224}$ & N & \#3236 \\
3558 & 0 & 0 & 0 & 0 & 0 & 0 & 1 & 2 & 0 & 4 & 0 & 7 & 1 & 8 & 3 & 9 & 4 & 4 & $w_{263}$ & N & \#3305 \\
3559 & 0 & 0 & 0 & 0 & 0 & 0 & 1 & 2 & 0 & 4 & 0 & 7 & 1 & 8 & 3 & 10 & 4 & 3 & $w_{251}$ & N & \#3328 \\
3560 & 0 & 0 & 0 & 0 & 0 & 0 & 1 & 2 & 0 & 4 & 0 & 7 & 1 & 8 & 3 & 13 & 2 & 4 & $w_{393}$ & N & can. \\
3561 & 0 & 0 & 0 & 0 & 0 & 0 & 1 & 2 & 0 & 4 & 0 & 7 & 1 & 8 & 8 & 2 & 8 & 4 & $w_{335}$ & N & can. \\
3562 & 0 & 0 & 0 & 0 & 0 & 0 & 1 & 2 & 0 & 4 & 0 & 7 & 1 & 8 & 8 & 4 & 2 & 4 & $w_{338}$ & N & can. \\
3563 & 0 & 0 & 0 & 0 & 0 & 0 & 1 & 2 & 0 & 4 & 0 & 7 & 1 & 8 & 8 & 13 & 1 & 4 & $w_{260}$ & N & can. \\
3564 & 0 & 0 & 0 & 0 & 0 & 0 & 1 & 2 & 0 & 4 & 0 & 7 & 1 & 8 & 9 & 3 & 2 & 4 & $w_{263}$ & N & \#3305 \\
3565 & 0 & 0 & 0 & 0 & 0 & 0 & 1 & 2 & 0 & 4 & 0 & 7 & 1 & 8 & 9 & 4 & 1 & 4 & $w_{391}$ & N & \#3248 \\
3566 & 0 & 0 & 0 & 0 & 0 & 0 & 1 & 2 & 0 & 4 & 0 & 7 & 1 & 8 & 9 & 13 & 1 & 4 & $w_{264}$ & N & \#3308 \\
3567 & 0 & 0 & 0 & 0 & 0 & 0 & 1 & 2 & 0 & 4 & 0 & 7 & 1 & 8 & 12 & 2 & 2 & 4 & $w_{262}$ & N & can. \\
3568 & 0 & 0 & 0 & 0 & 0 & 0 & 1 & 2 & 0 & 4 & 0 & 7 & 1 & 8 & 12 & 3 & 1 & 4 & $w_{260}$ & N & can. \\
3569 & 0 & 0 & 0 & 0 & 0 & 0 & 1 & 2 & 0 & 4 & 0 & 7 & 1 & 8 & 12 & 9 & 1 & 4 & $w_{264}$ & N & \#3308 \\
3570 & 0 & 0 & 0 & 0 & 0 & 0 & 1 & 2 & 0 & 4 & 0 & 7 & 1 & 8 & 12 & 10 & 1 & 4 & $w_{265}$ & N & \#3340 \\
3571 & 0 & 0 & 0 & 0 & 0 & 0 & 1 & 2 & 0 & 4 & 0 & 7 & 1 & 8 & 13 & 2 & 2 & 4 & $w_{352}$ & N & can. \\
3572 & 0 & 0 & 0 & 0 & 0 & 0 & 1 & 2 & 0 & 4 & 0 & 7 & 1 & 8 & 13 & 3 & 1 & 4 & $w_{393}$ & N & \#3560 \\
3573 & 0 & 0 & 0 & 0 & 0 & 0 & 1 & 2 & 0 & 4 & 0 & 7 & 1 & 8 & 13 & 4 & 2 & 3 & $w_{251}$ & N & \#3249 \\
3574 & 0 & 0 & 0 & 0 & 0 & 0 & 1 & 2 & 0 & 4 & 0 & 7 & 1 & 8 & 13 & 7 & 2 & 4 & $w_{264}$ & N & \#3250 \\
3575 & 0 & 0 & 0 & 0 & 0 & 0 & 1 & 2 & 0 & 4 & 0 & 7 & 1 & 8 & 13 & 9 & 1 & 4 & $w_{264}$ & N & \#3308 \\
3576 & 0 & 0 & 0 & 0 & 0 & 0 & 1 & 2 & 0 & 4 & 0 & 7 & 1 & 8 & 13 & 10 & 1 & 4 & $w_{265}$ & N & \#3340 \\
3577 & 0 & 0 & 0 & 0 & 0 & 0 & 1 & 2 & 0 & 4 & 0 & 7 & 3 & 5 & 5 & 8 & 8 & 4 & $w_{240}$ & N & can. \\
3578 & 0 & 0 & 0 & 0 & 0 & 0 & 1 & 2 & 0 & 4 & 0 & 7 & 3 & 5 & 6 & 3 & 192 & 4 & $w_{256}$ & N & can. \\
3579 & 0 & 0 & 0 & 0 & 0 & 0 & 1 & 2 & 0 & 4 & 0 & 7 & 3 & 5 & 6 & 8 & 4 & 4 & $w_{411}$ & N & can. \\
3580 & 0 & 0 & 0 & 0 & 0 & 0 & 1 & 2 & 0 & 4 & 0 & 7 & 3 & 5 & 8 & 3 & 4 & 4 & $w_{411}$ & N & can. \\
3581 & 0 & 0 & 0 & 0 & 0 & 0 & 1 & 2 & 0 & 4 & 0 & 7 & 3 & 5 & 8 & 12 & 1 & 4 & $w_{393}$ & N & can. \\
3582 & 0 & 0 & 0 & 0 & 0 & 0 & 1 & 2 & 0 & 4 & 0 & 7 & 3 & 5 & 8 & 13 & 2 & 4 & $w_{433}$ & N & can. \\
3583 & 0 & 0 & 0 & 0 & 0 & 0 & 1 & 2 & 0 & 4 & 0 & 7 & 3 & 5 & 8 & 14 & 2 & 3 & $w_{277}$ & Y & \#3494 \\
3584 & 0 & 0 & 0 & 0 & 0 & 0 & 1 & 2 & 0 & 4 & 0 & 7 & 3 & 8 & 8 & 5 & 4 & 4 & $w_{352}$ & N & can. \\
3585 & 0 & 0 & 0 & 0 & 0 & 0 & 1 & 2 & 0 & 4 & 0 & 7 & 3 & 8 & 8 & 13 & 4 & 4 & $w_{281}$ & N & can. \\
3586 & 0 & 0 & 0 & 0 & 0 & 0 & 1 & 2 & 0 & 4 & 0 & 7 & 3 & 8 & 9 & 5 & 1 & 4 & $w_{393}$ & N & \#3581 \\
3587 & 0 & 0 & 0 & 0 & 0 & 0 & 1 & 2 & 0 & 4 & 0 & 7 & 3 & 8 & 9 & 13 & 1 & 4 & $w_{278}$ & N & \#3398 \\
3588 & 0 & 0 & 0 & 0 & 0 & 0 & 1 & 2 & 0 & 4 & 0 & 7 & 3 & 8 & 11 & 3 & 48 & 4 & $w_{444}$ & N & can. \\
3589 & 0 & 0 & 0 & 0 & 0 & 0 & 1 & 2 & 0 & 4 & 0 & 7 & 3 & 8 & 11 & 5 & 4 & 4 & $w_{347}$ & N & can. \\
3590 & 0 & 0 & 0 & 0 & 0 & 0 & 1 & 2 & 0 & 4 & 0 & 7 & 3 & 8 & 11 & 13 & 4 & 4 & $w_{428}$ & N & can. \\
3591 & 0 & 0 & 0 & 0 & 0 & 0 & 1 & 2 & 0 & 4 & 0 & 7 & 3 & 8 & 13 & 3 & 8 & 4 & $w_{347}$ & N & can. \\
3592 & 0 & 0 & 0 & 0 & 0 & 0 & 1 & 2 & 0 & 4 & 0 & 7 & 3 & 8 & 13 & 5 & 4 & 4 & $w_{433}$ & N & \#3582 \\
3593 & 0 & 0 & 0 & 0 & 0 & 0 & 1 & 2 & 0 & 4 & 0 & 7 & 3 & 8 & 13 & 6 & 4 & 3 & $w_{277}$ & N & \#3494 \\
3594 & 0 & 0 & 0 & 0 & 0 & 0 & 1 & 2 & 0 & 4 & 0 & 7 & 3 & 8 & 13 & 9 & 2 & 4 & $w_{278}$ & N & \#3398 \\
3595 & 0 & 0 & 0 & 0 & 0 & 0 & 1 & 2 & 0 & 4 & 0 & 7 & 5 & 8 & 8 & 6 & 24 & 4 & $w_{335}$ & N & can. \\
3596 & 0 & 0 & 0 & 0 & 0 & 0 & 1 & 2 & 0 & 4 & 0 & 7 & 5 & 8 & 8 & 13 & 4 & 4 & $w_{258}$ & N & can. \\
3597 & 0 & 0 & 0 & 0 & 0 & 0 & 1 & 2 & 0 & 4 & 0 & 7 & 5 & 8 & 8 & 14 & 4 & 4 & $w_{268}$ & N & can. \\
3598 & 0 & 0 & 0 & 0 & 0 & 0 & 1 & 2 & 0 & 4 & 0 & 7 & 5 & 8 & 9 & 6 & 2 & 4 & $w_{262}$ & N & can. \\
3599 & 0 & 0 & 0 & 0 & 0 & 0 & 1 & 2 & 0 & 4 & 0 & 7 & 5 & 8 & 9 & 13 & 1 & 4 & $w_{264}$ & N & can. \\
3600 & 0 & 0 & 0 & 0 & 0 & 0 & 1 & 2 & 0 & 4 & 0 & 7 & 5 & 8 & 9 & 14 & 1 & 4 & $w_{265}$ & N & can. \\
3601 & 0 & 0 & 0 & 0 & 0 & 0 & 1 & 2 & 0 & 4 & 0 & 7 & 5 & 8 & 11 & 6 & 8 & 3 & $w_{371}$ & N & can. \\
3602 & 0 & 0 & 0 & 0 & 0 & 0 & 1 & 2 & 0 & 4 & 0 & 7 & 5 & 8 & 11 & 13 & 2 & 4 & $w_{281}$ & N & can. \\
3603 & 0 & 0 & 0 & 0 & 0 & 0 & 1 & 2 & 0 & 4 & 0 & 7 & 5 & 8 & 13 & 6 & 4 & 4 & $w_{352}$ & N & can. \\
3604 & 0 & 0 & 0 & 0 & 0 & 0 & 1 & 2 & 0 & 4 & 0 & 7 & 5 & 8 & 13 & 9 & 2 & 4 & $w_{264}$ & N & \#3599 \\
3605 & 0 & 0 & 0 & 0 & 0 & 0 & 1 & 2 & 0 & 4 & 0 & 7 & 5 & 8 & 14 & 5 & 16 & 4 & $w_{370}$ & N & can. \\
3606 & 0 & 0 & 0 & 0 & 0 & 0 & 1 & 2 & 0 & 4 & 0 & 7 & 5 & 8 & 14 & 9 & 2 & 4 & $w_{265}$ & N & \#3600 \\
3607 & 0 & 0 & 0 & 0 & 0 & 0 & 1 & 2 & 0 & 4 & 0 & 7 & 8 & 12 & 9 & 13 & 16 & 3 & $w_{274}$ & N & \#3410 \\
3608 & 0 & 0 & 0 & 0 & 0 & 0 & 1 & 2 & 0 & 4 & 0 & 7 & 8 & 12 & 9 & 14 & 8 & 4 & $w_{275}$ & N & \#3412 \\
3609 & 0 & 0 & 0 & 0 & 0 & 0 & 1 & 2 & 0 & 4 & 0 & 7 & 8 & 12 & 13 & 9 & 8 & 3 & $w_{274}$ & Y & \#3410 \\
3610 & 0 & 0 & 0 & 0 & 0 & 0 & 1 & 2 & 0 & 4 & 0 & 7 & 8 & 12 & 13 & 10 & 4 & 4 & $w_{275}$ & N & \#3412 \\
3611 & 0 & 0 & 0 & 0 & 0 & 0 & 1 & 2 & 0 & 4 & 0 & 7 & 8 & 13 & 13 & 11 & 24 & 4 & $w_{278}$ & N & can. \\
3612 & 0 & 0 & 0 & 0 & 0 & 0 & 1 & 2 & 0 & 4 & 0 & 7 & 8 & 13 & 14 & 11 & 16 & 3 & $w_{377}$ & N & can. \\
3613 & 0 & 0 & 0 & 0 & 0 & 0 & 1 & 2 & 0 & 4 & 0 & 8 & 0 & 12 & 3 & 5 & 8 & 4 & $w_{262}$ & N & can. \\
3614 & 0 & 0 & 0 & 0 & 0 & 0 & 1 & 2 & 0 & 4 & 0 & 8 & 0 & 12 & 7 & 9 & 2 & 4 & $w_{241}$ & N & can. \\
3615 & 0 & 0 & 0 & 0 & 0 & 0 & 1 & 2 & 0 & 4 & 0 & 8 & 1 & 5 & 10 & 2 & 6 & 4 & $w_{329}$ & N & can. \\
3616 & 0 & 0 & 0 & 0 & 0 & 0 & 1 & 2 & 0 & 4 & 0 & 8 & 1 & 5 & 10 & 3 & 1 & 4 & $w_{225}$ & N & can. \\
3617 & 0 & 0 & 0 & 0 & 0 & 0 & 1 & 2 & 0 & 4 & 0 & 8 & 1 & 5 & 10 & 4 & 12 & 4 & $w_{329}$ & N & can. \\
3618 & 0 & 0 & 0 & 0 & 0 & 0 & 1 & 2 & 0 & 4 & 0 & 8 & 1 & 5 & 10 & 7 & 2 & 4 & $w_{370}$ & N & can. \\
3619 & 0 & 0 & 0 & 0 & 0 & 0 & 1 & 2 & 0 & 4 & 0 & 8 & 1 & 5 & 10 & 12 & 1 & 4 & $w_{241}$ & N & can. \\
3620 & 0 & 0 & 0 & 0 & 0 & 0 & 1 & 2 & 0 & 4 & 0 & 8 & 1 & 5 & 11 & 2 & 1 & 4 & $w_{225}$ & N & can. \\
3621 & 0 & 0 & 0 & 0 & 0 & 0 & 1 & 2 & 0 & 4 & 0 & 8 & 1 & 5 & 11 & 3 & 1 & 4 & $w_{224}$ & N & \#3236 \\
3622 & 0 & 0 & 0 & 0 & 0 & 0 & 1 & 2 & 0 & 4 & 0 & 8 & 1 & 5 & 11 & 4 & 6 & 4 & $w_{390}$ & N & \#3240 \\
3623 & 0 & 0 & 0 & 0 & 0 & 0 & 1 & 2 & 0 & 4 & 0 & 8 & 1 & 5 & 11 & 6 & 1 & 4 & $w_{241}$ & N & can. \\
3624 & 0 & 0 & 0 & 0 & 0 & 0 & 1 & 2 & 0 & 4 & 0 & 8 & 1 & 5 & 11 & 7 & 1 & 4 & $w_{254}$ & N & \#3269 \\
3625 & 0 & 0 & 0 & 0 & 0 & 0 & 1 & 2 & 0 & 4 & 0 & 8 & 1 & 5 & 11 & 12 & 1 & 4 & $w_{264}$ & N & \#3250 \\
3626 & 0 & 0 & 0 & 0 & 0 & 0 & 1 & 2 & 0 & 4 & 0 & 8 & 1 & 5 & 11 & 15 & 2 & 4 & $w_{239}$ & N & \#3245 \\
3627 & 0 & 0 & 0 & 0 & 0 & 0 & 1 & 2 & 0 & 4 & 0 & 8 & 1 & 5 & 15 & 2 & 2 & 4 & $w_{370}$ & N & can. \\
3628 & 0 & 0 & 0 & 0 & 0 & 0 & 1 & 2 & 0 & 4 & 0 & 8 & 1 & 5 & 15 & 3 & 1 & 4 & $w_{254}$ & N & \#3264 \\
3629 & 0 & 0 & 0 & 0 & 0 & 0 & 1 & 2 & 0 & 4 & 0 & 8 & 1 & 5 & 15 & 4 & 12 & 3 & $w_{223}$ & N & \#3241 \\
3630 & 0 & 0 & 0 & 0 & 0 & 0 & 1 & 2 & 0 & 4 & 0 & 8 & 1 & 5 & 15 & 7 & 2 & 4 & $w_{239}$ & N & \#3270 \\
3631 & 0 & 0 & 0 & 0 & 0 & 0 & 1 & 2 & 0 & 4 & 0 & 8 & 1 & 5 & 15 & 8 & 2 & 4 & $w_{239}$ & N & \#3245 \\
3632 & 0 & 0 & 0 & 0 & 0 & 0 & 1 & 2 & 0 & 4 & 0 & 8 & 1 & 6 & 3 & 7 & 2 & 4 & $w_{250}$ & N & \#3301 \\
3633 & 0 & 0 & 0 & 0 & 0 & 0 & 1 & 2 & 0 & 4 & 0 & 8 & 1 & 6 & 3 & 11 & 2 & 3 & $w_{251}$ & N & \#3307 \\
3634 & 0 & 0 & 0 & 0 & 0 & 0 & 1 & 2 & 0 & 4 & 0 & 8 & 1 & 6 & 3 & 12 & 1 & 4 & $w_{264}$ & N & \#3308 \\
3635 & 0 & 0 & 0 & 0 & 0 & 0 & 1 & 2 & 0 & 4 & 0 & 8 & 1 & 6 & 3 & 13 & 1 & 4 & $w_{352}$ & N & can. \\
3636 & 0 & 0 & 0 & 0 & 0 & 0 & 1 & 2 & 0 & 4 & 0 & 8 & 1 & 6 & 3 & 15 & 1 & 4 & $w_{264}$ & N & can. \\
3637 & 0 & 0 & 0 & 0 & 0 & 0 & 1 & 2 & 0 & 4 & 0 & 8 & 1 & 6 & 7 & 3 & 2 & 4 & $w_{250}$ & N & \#3301 \\
3638 & 0 & 0 & 0 & 0 & 0 & 0 & 1 & 2 & 0 & 4 & 0 & 8 & 1 & 6 & 7 & 11 & 1 & 4 & $w_{264}$ & N & \#3636 \\
3639 & 0 & 0 & 0 & 0 & 0 & 0 & 1 & 2 & 0 & 4 & 0 & 8 & 1 & 6 & 7 & 12 & 1 & 4 & $w_{264}$ & N & \#3308 \\
3640 & 0 & 0 & 0 & 0 & 0 & 0 & 1 & 2 & 0 & 4 & 0 & 8 & 1 & 6 & 7 & 13 & 1 & 4 & $w_{260}$ & N & can. \\
3641 & 0 & 0 & 0 & 0 & 0 & 0 & 1 & 2 & 0 & 4 & 0 & 8 & 1 & 6 & 7 & 15 & 1 & 3 & $w_{251}$ & Y & \#3307 \\
3642 & 0 & 0 & 0 & 0 & 0 & 0 & 1 & 2 & 0 & 4 & 0 & 8 & 1 & 6 & 10 & 3 & 1 & 4 & $w_{262}$ & N & can. \\
3643 & 0 & 0 & 0 & 0 & 0 & 0 & 1 & 2 & 0 & 4 & 0 & 8 & 1 & 6 & 10 & 7 & 1 & 4 & $w_{372}$ & N & can. \\
3644 & 0 & 0 & 0 & 0 & 0 & 0 & 1 & 2 & 0 & 4 & 0 & 8 & 1 & 6 & 10 & 12 & 1 & 4 & $w_{260}$ & N & can. \\
3645 & 0 & 0 & 0 & 0 & 0 & 0 & 1 & 2 & 0 & 4 & 0 & 8 & 1 & 6 & 10 & 13 & 1 & 4 & $w_{372}$ & N & can. \\
3646 & 0 & 0 & 0 & 0 & 0 & 0 & 1 & 2 & 0 & 4 & 0 & 8 & 1 & 6 & 10 & 15 & 1 & 4 & $w_{268}$ & N & can. \\
3647 & 0 & 0 & 0 & 0 & 0 & 0 & 1 & 2 & 0 & 4 & 0 & 8 & 1 & 6 & 11 & 7 & 1 & 4 & $w_{264}$ & N & \#3636 \\
3648 & 0 & 0 & 0 & 0 & 0 & 0 & 1 & 2 & 0 & 4 & 0 & 8 & 1 & 6 & 11 & 12 & 1 & 4 & $w_{265}$ & N & \#3340 \\
3649 & 0 & 0 & 0 & 0 & 0 & 0 & 1 & 2 & 0 & 4 & 0 & 8 & 1 & 6 & 11 & 13 & 1 & 4 & $w_{267}$ & N & can. \\
3650 & 0 & 0 & 0 & 0 & 0 & 0 & 1 & 2 & 0 & 4 & 0 & 8 & 1 & 6 & 11 & 15 & 2 & 4 & $w_{265}$ & N & can. \\
3651 & 0 & 0 & 0 & 0 & 0 & 0 & 1 & 2 & 0 & 4 & 0 & 8 & 1 & 6 & 14 & 3 & 1 & 4 & $w_{241}$ & N & can. \\
3652 & 0 & 0 & 0 & 0 & 0 & 0 & 1 & 2 & 0 & 4 & 0 & 8 & 1 & 6 & 14 & 7 & 1 & 4 & $w_{391}$ & N & \#3306 \\
3653 & 0 & 0 & 0 & 0 & 0 & 0 & 1 & 2 & 0 & 4 & 0 & 8 & 1 & 6 & 14 & 11 & 1 & 4 & $w_{265}$ & N & \#3650 \\
3654 & 0 & 0 & 0 & 0 & 0 & 0 & 1 & 2 & 0 & 4 & 0 & 8 & 1 & 6 & 15 & 3 & 1 & 4 & $w_{264}$ & N & \#3636 \\
3655 & 0 & 0 & 0 & 0 & 0 & 0 & 1 & 2 & 0 & 4 & 0 & 8 & 1 & 6 & 15 & 7 & 2 & 3 & $w_{251}$ & N & \#3307 \\
3656 & 0 & 0 & 0 & 0 & 0 & 0 & 1 & 2 & 0 & 4 & 0 & 8 & 1 & 6 & 15 & 11 & 2 & 4 & $w_{265}$ & N & \#3650 \\
3657 & 0 & 0 & 0 & 0 & 0 & 0 & 1 & 2 & 0 & 4 & 0 & 8 & 1 & 7 & 2 & 12 & 1 & 4 & $w_{260}$ & N & can. \\
3658 & 0 & 0 & 0 & 0 & 0 & 0 & 1 & 2 & 0 & 4 & 0 & 8 & 1 & 7 & 3 & 6 & 2 & 4 & $w_{250}$ & N & \#3301 \\
3659 & 0 & 0 & 0 & 0 & 0 & 0 & 1 & 2 & 0 & 4 & 0 & 8 & 1 & 7 & 3 & 10 & 2 & 3 & $w_{251}$ & Y & \#3307 \\
3660 & 0 & 0 & 0 & 0 & 0 & 0 & 1 & 2 & 0 & 4 & 0 & 8 & 1 & 7 & 3 & 12 & 1 & 4 & $w_{260}$ & N & can. \\
3661 & 0 & 0 & 0 & 0 & 0 & 0 & 1 & 2 & 0 & 4 & 0 & 8 & 1 & 7 & 3 & 13 & 1 & 4 & $w_{260}$ & N & can. \\
3662 & 0 & 0 & 0 & 0 & 0 & 0 & 1 & 2 & 0 & 4 & 0 & 8 & 1 & 7 & 3 & 14 & 1 & 4 & $w_{260}$ & N & can. \\
3663 & 0 & 0 & 0 & 0 & 0 & 0 & 1 & 2 & 0 & 4 & 0 & 8 & 1 & 7 & 6 & 10 & 1 & 4 & $w_{241}$ & N & can. \\
3664 & 0 & 0 & 0 & 0 & 0 & 0 & 1 & 2 & 0 & 4 & 0 & 8 & 1 & 7 & 6 & 12 & 1 & 4 & $w_{260}$ & N & \#3300 \\
3665 & 0 & 0 & 0 & 0 & 0 & 0 & 1 & 2 & 0 & 4 & 0 & 8 & 1 & 7 & 6 & 13 & 1 & 4 & $w_{260}$ & N & can. \\
3666 & 0 & 0 & 0 & 0 & 0 & 0 & 1 & 2 & 0 & 4 & 0 & 8 & 1 & 7 & 7 & 10 & 1 & 4 & $w_{241}$ & N & can. \\
3667 & 0 & 0 & 0 & 0 & 0 & 0 & 1 & 2 & 0 & 4 & 0 & 8 & 1 & 7 & 7 & 12 & 1 & 4 & $w_{260}$ & N & can. \\
3668 & 0 & 0 & 0 & 0 & 0 & 0 & 1 & 2 & 0 & 4 & 0 & 8 & 1 & 7 & 7 & 14 & 2 & 3 & $w_{259}$ & Y & can. \\
3669 & 0 & 0 & 0 & 0 & 0 & 0 & 1 & 2 & 0 & 4 & 0 & 8 & 1 & 7 & 10 & 6 & 1 & 4 & $w_{241}$ & N & can. \\
3670 & 0 & 0 & 0 & 0 & 0 & 0 & 1 & 2 & 0 & 4 & 0 & 8 & 1 & 7 & 10 & 12 & 1 & 4 & $w_{268}$ & N & can. \\
3671 & 0 & 0 & 0 & 0 & 0 & 0 & 1 & 2 & 0 & 4 & 0 & 8 & 1 & 7 & 10 & 13 & 1 & 4 & $w_{267}$ & N & can. \\
3672 & 0 & 0 & 0 & 0 & 0 & 0 & 1 & 2 & 0 & 4 & 0 & 8 & 1 & 7 & 11 & 6 & 1 & 4 & $w_{260}$ & N & \#3662 \\
3673 & 0 & 0 & 0 & 0 & 0 & 0 & 1 & 2 & 0 & 4 & 0 & 8 & 1 & 7 & 11 & 12 & 1 & 4 & $w_{281}$ & N & can. \\
3674 & 0 & 0 & 0 & 0 & 0 & 0 & 1 & 2 & 0 & 4 & 0 & 8 & 1 & 7 & 11 & 13 & 1 & 4 & $w_{267}$ & N & can. \\
3675 & 0 & 0 & 0 & 0 & 0 & 0 & 1 & 2 & 0 & 4 & 0 & 8 & 1 & 7 & 11 & 14 & 2 & 4 & $w_{281}$ & N & can. \\
3676 & 0 & 0 & 0 & 0 & 0 & 0 & 1 & 2 & 0 & 4 & 0 & 8 & 1 & 7 & 15 & 6 & 2 & 3 & $w_{251}$ & Y & \#3307 \\
3677 & 0 & 0 & 0 & 0 & 0 & 0 & 1 & 2 & 0 & 4 & 0 & 8 & 1 & 7 & 15 & 10 & 2 & 4 & $w_{265}$ & N & \#3650 \\
3678 & 0 & 0 & 0 & 0 & 0 & 0 & 1 & 2 & 0 & 4 & 0 & 8 & 1 & 12 & 2 & 13 & 4 & 4 & $w_{262}$ & N & can. \\
3679 & 0 & 0 & 0 & 0 & 0 & 0 & 1 & 2 & 0 & 4 & 0 & 8 & 1 & 12 & 2 & 14 & 4 & 4 & $w_{262}$ & N & can. \\
3680 & 0 & 0 & 0 & 0 & 0 & 0 & 1 & 2 & 0 & 4 & 0 & 8 & 1 & 12 & 2 & 15 & 2 & 4 & $w_{241}$ & N & can. \\
3681 & 0 & 0 & 0 & 0 & 0 & 0 & 1 & 2 & 0 & 4 & 0 & 8 & 1 & 12 & 3 & 13 & 4 & 4 & $w_{344}$ & N & can. \\
3682 & 0 & 0 & 0 & 0 & 0 & 0 & 1 & 2 & 0 & 4 & 0 & 8 & 1 & 12 & 3 & 15 & 2 & 4 & $w_{266}$ & N & can. \\
3683 & 0 & 0 & 0 & 0 & 0 & 0 & 1 & 2 & 0 & 4 & 0 & 8 & 1 & 12 & 6 & 10 & 1 & 4 & $w_{260}$ & N & can. \\
3684 & 0 & 0 & 0 & 0 & 0 & 0 & 1 & 2 & 0 & 4 & 0 & 8 & 1 & 12 & 6 & 11 & 1 & 4 & $w_{356}$ & N & can. \\
3685 & 0 & 0 & 0 & 0 & 0 & 0 & 1 & 2 & 0 & 4 & 0 & 8 & 1 & 12 & 6 & 13 & 2 & 4 & $w_{352}$ & N & can. \\
3686 & 0 & 0 & 0 & 0 & 0 & 0 & 1 & 2 & 0 & 4 & 0 & 8 & 1 & 12 & 7 & 10 & 1 & 4 & $w_{268}$ & N & can. \\
3687 & 0 & 0 & 0 & 0 & 0 & 0 & 1 & 2 & 0 & 4 & 0 & 8 & 1 & 12 & 7 & 11 & 1 & 4 & $w_{281}$ & N & can. \\
3688 & 0 & 0 & 0 & 0 & 0 & 0 & 1 & 2 & 0 & 4 & 0 & 8 & 1 & 12 & 7 & 13 & 2 & 4 & $w_{266}$ & N & can. \\
3689 & 0 & 0 & 0 & 0 & 0 & 0 & 1 & 2 & 0 & 4 & 0 & 8 & 1 & 13 & 3 & 12 & 4 & 4 & $w_{263}$ & N & \#3305 \\
3690 & 0 & 0 & 0 & 0 & 0 & 0 & 1 & 2 & 0 & 4 & 0 & 8 & 1 & 13 & 3 & 14 & 2 & 4 & $w_{266}$ & N & can. \\
3691 & 0 & 0 & 0 & 0 & 0 & 0 & 1 & 2 & 0 & 4 & 0 & 8 & 1 & 13 & 3 & 15 & 2 & 4 & $w_{254}$ & N & \#3431 \\
3692 & 0 & 0 & 0 & 0 & 0 & 0 & 1 & 2 & 0 & 4 & 0 & 8 & 1 & 13 & 6 & 10 & 1 & 4 & $w_{241}$ & N & can. \\
3693 & 0 & 0 & 0 & 0 & 0 & 0 & 1 & 2 & 0 & 4 & 0 & 8 & 1 & 13 & 6 & 11 & 1 & 4 & $w_{267}$ & N & can. \\
3694 & 0 & 0 & 0 & 0 & 0 & 0 & 1 & 2 & 0 & 4 & 0 & 8 & 1 & 13 & 6 & 12 & 2 & 4 & $w_{260}$ & N & can. \\
3695 & 0 & 0 & 0 & 0 & 0 & 0 & 1 & 2 & 0 & 4 & 0 & 8 & 1 & 13 & 7 & 10 & 1 & 4 & $w_{267}$ & N & can. \\
3696 & 0 & 0 & 0 & 0 & 0 & 0 & 1 & 2 & 0 & 4 & 0 & 8 & 1 & 13 & 7 & 11 & 1 & 4 & $w_{374}$ & N & \#3480 \\
3697 & 0 & 0 & 0 & 0 & 0 & 0 & 1 & 2 & 0 & 4 & 0 & 8 & 1 & 13 & 7 & 12 & 2 & 4 & $w_{393}$ & N & \#3560 \\
3698 & 0 & 0 & 0 & 0 & 0 & 0 & 1 & 2 & 0 & 4 & 0 & 8 & 1 & 14 & 3 & 15 & 4 & 4 & $w_{264}$ & N & can. \\
3699 & 0 & 0 & 0 & 0 & 0 & 0 & 1 & 2 & 0 & 4 & 0 & 8 & 1 & 14 & 6 & 11 & 1 & 4 & $w_{268}$ & N & can. \\
3700 & 0 & 0 & 0 & 0 & 0 & 0 & 1 & 2 & 0 & 4 & 0 & 8 & 1 & 14 & 7 & 11 & 2 & 4 & $w_{278}$ & N & can. \\
3701 & 0 & 0 & 0 & 0 & 0 & 0 & 1 & 2 & 0 & 4 & 0 & 8 & 1 & 14 & 15 & 3 & 4 & 4 & $w_{264}$ & N & \#3698 \\
3702 & 0 & 0 & 0 & 0 & 0 & 0 & 1 & 2 & 0 & 4 & 0 & 8 & 1 & 15 & 3 & 14 & 4 & 4 & $w_{352}$ & N & can. \\
3703 & 0 & 0 & 0 & 0 & 0 & 0 & 1 & 2 & 0 & 4 & 0 & 8 & 1 & 15 & 7 & 10 & 2 & 4 & $w_{268}$ & N & can. \\
3704 & 0 & 0 & 0 & 0 & 0 & 0 & 1 & 2 & 0 & 4 & 0 & 8 & 3 & 5 & 6 & 10 & 1 & 4 & $w_{266}$ & N & can. \\
3705 & 0 & 0 & 0 & 0 & 0 & 0 & 1 & 2 & 0 & 4 & 0 & 8 & 3 & 5 & 6 & 11 & 1 & 4 & $w_{266}$ & N & can. \\
3706 & 0 & 0 & 0 & 0 & 0 & 0 & 1 & 2 & 0 & 4 & 0 & 8 & 3 & 5 & 6 & 12 & 1 & 4 & $w_{392}$ & N & can. \\
3707 & 0 & 0 & 0 & 0 & 0 & 0 & 1 & 2 & 0 & 4 & 0 & 8 & 3 & 5 & 6 & 14 & 1 & 4 & $w_{266}$ & N & can. \\
3708 & 0 & 0 & 0 & 0 & 0 & 0 & 1 & 2 & 0 & 4 & 0 & 8 & 3 & 5 & 6 & 15 & 1 & 3 & $w_{280}$ & N & can. \\
3709 & 0 & 0 & 0 & 0 & 0 & 0 & 1 & 2 & 0 & 4 & 0 & 8 & 3 & 5 & 9 & 6 & 1 & 4 & $w_{266}$ & N & can. \\
3710 & 0 & 0 & 0 & 0 & 0 & 0 & 1 & 2 & 0 & 4 & 0 & 8 & 3 & 5 & 9 & 7 & 1 & 4 & $w_{260}$ & N & can. \\
3711 & 0 & 0 & 0 & 0 & 0 & 0 & 1 & 2 & 0 & 4 & 0 & 8 & 3 & 5 & 9 & 14 & 1 & 4 & $w_{281}$ & N & can. \\
3712 & 0 & 0 & 0 & 0 & 0 & 0 & 1 & 2 & 0 & 4 & 0 & 8 & 3 & 5 & 9 & 15 & 1 & 4 & $w_{267}$ & N & can. \\
3713 & 0 & 0 & 0 & 0 & 0 & 0 & 1 & 2 & 0 & 4 & 0 & 8 & 3 & 5 & 10 & 3 & 4 & 3 & $w_{259}$ & Y & can. \\
3714 & 0 & 0 & 0 & 0 & 0 & 0 & 1 & 2 & 0 & 4 & 0 & 8 & 3 & 5 & 10 & 6 & 1 & 4 & $w_{266}$ & N & can. \\
3715 & 0 & 0 & 0 & 0 & 0 & 0 & 1 & 2 & 0 & 4 & 0 & 8 & 3 & 5 & 10 & 7 & 1 & 4 & $w_{260}$ & N & can. \\
3716 & 0 & 0 & 0 & 0 & 0 & 0 & 1 & 2 & 0 & 4 & 0 & 8 & 3 & 5 & 10 & 12 & 2 & 4 & $w_{267}$ & N & can. \\
3717 & 0 & 0 & 0 & 0 & 0 & 0 & 1 & 2 & 0 & 4 & 0 & 8 & 3 & 5 & 10 & 14 & 1 & 4 & $w_{267}$ & N & can. \\
3718 & 0 & 0 & 0 & 0 & 0 & 0 & 1 & 2 & 0 & 4 & 0 & 8 & 3 & 5 & 10 & 15 & 1 & 4 & $w_{281}$ & N & can. \\
3719 & 0 & 0 & 0 & 0 & 0 & 0 & 1 & 2 & 0 & 4 & 0 & 8 & 3 & 5 & 12 & 3 & 1 & 4 & $w_{266}$ & N & can. \\
3720 & 0 & 0 & 0 & 0 & 0 & 0 & 1 & 2 & 0 & 4 & 0 & 8 & 3 & 5 & 12 & 6 & 1 & 4 & $w_{433}$ & N & \#3582 \\
3721 & 0 & 0 & 0 & 0 & 0 & 0 & 1 & 2 & 0 & 4 & 0 & 8 & 3 & 5 & 12 & 7 & 1 & 4 & $w_{393}$ & N & \#3353 \\
3722 & 0 & 0 & 0 & 0 & 0 & 0 & 1 & 2 & 0 & 4 & 0 & 8 & 3 & 5 & 12 & 10 & 1 & 4 & $w_{374}$ & N & \#3495 \\
3723 & 0 & 0 & 0 & 0 & 0 & 0 & 1 & 2 & 0 & 4 & 0 & 8 & 3 & 5 & 12 & 11 & 1 & 4 & $w_{278}$ & N & can. \\
3724 & 0 & 0 & 0 & 0 & 0 & 0 & 1 & 2 & 0 & 4 & 0 & 8 & 3 & 5 & 13 & 6 & 1 & 4 & $w_{393}$ & N & \#3581 \\
3725 & 0 & 0 & 0 & 0 & 0 & 0 & 1 & 2 & 0 & 4 & 0 & 8 & 3 & 5 & 13 & 7 & 1 & 4 & $w_{264}$ & N & \#3351 \\
3726 & 0 & 0 & 0 & 0 & 0 & 0 & 1 & 2 & 0 & 4 & 0 & 8 & 3 & 5 & 13 & 10 & 1 & 4 & $w_{374}$ & N & can. \\
3727 & 0 & 0 & 0 & 0 & 0 & 0 & 1 & 2 & 0 & 4 & 0 & 8 & 3 & 5 & 13 & 11 & 1 & 4 & $w_{265}$ & N & \#3384 \\
3728 & 0 & 0 & 0 & 0 & 0 & 0 & 1 & 2 & 0 & 4 & 0 & 8 & 3 & 5 & 14 & 6 & 1 & 4 & $w_{393}$ & N & \#3581 \\
3729 & 0 & 0 & 0 & 0 & 0 & 0 & 1 & 2 & 0 & 4 & 0 & 8 & 3 & 5 & 14 & 7 & 1 & 3 & $w_{277}$ & Y & \#3354 \\
3730 & 0 & 0 & 0 & 0 & 0 & 0 & 1 & 2 & 0 & 4 & 0 & 8 & 3 & 5 & 14 & 10 & 1 & 4 & $w_{374}$ & N & \#3726 \\
3731 & 0 & 0 & 0 & 0 & 0 & 0 & 1 & 2 & 0 & 4 & 0 & 8 & 3 & 5 & 14 & 11 & 1 & 4 & $w_{278}$ & N & \#3723 \\
3732 & 0 & 0 & 0 & 0 & 0 & 0 & 1 & 2 & 0 & 4 & 0 & 8 & 3 & 7 & 5 & 9 & 1 & 4 & $w_{264}$ & N & \#3599 \\
3733 & 0 & 0 & 0 & 0 & 0 & 0 & 1 & 2 & 0 & 4 & 0 & 8 & 3 & 7 & 5 & 10 & 1 & 4 & $w_{264}$ & N & can. \\
3734 & 0 & 0 & 0 & 0 & 0 & 0 & 1 & 2 & 0 & 4 & 0 & 8 & 3 & 7 & 5 & 12 & 1 & 4 & $w_{266}$ & N & can. \\
3735 & 0 & 0 & 0 & 0 & 0 & 0 & 1 & 2 & 0 & 4 & 0 & 8 & 3 & 7 & 5 & 13 & 1 & 4 & $w_{264}$ & N & \#3351 \\
3736 & 0 & 0 & 0 & 0 & 0 & 0 & 1 & 2 & 0 & 4 & 0 & 8 & 3 & 7 & 5 & 14 & 1 & 3 & $w_{277}$ & Y & \#3354 \\
3737 & 0 & 0 & 0 & 0 & 0 & 0 & 1 & 2 & 0 & 4 & 0 & 8 & 3 & 7 & 9 & 5 & 1 & 4 & $w_{264}$ & N & \#3599 \\
3738 & 0 & 0 & 0 & 0 & 0 & 0 & 1 & 2 & 0 & 4 & 0 & 8 & 3 & 7 & 9 & 6 & 1 & 4 & $w_{264}$ & N & \#3733 \\
3739 & 0 & 0 & 0 & 0 & 0 & 0 & 1 & 2 & 0 & 4 & 0 & 8 & 3 & 7 & 9 & 12 & 1 & 4 & $w_{281}$ & N & can. \\
3740 & 0 & 0 & 0 & 0 & 0 & 0 & 1 & 2 & 0 & 4 & 0 & 8 & 3 & 7 & 9 & 13 & 1 & 4 & $w_{265}$ & N & \#3384 \\
3741 & 0 & 0 & 0 & 0 & 0 & 0 & 1 & 2 & 0 & 4 & 0 & 8 & 3 & 7 & 9 & 14 & 1 & 4 & $w_{278}$ & N & \#3723 \\
3742 & 0 & 0 & 0 & 0 & 0 & 0 & 1 & 2 & 0 & 4 & 0 & 8 & 3 & 7 & 12 & 5 & 1 & 4 & $w_{393}$ & N & \#3353 \\
3743 & 0 & 0 & 0 & 0 & 0 & 0 & 1 & 2 & 0 & 4 & 0 & 8 & 3 & 7 & 12 & 9 & 1 & 4 & $w_{278}$ & N & \#3723 \\
3744 & 0 & 0 & 0 & 0 & 0 & 0 & 1 & 2 & 0 & 4 & 0 & 8 & 3 & 7 & 13 & 5 & 1 & 4 & $w_{264}$ & N & \#3351 \\
3745 & 0 & 0 & 0 & 0 & 0 & 0 & 1 & 2 & 0 & 4 & 0 & 8 & 3 & 7 & 13 & 6 & 1 & 3 & $w_{277}$ & N & \#3354 \\
3746 & 0 & 0 & 0 & 0 & 0 & 0 & 1 & 2 & 0 & 4 & 0 & 8 & 3 & 7 & 13 & 9 & 1 & 4 & $w_{265}$ & N & \#3384 \\
3747 & 0 & 0 & 0 & 0 & 0 & 0 & 1 & 2 & 0 & 4 & 0 & 8 & 3 & 7 & 13 & 10 & 1 & 4 & $w_{278}$ & N & \#3723 \\
3748 & 0 & 0 & 0 & 0 & 0 & 0 & 1 & 2 & 0 & 4 & 0 & 8 & 3 & 12 & 5 & 11 & 1 & 4 & $w_{281}$ & N & can. \\
3749 & 0 & 0 & 0 & 0 & 0 & 0 & 1 & 2 & 0 & 4 & 0 & 8 & 3 & 12 & 5 & 15 & 1 & 4 & $w_{379}$ & N & can. \\
3750 & 0 & 0 & 0 & 0 & 0 & 0 & 1 & 2 & 0 & 4 & 0 & 8 & 3 & 12 & 13 & 7 & 1 & 4 & $w_{281}$ & N & can. \\
3751 & 0 & 0 & 0 & 0 & 0 & 0 & 1 & 2 & 0 & 4 & 0 & 8 & 3 & 13 & 5 & 10 & 1 & 4 & $w_{267}$ & N & can. \\
3752 & 0 & 0 & 0 & 0 & 0 & 0 & 1 & 2 & 0 & 4 & 0 & 8 & 3 & 13 & 5 & 14 & 1 & 4 & $w_{281}$ & N & can. \\
3753 & 0 & 0 & 0 & 0 & 0 & 0 & 1 & 2 & 0 & 4 & 0 & 8 & 3 & 13 & 5 & 15 & 1 & 4 & $w_{267}$ & N & \#3394 \\
3754 & 0 & 0 & 0 & 0 & 0 & 0 & 1 & 2 & 0 & 4 & 0 & 8 & 3 & 13 & 6 & 10 & 1 & 4 & $w_{281}$ & N & can. \\
3755 & 0 & 0 & 0 & 0 & 0 & 0 & 1 & 2 & 0 & 4 & 0 & 8 & 3 & 13 & 6 & 11 & 1 & 4 & $w_{281}$ & N & can. \\
3756 & 0 & 0 & 0 & 0 & 0 & 0 & 1 & 2 & 0 & 4 & 0 & 8 & 3 & 13 & 6 & 14 & 1 & 4 & $w_{281}$ & N & can. \\
3757 & 0 & 0 & 0 & 0 & 0 & 0 & 1 & 2 & 0 & 4 & 0 & 8 & 3 & 13 & 6 & 15 & 1 & 4 & $w_{281}$ & N & can. \\
3758 & 0 & 0 & 0 & 0 & 0 & 0 & 1 & 2 & 0 & 4 & 0 & 8 & 3 & 13 & 13 & 6 & 3 & 4 & $w_{373}$ & N & can. \\
3759 & 0 & 0 & 0 & 0 & 0 & 0 & 1 & 2 & 0 & 4 & 0 & 8 & 3 & 13 & 14 & 6 & 1 & 4 & $w_{354}$ & N & can. \\
3760 & 0 & 0 & 0 & 0 & 0 & 0 & 1 & 2 & 0 & 4 & 0 & 8 & 3 & 13 & 14 & 7 & 1 & 4 & $w_{379}$ & N & can. \\
3761 & 0 & 0 & 0 & 0 & 0 & 0 & 1 & 2 & 0 & 4 & 0 & 8 & 3 & 15 & 5 & 9 & 1 & 4 & $w_{374}$ & N & \#3485 \\
3762 & 0 & 0 & 0 & 0 & 0 & 0 & 1 & 2 & 0 & 4 & 0 & 8 & 3 & 15 & 5 & 10 & 1 & 4 & $w_{278}$ & N & can. \\
3763 & 0 & 0 & 0 & 0 & 0 & 0 & 1 & 2 & 0 & 4 & 0 & 8 & 3 & 15 & 5 & 12 & 2 & 4 & $w_{379}$ & N & can. \\
3764 & 0 & 0 & 0 & 0 & 0 & 0 & 1 & 2 & 0 & 4 & 0 & 8 & 3 & 15 & 5 & 13 & 1 & 4 & $w_{374}$ & N & \#3368 \\
3765 & 0 & 0 & 0 & 0 & 0 & 0 & 1 & 2 & 0 & 4 & 0 & 8 & 3 & 15 & 5 & 14 & 1 & 4 & $w_{278}$ & N & can. \\
3766 & 0 & 0 & 0 & 0 & 0 & 0 & 1 & 2 & 0 & 4 & 0 & 8 & 3 & 15 & 13 & 5 & 1 & 4 & $w_{374}$ & N & \#3368 \\
3767 & 0 & 0 & 0 & 0 & 0 & 0 & 1 & 2 & 0 & 4 & 0 & 8 & 3 & 15 & 13 & 6 & 1 & 4 & $w_{278}$ & N & \#3765 \\
3768 & 0 & 0 & 0 & 0 & 0 & 0 & 1 & 2 & 0 & 4 & 0 & 8 & 5 & 9 & 6 & 10 & 8 & 4 & $w_{262}$ & N & can. \\
3769 & 0 & 0 & 0 & 0 & 0 & 0 & 1 & 2 & 0 & 4 & 0 & 8 & 5 & 9 & 6 & 11 & 1 & 4 & $w_{260}$ & N & can. \\
3770 & 0 & 0 & 0 & 0 & 0 & 0 & 1 & 2 & 0 & 4 & 0 & 8 & 5 & 9 & 6 & 15 & 2 & 4 & $w_{267}$ & N & can. \\
3771 & 0 & 0 & 0 & 0 & 0 & 0 & 1 & 2 & 0 & 4 & 0 & 8 & 5 & 9 & 7 & 11 & 2 & 4 & $w_{254}$ & N & \#3434 \\
3772 & 0 & 0 & 0 & 0 & 0 & 0 & 1 & 2 & 0 & 4 & 0 & 8 & 5 & 9 & 7 & 14 & 1 & 4 & $w_{267}$ & N & can. \\
3773 & 0 & 0 & 0 & 0 & 0 & 0 & 1 & 2 & 0 & 4 & 0 & 8 & 5 & 9 & 7 & 15 & 1 & 4 & $w_{265}$ & N & \#3600 \\
3774 & 0 & 0 & 0 & 0 & 0 & 0 & 1 & 2 & 0 & 4 & 0 & 8 & 5 & 9 & 10 & 6 & 4 & 4 & $w_{262}$ & N & can. \\
3775 & 0 & 0 & 0 & 0 & 0 & 0 & 1 & 2 & 0 & 4 & 0 & 8 & 5 & 9 & 10 & 7 & 1 & 4 & $w_{260}$ & N & can. \\
3776 & 0 & 0 & 0 & 0 & 0 & 0 & 1 & 2 & 0 & 4 & 0 & 8 & 5 & 9 & 10 & 14 & 1 & 4 & $w_{260}$ & N & can. \\
3777 & 0 & 0 & 0 & 0 & 0 & 0 & 1 & 2 & 0 & 4 & 0 & 8 & 5 & 9 & 10 & 15 & 1 & 4 & $w_{267}$ & N & can. \\
3778 & 0 & 0 & 0 & 0 & 0 & 0 & 1 & 2 & 0 & 4 & 0 & 8 & 5 & 9 & 11 & 7 & 2 & 4 & $w_{254}$ & N & \#3434 \\
3779 & 0 & 0 & 0 & 0 & 0 & 0 & 1 & 2 & 0 & 4 & 0 & 8 & 5 & 9 & 11 & 14 & 1 & 4 & $w_{267}$ & N & can. \\
3780 & 0 & 0 & 0 & 0 & 0 & 0 & 1 & 2 & 0 & 4 & 0 & 8 & 5 & 9 & 11 & 15 & 1 & 4 & $w_{265}$ & N & \#3600 \\
3781 & 0 & 0 & 0 & 0 & 0 & 0 & 1 & 2 & 0 & 4 & 0 & 8 & 5 & 10 & 7 & 11 & 2 & 4 & $w_{264}$ & N & can. \\
3782 & 0 & 0 & 0 & 0 & 0 & 0 & 1 & 2 & 0 & 4 & 0 & 8 & 5 & 10 & 7 & 13 & 1 & 4 & $w_{260}$ & N & can. \\
3783 & 0 & 0 & 0 & 0 & 0 & 0 & 1 & 2 & 0 & 4 & 0 & 8 & 5 & 10 & 7 & 15 & 1 & 4 & $w_{264}$ & N & \#3733 \\
3784 & 0 & 0 & 0 & 0 & 0 & 0 & 1 & 2 & 0 & 4 & 0 & 8 & 5 & 10 & 11 & 7 & 2 & 4 & $w_{264}$ & N & \#3781 \\
3785 & 0 & 0 & 0 & 0 & 0 & 0 & 1 & 2 & 0 & 4 & 0 & 8 & 5 & 10 & 11 & 13 & 1 & 4 & $w_{268}$ & N & can. \\
3786 & 0 & 0 & 0 & 0 & 0 & 0 & 1 & 2 & 0 & 4 & 0 & 8 & 5 & 10 & 11 & 15 & 1 & 4 & $w_{264}$ & N & \#3733 \\
3787 & 0 & 0 & 0 & 0 & 0 & 0 & 1 & 2 & 0 & 4 & 0 & 8 & 5 & 11 & 7 & 10 & 2 & 4 & $w_{352}$ & N & can. \\
3788 & 0 & 0 & 0 & 0 & 0 & 0 & 1 & 2 & 0 & 4 & 0 & 8 & 5 & 11 & 7 & 13 & 1 & 4 & $w_{267}$ & N & \#3393 \\
3789 & 0 & 0 & 0 & 0 & 0 & 0 & 1 & 2 & 0 & 4 & 0 & 8 & 5 & 11 & 7 & 14 & 1 & 4 & $w_{281}$ & N & \#3757 \\
3790 & 0 & 0 & 0 & 0 & 0 & 0 & 1 & 2 & 0 & 4 & 0 & 8 & 5 & 11 & 14 & 6 & 1 & 4 & $w_{373}$ & N & can. \\
3791 & 0 & 0 & 0 & 0 & 0 & 0 & 1 & 2 & 0 & 4 & 0 & 8 & 5 & 11 & 14 & 10 & 2 & 4 & $w_{352}$ & N & can. \\
3792 & 0 & 0 & 0 & 0 & 0 & 0 & 1 & 2 & 0 & 4 & 0 & 8 & 5 & 11 & 15 & 6 & 1 & 4 & $w_{267}$ & N & can. \\
3793 & 0 & 0 & 0 & 0 & 0 & 0 & 1 & 2 & 0 & 4 & 0 & 8 & 5 & 11 & 15 & 9 & 1 & 4 & $w_{268}$ & N & can. \\
3794 & 0 & 0 & 0 & 0 & 0 & 0 & 1 & 2 & 0 & 4 & 0 & 8 & 5 & 11 & 15 & 10 & 1 & 4 & $w_{260}$ & N & \#3715 \\
3795 & 0 & 0 & 0 & 0 & 0 & 0 & 1 & 2 & 0 & 4 & 0 & 8 & 5 & 13 & 7 & 11 & 1 & 4 & $w_{374}$ & N & \#3364 \\
3796 & 0 & 0 & 0 & 0 & 0 & 0 & 1 & 2 & 0 & 4 & 0 & 8 & 5 & 13 & 7 & 15 & 4 & 4 & $w_{239}$ & N & \#3362 \\
3797 & 0 & 0 & 0 & 0 & 0 & 0 & 1 & 2 & 0 & 4 & 0 & 8 & 5 & 13 & 11 & 7 & 1 & 4 & $w_{374}$ & N & \#3364 \\
3798 & 0 & 0 & 0 & 0 & 0 & 0 & 1 & 2 & 0 & 4 & 0 & 8 & 5 & 13 & 11 & 14 & 1 & 4 & $w_{278}$ & N & \#3765 \\
3799 & 0 & 0 & 0 & 0 & 0 & 0 & 1 & 2 & 0 & 4 & 0 & 8 & 5 & 13 & 11 & 15 & 1 & 4 & $w_{265}$ & N & \#3369 \\
3800 & 0 & 0 & 0 & 0 & 0 & 0 & 1 & 2 & 0 & 4 & 0 & 8 & 5 & 13 & 15 & 7 & 2 & 4 & $w_{239}$ & N & \#3362 \\
3801 & 0 & 0 & 0 & 0 & 0 & 0 & 1 & 2 & 0 & 4 & 0 & 8 & 5 & 13 & 15 & 11 & 1 & 4 & $w_{265}$ & N & \#3369 \\
3802 & 0 & 0 & 0 & 0 & 0 & 0 & 1 & 2 & 0 & 4 & 0 & 8 & 5 & 14 & 11 & 7 & 1 & 4 & $w_{278}$ & N & \#3765 \\
3803 & 0 & 0 & 0 & 0 & 0 & 0 & 1 & 2 & 0 & 4 & 0 & 8 & 5 & 14 & 11 & 15 & 2 & 3 & $w_{277}$ & N & \#3354 \\
3804 & 0 & 0 & 0 & 0 & 0 & 0 & 1 & 2 & 0 & 4 & 0 & 8 & 5 & 14 & 15 & 7 & 1 & 4 & $w_{265}$ & N & \#3369 \\
3805 & 0 & 0 & 0 & 0 & 0 & 0 & 1 & 2 & 0 & 4 & 0 & 8 & 5 & 14 & 15 & 11 & 1 & 3 & $w_{277}$ & Y & \#3354 \\
3806 & 0 & 0 & 0 & 0 & 0 & 0 & 1 & 2 & 0 & 4 & 0 & 8 & 7 & 13 & 14 & 11 & 4 & 3 & $w_{440}$ & Y & can. \\
3807 & 0 & 0 & 0 & 0 & 0 & 0 & 1 & 2 & 0 & 4 & 1 & 5 & 2 & 7 & 6 & 3 & 192 & 4 & $w_{255}$ & N & can. \\
3808 & 0 & 0 & 0 & 0 & 0 & 0 & 1 & 2 & 0 & 4 & 1 & 5 & 2 & 7 & 6 & 8 & 2 & 4 & $w_{257}$ & N & can. \\
3809 & 0 & 0 & 0 & 0 & 0 & 0 & 1 & 2 & 0 & 4 & 1 & 5 & 2 & 7 & 8 & 13 & 2 & 4 & $w_{264}$ & N & \#3436 \\
3810 & 0 & 0 & 0 & 0 & 0 & 0 & 1 & 2 & 0 & 4 & 1 & 5 & 2 & 7 & 8 & 14 & 2 & 3 & $w_{277}$ & N & \#3439 \\
3811 & 0 & 0 & 0 & 0 & 0 & 0 & 1 & 2 & 0 & 4 & 1 & 5 & 2 & 7 & 8 & 15 & 2 & 4 & $w_{393}$ & N & \#3438 \\
3812 & 0 & 0 & 0 & 0 & 0 & 0 & 1 & 2 & 0 & 4 & 1 & 5 & 2 & 8 & 9 & 3 & 12 & 4 & $w_{335}$ & N & can. \\
3813 & 0 & 0 & 0 & 0 & 0 & 0 & 1 & 2 & 0 & 4 & 1 & 5 & 2 & 8 & 9 & 6 & 4 & 4 & $w_{352}$ & N & can. \\
3814 & 0 & 0 & 0 & 0 & 0 & 0 & 1 & 2 & 0 & 4 & 1 & 5 & 2 & 8 & 9 & 12 & 2 & 4 & $w_{266}$ & N & can. \\
3815 & 0 & 0 & 0 & 0 & 0 & 0 & 1 & 2 & 0 & 4 & 1 & 5 & 2 & 8 & 9 & 15 & 2 & 4 & $w_{267}$ & N & can. \\
3816 & 0 & 0 & 0 & 0 & 0 & 0 & 1 & 2 & 0 & 4 & 1 & 5 & 2 & 8 & 10 & 3 & 4 & 3 & $w_{371}$ & N & can. \\
3817 & 0 & 0 & 0 & 0 & 0 & 0 & 1 & 2 & 0 & 4 & 1 & 5 & 2 & 8 & 10 & 6 & 4 & 4 & $w_{352}$ & N & can. \\
3818 & 0 & 0 & 0 & 0 & 0 & 0 & 1 & 2 & 0 & 4 & 1 & 5 & 2 & 8 & 10 & 12 & 1 & 4 & $w_{267}$ & N & can. \\
3819 & 0 & 0 & 0 & 0 & 0 & 0 & 1 & 2 & 0 & 4 & 1 & 5 & 2 & 8 & 11 & 3 & 2 & 4 & $w_{258}$ & N & can. \\
3820 & 0 & 0 & 0 & 0 & 0 & 0 & 1 & 2 & 0 & 4 & 1 & 5 & 2 & 8 & 11 & 6 & 2 & 4 & $w_{393}$ & N & \#3438 \\
3821 & 0 & 0 & 0 & 0 & 0 & 0 & 1 & 2 & 0 & 4 & 1 & 5 & 2 & 8 & 11 & 12 & 1 & 4 & $w_{278}$ & N & \#3450 \\
3822 & 0 & 0 & 0 & 0 & 0 & 0 & 1 & 2 & 0 & 4 & 1 & 5 & 2 & 8 & 12 & 3 & 4 & 4 & $w_{352}$ & N & can. \\
3823 & 0 & 0 & 0 & 0 & 0 & 0 & 1 & 2 & 0 & 4 & 1 & 5 & 2 & 8 & 12 & 6 & 4 & 4 & $w_{264}$ & N & \#3436 \\
3824 & 0 & 0 & 0 & 0 & 0 & 0 & 1 & 2 & 0 & 4 & 1 & 5 & 2 & 8 & 12 & 11 & 2 & 4 & $w_{278}$ & N & \#3450 \\
3825 & 0 & 0 & 0 & 0 & 0 & 0 & 1 & 2 & 0 & 4 & 1 & 5 & 2 & 8 & 15 & 3 & 4 & 4 & $w_{352}$ & N & can. \\
3826 & 0 & 0 & 0 & 0 & 0 & 0 & 1 & 2 & 0 & 4 & 1 & 5 & 2 & 8 & 15 & 6 & 4 & 3 & $w_{277}$ & N & \#3439 \\
3827 & 0 & 0 & 0 & 0 & 0 & 0 & 1 & 2 & 0 & 4 & 1 & 5 & 8 & 13 & 11 & 14 & 16 & 4 & $w_{436}$ & N & \#3474 \\
3828 & 0 & 0 & 0 & 0 & 0 & 0 & 1 & 2 & 0 & 4 & 1 & 5 & 8 & 13 & 12 & 9 & 96 & 4 & $w_{276}$ & N & can. \\
3829 & 0 & 0 & 0 & 0 & 0 & 0 & 1 & 2 & 0 & 4 & 1 & 5 & 8 & 13 & 12 & 10 & 8 & 3 & $w_{277}$ & N & can. \\
3830 & 0 & 0 & 0 & 0 & 0 & 0 & 1 & 2 & 0 & 4 & 1 & 5 & 8 & 13 & 15 & 10 & 48 & 4 & $w_{265}$ & N & can. \\
3831 & 0 & 0 & 0 & 0 & 0 & 0 & 1 & 2 & 0 & 4 & 1 & 5 & 8 & 14 & 15 & 9 & 96 & 4 & $w_{265}$ & N & can. \\
3832 & 0 & 0 & 0 & 0 & 0 & 0 & 1 & 2 & 0 & 4 & 1 & 5 & 8 & 15 & 11 & 12 & 32 & 4 & $w_{436}$ & N & \#3474 \\
3833 & 0 & 0 & 0 & 0 & 0 & 0 & 1 & 2 & 0 & 4 & 1 & 6 & 2 & 5 & 7 & 3 & 128 & 4 & $w_{401}$ & N & can. \\
3834 & 0 & 0 & 0 & 0 & 0 & 0 & 1 & 2 & 0 & 4 & 1 & 6 & 2 & 5 & 7 & 8 & 2 & 4 & $w_{405}$ & N & can. \\
3835 & 0 & 0 & 0 & 0 & 0 & 0 & 1 & 2 & 0 & 4 & 1 & 6 & 2 & 5 & 8 & 12 & 4 & 4 & $w_{263}$ & N & \#3511 \\
3836 & 0 & 0 & 0 & 0 & 0 & 0 & 1 & 2 & 0 & 4 & 1 & 6 & 2 & 5 & 8 & 13 & 2 & 4 & $w_{393}$ & N & can. \\
3837 & 0 & 0 & 0 & 0 & 0 & 0 & 1 & 2 & 0 & 4 & 1 & 6 & 2 & 5 & 8 & 15 & 8 & 3 & $w_{251}$ & N & can. \\
3838 & 0 & 0 & 0 & 0 & 0 & 0 & 1 & 2 & 0 & 4 & 1 & 6 & 2 & 7 & 5 & 3 & 64 & 4 & $w_{256}$ & N & can. \\
3839 & 0 & 0 & 0 & 0 & 0 & 0 & 1 & 2 & 0 & 4 & 1 & 6 & 2 & 7 & 5 & 8 & 1 & 4 & $w_{411}$ & N & can. \\
3840 & 0 & 0 & 0 & 0 & 0 & 0 & 1 & 2 & 0 & 4 & 1 & 6 & 2 & 7 & 8 & 12 & 1 & 4 & $w_{393}$ & N & can. \\
3841 & 0 & 0 & 0 & 0 & 0 & 0 & 1 & 2 & 0 & 4 & 1 & 6 & 2 & 7 & 8 & 13 & 2 & 3 & $w_{277}$ & Y & can. \\
3842 & 0 & 0 & 0 & 0 & 0 & 0 & 1 & 2 & 0 & 4 & 1 & 6 & 2 & 7 & 8 & 14 & 2 & 4 & $w_{433}$ & N & can. \\
3843 & 0 & 0 & 0 & 0 & 0 & 0 & 1 & 2 & 0 & 4 & 1 & 6 & 2 & 8 & 5 & 11 & 2 & 4 & $w_{433}$ & N & \#3842 \\
3844 & 0 & 0 & 0 & 0 & 0 & 0 & 1 & 2 & 0 & 4 & 1 & 6 & 2 & 8 & 5 & 12 & 2 & 4 & $w_{433}$ & N & \#3842 \\
3845 & 0 & 0 & 0 & 0 & 0 & 0 & 1 & 2 & 0 & 4 & 1 & 6 & 2 & 8 & 5 & 13 & 1 & 4 & $w_{392}$ & N & can. \\
3846 & 0 & 0 & 0 & 0 & 0 & 0 & 1 & 2 & 0 & 4 & 1 & 6 & 2 & 8 & 5 & 15 & 2 & 3 & $w_{277}$ & N & \#3841 \\
3847 & 0 & 0 & 0 & 0 & 0 & 0 & 1 & 2 & 0 & 4 & 1 & 6 & 2 & 8 & 7 & 11 & 2 & 4 & $w_{347}$ & N & can. \\
3848 & 0 & 0 & 0 & 0 & 0 & 0 & 1 & 2 & 0 & 4 & 1 & 6 & 2 & 8 & 7 & 12 & 1 & 4 & $w_{392}$ & N & can. \\
3849 & 0 & 0 & 0 & 0 & 0 & 0 & 1 & 2 & 0 & 4 & 1 & 6 & 2 & 8 & 7 & 13 & 2 & 3 & $w_{345}$ & N & can. \\
3850 & 0 & 0 & 0 & 0 & 0 & 0 & 1 & 2 & 0 & 4 & 1 & 6 & 2 & 8 & 7 & 14 & 2 & 4 & $w_{418}$ & N & can. \\
3851 & 0 & 0 & 0 & 0 & 0 & 0 & 1 & 2 & 0 & 4 & 1 & 6 & 2 & 8 & 9 & 5 & 1 & 4 & $w_{353}$ & N & can. \\
3852 & 0 & 0 & 0 & 0 & 0 & 0 & 1 & 2 & 0 & 4 & 1 & 6 & 2 & 8 & 9 & 7 & 1 & 4 & $w_{258}$ & N & can. \\
3853 & 0 & 0 & 0 & 0 & 0 & 0 & 1 & 2 & 0 & 4 & 1 & 6 & 2 & 8 & 9 & 12 & 1 & 4 & $w_{281}$ & N & can. \\
3854 & 0 & 0 & 0 & 0 & 0 & 0 & 1 & 2 & 0 & 4 & 1 & 6 & 2 & 8 & 9 & 13 & 1 & 4 & $w_{353}$ & N & can. \\
3855 & 0 & 0 & 0 & 0 & 0 & 0 & 1 & 2 & 0 & 4 & 1 & 6 & 2 & 8 & 9 & 14 & 1 & 4 & $w_{353}$ & N & can. \\
3856 & 0 & 0 & 0 & 0 & 0 & 0 & 1 & 2 & 0 & 4 & 1 & 6 & 2 & 8 & 9 & 15 & 1 & 4 & $w_{281}$ & N & \#3853 \\
3857 & 0 & 0 & 0 & 0 & 0 & 0 & 1 & 2 & 0 & 4 & 1 & 6 & 2 & 8 & 10 & 3 & 4 & 4 & $w_{407}$ & N & can. \\
3858 & 0 & 0 & 0 & 0 & 0 & 0 & 1 & 2 & 0 & 4 & 1 & 6 & 2 & 8 & 10 & 12 & 1 & 4 & $w_{392}$ & N & can. \\
3859 & 0 & 0 & 0 & 0 & 0 & 0 & 1 & 2 & 0 & 4 & 1 & 6 & 2 & 8 & 10 & 13 & 1 & 4 & $w_{353}$ & N & can. \\
3860 & 0 & 0 & 0 & 0 & 0 & 0 & 1 & 2 & 0 & 4 & 1 & 6 & 2 & 8 & 10 & 14 & 1 & 4 & $w_{409}$ & N & can. \\
3861 & 0 & 0 & 0 & 0 & 0 & 0 & 1 & 2 & 0 & 4 & 1 & 6 & 2 & 8 & 10 & 15 & 1 & 4 & $w_{281}$ & N & can. \\
3862 & 0 & 0 & 0 & 0 & 0 & 0 & 1 & 2 & 0 & 4 & 1 & 6 & 2 & 8 & 11 & 3 & 2 & 4 & $w_{351}$ & N & can. \\
3863 & 0 & 0 & 0 & 0 & 0 & 0 & 1 & 2 & 0 & 4 & 1 & 6 & 2 & 8 & 11 & 5 & 2 & 4 & $w_{433}$ & N & \#3842 \\
3864 & 0 & 0 & 0 & 0 & 0 & 0 & 1 & 2 & 0 & 4 & 1 & 6 & 2 & 8 & 11 & 7 & 2 & 4 & $w_{266}$ & N & can. \\
3865 & 0 & 0 & 0 & 0 & 0 & 0 & 1 & 2 & 0 & 4 & 1 & 6 & 2 & 8 & 11 & 12 & 1 & 4 & $w_{278}$ & N & can. \\
3866 & 0 & 0 & 0 & 0 & 0 & 0 & 1 & 2 & 0 & 4 & 1 & 6 & 2 & 8 & 11 & 13 & 1 & 4 & $w_{379}$ & N & can. \\
3867 & 0 & 0 & 0 & 0 & 0 & 0 & 1 & 2 & 0 & 4 & 1 & 6 & 2 & 8 & 11 & 14 & 1 & 4 & $w_{379}$ & N & can. \\
3868 & 0 & 0 & 0 & 0 & 0 & 0 & 1 & 2 & 0 & 4 & 1 & 6 & 2 & 8 & 11 & 15 & 1 & 4 & $w_{278}$ & N & \#3513 \\
3869 & 0 & 0 & 0 & 0 & 0 & 0 & 1 & 2 & 0 & 4 & 1 & 6 & 2 & 8 & 12 & 3 & 1 & 4 & $w_{353}$ & N & can. \\
3870 & 0 & 0 & 0 & 0 & 0 & 0 & 1 & 2 & 0 & 4 & 1 & 6 & 2 & 8 & 12 & 5 & 2 & 4 & $w_{433}$ & N & \#3842 \\
3871 & 0 & 0 & 0 & 0 & 0 & 0 & 1 & 2 & 0 & 4 & 1 & 6 & 2 & 8 & 12 & 7 & 1 & 4 & $w_{393}$ & N & \#3836 \\
3872 & 0 & 0 & 0 & 0 & 0 & 0 & 1 & 2 & 0 & 4 & 1 & 6 & 2 & 8 & 12 & 11 & 1 & 4 & $w_{278}$ & N & \#3865 \\
3873 & 0 & 0 & 0 & 0 & 0 & 0 & 1 & 2 & 0 & 4 & 1 & 6 & 2 & 8 & 13 & 3 & 1 & 4 & $w_{258}$ & N & can. \\
3874 & 0 & 0 & 0 & 0 & 0 & 0 & 1 & 2 & 0 & 4 & 1 & 6 & 2 & 8 & 13 & 5 & 1 & 4 & $w_{393}$ & N & \#3840 \\
3875 & 0 & 0 & 0 & 0 & 0 & 0 & 1 & 2 & 0 & 4 & 1 & 6 & 2 & 8 & 13 & 7 & 4 & 3 & $w_{251}$ & Y & \#3837 \\
3876 & 0 & 0 & 0 & 0 & 0 & 0 & 1 & 2 & 0 & 4 & 1 & 6 & 2 & 8 & 13 & 11 & 1 & 4 & $w_{278}$ & N & \#3865 \\
3877 & 0 & 0 & 0 & 0 & 0 & 0 & 1 & 2 & 0 & 4 & 1 & 6 & 2 & 8 & 14 & 7 & 2 & 4 & $w_{263}$ & N & \#3511 \\
3878 & 0 & 0 & 0 & 0 & 0 & 0 & 1 & 2 & 0 & 4 & 1 & 6 & 2 & 8 & 14 & 11 & 1 & 4 & $w_{278}$ & N & \#3513 \\
3879 & 0 & 0 & 0 & 0 & 0 & 0 & 1 & 2 & 0 & 4 & 1 & 6 & 2 & 8 & 15 & 5 & 2 & 3 & $w_{277}$ & N & \#3841 \\
3880 & 0 & 0 & 0 & 0 & 0 & 0 & 1 & 2 & 0 & 4 & 1 & 6 & 2 & 8 & 15 & 11 & 1 & 4 & $w_{278}$ & N & \#3513 \\
3881 & 0 & 0 & 0 & 0 & 0 & 0 & 1 & 2 & 0 & 4 & 1 & 6 & 8 & 12 & 11 & 15 & 16 & 3 & $w_{376}$ & N & \#3476 \\
3882 & 0 & 0 & 0 & 0 & 0 & 0 & 1 & 2 & 0 & 4 & 1 & 6 & 8 & 12 & 13 & 10 & 8 & 4 & $w_{344}$ & N & can. \\
3883 & 0 & 0 & 0 & 0 & 0 & 0 & 1 & 2 & 0 & 4 & 1 & 6 & 8 & 12 & 13 & 11 & 1 & 4 & $w_{278}$ & N & can. \\
3884 & 0 & 0 & 0 & 0 & 0 & 0 & 1 & 2 & 0 & 4 & 1 & 6 & 8 & 12 & 15 & 11 & 16 & 3 & $w_{376}$ & N & \#3476 \\
3885 & 0 & 0 & 0 & 0 & 0 & 0 & 1 & 2 & 0 & 4 & 1 & 6 & 8 & 13 & 11 & 14 & 16 & 3 & $w_{376}$ & Y & \#3476 \\
3886 & 0 & 0 & 0 & 0 & 0 & 0 & 1 & 2 & 0 & 4 & 1 & 6 & 8 & 13 & 12 & 10 & 8 & 4 & $w_{278}$ & N & can. \\
3887 & 0 & 0 & 0 & 0 & 0 & 0 & 1 & 2 & 0 & 4 & 1 & 6 & 8 & 13 & 14 & 11 & 8 & 3 & $w_{377}$ & N & can. \\
3888 & 0 & 0 & 0 & 0 & 0 & 0 & 1 & 2 & 0 & 4 & 1 & 7 & 2 & 8 & 5 & 10 & 1 & 4 & $w_{266}$ & N & can. \\
3889 & 0 & 0 & 0 & 0 & 0 & 0 & 1 & 2 & 0 & 4 & 1 & 7 & 2 & 8 & 5 & 11 & 1 & 4 & $w_{266}$ & N & can. \\
3890 & 0 & 0 & 0 & 0 & 0 & 0 & 1 & 2 & 0 & 4 & 1 & 7 & 2 & 8 & 5 & 12 & 1 & 4 & $w_{392}$ & N & can. \\
3891 & 0 & 0 & 0 & 0 & 0 & 0 & 1 & 2 & 0 & 4 & 1 & 7 & 2 & 8 & 5 & 13 & 1 & 4 & $w_{392}$ & N & can. \\
3892 & 0 & 0 & 0 & 0 & 0 & 0 & 1 & 2 & 0 & 4 & 1 & 7 & 2 & 8 & 5 & 14 & 1 & 3 & $w_{280}$ & N & can. \\
3893 & 0 & 0 & 0 & 0 & 0 & 0 & 1 & 2 & 0 & 4 & 1 & 7 & 2 & 8 & 5 & 15 & 1 & 4 & $w_{266}$ & N & can. \\
3894 & 0 & 0 & 0 & 0 & 0 & 0 & 1 & 2 & 0 & 4 & 1 & 7 & 2 & 8 & 6 & 10 & 1 & 4 & $w_{266}$ & N & can. \\
3895 & 0 & 0 & 0 & 0 & 0 & 0 & 1 & 2 & 0 & 4 & 1 & 7 & 2 & 8 & 6 & 11 & 1 & 4 & $w_{266}$ & N & can. \\
3896 & 0 & 0 & 0 & 0 & 0 & 0 & 1 & 2 & 0 & 4 & 1 & 7 & 2 & 8 & 6 & 12 & 1 & 4 & $w_{266}$ & N & can. \\
3897 & 0 & 0 & 0 & 0 & 0 & 0 & 1 & 2 & 0 & 4 & 1 & 7 & 2 & 8 & 6 & 13 & 1 & 3 & $w_{280}$ & Y & can. \\
3898 & 0 & 0 & 0 & 0 & 0 & 0 & 1 & 2 & 0 & 4 & 1 & 7 & 2 & 8 & 6 & 14 & 1 & 4 & $w_{392}$ & N & can. \\
3899 & 0 & 0 & 0 & 0 & 0 & 0 & 1 & 2 & 0 & 4 & 1 & 7 & 2 & 8 & 6 & 15 & 1 & 4 & $w_{392}$ & N & can. \\
3900 & 0 & 0 & 0 & 0 & 0 & 0 & 1 & 2 & 0 & 4 & 1 & 7 & 2 & 8 & 9 & 6 & 1 & 4 & $w_{266}$ & N & can. \\
3901 & 0 & 0 & 0 & 0 & 0 & 0 & 1 & 2 & 0 & 4 & 1 & 7 & 2 & 8 & 9 & 12 & 1 & 4 & $w_{379}$ & N & can. \\
3902 & 0 & 0 & 0 & 0 & 0 & 0 & 1 & 2 & 0 & 4 & 1 & 7 & 2 & 8 & 9 & 13 & 1 & 4 & $w_{281}$ & N & can. \\
3903 & 0 & 0 & 0 & 0 & 0 & 0 & 1 & 2 & 0 & 4 & 1 & 7 & 2 & 8 & 9 & 14 & 1 & 4 & $w_{379}$ & N & can. \\
3904 & 0 & 0 & 0 & 0 & 0 & 0 & 1 & 2 & 0 & 4 & 1 & 7 & 2 & 8 & 9 & 15 & 1 & 4 & $w_{281}$ & N & can. \\
3905 & 0 & 0 & 0 & 0 & 0 & 0 & 1 & 2 & 0 & 4 & 1 & 7 & 2 & 8 & 10 & 6 & 1 & 4 & $w_{266}$ & N & can. \\
3906 & 0 & 0 & 0 & 0 & 0 & 0 & 1 & 2 & 0 & 4 & 1 & 7 & 2 & 8 & 10 & 12 & 1 & 4 & $w_{281}$ & N & can. \\
3907 & 0 & 0 & 0 & 0 & 0 & 0 & 1 & 2 & 0 & 4 & 1 & 7 & 2 & 8 & 10 & 13 & 1 & 4 & $w_{379}$ & N & can. \\
3908 & 0 & 0 & 0 & 0 & 0 & 0 & 1 & 2 & 0 & 4 & 1 & 7 & 2 & 8 & 10 & 14 & 1 & 4 & $w_{281}$ & N & can. \\
3909 & 0 & 0 & 0 & 0 & 0 & 0 & 1 & 2 & 0 & 4 & 1 & 7 & 2 & 8 & 10 & 15 & 1 & 4 & $w_{379}$ & N & can. \\
3910 & 0 & 0 & 0 & 0 & 0 & 0 & 1 & 2 & 0 & 4 & 1 & 7 & 2 & 8 & 11 & 3 & 4 & 4 & $w_{413}$ & N & can. \\
3911 & 0 & 0 & 0 & 0 & 0 & 0 & 1 & 2 & 0 & 4 & 1 & 7 & 2 & 8 & 11 & 6 & 1 & 4 & $w_{392}$ & N & can. \\
3912 & 0 & 0 & 0 & 0 & 0 & 0 & 1 & 2 & 0 & 4 & 1 & 7 & 2 & 8 & 11 & 12 & 1 & 4 & $w_{428}$ & N & can. \\
3913 & 0 & 0 & 0 & 0 & 0 & 0 & 1 & 2 & 0 & 4 & 1 & 7 & 2 & 8 & 11 & 13 & 1 & 4 & $w_{379}$ & N & can. \\
3914 & 0 & 0 & 0 & 0 & 0 & 0 & 1 & 2 & 0 & 4 & 1 & 7 & 2 & 8 & 11 & 14 & 1 & 4 & $w_{428}$ & N & can. \\
3915 & 0 & 0 & 0 & 0 & 0 & 0 & 1 & 2 & 0 & 4 & 1 & 7 & 2 & 8 & 11 & 15 & 1 & 4 & $w_{379}$ & N & can. \\
3916 & 0 & 0 & 0 & 0 & 0 & 0 & 1 & 2 & 0 & 4 & 1 & 7 & 2 & 8 & 12 & 3 & 1 & 4 & $w_{392}$ & N & can. \\
3917 & 0 & 0 & 0 & 0 & 0 & 0 & 1 & 2 & 0 & 4 & 1 & 7 & 2 & 8 & 12 & 6 & 1 & 4 & $w_{393}$ & N & \#3840 \\
3918 & 0 & 0 & 0 & 0 & 0 & 0 & 1 & 2 & 0 & 4 & 1 & 7 & 2 & 8 & 12 & 10 & 1 & 4 & $w_{278}$ & N & \#3865 \\
3919 & 0 & 0 & 0 & 0 & 0 & 0 & 1 & 2 & 0 & 4 & 1 & 7 & 2 & 8 & 12 & 11 & 1 & 4 & $w_{432}$ & N & can. \\
3920 & 0 & 0 & 0 & 0 & 0 & 0 & 1 & 2 & 0 & 4 & 1 & 7 & 2 & 8 & 13 & 3 & 1 & 4 & $w_{266}$ & N & can. \\
3921 & 0 & 0 & 0 & 0 & 0 & 0 & 1 & 2 & 0 & 4 & 1 & 7 & 2 & 8 & 13 & 6 & 1 & 3 & $w_{277}$ & Y & \#3841 \\
3922 & 0 & 0 & 0 & 0 & 0 & 0 & 1 & 2 & 0 & 4 & 1 & 7 & 2 & 8 & 13 & 10 & 1 & 4 & $w_{278}$ & N & \#3865 \\
3923 & 0 & 0 & 0 & 0 & 0 & 0 & 1 & 2 & 0 & 4 & 1 & 7 & 2 & 8 & 14 & 6 & 1 & 4 & $w_{433}$ & N & \#3842 \\
3924 & 0 & 0 & 0 & 0 & 0 & 0 & 1 & 2 & 0 & 4 & 1 & 7 & 2 & 8 & 14 & 10 & 1 & 4 & $w_{278}$ & N & \#3513 \\
3925 & 0 & 0 & 0 & 0 & 0 & 0 & 1 & 2 & 0 & 4 & 1 & 7 & 2 & 8 & 14 & 11 & 1 & 4 & $w_{432}$ & N & can. \\
3926 & 0 & 0 & 0 & 0 & 0 & 0 & 1 & 2 & 0 & 4 & 1 & 7 & 2 & 8 & 15 & 6 & 1 & 4 & $w_{393}$ & N & \#3840 \\
3927 & 0 & 0 & 0 & 0 & 0 & 0 & 1 & 2 & 0 & 4 & 1 & 7 & 2 & 8 & 15 & 10 & 1 & 4 & $w_{278}$ & N & \#3513 \\
3928 & 0 & 0 & 0 & 0 & 0 & 0 & 1 & 2 & 0 & 4 & 1 & 7 & 5 & 8 & 9 & 6 & 4 & 4 & $w_{258}$ & N & can. \\
3929 & 0 & 0 & 0 & 0 & 0 & 0 & 1 & 2 & 0 & 4 & 1 & 7 & 5 & 8 & 9 & 12 & 1 & 4 & $w_{392}$ & N & can. \\
3930 & 0 & 0 & 0 & 0 & 0 & 0 & 1 & 2 & 0 & 4 & 1 & 7 & 5 & 8 & 9 & 13 & 1 & 4 & $w_{266}$ & N & can. \\
3931 & 0 & 0 & 0 & 0 & 0 & 0 & 1 & 2 & 0 & 4 & 1 & 7 & 5 & 8 & 9 & 14 & 1 & 4 & $w_{281}$ & N & can. \\
3932 & 0 & 0 & 0 & 0 & 0 & 0 & 1 & 2 & 0 & 4 & 1 & 7 & 5 & 8 & 9 & 15 & 1 & 4 & $w_{267}$ & N & can. \\
3933 & 0 & 0 & 0 & 0 & 0 & 0 & 1 & 2 & 0 & 4 & 1 & 7 & 5 & 8 & 10 & 6 & 4 & 3 & $w_{259}$ & N & can. \\
3934 & 0 & 0 & 0 & 0 & 0 & 0 & 1 & 2 & 0 & 4 & 1 & 7 & 5 & 8 & 10 & 12 & 1 & 4 & $w_{267}$ & N & can. \\
3935 & 0 & 0 & 0 & 0 & 0 & 0 & 1 & 2 & 0 & 4 & 1 & 7 & 5 & 8 & 10 & 13 & 1 & 4 & $w_{281}$ & N & can. \\
3936 & 0 & 0 & 0 & 0 & 0 & 0 & 1 & 2 & 0 & 4 & 1 & 7 & 5 & 8 & 10 & 14 & 1 & 4 & $w_{267}$ & N & \#3934 \\
3937 & 0 & 0 & 0 & 0 & 0 & 0 & 1 & 2 & 0 & 4 & 1 & 7 & 5 & 8 & 10 & 15 & 1 & 4 & $w_{281}$ & N & \#3935 \\
3938 & 0 & 0 & 0 & 0 & 0 & 0 & 1 & 2 & 0 & 4 & 1 & 7 & 5 & 8 & 11 & 6 & 2 & 4 & $w_{258}$ & N & can. \\
3939 & 0 & 0 & 0 & 0 & 0 & 0 & 1 & 2 & 0 & 4 & 1 & 7 & 5 & 8 & 11 & 12 & 1 & 4 & $w_{379}$ & N & can. \\
3940 & 0 & 0 & 0 & 0 & 0 & 0 & 1 & 2 & 0 & 4 & 1 & 7 & 5 & 8 & 11 & 13 & 1 & 4 & $w_{281}$ & N & can. \\
3941 & 0 & 0 & 0 & 0 & 0 & 0 & 1 & 2 & 0 & 4 & 1 & 7 & 5 & 8 & 11 & 14 & 1 & 4 & $w_{379}$ & N & can. \\
3942 & 0 & 0 & 0 & 0 & 0 & 0 & 1 & 2 & 0 & 4 & 1 & 7 & 5 & 8 & 11 & 15 & 1 & 4 & $w_{281}$ & N & can. \\
3943 & 0 & 0 & 0 & 0 & 0 & 0 & 1 & 2 & 0 & 4 & 1 & 7 & 5 & 8 & 12 & 6 & 1 & 4 & $w_{266}$ & N & can. \\
3944 & 0 & 0 & 0 & 0 & 0 & 0 & 1 & 2 & 0 & 4 & 1 & 7 & 5 & 8 & 12 & 10 & 1 & 4 & $w_{374}$ & N & can. \\
3945 & 0 & 0 & 0 & 0 & 0 & 0 & 1 & 2 & 0 & 4 & 1 & 7 & 5 & 8 & 12 & 11 & 1 & 4 & $w_{278}$ & N & can. \\
3946 & 0 & 0 & 0 & 0 & 0 & 0 & 1 & 2 & 0 & 4 & 1 & 7 & 5 & 8 & 13 & 6 & 1 & 4 & $w_{260}$ & N & can. \\
3947 & 0 & 0 & 0 & 0 & 0 & 0 & 1 & 2 & 0 & 4 & 1 & 7 & 5 & 8 & 13 & 10 & 1 & 4 & $w_{374}$ & N & \#3944 \\
3948 & 0 & 0 & 0 & 0 & 0 & 0 & 1 & 2 & 0 & 4 & 1 & 7 & 5 & 8 & 13 & 11 & 2 & 4 & $w_{265}$ & N & can. \\
3949 & 0 & 0 & 0 & 0 & 0 & 0 & 1 & 2 & 0 & 4 & 1 & 7 & 5 & 8 & 14 & 10 & 1 & 4 & $w_{374}$ & N & \#3944 \\
3950 & 0 & 0 & 0 & 0 & 0 & 0 & 1 & 2 & 0 & 4 & 1 & 7 & 5 & 8 & 14 & 11 & 1 & 4 & $w_{278}$ & N & \#3945 \\
3951 & 0 & 0 & 0 & 0 & 0 & 0 & 1 & 2 & 0 & 4 & 1 & 7 & 5 & 8 & 15 & 10 & 1 & 4 & $w_{374}$ & N & \#3944 \\
3952 & 0 & 0 & 0 & 0 & 0 & 0 & 1 & 2 & 0 & 4 & 1 & 7 & 5 & 8 & 15 & 11 & 1 & 4 & $w_{265}$ & N & \#3948 \\
3953 & 0 & 0 & 0 & 0 & 0 & 0 & 1 & 2 & 0 & 4 & 1 & 7 & 8 & 12 & 10 & 15 & 4 & 3 & $w_{376}$ & Y & \#3476 \\
3954 & 0 & 0 & 0 & 0 & 0 & 0 & 1 & 2 & 0 & 4 & 1 & 7 & 8 & 12 & 13 & 11 & 4 & 4 & $w_{278}$ & N & can. \\
3955 & 0 & 0 & 0 & 0 & 0 & 0 & 1 & 2 & 0 & 4 & 1 & 7 & 8 & 12 & 14 & 11 & 2 & 3 & $w_{377}$ & N & can. \\
3956 & 0 & 0 & 0 & 0 & 0 & 0 & 1 & 2 & 0 & 4 & 1 & 7 & 8 & 13 & 12 & 11 & 4 & 4 & $w_{434}$ & N & can. \\
3957 & 0 & 0 & 0 & 0 & 0 & 0 & 1 & 2 & 0 & 4 & 1 & 7 & 8 & 13 & 14 & 11 & 4 & 4 & $w_{434}$ & N & can. \\
3958 & 0 & 0 & 0 & 0 & 0 & 0 & 1 & 2 & 0 & 4 & 1 & 8 & 2 & 11 & 5 & 9 & 2 & 4 & $w_{260}$ & N & can. \\
3959 & 0 & 0 & 0 & 0 & 0 & 0 & 1 & 2 & 0 & 4 & 1 & 8 & 2 & 11 & 5 & 12 & 2 & 4 & $w_{265}$ & N & \#3948 \\
3960 & 0 & 0 & 0 & 0 & 0 & 0 & 1 & 2 & 0 & 4 & 1 & 8 & 2 & 11 & 5 & 13 & 1 & 4 & $w_{281}$ & N & can. \\
3961 & 0 & 0 & 0 & 0 & 0 & 0 & 1 & 2 & 0 & 4 & 1 & 8 & 2 & 11 & 5 & 14 & 1 & 4 & $w_{281}$ & N & \#3942 \\
3962 & 0 & 0 & 0 & 0 & 0 & 0 & 1 & 2 & 0 & 4 & 1 & 8 & 2 & 11 & 5 & 15 & 1 & 4 & $w_{278}$ & N & \#3945 \\
3963 & 0 & 0 & 0 & 0 & 0 & 0 & 1 & 2 & 0 & 4 & 1 & 8 & 2 & 11 & 7 & 3 & 8 & 4 & $w_{407}$ & N & can. \\
3964 & 0 & 0 & 0 & 0 & 0 & 0 & 1 & 2 & 0 & 4 & 1 & 8 & 2 & 11 & 7 & 9 & 4 & 4 & $w_{351}$ & N & can. \\
3965 & 0 & 0 & 0 & 0 & 0 & 0 & 1 & 2 & 0 & 4 & 1 & 8 & 2 & 11 & 7 & 12 & 1 & 4 & $w_{379}$ & N & can. \\
3966 & 0 & 0 & 0 & 0 & 0 & 0 & 1 & 2 & 0 & 4 & 1 & 8 & 2 & 11 & 7 & 13 & 2 & 4 & $w_{419}$ & N & can. \\
3967 & 0 & 0 & 0 & 0 & 0 & 0 & 1 & 2 & 0 & 4 & 1 & 8 & 2 & 11 & 9 & 5 & 2 & 4 & $w_{352}$ & N & can. \\
3968 & 0 & 0 & 0 & 0 & 0 & 0 & 1 & 2 & 0 & 4 & 1 & 8 & 2 & 11 & 9 & 7 & 4 & 4 & $w_{262}$ & N & can. \\
3969 & 0 & 0 & 0 & 0 & 0 & 0 & 1 & 2 & 0 & 4 & 1 & 8 & 2 & 11 & 9 & 12 & 2 & 4 & $w_{266}$ & N & can. \\
3970 & 0 & 0 & 0 & 0 & 0 & 0 & 1 & 2 & 0 & 4 & 1 & 8 & 2 & 11 & 9 & 13 & 4 & 4 & $w_{344}$ & N & can. \\
3971 & 0 & 0 & 0 & 0 & 0 & 0 & 1 & 2 & 0 & 4 & 1 & 8 & 2 & 11 & 9 & 14 & 4 & 3 & $w_{345}$ & N & can. \\
3972 & 0 & 0 & 0 & 0 & 0 & 0 & 1 & 2 & 0 & 4 & 1 & 8 & 2 & 11 & 10 & 5 & 2 & 4 & $w_{352}$ & N & can. \\
3973 & 0 & 0 & 0 & 0 & 0 & 0 & 1 & 2 & 0 & 4 & 1 & 8 & 2 & 11 & 10 & 7 & 8 & 4 & $w_{262}$ & N & can. \\
3974 & 0 & 0 & 0 & 0 & 0 & 0 & 1 & 2 & 0 & 4 & 1 & 8 & 2 & 11 & 10 & 12 & 2 & 4 & $w_{266}$ & N & can. \\
3975 & 0 & 0 & 0 & 0 & 0 & 0 & 1 & 2 & 0 & 4 & 1 & 8 & 2 & 11 & 10 & 13 & 4 & 3 & $w_{345}$ & N & can. \\
3976 & 0 & 0 & 0 & 0 & 0 & 0 & 1 & 2 & 0 & 4 & 1 & 8 & 2 & 11 & 10 & 14 & 4 & 4 & $w_{344}$ & N & can. \\
3977 & 0 & 0 & 0 & 0 & 0 & 0 & 1 & 2 & 0 & 4 & 1 & 8 & 2 & 11 & 12 & 3 & 2 & 4 & $w_{353}$ & N & can. \\
3978 & 0 & 0 & 0 & 0 & 0 & 0 & 1 & 2 & 0 & 4 & 1 & 8 & 2 & 11 & 12 & 5 & 2 & 4 & $w_{265}$ & N & \#3948 \\
3979 & 0 & 0 & 0 & 0 & 0 & 0 & 1 & 2 & 0 & 4 & 1 & 8 & 2 & 11 & 12 & 6 & 1 & 4 & $w_{278}$ & N & \#3945 \\
3980 & 0 & 0 & 0 & 0 & 0 & 0 & 1 & 2 & 0 & 4 & 1 & 8 & 2 & 11 & 12 & 9 & 2 & 4 & $w_{393}$ & N & \#3836 \\
3981 & 0 & 0 & 0 & 0 & 0 & 0 & 1 & 2 & 0 & 4 & 1 & 8 & 2 & 11 & 13 & 3 & 4 & 4 & $w_{351}$ & N & can. \\
3982 & 0 & 0 & 0 & 0 & 0 & 0 & 1 & 2 & 0 & 4 & 1 & 8 & 2 & 11 & 13 & 5 & 1 & 4 & $w_{278}$ & N & \#3945 \\
3983 & 0 & 0 & 0 & 0 & 0 & 0 & 1 & 2 & 0 & 4 & 1 & 8 & 2 & 11 & 13 & 9 & 4 & 4 & $w_{263}$ & N & \#3511 \\
3984 & 0 & 0 & 0 & 0 & 0 & 0 & 1 & 2 & 0 & 4 & 1 & 8 & 2 & 11 & 14 & 9 & 8 & 3 & $w_{251}$ & Y & \#3837 \\
3985 & 0 & 0 & 0 & 0 & 0 & 0 & 1 & 2 & 0 & 4 & 1 & 8 & 2 & 12 & 5 & 9 & 1 & 4 & $w_{392}$ & N & can. \\
3986 & 0 & 0 & 0 & 0 & 0 & 0 & 1 & 2 & 0 & 4 & 1 & 8 & 2 & 12 & 5 & 11 & 1 & 4 & $w_{278}$ & N & can. \\
3987 & 0 & 0 & 0 & 0 & 0 & 0 & 1 & 2 & 0 & 4 & 1 & 8 & 2 & 12 & 5 & 14 & 1 & 4 & $w_{281}$ & N & can. \\
3988 & 0 & 0 & 0 & 0 & 0 & 0 & 1 & 2 & 0 & 4 & 1 & 8 & 2 & 12 & 5 & 15 & 1 & 4 & $w_{432}$ & N & can. \\
3989 & 0 & 0 & 0 & 0 & 0 & 0 & 1 & 2 & 0 & 4 & 1 & 8 & 2 & 12 & 7 & 9 & 1 & 4 & $w_{392}$ & N & can. \\
3990 & 0 & 0 & 0 & 0 & 0 & 0 & 1 & 2 & 0 & 4 & 1 & 8 & 2 & 12 & 7 & 11 & 1 & 4 & $w_{428}$ & N & can. \\
3991 & 0 & 0 & 0 & 0 & 0 & 0 & 1 & 2 & 0 & 4 & 1 & 8 & 2 & 12 & 9 & 5 & 1 & 4 & $w_{266}$ & N & can. \\
3992 & 0 & 0 & 0 & 0 & 0 & 0 & 1 & 2 & 0 & 4 & 1 & 8 & 2 & 12 & 9 & 6 & 1 & 4 & $w_{267}$ & N & can. \\
3993 & 0 & 0 & 0 & 0 & 0 & 0 & 1 & 2 & 0 & 4 & 1 & 8 & 2 & 12 & 9 & 14 & 1 & 4 & $w_{266}$ & N & can. \\
3994 & 0 & 0 & 0 & 0 & 0 & 0 & 1 & 2 & 0 & 4 & 1 & 8 & 2 & 12 & 9 & 15 & 1 & 4 & $w_{281}$ & N & can. \\
3995 & 0 & 0 & 0 & 0 & 0 & 0 & 1 & 2 & 0 & 4 & 1 & 8 & 2 & 12 & 10 & 5 & 1 & 4 & $w_{281}$ & N & can. \\
3996 & 0 & 0 & 0 & 0 & 0 & 0 & 1 & 2 & 0 & 4 & 1 & 8 & 2 & 12 & 10 & 6 & 1 & 4 & $w_{392}$ & N & can. \\
3997 & 0 & 0 & 0 & 0 & 0 & 0 & 1 & 2 & 0 & 4 & 1 & 8 & 2 & 12 & 10 & 7 & 1 & 4 & $w_{392}$ & N & can. \\
3998 & 0 & 0 & 0 & 0 & 0 & 0 & 1 & 2 & 0 & 4 & 1 & 8 & 2 & 12 & 10 & 14 & 1 & 4 & $w_{351}$ & N & can. \\
3999 & 0 & 0 & 0 & 0 & 0 & 0 & 1 & 2 & 0 & 4 & 1 & 8 & 2 & 12 & 10 & 15 & 1 & 4 & $w_{379}$ & N & can. \\
4000 & 0 & 0 & 0 & 0 & 0 & 0 & 1 & 2 & 0 & 4 & 1 & 8 & 2 & 12 & 11 & 5 & 1 & 4 & $w_{278}$ & N & \#3986 \\
4001 & 0 & 0 & 0 & 0 & 0 & 0 & 1 & 2 & 0 & 4 & 1 & 8 & 2 & 12 & 11 & 6 & 1 & 4 & $w_{432}$ & N & \#3988 \\
4002 & 0 & 0 & 0 & 0 & 0 & 0 & 1 & 2 & 0 & 4 & 1 & 8 & 2 & 12 & 11 & 7 & 1 & 4 & $w_{379}$ & N & can. \\
4003 & 0 & 0 & 0 & 0 & 0 & 0 & 1 & 2 & 0 & 4 & 1 & 8 & 2 & 12 & 11 & 14 & 1 & 4 & $w_{379}$ & N & can. \\
4004 & 0 & 0 & 0 & 0 & 0 & 0 & 1 & 2 & 0 & 4 & 1 & 8 & 2 & 12 & 11 & 15 & 1 & 4 & $w_{432}$ & N & \#3529 \\
4005 & 0 & 0 & 0 & 0 & 0 & 0 & 1 & 2 & 0 & 4 & 1 & 8 & 2 & 13 & 5 & 9 & 1 & 4 & $w_{266}$ & N & can. \\
4006 & 0 & 0 & 0 & 0 & 0 & 0 & 1 & 2 & 0 & 4 & 1 & 8 & 2 & 13 & 5 & 11 & 1 & 4 & $w_{281}$ & N & can. \\
4007 & 0 & 0 & 0 & 0 & 0 & 0 & 1 & 2 & 0 & 4 & 1 & 8 & 2 & 13 & 5 & 14 & 1 & 4 & $w_{281}$ & N & \#3521 \\
4008 & 0 & 0 & 0 & 0 & 0 & 0 & 1 & 2 & 0 & 4 & 1 & 8 & 2 & 13 & 5 & 15 & 1 & 4 & $w_{267}$ & N & can. \\
4009 & 0 & 0 & 0 & 0 & 0 & 0 & 1 & 2 & 0 & 4 & 1 & 8 & 2 & 13 & 6 & 9 & 1 & 4 & $w_{267}$ & N & can. \\
4010 & 0 & 0 & 0 & 0 & 0 & 0 & 1 & 2 & 0 & 4 & 1 & 8 & 2 & 13 & 6 & 11 & 1 & 4 & $w_{379}$ & N & can. \\
4011 & 0 & 0 & 0 & 0 & 0 & 0 & 1 & 2 & 0 & 4 & 1 & 8 & 2 & 13 & 6 & 14 & 1 & 4 & $w_{379}$ & N & can. \\
4012 & 0 & 0 & 0 & 0 & 0 & 0 & 1 & 2 & 0 & 4 & 1 & 8 & 2 & 13 & 6 & 15 & 1 & 4 & $w_{267}$ & N & can. \\
4013 & 0 & 0 & 0 & 0 & 0 & 0 & 1 & 2 & 0 & 4 & 1 & 8 & 2 & 13 & 7 & 9 & 1 & 4 & $w_{379}$ & N & \#4003 \\
4014 & 0 & 0 & 0 & 0 & 0 & 0 & 1 & 2 & 0 & 4 & 1 & 8 & 2 & 13 & 7 & 11 & 1 & 4 & $w_{427}$ & N & can. \\
4015 & 0 & 0 & 0 & 0 & 0 & 0 & 1 & 2 & 0 & 4 & 1 & 8 & 2 & 13 & 7 & 14 & 1 & 4 & $w_{355}$ & N & can. \\
4016 & 0 & 0 & 0 & 0 & 0 & 0 & 1 & 2 & 0 & 4 & 1 & 8 & 2 & 13 & 7 & 15 & 1 & 4 & $w_{379}$ & N & can. \\
4017 & 0 & 0 & 0 & 0 & 0 & 0 & 1 & 2 & 0 & 4 & 1 & 8 & 2 & 13 & 9 & 5 & 1 & 4 & $w_{347}$ & N & can. \\
4018 & 0 & 0 & 0 & 0 & 0 & 0 & 1 & 2 & 0 & 4 & 1 & 8 & 2 & 13 & 9 & 6 & 1 & 4 & $w_{373}$ & N & can. \\
4019 & 0 & 0 & 0 & 0 & 0 & 0 & 1 & 2 & 0 & 4 & 1 & 8 & 2 & 13 & 9 & 7 & 1 & 4 & $w_{281}$ & N & \#3994 \\
4020 & 0 & 0 & 0 & 0 & 0 & 0 & 1 & 2 & 0 & 4 & 1 & 8 & 2 & 13 & 9 & 14 & 1 & 4 & $w_{355}$ & N & can. \\
4021 & 0 & 0 & 0 & 0 & 0 & 0 & 1 & 2 & 0 & 4 & 1 & 8 & 2 & 13 & 9 & 15 & 1 & 4 & $w_{281}$ & N & can. \\
4022 & 0 & 0 & 0 & 0 & 0 & 0 & 1 & 2 & 0 & 4 & 1 & 8 & 2 & 13 & 10 & 5 & 1 & 4 & $w_{373}$ & N & can. \\
4023 & 0 & 0 & 0 & 0 & 0 & 0 & 1 & 2 & 0 & 4 & 1 & 8 & 2 & 13 & 10 & 6 & 1 & 4 & $w_{354}$ & N & can. \\
4024 & 0 & 0 & 0 & 0 & 0 & 0 & 1 & 2 & 0 & 4 & 1 & 8 & 2 & 13 & 10 & 7 & 1 & 4 & $w_{281}$ & N & can. \\
4025 & 0 & 0 & 0 & 0 & 0 & 0 & 1 & 2 & 0 & 4 & 1 & 8 & 2 & 13 & 10 & 14 & 1 & 4 & $w_{354}$ & N & can. \\
4026 & 0 & 0 & 0 & 0 & 0 & 0 & 1 & 2 & 0 & 4 & 1 & 8 & 2 & 13 & 10 & 15 & 1 & 4 & $w_{281}$ & N & can. \\
4027 & 0 & 0 & 0 & 0 & 0 & 0 & 1 & 2 & 0 & 4 & 1 & 8 & 2 & 13 & 11 & 5 & 1 & 4 & $w_{379}$ & N & can. \\
4028 & 0 & 0 & 0 & 0 & 0 & 0 & 1 & 2 & 0 & 4 & 1 & 8 & 2 & 13 & 11 & 6 & 1 & 4 & $w_{379}$ & N & can. \\
4029 & 0 & 0 & 0 & 0 & 0 & 0 & 1 & 2 & 0 & 4 & 1 & 8 & 2 & 13 & 11 & 7 & 1 & 4 & $w_{379}$ & N & can. \\
4030 & 0 & 0 & 0 & 0 & 0 & 0 & 1 & 2 & 0 & 4 & 1 & 8 & 2 & 13 & 11 & 14 & 1 & 4 & $w_{428}$ & N & can. \\
4031 & 0 & 0 & 0 & 0 & 0 & 0 & 1 & 2 & 0 & 4 & 1 & 8 & 2 & 13 & 11 & 15 & 1 & 4 & $w_{281}$ & N & can. \\
4032 & 0 & 0 & 0 & 0 & 0 & 0 & 1 & 2 & 0 & 4 & 1 & 8 & 2 & 13 & 12 & 5 & 1 & 4 & $w_{353}$ & N & can. \\
4033 & 0 & 0 & 0 & 0 & 0 & 0 & 1 & 2 & 0 & 4 & 1 & 8 & 2 & 13 & 12 & 6 & 1 & 4 & $w_{354}$ & N & can. \\
4034 & 0 & 0 & 0 & 0 & 0 & 0 & 1 & 2 & 0 & 4 & 1 & 8 & 2 & 13 & 12 & 7 & 1 & 4 & $w_{281}$ & N & can. \\
4035 & 0 & 0 & 0 & 0 & 0 & 0 & 1 & 2 & 0 & 4 & 1 & 8 & 2 & 13 & 12 & 9 & 1 & 4 & $w_{392}$ & N & can. \\
4036 & 0 & 0 & 0 & 0 & 0 & 0 & 1 & 2 & 0 & 4 & 1 & 8 & 2 & 13 & 12 & 11 & 1 & 4 & $w_{355}$ & N & can. \\
4037 & 0 & 0 & 0 & 0 & 0 & 0 & 1 & 2 & 0 & 4 & 1 & 8 & 2 & 13 & 14 & 5 & 1 & 4 & $w_{281}$ & N & can. \\
4038 & 0 & 0 & 0 & 0 & 0 & 0 & 1 & 2 & 0 & 4 & 1 & 8 & 2 & 13 & 14 & 6 & 1 & 4 & $w_{281}$ & N & can. \\
4039 & 0 & 0 & 0 & 0 & 0 & 0 & 1 & 2 & 0 & 4 & 1 & 8 & 2 & 13 & 14 & 9 & 1 & 4 & $w_{281}$ & N & can. \\
4040 & 0 & 0 & 0 & 0 & 0 & 0 & 1 & 2 & 0 & 4 & 1 & 8 & 2 & 13 & 14 & 11 & 1 & 4 & $w_{379}$ & N & can. \\
4041 & 0 & 0 & 0 & 0 & 0 & 0 & 1 & 2 & 0 & 4 & 1 & 8 & 2 & 13 & 15 & 5 & 1 & 4 & $w_{354}$ & N & can. \\
4042 & 0 & 0 & 0 & 0 & 0 & 0 & 1 & 2 & 0 & 4 & 1 & 8 & 2 & 13 & 15 & 6 & 1 & 4 & $w_{373}$ & N & can. \\
4043 & 0 & 0 & 0 & 0 & 0 & 0 & 1 & 2 & 0 & 4 & 1 & 8 & 2 & 13 & 15 & 7 & 1 & 4 & $w_{281}$ & N & can. \\
4044 & 0 & 0 & 0 & 0 & 0 & 0 & 1 & 2 & 0 & 4 & 1 & 8 & 2 & 13 & 15 & 9 & 1 & 4 & $w_{281}$ & N & can. \\
4045 & 0 & 0 & 0 & 0 & 0 & 0 & 1 & 2 & 0 & 4 & 1 & 8 & 2 & 13 & 15 & 11 & 1 & 4 & $w_{354}$ & N & can. \\
4046 & 0 & 0 & 0 & 0 & 0 & 0 & 1 & 2 & 0 & 4 & 1 & 8 & 2 & 15 & 5 & 9 & 1 & 4 & $w_{281}$ & N & can. \\
4047 & 0 & 0 & 0 & 0 & 0 & 0 & 1 & 2 & 0 & 4 & 1 & 8 & 2 & 15 & 5 & 11 & 1 & 4 & $w_{432}$ & N & \#3988 \\
4048 & 0 & 0 & 0 & 0 & 0 & 0 & 1 & 2 & 0 & 4 & 1 & 8 & 2 & 15 & 5 & 12 & 1 & 4 & $w_{278}$ & N & \#3945 \\
4049 & 0 & 0 & 0 & 0 & 0 & 0 & 1 & 2 & 0 & 4 & 1 & 8 & 2 & 15 & 5 & 13 & 1 & 4 & $w_{281}$ & N & can. \\
4050 & 0 & 0 & 0 & 0 & 0 & 0 & 1 & 2 & 0 & 4 & 1 & 8 & 2 & 15 & 7 & 9 & 1 & 4 & $w_{392}$ & N & can. \\
4051 & 0 & 0 & 0 & 0 & 0 & 0 & 1 & 2 & 0 & 4 & 1 & 8 & 2 & 15 & 7 & 11 & 1 & 4 & $w_{379}$ & N & can. \\
4052 & 0 & 0 & 0 & 0 & 0 & 0 & 1 & 2 & 0 & 4 & 1 & 8 & 2 & 15 & 9 & 5 & 1 & 4 & $w_{281}$ & N & can. \\
4053 & 0 & 0 & 0 & 0 & 0 & 0 & 1 & 2 & 0 & 4 & 1 & 8 & 2 & 15 & 9 & 6 & 1 & 4 & $w_{281}$ & N & can. \\
4054 & 0 & 0 & 0 & 0 & 0 & 0 & 1 & 2 & 0 & 4 & 1 & 8 & 2 & 15 & 9 & 7 & 1 & 4 & $w_{392}$ & N & can. \\
4055 & 0 & 0 & 0 & 0 & 0 & 0 & 1 & 2 & 0 & 4 & 1 & 8 & 2 & 15 & 9 & 12 & 1 & 4 & $w_{379}$ & N & can. \\
4056 & 0 & 0 & 0 & 0 & 0 & 0 & 1 & 2 & 0 & 4 & 1 & 8 & 2 & 15 & 9 & 13 & 1 & 4 & $w_{266}$ & N & can. \\
4057 & 0 & 0 & 0 & 0 & 0 & 0 & 1 & 2 & 0 & 4 & 1 & 8 & 2 & 15 & 10 & 5 & 1 & 4 & $w_{267}$ & N & can. \\
4058 & 0 & 0 & 0 & 0 & 0 & 0 & 1 & 2 & 0 & 4 & 1 & 8 & 2 & 15 & 10 & 6 & 1 & 4 & $w_{267}$ & N & can. \\
4059 & 0 & 0 & 0 & 0 & 0 & 0 & 1 & 2 & 0 & 4 & 1 & 8 & 2 & 15 & 10 & 12 & 1 & 4 & $w_{281}$ & N & can. \\
4060 & 0 & 0 & 0 & 0 & 0 & 0 & 1 & 2 & 0 & 4 & 1 & 8 & 2 & 15 & 10 & 13 & 1 & 4 & $w_{266}$ & N & can. \\
4061 & 0 & 0 & 0 & 0 & 0 & 0 & 1 & 2 & 0 & 4 & 1 & 8 & 2 & 15 & 11 & 5 & 1 & 4 & $w_{432}$ & N & \#3988 \\
4062 & 0 & 0 & 0 & 0 & 0 & 0 & 1 & 2 & 0 & 4 & 1 & 8 & 2 & 15 & 11 & 6 & 1 & 4 & $w_{278}$ & N & \#3945 \\
4063 & 0 & 0 & 0 & 0 & 0 & 0 & 1 & 2 & 0 & 4 & 1 & 8 & 2 & 15 & 11 & 7 & 1 & 4 & $w_{379}$ & N & \#3517 \\
4064 & 0 & 0 & 0 & 0 & 0 & 0 & 1 & 2 & 0 & 4 & 1 & 8 & 2 & 15 & 11 & 12 & 1 & 4 & $w_{434}$ & N & can. \\
4065 & 0 & 0 & 0 & 0 & 0 & 0 & 1 & 2 & 0 & 4 & 1 & 8 & 2 & 15 & 11 & 13 & 1 & 4 & $w_{379}$ & N & can. \\
4066 & 0 & 0 & 0 & 0 & 0 & 0 & 1 & 2 & 0 & 4 & 1 & 8 & 5 & 12 & 10 & 14 & 4 & 4 & $w_{264}$ & N & can. \\
4067 & 0 & 0 & 0 & 0 & 0 & 0 & 1 & 2 & 0 & 4 & 1 & 8 & 5 & 12 & 10 & 15 & 1 & 4 & $w_{278}$ & N & can. \\
4068 & 0 & 0 & 0 & 0 & 0 & 0 & 1 & 2 & 0 & 4 & 1 & 8 & 5 & 12 & 11 & 15 & 2 & 4 & $w_{278}$ & N & \#3525 \\
4069 & 0 & 0 & 0 & 0 & 0 & 0 & 1 & 2 & 0 & 4 & 1 & 8 & 5 & 12 & 13 & 9 & 8 & 4 & $w_{403}$ & N & can. \\
4070 & 0 & 0 & 0 & 0 & 0 & 0 & 1 & 2 & 0 & 4 & 1 & 8 & 5 & 12 & 13 & 10 & 1 & 3 & $w_{345}$ & N & can. \\
4071 & 0 & 0 & 0 & 0 & 0 & 0 & 1 & 2 & 0 & 4 & 1 & 8 & 5 & 12 & 13 & 11 & 1 & 4 & $w_{392}$ & N & can. \\
4072 & 0 & 0 & 0 & 0 & 0 & 0 & 1 & 2 & 0 & 4 & 1 & 8 & 5 & 12 & 14 & 10 & 4 & 4 & $w_{352}$ & N & can. \\
4073 & 0 & 0 & 0 & 0 & 0 & 0 & 1 & 2 & 0 & 4 & 1 & 8 & 5 & 12 & 14 & 11 & 1 & 4 & $w_{281}$ & N & can. \\
4074 & 0 & 0 & 0 & 0 & 0 & 0 & 1 & 2 & 0 & 4 & 1 & 8 & 5 & 12 & 15 & 11 & 2 & 4 & $w_{278}$ & N & \#3525 \\
4075 & 0 & 0 & 0 & 0 & 0 & 0 & 1 & 2 & 0 & 4 & 1 & 8 & 5 & 13 & 10 & 14 & 1 & 4 & $w_{278}$ & N & \#4067 \\
4076 & 0 & 0 & 0 & 0 & 0 & 0 & 1 & 2 & 0 & 4 & 1 & 8 & 5 & 13 & 10 & 15 & 2 & 4 & $w_{278}$ & N & can. \\
4077 & 0 & 0 & 0 & 0 & 0 & 0 & 1 & 2 & 0 & 4 & 1 & 8 & 5 & 13 & 11 & 14 & 2 & 4 & $w_{434}$ & N & \#4064 \\
4078 & 0 & 0 & 0 & 0 & 0 & 0 & 1 & 2 & 0 & 4 & 1 & 8 & 5 & 13 & 12 & 9 & 8 & 4 & $w_{279}$ & N & can. \\
4079 & 0 & 0 & 0 & 0 & 0 & 0 & 1 & 2 & 0 & 4 & 1 & 8 & 5 & 13 & 12 & 10 & 1 & 3 & $w_{280}$ & N & can. \\
4080 & 0 & 0 & 0 & 0 & 0 & 0 & 1 & 2 & 0 & 4 & 1 & 8 & 5 & 13 & 12 & 11 & 1 & 4 & $w_{423}$ & N & can. \\
4081 & 0 & 0 & 0 & 0 & 0 & 0 & 1 & 2 & 0 & 4 & 1 & 8 & 5 & 13 & 14 & 10 & 1 & 4 & $w_{281}$ & N & \#4073 \\
4082 & 0 & 0 & 0 & 0 & 0 & 0 & 1 & 2 & 0 & 4 & 1 & 8 & 5 & 13 & 14 & 11 & 2 & 4 & $w_{428}$ & N & can. \\
4083 & 0 & 0 & 0 & 0 & 0 & 0 & 1 & 2 & 0 & 4 & 1 & 8 & 5 & 13 & 15 & 10 & 2 & 4 & $w_{281}$ & N & can. \\
4084 & 0 & 0 & 0 & 0 & 0 & 0 & 1 & 2 & 0 & 4 & 1 & 8 & 5 & 14 & 10 & 13 & 1 & 4 & $w_{278}$ & N & \#4067 \\
4085 & 0 & 0 & 0 & 0 & 0 & 0 & 1 & 2 & 0 & 4 & 1 & 8 & 5 & 14 & 11 & 13 & 2 & 4 & $w_{278}$ & N & \#3525 \\
4086 & 0 & 0 & 0 & 0 & 0 & 0 & 1 & 2 & 0 & 4 & 1 & 8 & 5 & 14 & 12 & 11 & 1 & 4 & $w_{379}$ & N & can. \\
4087 & 0 & 0 & 0 & 0 & 0 & 0 & 1 & 2 & 0 & 4 & 1 & 8 & 5 & 14 & 13 & 11 & 2 & 4 & $w_{428}$ & N & can. \\
4088 & 0 & 0 & 0 & 0 & 0 & 0 & 1 & 2 & 0 & 4 & 1 & 8 & 5 & 14 & 15 & 9 & 8 & 4 & $w_{281}$ & N & can. \\
4089 & 0 & 0 & 0 & 0 & 0 & 0 & 1 & 2 & 0 & 4 & 1 & 8 & 5 & 15 & 10 & 12 & 1 & 4 & $w_{278}$ & N & \#4067 \\
4090 & 0 & 0 & 0 & 0 & 0 & 0 & 1 & 2 & 0 & 4 & 1 & 8 & 5 & 15 & 10 & 13 & 4 & 4 & $w_{264}$ & N & \#4066 \\
4091 & 0 & 0 & 0 & 0 & 0 & 0 & 1 & 2 & 0 & 4 & 1 & 8 & 5 & 15 & 11 & 12 & 2 & 4 & $w_{434}$ & N & \#4064 \\
4092 & 0 & 0 & 0 & 0 & 0 & 0 & 1 & 2 & 0 & 4 & 1 & 8 & 5 & 15 & 12 & 10 & 1 & 4 & $w_{379}$ & N & \#4086 \\
4093 & 0 & 0 & 0 & 0 & 0 & 0 & 1 & 2 & 0 & 4 & 1 & 8 & 5 & 15 & 12 & 11 & 2 & 4 & $w_{434}$ & N & \#4064 \\
4094 & 0 & 0 & 0 & 0 & 0 & 0 & 1 & 2 & 0 & 4 & 1 & 8 & 5 & 15 & 13 & 10 & 4 & 4 & $w_{347}$ & N & can. \\
4095 & 0 & 0 & 0 & 0 & 0 & 0 & 1 & 2 & 0 & 4 & 1 & 8 & 5 & 15 & 14 & 9 & 8 & 4 & $w_{347}$ & N & can. \\
4096 & 0 & 0 & 0 & 0 & 0 & 0 & 1 & 2 & 0 & 4 & 1 & 8 & 6 & 12 & 11 & 15 & 2 & 3 & $w_{377}$ & N & \#3531 \\
4097 & 0 & 0 & 0 & 0 & 0 & 0 & 1 & 2 & 0 & 4 & 1 & 8 & 6 & 12 & 13 & 10 & 8 & 4 & $w_{347}$ & N & can. \\
4098 & 0 & 0 & 0 & 0 & 0 & 0 & 1 & 2 & 0 & 4 & 1 & 8 & 6 & 12 & 13 & 11 & 1 & 4 & $w_{379}$ & N & can. \\
4099 & 0 & 0 & 0 & 0 & 0 & 0 & 1 & 2 & 0 & 4 & 1 & 8 & 6 & 12 & 15 & 11 & 2 & 3 & $w_{377}$ & Y & \#3531 \\
4100 & 0 & 0 & 0 & 0 & 0 & 0 & 1 & 2 & 0 & 4 & 1 & 8 & 6 & 13 & 11 & 14 & 1 & 3 & $w_{377}$ & Y & \#3531 \\
4101 & 0 & 0 & 0 & 0 & 0 & 0 & 1 & 2 & 0 & 4 & 1 & 8 & 6 & 13 & 12 & 10 & 4 & 4 & $w_{281}$ & N & can. \\
4102 & 0 & 0 & 0 & 0 & 0 & 0 & 1 & 2 & 0 & 4 & 1 & 8 & 6 & 13 & 14 & 10 & 1 & 4 & $w_{281}$ & N & can. \\
4103 & 0 & 0 & 0 & 0 & 0 & 0 & 1 & 2 & 0 & 4 & 1 & 8 & 6 & 13 & 14 & 11 & 1 & 3 & $w_{440}$ & Y & can. \\
4104 & 0 & 0 & 0 & 0 & 0 & 0 & 1 & 2 & 0 & 4 & 1 & 8 & 6 & 15 & 11 & 12 & 2 & 3 & $w_{377}$ & N & \#3531 \\
4105 & 0 & 0 & 0 & 0 & 0 & 0 & 1 & 2 & 0 & 4 & 1 & 8 & 6 & 15 & 12 & 11 & 2 & 3 & $w_{377}$ & N & \#3531 \\
4106 & 0 & 0 & 0 & 0 & 0 & 0 & 1 & 2 & 0 & 4 & 1 & 8 & 6 & 15 & 14 & 10 & 8 & 4 & $w_{352}$ & N & can. \\
4107 & 0 & 0 & 0 & 0 & 0 & 0 & 1 & 2 & 0 & 4 & 1 & 8 & 7 & 12 & 11 & 14 & 2 & 3 & $w_{377}$ & Y & can. \\
4108 & 0 & 0 & 0 & 0 & 0 & 0 & 1 & 2 & 0 & 4 & 1 & 8 & 7 & 12 & 13 & 11 & 4 & 4 & $w_{428}$ & N & can. \\
4109 & 0 & 0 & 0 & 0 & 0 & 0 & 1 & 2 & 0 & 4 & 1 & 8 & 7 & 12 & 14 & 11 & 2 & 3 & $w_{440}$ & N & can. \\
4110 & 0 & 0 & 0 & 0 & 0 & 0 & 1 & 2 & 0 & 4 & 1 & 8 & 7 & 12 & 15 & 11 & 1 & 4 & $w_{428}$ & N & can. \\
4111 & 0 & 0 & 0 & 0 & 0 & 0 & 1 & 2 & 0 & 4 & 1 & 8 & 7 & 13 & 11 & 14 & 1 & 4 & $w_{434}$ & N & can. \\
4112 & 0 & 0 & 0 & 0 & 0 & 0 & 1 & 2 & 0 & 4 & 1 & 8 & 7 & 13 & 12 & 11 & 4 & 4 & $w_{426}$ & N & can. \\
4113 & 0 & 0 & 0 & 0 & 0 & 0 & 1 & 2 & 0 & 4 & 1 & 8 & 7 & 13 & 14 & 11 & 1 & 4 & $w_{430}$ & N & can. \\
4114 & 0 & 0 & 0 & 0 & 0 & 0 & 1 & 2 & 0 & 4 & 1 & 8 & 7 & 13 & 15 & 11 & 2 & 3 & $w_{378}$ & N & can. \\
4115 & 0 & 0 & 0 & 0 & 0 & 0 & 1 & 2 & 0 & 4 & 1 & 8 & 7 & 14 & 15 & 11 & 4 & 4 & $w_{347}$ & N & can. \\
4116 & 0 & 0 & 0 & 0 & 0 & 0 & 1 & 2 & 0 & 4 & 1 & 8 & 7 & 15 & 14 & 11 & 4 & 4 & $w_{428}$ & N & can. \\
4117 & 0 & 0 & 0 & 0 & 0 & 0 & 1 & 2 & 0 & 4 & 3 & 7 & 8 & 13 & 14 & 11 & 32 & 4 & $w_{434}$ & N & can. \\
4118 & 0 & 0 & 0 & 0 & 0 & 0 & 1 & 2 & 0 & 4 & 3 & 7 & 8 & 15 & 12 & 11 & 64 & 4 & $w_{445}$ & N & can. \\
4119 & 0 & 0 & 0 & 0 & 0 & 0 & 1 & 2 & 0 & 4 & 3 & 8 & 7 & 12 & 15 & 11 & 16 & 4 & $w_{426}$ & N & can. \\
4120 & 0 & 0 & 0 & 0 & 0 & 0 & 1 & 2 & 0 & 4 & 3 & 8 & 7 & 13 & 11 & 14 & 8 & 4 & $w_{434}$ & N & can. \\
4121 & 0 & 0 & 0 & 0 & 0 & 0 & 1 & 2 & 0 & 4 & 3 & 8 & 7 & 13 & 14 & 11 & 8 & 4 & $w_{441}$ & N & can. \\
4122 & 0 & 0 & 0 & 0 & 0 & 0 & 1 & 2 & 0 & 4 & 3 & 8 & 7 & 15 & 11 & 12 & 16 & 4 & $w_{437}$ & N & can. \\
4123 & 0 & 0 & 0 & 0 & 0 & 0 & 1 & 2 & 0 & 4 & 3 & 8 & 7 & 15 & 12 & 11 & 16 & 4 & $w_{446}$ & N & can. \\
4124 & 0 & 0 & 0 & 0 & 0 & 1 & 2 & 3 & 0 & 2 & 3 & 1 & 0 & 3 & 1 & 2 & 552960 & 2 & $w_{447}$ & N & can. \\
4125 & 0 & 0 & 0 & 0 & 0 & 1 & 2 & 3 & 0 & 2 & 3 & 1 & 0 & 3 & 1 & 4 & 2880 & 4 & $w_{448}$ & N & can. \\
4126 & 0 & 0 & 0 & 0 & 0 & 1 & 2 & 3 & 0 & 2 & 3 & 1 & 0 & 3 & 4 & 7 & 384 & 3 & $w_{449}$ & N & can. \\
4127 & 0 & 0 & 0 & 0 & 0 & 1 & 2 & 3 & 0 & 2 & 3 & 1 & 0 & 3 & 4 & 8 & 48 & 4 & $w_{450}$ & N & can. \\
4128 & 0 & 0 & 0 & 0 & 0 & 1 & 2 & 3 & 0 & 2 & 3 & 1 & 0 & 4 & 6 & 5 & 576 & 4 & $w_{451}$ & N & can. \\
4129 & 0 & 0 & 0 & 0 & 0 & 1 & 2 & 3 & 0 & 2 & 3 & 1 & 0 & 4 & 6 & 8 & 24 & 4 & $w_{452}$ & N & can. \\
4130 & 0 & 0 & 0 & 0 & 0 & 1 & 2 & 3 & 0 & 2 & 3 & 1 & 0 & 4 & 8 & 12 & 72 & 3 & $w_{453}$ & N & can. \\
4131 & 0 & 0 & 0 & 0 & 0 & 1 & 2 & 3 & 0 & 2 & 3 & 1 & 0 & 4 & 8 & 13 & 24 & 4 & $w_{444}$ & N & can. \\
4132 & 0 & 0 & 0 & 0 & 0 & 1 & 2 & 3 & 0 & 2 & 3 & 1 & 4 & 7 & 5 & 6 & 36864 & 2 & $w_{454}$ & N & can. \\
4133 & 0 & 0 & 0 & 0 & 0 & 1 & 2 & 3 & 0 & 2 & 3 & 1 & 4 & 7 & 5 & 8 & 288 & 4 & $w_{455}$ & N & can. \\
4134 & 0 & 0 & 0 & 0 & 0 & 1 & 2 & 3 & 0 & 2 & 3 & 1 & 4 & 7 & 8 & 11 & 384 & 3 & $w_{456}$ & N & can. \\
4135 & 0 & 0 & 0 & 0 & 0 & 1 & 2 & 3 & 0 & 2 & 3 & 1 & 4 & 7 & 8 & 12 & 48 & 4 & $w_{457}$ & N & can. \\
4136 & 0 & 0 & 0 & 0 & 0 & 1 & 2 & 3 & 0 & 2 & 3 & 4 & 0 & 3 & 4 & 6 & 32 & 4 & $w_{402}$ & N & can. \\
4137 & 0 & 0 & 0 & 0 & 0 & 1 & 2 & 3 & 0 & 2 & 3 & 4 & 0 & 3 & 4 & 7 & 288 & 4 & $w_{458}$ & N & can. \\
4138 & 0 & 0 & 0 & 0 & 0 & 1 & 2 & 3 & 0 & 2 & 3 & 4 & 0 & 3 & 4 & 8 & 4 & 4 & $w_{459}$ & N & can. \\
4139 & 0 & 0 & 0 & 0 & 0 & 1 & 2 & 3 & 0 & 2 & 3 & 4 & 0 & 3 & 5 & 8 & 4 & 4 & $w_{403}$ & N & can. \\
4140 & 0 & 0 & 0 & 0 & 0 & 1 & 2 & 3 & 0 & 2 & 3 & 4 & 0 & 3 & 8 & 12 & 6 & 4 & $w_{460}$ & N & can. \\
4141 & 0 & 0 & 0 & 0 & 0 & 1 & 2 & 3 & 0 & 2 & 3 & 4 & 0 & 3 & 8 & 14 & 12 & 3 & $w_{453}$ & N & can. \\
4142 & 0 & 0 & 0 & 0 & 0 & 1 & 2 & 3 & 0 & 2 & 3 & 4 & 0 & 3 & 8 & 15 & 12 & 4 & $w_{407}$ & N & can. \\
4143 & 0 & 0 & 0 & 0 & 0 & 1 & 2 & 3 & 0 & 2 & 3 & 4 & 0 & 4 & 5 & 8 & 2 & 4 & $w_{421}$ & N & can. \\
4144 & 0 & 0 & 0 & 0 & 0 & 1 & 2 & 3 & 0 & 2 & 3 & 4 & 0 & 4 & 8 & 5 & 2 & 4 & $w_{417}$ & N & can. \\
4145 & 0 & 0 & 0 & 0 & 0 & 1 & 2 & 3 & 0 & 2 & 3 & 4 & 0 & 4 & 8 & 7 & 1 & 4 & $w_{412}$ & N & can. \\
4146 & 0 & 0 & 0 & 0 & 0 & 1 & 2 & 3 & 0 & 2 & 3 & 4 & 0 & 4 & 8 & 10 & 1 & 4 & $w_{413}$ & N & can. \\
4147 & 0 & 0 & 0 & 0 & 0 & 1 & 2 & 3 & 0 & 2 & 3 & 4 & 0 & 4 & 8 & 11 & 1 & 4 & $w_{424}$ & N & can. \\
4148 & 0 & 0 & 0 & 0 & 0 & 1 & 2 & 3 & 0 & 2 & 3 & 4 & 0 & 4 & 8 & 12 & 1 & 4 & $w_{347}$ & N & can. \\
4149 & 0 & 0 & 0 & 0 & 0 & 1 & 2 & 3 & 0 & 2 & 3 & 4 & 0 & 4 & 8 & 13 & 1 & 4 & $w_{423}$ & N & can. \\
4150 & 0 & 0 & 0 & 0 & 0 & 1 & 2 & 3 & 0 & 2 & 3 & 4 & 0 & 4 & 8 & 14 & 1 & 4 & $w_{419}$ & N & can. \\
4151 & 0 & 0 & 0 & 0 & 0 & 1 & 2 & 3 & 0 & 2 & 3 & 4 & 0 & 4 & 8 & 15 & 1 & 4 & $w_{392}$ & N & can. \\
4152 & 0 & 0 & 0 & 0 & 0 & 1 & 2 & 3 & 0 & 2 & 3 & 4 & 0 & 6 & 4 & 7 & 384 & 3 & $w_{461}$ & N & can. \\
4153 & 0 & 0 & 0 & 0 & 0 & 1 & 2 & 3 & 0 & 2 & 3 & 4 & 0 & 6 & 4 & 8 & 6 & 4 & $w_{462}$ & N & can. \\
4154 & 0 & 0 & 0 & 0 & 0 & 1 & 2 & 3 & 0 & 2 & 3 & 4 & 0 & 6 & 8 & 10 & 2 & 4 & $w_{418}$ & N & can. \\
4155 & 0 & 0 & 0 & 0 & 0 & 1 & 2 & 3 & 0 & 2 & 3 & 4 & 0 & 6 & 8 & 11 & 2 & 3 & $w_{463}$ & N & can. \\
4156 & 0 & 0 & 0 & 0 & 0 & 1 & 2 & 3 & 0 & 2 & 3 & 4 & 0 & 6 & 8 & 12 & 1 & 4 & $w_{419}$ & N & can. \\
4157 & 0 & 0 & 0 & 0 & 0 & 1 & 2 & 3 & 0 & 2 & 3 & 4 & 0 & 6 & 8 & 14 & 2 & 4 & $w_{464}$ & N & can. \\
4158 & 0 & 0 & 0 & 0 & 0 & 1 & 2 & 3 & 0 & 2 & 3 & 4 & 0 & 6 & 8 & 15 & 2 & 4 & $w_{347}$ & N & can. \\
4159 & 0 & 0 & 0 & 0 & 0 & 1 & 2 & 3 & 0 & 2 & 3 & 4 & 0 & 8 & 9 & 12 & 1 & 4 & $w_{419}$ & N & can. \\
4160 & 0 & 0 & 0 & 0 & 0 & 1 & 2 & 3 & 0 & 2 & 3 & 4 & 0 & 8 & 9 & 13 & 2 & 4 & $w_{423}$ & N & can. \\
4161 & 0 & 0 & 0 & 0 & 0 & 1 & 2 & 3 & 0 & 2 & 3 & 4 & 0 & 8 & 10 & 9 & 6 & 4 & $w_{465}$ & N & can. \\
4162 & 0 & 0 & 0 & 0 & 0 & 1 & 2 & 3 & 0 & 2 & 3 & 4 & 0 & 8 & 10 & 12 & 2 & 4 & $w_{466}$ & N & can. \\
4163 & 0 & 0 & 0 & 0 & 0 & 1 & 2 & 3 & 0 & 2 & 3 & 4 & 0 & 8 & 10 & 13 & 1 & 4 & $w_{419}$ & N & can. \\
4164 & 0 & 0 & 0 & 0 & 0 & 1 & 2 & 3 & 0 & 2 & 3 & 4 & 0 & 8 & 10 & 15 & 2 & 4 & $w_{426}$ & N & can. \\
4165 & 0 & 0 & 0 & 0 & 0 & 1 & 2 & 3 & 0 & 2 & 3 & 4 & 4 & 6 & 8 & 11 & 4 & 4 & $w_{426}$ & N & can. \\
4166 & 0 & 0 & 0 & 0 & 0 & 1 & 2 & 3 & 0 & 2 & 3 & 4 & 4 & 6 & 8 & 12 & 2 & 4 & $w_{425}$ & N & can. \\
4167 & 0 & 0 & 0 & 0 & 0 & 1 & 2 & 3 & 0 & 2 & 3 & 4 & 4 & 6 & 8 & 14 & 4 & 4 & $w_{426}$ & N & can. \\
4168 & 0 & 0 & 0 & 0 & 0 & 1 & 2 & 3 & 0 & 2 & 3 & 4 & 4 & 6 & 8 & 15 & 4 & 3 & $w_{378}$ & N & can. \\
4169 & 0 & 0 & 0 & 0 & 0 & 1 & 2 & 3 & 0 & 2 & 3 & 4 & 4 & 7 & 8 & 10 & 4 & 4 & $w_{426}$ & N & can. \\
4170 & 0 & 0 & 0 & 0 & 0 & 1 & 2 & 3 & 0 & 2 & 3 & 4 & 4 & 7 & 8 & 11 & 4 & 4 & $w_{466}$ & N & can. \\
4171 & 0 & 0 & 0 & 0 & 0 & 1 & 2 & 3 & 0 & 2 & 3 & 4 & 4 & 7 & 8 & 12 & 2 & 4 & $w_{467}$ & N & can. \\
4172 & 0 & 0 & 0 & 0 & 0 & 1 & 2 & 3 & 0 & 2 & 3 & 4 & 4 & 7 & 8 & 14 & 4 & 3 & $w_{468}$ & N & can. \\
4173 & 0 & 0 & 0 & 0 & 0 & 1 & 2 & 3 & 0 & 2 & 3 & 4 & 4 & 7 & 8 & 15 & 4 & 4 & $w_{426}$ & N & can. \\
4174 & 0 & 0 & 0 & 0 & 0 & 1 & 2 & 3 & 0 & 2 & 3 & 4 & 4 & 8 & 9 & 12 & 1 & 4 & $w_{429}$ & N & can. \\
4175 & 0 & 0 & 0 & 0 & 0 & 1 & 2 & 3 & 0 & 2 & 3 & 4 & 4 & 8 & 9 & 13 & 2 & 4 & $w_{430}$ & N & can. \\
4176 & 0 & 0 & 0 & 0 & 0 & 1 & 2 & 3 & 0 & 2 & 3 & 4 & 4 & 8 & 10 & 9 & 12 & 4 & $w_{457}$ & N & can. \\
4177 & 0 & 0 & 0 & 0 & 0 & 1 & 2 & 3 & 0 & 2 & 3 & 4 & 4 & 8 & 10 & 12 & 2 & 4 & $w_{469}$ & N & can. \\
4178 & 0 & 0 & 0 & 0 & 0 & 1 & 2 & 3 & 0 & 2 & 3 & 4 & 4 & 8 & 10 & 13 & 1 & 4 & $w_{429}$ & N & can. \\
4179 & 0 & 0 & 0 & 0 & 0 & 1 & 2 & 3 & 0 & 2 & 3 & 4 & 4 & 8 & 10 & 15 & 2 & 4 & $w_{470}$ & N & can. \\
4180 & 0 & 0 & 0 & 0 & 0 & 1 & 2 & 3 & 0 & 2 & 3 & 4 & 8 & 10 & 12 & 15 & 8 & 4 & $w_{470}$ & N & can. \\
4181 & 0 & 0 & 0 & 0 & 0 & 1 & 2 & 3 & 0 & 2 & 3 & 4 & 8 & 11 & 9 & 12 & 16 & 4 & $w_{446}$ & N & can. \\
4182 & 0 & 0 & 0 & 0 & 0 & 1 & 2 & 3 & 0 & 2 & 3 & 4 & 8 & 11 & 9 & 15 & 48 & 3 & $w_{471}$ & N & can. \\
4183 & 0 & 0 & 0 & 0 & 0 & 1 & 2 & 3 & 0 & 2 & 3 & 4 & 8 & 11 & 12 & 15 & 96 & 4 & $w_{472}$ & N & can. \\
4184 & 0 & 0 & 0 & 0 & 0 & 1 & 2 & 3 & 0 & 2 & 3 & 4 & 8 & 11 & 13 & 14 & 32 & 4 & $w_{470}$ & N & can. \\
4185 & 0 & 0 & 0 & 0 & 0 & 1 & 2 & 3 & 0 & 2 & 4 & 6 & 0 & 3 & 6 & 5 & 4608 & 2 & $w_{342}$ & N & can. \\
4186 & 0 & 0 & 0 & 0 & 0 & 1 & 2 & 3 & 0 & 2 & 4 & 6 & 0 & 3 & 6 & 8 & 36 & 4 & $w_{343}$ & N & can. \\
4187 & 0 & 0 & 0 & 0 & 0 & 1 & 2 & 3 & 0 & 2 & 4 & 6 & 0 & 3 & 7 & 4 & 512 & 3 & $w_{342}$ & N & \#4185 \\
4188 & 0 & 0 & 0 & 0 & 0 & 1 & 2 & 3 & 0 & 2 & 4 & 6 & 0 & 3 & 7 & 8 & 4 & 4 & $w_{343}$ & N & \#4186 \\
4189 & 0 & 0 & 0 & 0 & 0 & 1 & 2 & 3 & 0 & 2 & 4 & 6 & 0 & 3 & 8 & 11 & 16 & 3 & $w_{345}$ & N & can. \\
4190 & 0 & 0 & 0 & 0 & 0 & 1 & 2 & 3 & 0 & 2 & 4 & 6 & 0 & 3 & 8 & 12 & 8 & 4 & $w_{464}$ & N & can. \\
4191 & 0 & 0 & 0 & 0 & 0 & 1 & 2 & 3 & 0 & 2 & 4 & 6 & 0 & 3 & 8 & 13 & 8 & 4 & $w_{433}$ & N & can. \\
4192 & 0 & 0 & 0 & 0 & 0 & 1 & 2 & 3 & 0 & 2 & 4 & 6 & 0 & 3 & 8 & 14 & 4 & 4 & $w_{347}$ & N & can. \\
4193 & 0 & 0 & 0 & 0 & 0 & 1 & 2 & 3 & 0 & 2 & 4 & 6 & 0 & 4 & 3 & 7 & 192 & 3 & $w_{342}$ & N & can. \\
4194 & 0 & 0 & 0 & 0 & 0 & 1 & 2 & 3 & 0 & 2 & 4 & 6 & 0 & 4 & 3 & 8 & 2 & 4 & $w_{343}$ & N & can. \\
4195 & 0 & 0 & 0 & 0 & 0 & 1 & 2 & 3 & 0 & 2 & 4 & 6 & 0 & 4 & 8 & 11 & 8 & 4 & $w_{464}$ & N & can. \\
4196 & 0 & 0 & 0 & 0 & 0 & 1 & 2 & 3 & 0 & 2 & 4 & 6 & 0 & 4 & 8 & 12 & 24 & 3 & $w_{453}$ & N & can. \\
4197 & 0 & 0 & 0 & 0 & 0 & 1 & 2 & 3 & 0 & 2 & 4 & 6 & 0 & 4 & 8 & 13 & 8 & 4 & $w_{347}$ & N & can. \\
4198 & 0 & 0 & 0 & 0 & 0 & 1 & 2 & 3 & 0 & 2 & 4 & 6 & 0 & 4 & 8 & 14 & 4 & 4 & $w_{418}$ & N & can. \\
4199 & 0 & 0 & 0 & 0 & 0 & 1 & 2 & 3 & 0 & 2 & 4 & 6 & 0 & 5 & 8 & 11 & 8 & 4 & $w_{433}$ & N & can. \\
4200 & 0 & 0 & 0 & 0 & 0 & 1 & 2 & 3 & 0 & 2 & 4 & 6 & 0 & 5 & 8 & 12 & 8 & 4 & $w_{347}$ & N & can. \\
4201 & 0 & 0 & 0 & 0 & 0 & 1 & 2 & 3 & 0 & 2 & 4 & 6 & 0 & 5 & 8 & 13 & 8 & 3 & $w_{277}$ & N & can. \\
4202 & 0 & 0 & 0 & 0 & 0 & 1 & 2 & 3 & 0 & 2 & 4 & 6 & 0 & 5 & 8 & 14 & 4 & 4 & $w_{433}$ & N & \#4199 \\
4203 & 0 & 0 & 0 & 0 & 0 & 1 & 2 & 3 & 0 & 2 & 4 & 6 & 0 & 6 & 7 & 8 & 6 & 4 & $w_{343}$ & N & can. \\
4204 & 0 & 0 & 0 & 0 & 0 & 1 & 2 & 3 & 0 & 2 & 4 & 6 & 0 & 6 & 8 & 11 & 4 & 4 & $w_{347}$ & N & \#4192 \\
4205 & 0 & 0 & 0 & 0 & 0 & 1 & 2 & 3 & 0 & 2 & 4 & 6 & 0 & 6 & 8 & 12 & 4 & 4 & $w_{418}$ & N & can. \\
4206 & 0 & 0 & 0 & 0 & 0 & 1 & 2 & 3 & 0 & 2 & 4 & 6 & 0 & 6 & 8 & 13 & 4 & 4 & $w_{433}$ & N & \#4191 \\
4207 & 0 & 0 & 0 & 0 & 0 & 1 & 2 & 3 & 0 & 2 & 4 & 6 & 0 & 6 & 8 & 14 & 8 & 3 & $w_{345}$ & N & \#4189 \\
4208 & 0 & 0 & 0 & 0 & 0 & 1 & 2 & 3 & 0 & 2 & 4 & 6 & 0 & 6 & 8 & 15 & 4 & 4 & $w_{347}$ & N & can. \\
4209 & 0 & 0 & 0 & 0 & 0 & 1 & 2 & 3 & 0 & 2 & 4 & 6 & 0 & 8 & 9 & 13 & 2 & 4 & $w_{427}$ & N & can. \\
4210 & 0 & 0 & 0 & 0 & 0 & 1 & 2 & 3 & 0 & 2 & 4 & 6 & 0 & 8 & 9 & 14 & 4 & 4 & $w_{427}$ & N & can. \\
4211 & 0 & 0 & 0 & 0 & 0 & 1 & 2 & 3 & 0 & 2 & 4 & 6 & 0 & 8 & 9 & 15 & 6 & 4 & $w_{427}$ & N & can. \\
4212 & 0 & 0 & 0 & 0 & 0 & 1 & 2 & 3 & 0 & 2 & 4 & 6 & 0 & 8 & 13 & 14 & 12 & 4 & $w_{427}$ & N & \#4210 \\
4213 & 0 & 0 & 0 & 0 & 0 & 1 & 2 & 3 & 0 & 2 & 4 & 6 & 8 & 11 & 14 & 13 & 1536 & 2 & $w_{438}$ & N & can. \\
4214 & 0 & 0 & 0 & 0 & 0 & 1 & 2 & 3 & 0 & 2 & 4 & 6 & 8 & 11 & 15 & 12 & 512 & 3 & $w_{438}$ & N & \#4213 \\
4215 & 0 & 0 & 0 & 0 & 0 & 1 & 2 & 3 & 0 & 2 & 4 & 6 & 8 & 12 & 11 & 15 & 192 & 3 & $w_{438}$ & N & can. \\
4216 & 0 & 0 & 0 & 0 & 0 & 1 & 2 & 3 & 0 & 2 & 4 & 7 & 0 & 3 & 8 & 10 & 16 & 3 & $w_{345}$ & N & \#4189 \\
4217 & 0 & 0 & 0 & 0 & 0 & 1 & 2 & 3 & 0 & 2 & 4 & 7 & 0 & 3 & 8 & 12 & 4 & 4 & $w_{347}$ & N & can. \\
4218 & 0 & 0 & 0 & 0 & 0 & 1 & 2 & 3 & 0 & 2 & 4 & 7 & 0 & 3 & 8 & 14 & 4 & 4 & $w_{347}$ & N & can. \\
4219 & 0 & 0 & 0 & 0 & 0 & 1 & 2 & 3 & 0 & 2 & 4 & 7 & 0 & 4 & 1 & 8 & 2 & 4 & $w_{412}$ & N & can. \\
4220 & 0 & 0 & 0 & 0 & 0 & 1 & 2 & 3 & 0 & 2 & 4 & 7 & 0 & 4 & 8 & 10 & 2 & 4 & $w_{347}$ & N & can. \\
4221 & 0 & 0 & 0 & 0 & 0 & 1 & 2 & 3 & 0 & 2 & 4 & 7 & 0 & 4 & 8 & 11 & 2 & 4 & $w_{347}$ & N & can. \\
4222 & 0 & 0 & 0 & 0 & 0 & 1 & 2 & 3 & 0 & 2 & 4 & 7 & 0 & 4 & 8 & 12 & 2 & 4 & $w_{347}$ & N & can. \\
4223 & 0 & 0 & 0 & 0 & 0 & 1 & 2 & 3 & 0 & 2 & 4 & 7 & 0 & 4 & 8 & 13 & 2 & 3 & $w_{378}$ & N & can. \\
4224 & 0 & 0 & 0 & 0 & 0 & 1 & 2 & 3 & 0 & 2 & 4 & 7 & 0 & 4 & 8 & 14 & 1 & 4 & $w_{423}$ & N & can. \\
4225 & 0 & 0 & 0 & 0 & 0 & 1 & 2 & 3 & 0 & 2 & 4 & 7 & 0 & 8 & 1 & 11 & 8 & 4 & $w_{422}$ & N & can. \\
4226 & 0 & 0 & 0 & 0 & 0 & 1 & 2 & 3 & 0 & 2 & 4 & 7 & 0 & 8 & 1 & 12 & 2 & 4 & $w_{392}$ & N & can. \\
4227 & 0 & 0 & 0 & 0 & 0 & 1 & 2 & 3 & 0 & 2 & 4 & 7 & 0 & 8 & 1 & 14 & 2 & 4 & $w_{419}$ & N & can. \\
4228 & 0 & 0 & 0 & 0 & 0 & 1 & 2 & 3 & 0 & 2 & 4 & 7 & 0 & 8 & 3 & 9 & 8 & 4 & $w_{422}$ & N & can. \\
4229 & 0 & 0 & 0 & 0 & 0 & 1 & 2 & 3 & 0 & 2 & 4 & 7 & 0 & 8 & 3 & 12 & 2 & 4 & $w_{392}$ & N & can. \\
4230 & 0 & 0 & 0 & 0 & 0 & 1 & 2 & 3 & 0 & 2 & 4 & 7 & 0 & 8 & 3 & 14 & 2 & 4 & $w_{419}$ & N & can. \\
4231 & 0 & 0 & 0 & 0 & 0 & 1 & 2 & 3 & 0 & 2 & 4 & 7 & 0 & 8 & 6 & 9 & 1 & 4 & $w_{419}$ & N & can. \\
4232 & 0 & 0 & 0 & 0 & 0 & 1 & 2 & 3 & 0 & 2 & 4 & 7 & 0 & 8 & 6 & 12 & 2 & 4 & $w_{433}$ & N & can. \\
4233 & 0 & 0 & 0 & 0 & 0 & 1 & 2 & 3 & 0 & 2 & 4 & 7 & 0 & 8 & 6 & 13 & 2 & 4 & $w_{433}$ & N & \#4232 \\
4234 & 0 & 0 & 0 & 0 & 0 & 1 & 2 & 3 & 0 & 2 & 4 & 7 & 0 & 8 & 6 & 14 & 2 & 4 & $w_{347}$ & N & can. \\
4235 & 0 & 0 & 0 & 0 & 0 & 1 & 2 & 3 & 0 & 2 & 4 & 7 & 0 & 8 & 6 & 15 & 2 & 3 & $w_{378}$ & N & can. \\
4236 & 0 & 0 & 0 & 0 & 0 & 1 & 2 & 3 & 0 & 2 & 4 & 7 & 0 & 8 & 9 & 12 & 1 & 4 & $w_{428}$ & N & can. \\
4237 & 0 & 0 & 0 & 0 & 0 & 1 & 2 & 3 & 0 & 2 & 4 & 7 & 0 & 8 & 9 & 14 & 1 & 4 & $w_{429}$ & N & can. \\
4238 & 0 & 0 & 0 & 0 & 0 & 1 & 2 & 3 & 0 & 2 & 4 & 7 & 0 & 8 & 10 & 12 & 1 & 4 & $w_{428}$ & N & can. \\
4239 & 0 & 0 & 0 & 0 & 0 & 1 & 2 & 3 & 0 & 2 & 4 & 7 & 0 & 8 & 10 & 14 & 1 & 4 & $w_{429}$ & N & can. \\
4240 & 0 & 0 & 0 & 0 & 0 & 1 & 2 & 3 & 0 & 2 & 4 & 7 & 0 & 8 & 12 & 14 & 2 & 4 & $w_{427}$ & N & \#4210 \\
4241 & 0 & 0 & 0 & 0 & 0 & 1 & 2 & 3 & 0 & 2 & 4 & 7 & 0 & 8 & 12 & 15 & 2 & 4 & $w_{427}$ & N & can. \\
4242 & 0 & 0 & 0 & 0 & 0 & 1 & 2 & 3 & 0 & 2 & 4 & 7 & 8 & 10 & 14 & 13 & 128 & 3 & $w_{438}$ & N & \#4213 \\
4243 & 0 & 0 & 0 & 0 & 0 & 1 & 2 & 3 & 0 & 2 & 4 & 7 & 8 & 12 & 9 & 15 & 16 & 4 & $w_{473}$ & N & can. \\
4244 & 0 & 0 & 0 & 0 & 0 & 1 & 2 & 3 & 0 & 2 & 4 & 8 & 0 & 4 & 1 & 9 & 1 & 4 & $w_{409}$ & N & can. \\
4245 & 0 & 0 & 0 & 0 & 0 & 1 & 2 & 3 & 0 & 2 & 4 & 8 & 0 & 4 & 1 & 11 & 12 & 3 & $w_{453}$ & N & can. \\
4246 & 0 & 0 & 0 & 0 & 0 & 1 & 2 & 3 & 0 & 2 & 4 & 8 & 0 & 4 & 1 & 12 & 4 & 4 & $w_{464}$ & N & can. \\
4247 & 0 & 0 & 0 & 0 & 0 & 1 & 2 & 3 & 0 & 2 & 4 & 8 & 0 & 4 & 1 & 15 & 4 & 4 & $w_{347}$ & N & can. \\
4248 & 0 & 0 & 0 & 0 & 0 & 1 & 2 & 3 & 0 & 2 & 4 & 8 & 0 & 4 & 3 & 9 & 2 & 3 & $w_{345}$ & N & can. \\
4249 & 0 & 0 & 0 & 0 & 0 & 1 & 2 & 3 & 0 & 2 & 4 & 8 & 0 & 4 & 3 & 10 & 2 & 4 & $w_{351}$ & N & can. \\
4250 & 0 & 0 & 0 & 0 & 0 & 1 & 2 & 3 & 0 & 2 & 4 & 8 & 0 & 4 & 3 & 11 & 1 & 4 & $w_{409}$ & N & can. \\
4251 & 0 & 0 & 0 & 0 & 0 & 1 & 2 & 3 & 0 & 2 & 4 & 8 & 0 & 4 & 3 & 12 & 1 & 4 & $w_{347}$ & N & can. \\
4252 & 0 & 0 & 0 & 0 & 0 & 1 & 2 & 3 & 0 & 2 & 4 & 8 & 0 & 4 & 3 & 13 & 1 & 4 & $w_{392}$ & N & can. \\
4253 & 0 & 0 & 0 & 0 & 0 & 1 & 2 & 3 & 0 & 2 & 4 & 8 & 0 & 4 & 3 & 14 & 1 & 4 & $w_{353}$ & N & can. \\
4254 & 0 & 0 & 0 & 0 & 0 & 1 & 2 & 3 & 0 & 2 & 4 & 8 & 0 & 4 & 3 & 15 & 1 & 4 & $w_{266}$ & N & can. \\
4255 & 0 & 0 & 0 & 0 & 0 & 1 & 2 & 3 & 0 & 2 & 4 & 8 & 0 & 4 & 5 & 10 & 1 & 4 & $w_{392}$ & N & can. \\
4256 & 0 & 0 & 0 & 0 & 0 & 1 & 2 & 3 & 0 & 2 & 4 & 8 & 0 & 4 & 5 & 11 & 1 & 4 & $w_{347}$ & N & can. \\
4257 & 0 & 0 & 0 & 0 & 0 & 1 & 2 & 3 & 0 & 2 & 4 & 8 & 0 & 4 & 5 & 12 & 1 & 4 & $w_{419}$ & N & can. \\
4258 & 0 & 0 & 0 & 0 & 0 & 1 & 2 & 3 & 0 & 2 & 4 & 8 & 0 & 4 & 5 & 13 & 1 & 4 & $w_{423}$ & N & \#4224 \\
4259 & 0 & 0 & 0 & 0 & 0 & 1 & 2 & 3 & 0 & 2 & 4 & 8 & 0 & 4 & 5 & 14 & 1 & 4 & $w_{392}$ & N & can. \\
4260 & 0 & 0 & 0 & 0 & 0 & 1 & 2 & 3 & 0 & 2 & 4 & 8 & 0 & 4 & 5 & 15 & 1 & 3 & $w_{378}$ & N & \#4223 \\
4261 & 0 & 0 & 0 & 0 & 0 & 1 & 2 & 3 & 0 & 2 & 4 & 8 & 0 & 4 & 8 & 11 & 12 & 4 & $w_{464}$ & N & can. \\
4262 & 0 & 0 & 0 & 0 & 0 & 1 & 2 & 3 & 0 & 2 & 4 & 8 & 0 & 4 & 8 & 13 & 1 & 4 & $w_{355}$ & N & can. \\
4263 & 0 & 0 & 0 & 0 & 0 & 1 & 2 & 3 & 0 & 2 & 4 & 8 & 0 & 4 & 9 & 10 & 2 & 4 & $w_{347}$ & N & can. \\
4264 & 0 & 0 & 0 & 0 & 0 & 1 & 2 & 3 & 0 & 2 & 4 & 8 & 0 & 4 & 9 & 12 & 1 & 4 & $w_{355}$ & N & can. \\
4265 & 0 & 0 & 0 & 0 & 0 & 1 & 2 & 3 & 0 & 2 & 4 & 8 & 0 & 4 & 9 & 14 & 1 & 4 & $w_{392}$ & N & can. \\
4266 & 0 & 0 & 0 & 0 & 0 & 1 & 2 & 3 & 0 & 2 & 4 & 8 & 0 & 4 & 9 & 15 & 1 & 4 & $w_{379}$ & N & can. \\
4267 & 0 & 0 & 0 & 0 & 0 & 1 & 2 & 3 & 0 & 2 & 4 & 8 & 0 & 4 & 11 & 9 & 2 & 4 & $w_{266}$ & N & can. \\
4268 & 0 & 0 & 0 & 0 & 0 & 1 & 2 & 3 & 0 & 2 & 4 & 8 & 0 & 4 & 11 & 12 & 1 & 4 & $w_{355}$ & N & can. \\
4269 & 0 & 0 & 0 & 0 & 0 & 1 & 2 & 3 & 0 & 2 & 4 & 8 & 0 & 4 & 11 & 13 & 1 & 4 & $w_{379}$ & N & can. \\
4270 & 0 & 0 & 0 & 0 & 0 & 1 & 2 & 3 & 0 & 2 & 4 & 8 & 0 & 4 & 11 & 14 & 1 & 4 & $w_{281}$ & N & can. \\
4271 & 0 & 0 & 0 & 0 & 0 & 1 & 2 & 3 & 0 & 2 & 4 & 8 & 0 & 4 & 11 & 15 & 3 & 4 & $w_{354}$ & N & can. \\
4272 & 0 & 0 & 0 & 0 & 0 & 1 & 2 & 3 & 0 & 2 & 4 & 8 & 0 & 4 & 12 & 15 & 1 & 4 & $w_{355}$ & N & can. \\
4273 & 0 & 0 & 0 & 0 & 0 & 1 & 2 & 3 & 0 & 2 & 4 & 8 & 0 & 4 & 13 & 15 & 2 & 4 & $w_{379}$ & N & can. \\
4274 & 0 & 0 & 0 & 0 & 0 & 1 & 2 & 3 & 0 & 2 & 4 & 8 & 0 & 4 & 14 & 12 & 1 & 4 & $w_{419}$ & N & can. \\
4275 & 0 & 0 & 0 & 0 & 0 & 1 & 2 & 3 & 0 & 2 & 4 & 8 & 0 & 4 & 14 & 13 & 2 & 4 & $w_{427}$ & N & can. \\
4276 & 0 & 0 & 0 & 0 & 0 & 1 & 2 & 3 & 0 & 2 & 4 & 8 & 0 & 4 & 15 & 12 & 2 & 4 & $w_{474}$ & N & can. \\
4277 & 0 & 0 & 0 & 0 & 0 & 1 & 2 & 3 & 0 & 2 & 4 & 8 & 0 & 5 & 1 & 9 & 2 & 4 & $w_{351}$ & N & can. \\
4278 & 0 & 0 & 0 & 0 & 0 & 1 & 2 & 3 & 0 & 2 & 4 & 8 & 0 & 5 & 1 & 10 & 2 & 3 & $w_{345}$ & N & can. \\
4279 & 0 & 0 & 0 & 0 & 0 & 1 & 2 & 3 & 0 & 2 & 4 & 8 & 0 & 5 & 1 & 12 & 1 & 4 & $w_{347}$ & N & can. \\
4280 & 0 & 0 & 0 & 0 & 0 & 1 & 2 & 3 & 0 & 2 & 4 & 8 & 0 & 5 & 1 & 13 & 1 & 4 & $w_{353}$ & N & can. \\
4281 & 0 & 0 & 0 & 0 & 0 & 1 & 2 & 3 & 0 & 2 & 4 & 8 & 0 & 5 & 1 & 14 & 1 & 4 & $w_{392}$ & N & can. \\
4282 & 0 & 0 & 0 & 0 & 0 & 1 & 2 & 3 & 0 & 2 & 4 & 8 & 0 & 5 & 1 & 15 & 1 & 4 & $w_{266}$ & N & can. \\
4283 & 0 & 0 & 0 & 0 & 0 & 1 & 2 & 3 & 0 & 2 & 4 & 8 & 0 & 5 & 7 & 9 & 1 & 4 & $w_{392}$ & N & can. \\
4284 & 0 & 0 & 0 & 0 & 0 & 1 & 2 & 3 & 0 & 2 & 4 & 8 & 0 & 5 & 7 & 10 & 1 & 4 & $w_{347}$ & N & can. \\
4285 & 0 & 0 & 0 & 0 & 0 & 1 & 2 & 3 & 0 & 2 & 4 & 8 & 0 & 5 & 7 & 12 & 1 & 3 & $w_{378}$ & N & can. \\
4286 & 0 & 0 & 0 & 0 & 0 & 1 & 2 & 3 & 0 & 2 & 4 & 8 & 0 & 5 & 7 & 13 & 1 & 4 & $w_{419}$ & N & can. \\
4287 & 0 & 0 & 0 & 0 & 0 & 1 & 2 & 3 & 0 & 2 & 4 & 8 & 0 & 5 & 7 & 14 & 1 & 4 & $w_{423}$ & N & can. \\
4288 & 0 & 0 & 0 & 0 & 0 & 1 & 2 & 3 & 0 & 2 & 4 & 8 & 0 & 5 & 7 & 15 & 1 & 4 & $w_{392}$ & N & can. \\
4289 & 0 & 0 & 0 & 0 & 0 & 1 & 2 & 3 & 0 & 2 & 4 & 8 & 0 & 5 & 8 & 10 & 1 & 4 & $w_{353}$ & N & can. \\
4290 & 0 & 0 & 0 & 0 & 0 & 1 & 2 & 3 & 0 & 2 & 4 & 8 & 0 & 5 & 8 & 11 & 4 & 4 & $w_{433}$ & N & \#4232 \\
4291 & 0 & 0 & 0 & 0 & 0 & 1 & 2 & 3 & 0 & 2 & 4 & 8 & 0 & 5 & 8 & 12 & 1 & 4 & $w_{355}$ & N & can. \\
4292 & 0 & 0 & 0 & 0 & 0 & 1 & 2 & 3 & 0 & 2 & 4 & 8 & 0 & 5 & 8 & 13 & 1 & 4 & $w_{278}$ & N & can. \\
4293 & 0 & 0 & 0 & 0 & 0 & 1 & 2 & 3 & 0 & 2 & 4 & 8 & 0 & 5 & 8 & 14 & 1 & 4 & $w_{432}$ & N & can. \\
4294 & 0 & 0 & 0 & 0 & 0 & 1 & 2 & 3 & 0 & 2 & 4 & 8 & 0 & 5 & 8 & 15 & 1 & 4 & $w_{278}$ & N & can. \\
4295 & 0 & 0 & 0 & 0 & 0 & 1 & 2 & 3 & 0 & 2 & 4 & 8 & 0 & 5 & 9 & 12 & 1 & 4 & $w_{354}$ & N & can. \\
4296 & 0 & 0 & 0 & 0 & 0 & 1 & 2 & 3 & 0 & 2 & 4 & 8 & 0 & 5 & 9 & 13 & 1 & 4 & $w_{355}$ & N & can. \\
4297 & 0 & 0 & 0 & 0 & 0 & 1 & 2 & 3 & 0 & 2 & 4 & 8 & 0 & 5 & 9 & 14 & 1 & 4 & $w_{379}$ & N & can. \\
4298 & 0 & 0 & 0 & 0 & 0 & 1 & 2 & 3 & 0 & 2 & 4 & 8 & 0 & 5 & 9 & 15 & 1 & 4 & $w_{281}$ & N & can. \\
4299 & 0 & 0 & 0 & 0 & 0 & 1 & 2 & 3 & 0 & 2 & 4 & 8 & 0 & 5 & 11 & 12 & 1 & 4 & $w_{379}$ & N & can. \\
4300 & 0 & 0 & 0 & 0 & 0 & 1 & 2 & 3 & 0 & 2 & 4 & 8 & 0 & 5 & 11 & 13 & 1 & 4 & $w_{432}$ & N & can. \\
4301 & 0 & 0 & 0 & 0 & 0 & 1 & 2 & 3 & 0 & 2 & 4 & 8 & 0 & 5 & 11 & 14 & 2 & 4 & $w_{278}$ & N & can. \\
4302 & 0 & 0 & 0 & 0 & 0 & 1 & 2 & 3 & 0 & 2 & 4 & 8 & 0 & 5 & 11 & 15 & 1 & 4 & $w_{281}$ & N & can. \\
4303 & 0 & 0 & 0 & 0 & 0 & 1 & 2 & 3 & 0 & 2 & 4 & 8 & 0 & 5 & 12 & 14 & 2 & 4 & $w_{379}$ & N & can. \\
4304 & 0 & 0 & 0 & 0 & 0 & 1 & 2 & 3 & 0 & 2 & 4 & 8 & 0 & 5 & 12 & 15 & 1 & 4 & $w_{432}$ & N & can. \\
4305 & 0 & 0 & 0 & 0 & 0 & 1 & 2 & 3 & 0 & 2 & 4 & 8 & 0 & 5 & 14 & 12 & 1 & 4 & $w_{355}$ & N & can. \\
4306 & 0 & 0 & 0 & 0 & 0 & 1 & 2 & 3 & 0 & 2 & 4 & 8 & 0 & 5 & 14 & 13 & 2 & 4 & $w_{434}$ & N & can. \\
4307 & 0 & 0 & 0 & 0 & 0 & 1 & 2 & 3 & 0 & 2 & 4 & 8 & 0 & 5 & 15 & 12 & 2 & 4 & $w_{427}$ & N & can. \\
4308 & 0 & 0 & 0 & 0 & 0 & 1 & 2 & 3 & 0 & 2 & 4 & 8 & 0 & 6 & 3 & 11 & 4 & 3 & $w_{345}$ & N & \#4189 \\
4309 & 0 & 0 & 0 & 0 & 0 & 1 & 2 & 3 & 0 & 2 & 4 & 8 & 0 & 6 & 3 & 13 & 2 & 4 & $w_{433}$ & N & \#4191 \\
4310 & 0 & 0 & 0 & 0 & 0 & 1 & 2 & 3 & 0 & 2 & 4 & 8 & 0 & 6 & 3 & 14 & 2 & 4 & $w_{347}$ & N & \#4192 \\
4311 & 0 & 0 & 0 & 0 & 0 & 1 & 2 & 3 & 0 & 2 & 4 & 8 & 0 & 6 & 8 & 11 & 4 & 4 & $w_{347}$ & N & \#4234 \\
4312 & 0 & 0 & 0 & 0 & 0 & 1 & 2 & 3 & 0 & 2 & 4 & 8 & 0 & 6 & 8 & 13 & 2 & 4 & $w_{432}$ & N & can. \\
4313 & 0 & 0 & 0 & 0 & 0 & 1 & 2 & 3 & 0 & 2 & 4 & 8 & 0 & 6 & 8 & 15 & 2 & 4 & $w_{355}$ & N & can. \\
4314 & 0 & 0 & 0 & 0 & 0 & 1 & 2 & 3 & 0 & 2 & 4 & 8 & 0 & 6 & 9 & 10 & 2 & 4 & $w_{347}$ & N & can. \\
4315 & 0 & 0 & 0 & 0 & 0 & 1 & 2 & 3 & 0 & 2 & 4 & 8 & 0 & 6 & 9 & 12 & 1 & 4 & $w_{281}$ & N & can. \\
4316 & 0 & 0 & 0 & 0 & 0 & 1 & 2 & 3 & 0 & 2 & 4 & 8 & 0 & 6 & 9 & 13 & 1 & 4 & $w_{379}$ & N & can. \\
4317 & 0 & 0 & 0 & 0 & 0 & 1 & 2 & 3 & 0 & 2 & 4 & 8 & 0 & 6 & 9 & 14 & 1 & 4 & $w_{355}$ & N & can. \\
4318 & 0 & 0 & 0 & 0 & 0 & 1 & 2 & 3 & 0 & 2 & 4 & 8 & 0 & 6 & 11 & 9 & 2 & 4 & $w_{266}$ & N & can. \\
4319 & 0 & 0 & 0 & 0 & 0 & 1 & 2 & 3 & 0 & 2 & 4 & 8 & 0 & 6 & 11 & 12 & 1 & 4 & $w_{281}$ & N & can. \\
4320 & 0 & 0 & 0 & 0 & 0 & 1 & 2 & 3 & 0 & 2 & 4 & 8 & 0 & 6 & 11 & 13 & 1 & 4 & $w_{278}$ & N & \#4292 \\
4321 & 0 & 0 & 0 & 0 & 0 & 1 & 2 & 3 & 0 & 2 & 4 & 8 & 0 & 6 & 11 & 14 & 1 & 4 & $w_{432}$ & N & \#4293 \\
4322 & 0 & 0 & 0 & 0 & 0 & 1 & 2 & 3 & 0 & 2 & 4 & 8 & 0 & 6 & 11 & 15 & 1 & 4 & $w_{379}$ & N & can. \\
4323 & 0 & 0 & 0 & 0 & 0 & 1 & 2 & 3 & 0 & 2 & 4 & 8 & 0 & 6 & 12 & 9 & 1 & 4 & $w_{379}$ & N & \#4297 \\
4324 & 0 & 0 & 0 & 0 & 0 & 1 & 2 & 3 & 0 & 2 & 4 & 8 & 0 & 6 & 12 & 11 & 1 & 4 & $w_{355}$ & N & can. \\
4325 & 0 & 0 & 0 & 0 & 0 & 1 & 2 & 3 & 0 & 2 & 4 & 8 & 0 & 6 & 12 & 14 & 1 & 4 & $w_{419}$ & N & can. \\
4326 & 0 & 0 & 0 & 0 & 0 & 1 & 2 & 3 & 0 & 2 & 4 & 8 & 0 & 6 & 12 & 15 & 1 & 4 & $w_{427}$ & N & can. \\
4327 & 0 & 0 & 0 & 0 & 0 & 1 & 2 & 3 & 0 & 2 & 4 & 8 & 0 & 6 & 13 & 9 & 1 & 4 & $w_{355}$ & N & \#4296 \\
4328 & 0 & 0 & 0 & 0 & 0 & 1 & 2 & 3 & 0 & 2 & 4 & 8 & 0 & 6 & 13 & 10 & 1 & 4 & $w_{355}$ & N & can. \\
4329 & 0 & 0 & 0 & 0 & 0 & 1 & 2 & 3 & 0 & 2 & 4 & 8 & 0 & 6 & 13 & 11 & 2 & 4 & $w_{354}$ & N & can. \\
4330 & 0 & 0 & 0 & 0 & 0 & 1 & 2 & 3 & 0 & 2 & 4 & 8 & 0 & 6 & 13 & 14 & 1 & 4 & $w_{427}$ & N & can. \\
4331 & 0 & 0 & 0 & 0 & 0 & 1 & 2 & 3 & 0 & 2 & 4 & 8 & 0 & 6 & 13 & 15 & 1 & 4 & $w_{355}$ & N & can. \\
4332 & 0 & 0 & 0 & 0 & 0 & 1 & 2 & 3 & 0 & 2 & 4 & 8 & 0 & 6 & 14 & 9 & 1 & 4 & $w_{281}$ & N & \#4298 \\
4333 & 0 & 0 & 0 & 0 & 0 & 1 & 2 & 3 & 0 & 2 & 4 & 8 & 0 & 6 & 14 & 10 & 1 & 4 & $w_{392}$ & N & can. \\
4334 & 0 & 0 & 0 & 0 & 0 & 1 & 2 & 3 & 0 & 2 & 4 & 8 & 0 & 6 & 14 & 11 & 1 & 4 & $w_{278}$ & N & \#4292 \\
4335 & 0 & 0 & 0 & 0 & 0 & 1 & 2 & 3 & 0 & 2 & 4 & 8 & 0 & 6 & 14 & 13 & 1 & 4 & $w_{432}$ & N & \#4293 \\
4336 & 0 & 0 & 0 & 0 & 0 & 1 & 2 & 3 & 0 & 2 & 4 & 8 & 0 & 6 & 15 & 10 & 1 & 4 & $w_{379}$ & N & can. \\
4337 & 0 & 0 & 0 & 0 & 0 & 1 & 2 & 3 & 0 & 2 & 4 & 8 & 0 & 6 & 15 & 11 & 1 & 4 & $w_{355}$ & N & can. \\
4338 & 0 & 0 & 0 & 0 & 0 & 1 & 2 & 3 & 0 & 2 & 4 & 8 & 0 & 6 & 15 & 12 & 1 & 4 & $w_{355}$ & N & can. \\
4339 & 0 & 0 & 0 & 0 & 0 & 1 & 2 & 3 & 0 & 2 & 4 & 8 & 0 & 6 & 15 & 13 & 1 & 4 & $w_{379}$ & N & can. \\
4340 & 0 & 0 & 0 & 0 & 0 & 1 & 2 & 3 & 0 & 2 & 4 & 8 & 0 & 12 & 5 & 7 & 4 & 3 & $w_{277}$ & Y & \#4201 \\
4341 & 0 & 0 & 0 & 0 & 0 & 1 & 2 & 3 & 0 & 2 & 4 & 8 & 0 & 12 & 5 & 10 & 1 & 4 & $w_{278}$ & N & \#4292 \\
4342 & 0 & 0 & 0 & 0 & 0 & 1 & 2 & 3 & 0 & 2 & 4 & 8 & 0 & 12 & 5 & 13 & 1 & 4 & $w_{379}$ & N & can. \\
4343 & 0 & 0 & 0 & 0 & 0 & 1 & 2 & 3 & 0 & 2 & 4 & 8 & 0 & 12 & 5 & 15 & 1 & 4 & $w_{379}$ & N & can. \\
4344 & 0 & 0 & 0 & 0 & 0 & 1 & 2 & 3 & 0 & 2 & 4 & 8 & 0 & 12 & 7 & 4 & 4 & 4 & $w_{347}$ & N & \#4234 \\
4345 & 0 & 0 & 0 & 0 & 0 & 1 & 2 & 3 & 0 & 2 & 4 & 8 & 0 & 12 & 7 & 10 & 4 & 4 & $w_{354}$ & N & \#4329 \\
4346 & 0 & 0 & 0 & 0 & 0 & 1 & 2 & 3 & 0 & 2 & 4 & 8 & 0 & 12 & 7 & 13 & 1 & 4 & $w_{419}$ & N & can. \\
4347 & 0 & 0 & 0 & 0 & 0 & 1 & 2 & 3 & 0 & 2 & 4 & 8 & 0 & 12 & 7 & 15 & 1 & 4 & $w_{355}$ & N & can. \\
4348 & 0 & 0 & 0 & 0 & 0 & 1 & 2 & 3 & 0 & 2 & 4 & 8 & 0 & 13 & 5 & 10 & 1 & 4 & $w_{278}$ & N & \#4294 \\
4349 & 0 & 0 & 0 & 0 & 0 & 1 & 2 & 3 & 0 & 2 & 4 & 8 & 0 & 13 & 5 & 11 & 2 & 4 & $w_{278}$ & N & \#4294 \\
4350 & 0 & 0 & 0 & 0 & 0 & 1 & 2 & 3 & 0 & 2 & 4 & 8 & 0 & 13 & 5 & 12 & 1 & 4 & $w_{355}$ & N & can. \\
4351 & 0 & 0 & 0 & 0 & 0 & 1 & 2 & 3 & 0 & 2 & 4 & 8 & 0 & 13 & 5 & 14 & 1 & 4 & $w_{432}$ & N & \#4304 \\
4352 & 0 & 0 & 0 & 0 & 0 & 1 & 2 & 3 & 0 & 2 & 4 & 8 & 0 & 13 & 6 & 5 & 12 & 3 & $w_{378}$ & N & \#4235 \\
4353 & 0 & 0 & 0 & 0 & 0 & 1 & 2 & 3 & 0 & 2 & 4 & 8 & 0 & 13 & 6 & 11 & 12 & 4 & $w_{354}$ & N & \#4329 \\
4354 & 0 & 0 & 0 & 0 & 0 & 1 & 2 & 3 & 0 & 2 & 4 & 8 & 0 & 13 & 6 & 12 & 2 & 4 & $w_{474}$ & N & can. \\
4355 & 0 & 0 & 0 & 0 & 0 & 1 & 2 & 3 & 0 & 2 & 4 & 8 & 0 & 13 & 6 & 14 & 2 & 4 & $w_{427}$ & N & \#4330 \\
4356 & 0 & 0 & 0 & 0 & 0 & 1 & 2 & 3 & 0 & 2 & 4 & 8 & 0 & 13 & 7 & 4 & 4 & 3 & $w_{378}$ & N & \#4235 \\
4357 & 0 & 0 & 0 & 0 & 0 & 1 & 2 & 3 & 0 & 2 & 4 & 8 & 0 & 13 & 7 & 12 & 1 & 4 & $w_{427}$ & N & can. \\
4358 & 0 & 0 & 0 & 0 & 0 & 1 & 2 & 3 & 0 & 2 & 4 & 8 & 0 & 13 & 7 & 14 & 1 & 4 & $w_{427}$ & N & can. \\
4359 & 0 & 0 & 0 & 0 & 0 & 1 & 2 & 3 & 0 & 2 & 4 & 8 & 0 & 14 & 5 & 7 & 2 & 4 & $w_{433}$ & N & \#4199 \\
4360 & 0 & 0 & 0 & 0 & 0 & 1 & 2 & 3 & 0 & 2 & 4 & 8 & 0 & 14 & 5 & 10 & 1 & 4 & $w_{278}$ & N & \#4294 \\
4361 & 0 & 0 & 0 & 0 & 0 & 1 & 2 & 3 & 0 & 2 & 4 & 8 & 0 & 14 & 5 & 13 & 1 & 4 & $w_{434}$ & N & \#4306 \\
4362 & 0 & 0 & 0 & 0 & 0 & 1 & 2 & 3 & 0 & 2 & 4 & 8 & 0 & 14 & 5 & 15 & 1 & 4 & $w_{427}$ & N & can. \\
4363 & 0 & 0 & 0 & 0 & 0 & 1 & 2 & 3 & 0 & 2 & 4 & 8 & 0 & 14 & 7 & 4 & 2 & 4 & $w_{433}$ & N & \#4232 \\
4364 & 0 & 0 & 0 & 0 & 0 & 1 & 2 & 3 & 0 & 2 & 4 & 8 & 0 & 14 & 7 & 10 & 1 & 4 & $w_{278}$ & N & \#4292 \\
4365 & 0 & 0 & 0 & 0 & 0 & 1 & 2 & 3 & 0 & 2 & 4 & 8 & 0 & 14 & 7 & 13 & 1 & 4 & $w_{428}$ & N & can. \\
4366 & 0 & 0 & 0 & 0 & 0 & 1 & 2 & 3 & 0 & 2 & 4 & 8 & 0 & 14 & 7 & 15 & 1 & 4 & $w_{428}$ & N & can. \\
4367 & 0 & 0 & 0 & 0 & 0 & 1 & 2 & 3 & 0 & 2 & 4 & 8 & 0 & 14 & 12 & 4 & 1 & 4 & $w_{423}$ & N & can. \\
4368 & 0 & 0 & 0 & 0 & 0 & 1 & 2 & 3 & 0 & 2 & 4 & 8 & 0 & 14 & 12 & 5 & 1 & 4 & $w_{427}$ & N & can. \\
4369 & 0 & 0 & 0 & 0 & 0 & 1 & 2 & 3 & 0 & 2 & 4 & 8 & 0 & 14 & 12 & 7 & 1 & 4 & $w_{428}$ & N & can. \\
4370 & 0 & 0 & 0 & 0 & 0 & 1 & 2 & 3 & 0 & 2 & 4 & 8 & 0 & 14 & 15 & 4 & 1 & 4 & $w_{379}$ & N & can. \\
4371 & 0 & 0 & 0 & 0 & 0 & 1 & 2 & 3 & 0 & 2 & 4 & 8 & 0 & 14 & 15 & 7 & 1 & 4 & $w_{428}$ & N & can. \\
4372 & 0 & 0 & 0 & 0 & 0 & 1 & 2 & 3 & 0 & 2 & 4 & 8 & 4 & 9 & 12 & 14 & 8 & 4 & $w_{427}$ & N & can. \\
4373 & 0 & 0 & 0 & 0 & 0 & 1 & 2 & 3 & 0 & 2 & 4 & 8 & 4 & 9 & 12 & 15 & 16 & 3 & $w_{438}$ & N & can. \\
4374 & 0 & 0 & 0 & 0 & 0 & 1 & 2 & 3 & 0 & 2 & 4 & 8 & 4 & 9 & 13 & 14 & 16 & 3 & $w_{438}$ & N & \#4373 \\
4375 & 0 & 0 & 0 & 0 & 0 & 1 & 2 & 3 & 0 & 2 & 4 & 8 & 4 & 9 & 14 & 13 & 8 & 3 & $w_{377}$ & N & can. \\
4376 & 0 & 0 & 0 & 0 & 0 & 1 & 2 & 3 & 0 & 2 & 4 & 8 & 4 & 10 & 12 & 15 & 2 & 4 & $w_{434}$ & N & can. \\
4377 & 0 & 0 & 0 & 0 & 0 & 1 & 2 & 3 & 0 & 2 & 4 & 8 & 4 & 10 & 13 & 14 & 2 & 4 & $w_{434}$ & N & \#4376 \\
4378 & 0 & 0 & 0 & 0 & 0 & 1 & 2 & 3 & 0 & 2 & 4 & 8 & 4 & 10 & 13 & 15 & 2 & 4 & $w_{434}$ & N & can. \\
4379 & 0 & 0 & 0 & 0 & 0 & 1 & 2 & 3 & 0 & 2 & 4 & 8 & 4 & 12 & 8 & 13 & 2 & 4 & $w_{430}$ & N & can. \\
4380 & 0 & 0 & 0 & 0 & 0 & 1 & 2 & 3 & 0 & 2 & 4 & 8 & 4 & 12 & 8 & 15 & 1 & 4 & $w_{429}$ & N & can. \\
4381 & 0 & 0 & 0 & 0 & 0 & 1 & 2 & 3 & 0 & 2 & 4 & 8 & 4 & 12 & 9 & 13 & 1 & 4 & $w_{429}$ & N & can. \\
4382 & 0 & 0 & 0 & 0 & 0 & 1 & 2 & 3 & 0 & 2 & 4 & 8 & 4 & 12 & 9 & 15 & 4 & 3 & $w_{438}$ & N & \#4373 \\
4383 & 0 & 0 & 0 & 0 & 0 & 1 & 2 & 3 & 0 & 2 & 4 & 8 & 4 & 12 & 10 & 13 & 1 & 4 & $w_{428}$ & N & can. \\
4384 & 0 & 0 & 0 & 0 & 0 & 1 & 2 & 3 & 0 & 2 & 4 & 8 & 4 & 12 & 10 & 15 & 2 & 4 & $w_{434}$ & N & \#4376 \\
4385 & 0 & 0 & 0 & 0 & 0 & 1 & 2 & 3 & 0 & 2 & 4 & 8 & 4 & 12 & 11 & 13 & 2 & 3 & $w_{438}$ & N & can. \\
4386 & 0 & 0 & 0 & 0 & 0 & 1 & 2 & 3 & 0 & 2 & 4 & 8 & 4 & 12 & 13 & 8 & 1 & 4 & $w_{428}$ & N & can. \\
4387 & 0 & 0 & 0 & 0 & 0 & 1 & 2 & 3 & 0 & 2 & 4 & 8 & 4 & 12 & 13 & 9 & 2 & 4 & $w_{430}$ & N & can. \\
4388 & 0 & 0 & 0 & 0 & 0 & 1 & 2 & 3 & 0 & 2 & 4 & 8 & 4 & 12 & 13 & 11 & 2 & 3 & $w_{438}$ & N & can. \\
4389 & 0 & 0 & 0 & 0 & 0 & 1 & 2 & 3 & 0 & 2 & 4 & 8 & 4 & 12 & 14 & 10 & 2 & 4 & $w_{426}$ & N & can. \\
4390 & 0 & 0 & 0 & 0 & 0 & 1 & 2 & 3 & 0 & 2 & 4 & 8 & 4 & 14 & 11 & 15 & 4 & 3 & $w_{468}$ & N & can. \\
4391 & 0 & 0 & 0 & 0 & 0 & 1 & 2 & 3 & 0 & 2 & 4 & 8 & 4 & 15 & 9 & 12 & 8 & 3 & $w_{377}$ & N & \#4375 \\
4392 & 0 & 0 & 0 & 0 & 0 & 1 & 2 & 3 & 0 & 4 & 8 & 12 & 0 & 5 & 10 & 15 & 1152 & 2 & $w_{376}$ & N & can. \\
4393 & 0 & 0 & 0 & 0 & 0 & 1 & 2 & 3 & 0 & 4 & 8 & 12 & 0 & 5 & 11 & 14 & 64 & 3 & $w_{376}$ & N & \#4392 \\
4394 & 0 & 0 & 0 & 0 & 0 & 1 & 2 & 3 & 0 & 4 & 8 & 12 & 0 & 6 & 11 & 13 & 96 & 3 & $w_{376}$ & N & \#4392 \\
4395 & 0 & 0 & 0 & 0 & 0 & 1 & 2 & 3 & 0 & 4 & 8 & 13 & 0 & 5 & 11 & 15 & 128 & 3 & $w_{376}$ & N & \#4392 \\
4396 & 0 & 0 & 0 & 0 & 0 & 1 & 2 & 3 & 0 & 4 & 8 & 13 & 0 & 5 & 14 & 10 & 32 & 3 & $w_{376}$ & Y & \#4392 \\
4397 & 0 & 0 & 0 & 0 & 0 & 1 & 2 & 3 & 0 & 4 & 8 & 13 & 0 & 6 & 11 & 12 & 32 & 3 & $w_{376}$ & Y & \#4392 \\
4398 & 0 & 0 & 0 & 0 & 0 & 1 & 2 & 3 & 0 & 4 & 8 & 13 & 0 & 6 & 12 & 11 & 4 & 3 & $w_{377}$ & Y & can. \\
4399 & 0 & 0 & 0 & 0 & 0 & 1 & 2 & 3 & 0 & 4 & 8 & 13 & 0 & 8 & 5 & 15 & 4 & 4 & $w_{278}$ & N & \#4301 \\
4400 & 0 & 0 & 0 & 0 & 0 & 1 & 2 & 3 & 0 & 4 & 8 & 13 & 0 & 8 & 6 & 15 & 4 & 3 & $w_{377}$ & N & \#4398 \\
4401 & 0 & 0 & 0 & 0 & 0 & 1 & 2 & 3 & 0 & 4 & 8 & 13 & 0 & 8 & 7 & 14 & 24 & 3 & $w_{377}$ & N & can. \\
4402 & 0 & 0 & 0 & 0 & 0 & 1 & 2 & 3 & 0 & 4 & 8 & 13 & 0 & 8 & 12 & 5 & 4 & 3 & $w_{378}$ & N & can. \\
4403 & 0 & 0 & 0 & 0 & 0 & 1 & 2 & 3 & 0 & 4 & 8 & 13 & 0 & 8 & 12 & 6 & 2 & 4 & $w_{379}$ & N & can. \\
4404 & 0 & 0 & 0 & 0 & 0 & 1 & 2 & 3 & 0 & 4 & 8 & 13 & 0 & 8 & 12 & 7 & 2 & 4 & $w_{379}$ & N & can. \\
4405 & 0 & 0 & 0 & 0 & 0 & 1 & 2 & 3 & 0 & 4 & 8 & 13 & 0 & 8 & 13 & 4 & 8 & 3 & $w_{378}$ & N & can. \\
4406 & 0 & 0 & 0 & 0 & 0 & 1 & 2 & 3 & 0 & 4 & 8 & 13 & 0 & 8 & 13 & 6 & 2 & 4 & $w_{379}$ & N & \#4403 \\
4407 & 0 & 0 & 0 & 0 & 0 & 1 & 2 & 3 & 0 & 4 & 8 & 13 & 0 & 8 & 14 & 7 & 4 & 3 & $w_{377}$ & N & \#4398 \\
4408 & 0 & 0 & 0 & 0 & 0 & 1 & 2 & 3 & 0 & 4 & 8 & 13 & 0 & 8 & 15 & 6 & 8 & 3 & $w_{377}$ & N & \#4401 \\
4409 & 0 & 0 & 0 & 0 & 0 & 1 & 2 & 3 & 0 & 4 & 8 & 13 & 0 & 9 & 4 & 15 & 4 & 4 & $w_{278}$ & N & \#4301 \\
4410 & 0 & 0 & 0 & 0 & 0 & 1 & 2 & 3 & 0 & 4 & 8 & 13 & 0 & 9 & 12 & 7 & 2 & 4 & $w_{379}$ & N & \#4404 \\
4411 & 0 & 0 & 0 & 0 & 0 & 1 & 2 & 3 & 0 & 4 & 8 & 13 & 0 & 9 & 13 & 5 & 8 & 3 & $w_{378}$ & N & can. \\
4412 & 0 & 0 & 0 & 0 & 0 & 1 & 2 & 3 & 0 & 4 & 8 & 13 & 0 & 9 & 15 & 7 & 8 & 3 & $w_{377}$ & N & \#4375 \\
4413 & 0 & 0 & 0 & 0 & 0 & 1 & 2 & 4 & 0 & 1 & 3 & 6 & 2 & 8 & 6 & 5 & 4 & 4 & $w_{279}$ & N & can. \\
4414 & 0 & 0 & 0 & 0 & 0 & 1 & 2 & 4 & 0 & 1 & 3 & 6 & 2 & 8 & 6 & 9 & 1 & 4 & $w_{423}$ & N & can. \\
4415 & 0 & 0 & 0 & 0 & 0 & 1 & 2 & 4 & 0 & 1 & 3 & 6 & 2 & 8 & 6 & 10 & 1 & 4 & $w_{392}$ & N & can. \\
4416 & 0 & 0 & 0 & 0 & 0 & 1 & 2 & 4 & 0 & 1 & 3 & 6 & 2 & 8 & 6 & 11 & 1 & 4 & $w_{392}$ & N & can. \\
4417 & 0 & 0 & 0 & 0 & 0 & 1 & 2 & 4 & 0 & 1 & 3 & 6 & 2 & 8 & 6 & 13 & 1 & 4 & $w_{392}$ & N & can. \\
4418 & 0 & 0 & 0 & 0 & 0 & 1 & 2 & 4 & 0 & 1 & 3 & 6 & 2 & 8 & 6 & 14 & 1 & 4 & $w_{423}$ & N & can. \\
4419 & 0 & 0 & 0 & 0 & 0 & 1 & 2 & 4 & 0 & 1 & 3 & 6 & 2 & 8 & 6 & 15 & 1 & 3 & $w_{280}$ & Y & can. \\
4420 & 0 & 0 & 0 & 0 & 0 & 1 & 2 & 4 & 0 & 1 & 3 & 6 & 2 & 8 & 9 & 10 & 1 & 4 & $w_{419}$ & N & can. \\
4421 & 0 & 0 & 0 & 0 & 0 & 1 & 2 & 4 & 0 & 1 & 3 & 6 & 2 & 8 & 9 & 11 & 1 & 4 & $w_{423}$ & N & can. \\
4422 & 0 & 0 & 0 & 0 & 0 & 1 & 2 & 4 & 0 & 1 & 3 & 6 & 2 & 8 & 9 & 12 & 1 & 4 & $w_{379}$ & N & can. \\
4423 & 0 & 0 & 0 & 0 & 0 & 1 & 2 & 4 & 0 & 1 & 3 & 6 & 2 & 8 & 9 & 13 & 1 & 4 & $w_{355}$ & N & can. \\
4424 & 0 & 0 & 0 & 0 & 0 & 1 & 2 & 4 & 0 & 1 & 3 & 6 & 2 & 8 & 9 & 14 & 1 & 4 & $w_{427}$ & N & can. \\
4425 & 0 & 0 & 0 & 0 & 0 & 1 & 2 & 4 & 0 & 1 & 3 & 6 & 2 & 8 & 9 & 15 & 1 & 4 & $w_{281}$ & N & can. \\
4426 & 0 & 0 & 0 & 0 & 0 & 1 & 2 & 4 & 0 & 1 & 3 & 6 & 2 & 8 & 10 & 12 & 1 & 4 & $w_{281}$ & N & \#4425 \\
4427 & 0 & 0 & 0 & 0 & 0 & 1 & 2 & 4 & 0 & 1 & 3 & 6 & 2 & 8 & 10 & 13 & 2 & 4 & $w_{354}$ & N & can. \\
4428 & 0 & 0 & 0 & 0 & 0 & 1 & 2 & 4 & 0 & 1 & 3 & 6 & 2 & 8 & 10 & 14 & 1 & 4 & $w_{355}$ & N & can. \\
4429 & 0 & 0 & 0 & 0 & 0 & 1 & 2 & 4 & 0 & 1 & 3 & 6 & 2 & 8 & 11 & 12 & 1 & 4 & $w_{379}$ & N & can. \\
4430 & 0 & 0 & 0 & 0 & 0 & 1 & 2 & 4 & 0 & 1 & 3 & 6 & 2 & 8 & 11 & 14 & 1 & 4 & $w_{392}$ & N & can. \\
4431 & 0 & 0 & 0 & 0 & 0 & 1 & 2 & 4 & 0 & 1 & 3 & 6 & 2 & 8 & 11 & 15 & 1 & 4 & $w_{281}$ & N & can. \\
4432 & 0 & 0 & 0 & 0 & 0 & 1 & 2 & 4 & 0 & 1 & 3 & 6 & 2 & 8 & 12 & 14 & 1 & 4 & $w_{379}$ & N & can. \\
4433 & 0 & 0 & 0 & 0 & 0 & 1 & 2 & 4 & 0 & 1 & 3 & 6 & 2 & 8 & 12 & 15 & 2 & 4 & $w_{281}$ & N & can. \\
4434 & 0 & 0 & 0 & 0 & 0 & 1 & 2 & 4 & 0 & 1 & 3 & 6 & 2 & 8 & 13 & 14 & 2 & 4 & $w_{426}$ & N & can. \\
4435 & 0 & 0 & 0 & 0 & 0 & 1 & 2 & 4 & 0 & 1 & 3 & 6 & 2 & 8 & 14 & 12 & 1 & 4 & $w_{379}$ & N & can. \\
4436 & 0 & 0 & 0 & 0 & 0 & 1 & 2 & 4 & 0 & 1 & 3 & 6 & 2 & 8 & 14 & 13 & 2 & 4 & $w_{347}$ & N & can. \\
4437 & 0 & 0 & 0 & 0 & 0 & 1 & 2 & 4 & 0 & 1 & 3 & 6 & 2 & 8 & 15 & 12 & 2 & 4 & $w_{428}$ & N & can. \\
4438 & 0 & 0 & 0 & 0 & 0 & 1 & 2 & 4 & 0 & 1 & 3 & 6 & 3 & 8 & 5 & 6 & 6 & 4 & $w_{439}$ & N & can. \\
4439 & 0 & 0 & 0 & 0 & 0 & 1 & 2 & 4 & 0 & 1 & 3 & 6 & 3 & 8 & 5 & 9 & 1 & 4 & $w_{426}$ & N & can. \\
4440 & 0 & 0 & 0 & 0 & 0 & 1 & 2 & 4 & 0 & 1 & 3 & 6 & 3 & 8 & 5 & 10 & 1 & 4 & $w_{419}$ & N & can. \\
4441 & 0 & 0 & 0 & 0 & 0 & 1 & 2 & 4 & 0 & 1 & 3 & 6 & 3 & 8 & 5 & 11 & 1 & 4 & $w_{423}$ & N & can. \\
4442 & 0 & 0 & 0 & 0 & 0 & 1 & 2 & 4 & 0 & 1 & 3 & 6 & 3 & 8 & 5 & 12 & 1 & 4 & $w_{425}$ & N & can. \\
4443 & 0 & 0 & 0 & 0 & 0 & 1 & 2 & 4 & 0 & 1 & 3 & 6 & 3 & 8 & 5 & 13 & 1 & 3 & $w_{378}$ & N & can. \\
4444 & 0 & 0 & 0 & 0 & 0 & 1 & 2 & 4 & 0 & 1 & 3 & 6 & 3 & 8 & 5 & 15 & 1 & 4 & $w_{423}$ & N & can. \\
4445 & 0 & 0 & 0 & 0 & 0 & 1 & 2 & 4 & 0 & 1 & 3 & 6 & 3 & 8 & 7 & 9 & 1 & 4 & $w_{423}$ & N & can. \\
4446 & 0 & 0 & 0 & 0 & 0 & 1 & 2 & 4 & 0 & 1 & 3 & 6 & 3 & 8 & 7 & 10 & 1 & 4 & $w_{392}$ & N & can. \\
4447 & 0 & 0 & 0 & 0 & 0 & 1 & 2 & 4 & 0 & 1 & 3 & 6 & 3 & 8 & 7 & 11 & 1 & 4 & $w_{392}$ & N & can. \\
4448 & 0 & 0 & 0 & 0 & 0 & 1 & 2 & 4 & 0 & 1 & 3 & 6 & 3 & 8 & 7 & 13 & 1 & 4 & $w_{392}$ & N & \#4415 \\
4449 & 0 & 0 & 0 & 0 & 0 & 1 & 2 & 4 & 0 & 1 & 3 & 6 & 3 & 8 & 7 & 14 & 1 & 4 & $w_{423}$ & N & can. \\
4450 & 0 & 0 & 0 & 0 & 0 & 1 & 2 & 4 & 0 & 1 & 3 & 6 & 3 & 8 & 7 & 15 & 1 & 3 & $w_{280}$ & Y & \#4419 \\
4451 & 0 & 0 & 0 & 0 & 0 & 1 & 2 & 4 & 0 & 1 & 3 & 6 & 3 & 8 & 9 & 10 & 1 & 4 & $w_{426}$ & N & can. \\
4452 & 0 & 0 & 0 & 0 & 0 & 1 & 2 & 4 & 0 & 1 & 3 & 6 & 3 & 8 & 9 & 11 & 1 & 4 & $w_{425}$ & N & can. \\
4453 & 0 & 0 & 0 & 0 & 0 & 1 & 2 & 4 & 0 & 1 & 3 & 6 & 3 & 8 & 9 & 12 & 1 & 4 & $w_{428}$ & N & can. \\
4454 & 0 & 0 & 0 & 0 & 0 & 1 & 2 & 4 & 0 & 1 & 3 & 6 & 3 & 8 & 9 & 13 & 1 & 4 & $w_{427}$ & N & can. \\
4455 & 0 & 0 & 0 & 0 & 0 & 1 & 2 & 4 & 0 & 1 & 3 & 6 & 3 & 8 & 9 & 14 & 1 & 4 & $w_{429}$ & N & can. \\
4456 & 0 & 0 & 0 & 0 & 0 & 1 & 2 & 4 & 0 & 1 & 3 & 6 & 3 & 8 & 9 & 15 & 1 & 4 & $w_{379}$ & N & can. \\
4457 & 0 & 0 & 0 & 0 & 0 & 1 & 2 & 4 & 0 & 1 & 3 & 6 & 3 & 8 & 10 & 12 & 1 & 4 & $w_{379}$ & N & can. \\
4458 & 0 & 0 & 0 & 0 & 0 & 1 & 2 & 4 & 0 & 1 & 3 & 6 & 3 & 8 & 10 & 13 & 1 & 4 & $w_{355}$ & N & can. \\
4459 & 0 & 0 & 0 & 0 & 0 & 1 & 2 & 4 & 0 & 1 & 3 & 6 & 3 & 8 & 10 & 14 & 1 & 4 & $w_{427}$ & N & can. \\
4460 & 0 & 0 & 0 & 0 & 0 & 1 & 2 & 4 & 0 & 1 & 3 & 6 & 3 & 8 & 11 & 12 & 1 & 4 & $w_{428}$ & N & can. \\
4461 & 0 & 0 & 0 & 0 & 0 & 1 & 2 & 4 & 0 & 1 & 3 & 6 & 3 & 8 & 11 & 14 & 1 & 4 & $w_{379}$ & N & can. \\
4462 & 0 & 0 & 0 & 0 & 0 & 1 & 2 & 4 & 0 & 1 & 3 & 6 & 3 & 8 & 11 & 15 & 1 & 4 & $w_{379}$ & N & can. \\
4463 & 0 & 0 & 0 & 0 & 0 & 1 & 2 & 4 & 0 & 1 & 3 & 6 & 3 & 8 & 12 & 14 & 1 & 4 & $w_{428}$ & N & can. \\
4464 & 0 & 0 & 0 & 0 & 0 & 1 & 2 & 4 & 0 & 1 & 3 & 6 & 3 & 8 & 12 & 15 & 2 & 4 & $w_{379}$ & N & can. \\
4465 & 0 & 0 & 0 & 0 & 0 & 1 & 2 & 4 & 0 & 1 & 3 & 6 & 3 & 8 & 13 & 14 & 2 & 4 & $w_{429}$ & N & can. \\
4466 & 0 & 0 & 0 & 0 & 0 & 1 & 2 & 4 & 0 & 1 & 3 & 6 & 3 & 8 & 14 & 12 & 1 & 4 & $w_{428}$ & N & can. \\
4467 & 0 & 0 & 0 & 0 & 0 & 1 & 2 & 4 & 0 & 1 & 3 & 6 & 3 & 8 & 14 & 13 & 2 & 4 & $w_{355}$ & N & can. \\
4468 & 0 & 0 & 0 & 0 & 0 & 1 & 2 & 4 & 0 & 1 & 3 & 6 & 3 & 8 & 15 & 12 & 2 & 4 & $w_{430}$ & N & can. \\
4469 & 0 & 0 & 0 & 0 & 0 & 1 & 2 & 4 & 0 & 1 & 3 & 6 & 4 & 6 & 5 & 8 & 7 & 4 & $w_{279}$ & N & can. \\
4470 & 0 & 0 & 0 & 0 & 0 & 1 & 2 & 4 & 0 & 1 & 3 & 6 & 4 & 6 & 8 & 11 & 1 & 4 & $w_{425}$ & N & can. \\
4471 & 0 & 0 & 0 & 0 & 0 & 1 & 2 & 4 & 0 & 1 & 3 & 6 & 4 & 6 & 8 & 12 & 1 & 4 & $w_{425}$ & N & can. \\
4472 & 0 & 0 & 0 & 0 & 0 & 1 & 2 & 4 & 0 & 1 & 3 & 6 & 4 & 6 & 8 & 13 & 1 & 4 & $w_{423}$ & N & can. \\
4473 & 0 & 0 & 0 & 0 & 0 & 1 & 2 & 4 & 0 & 1 & 3 & 6 & 4 & 6 & 8 & 14 & 1 & 4 & $w_{423}$ & N & can. \\
4474 & 0 & 0 & 0 & 0 & 0 & 1 & 2 & 4 & 0 & 1 & 3 & 6 & 4 & 6 & 8 & 15 & 1 & 4 & $w_{425}$ & N & can. \\
4475 & 0 & 0 & 0 & 0 & 0 & 1 & 2 & 4 & 0 & 1 & 3 & 6 & 4 & 8 & 5 & 10 & 1 & 3 & $w_{378}$ & N & can. \\
4476 & 0 & 0 & 0 & 0 & 0 & 1 & 2 & 4 & 0 & 1 & 3 & 6 & 4 & 8 & 5 & 11 & 1 & 4 & $w_{423}$ & N & can. \\
4477 & 0 & 0 & 0 & 0 & 0 & 1 & 2 & 4 & 0 & 1 & 3 & 6 & 4 & 8 & 5 & 14 & 1 & 4 & $w_{426}$ & N & can. \\
4478 & 0 & 0 & 0 & 0 & 0 & 1 & 2 & 4 & 0 & 1 & 3 & 6 & 4 & 8 & 5 & 15 & 1 & 4 & $w_{423}$ & N & can. \\
4479 & 0 & 0 & 0 & 0 & 0 & 1 & 2 & 4 & 0 & 1 & 3 & 6 & 4 & 8 & 9 & 10 & 1 & 4 & $w_{427}$ & N & can. \\
4480 & 0 & 0 & 0 & 0 & 0 & 1 & 2 & 4 & 0 & 1 & 3 & 6 & 4 & 8 & 9 & 11 & 1 & 4 & $w_{428}$ & N & can. \\
4481 & 0 & 0 & 0 & 0 & 0 & 1 & 2 & 4 & 0 & 1 & 3 & 6 & 4 & 8 & 9 & 12 & 1 & 4 & $w_{428}$ & N & can. \\
4482 & 0 & 0 & 0 & 0 & 0 & 1 & 2 & 4 & 0 & 1 & 3 & 6 & 4 & 8 & 9 & 13 & 4 & 4 & $w_{427}$ & N & can. \\
4483 & 0 & 0 & 0 & 0 & 0 & 1 & 2 & 4 & 0 & 1 & 3 & 6 & 4 & 8 & 9 & 14 & 1 & 4 & $w_{429}$ & N & can. \\
4484 & 0 & 0 & 0 & 0 & 0 & 1 & 2 & 4 & 0 & 1 & 3 & 6 & 4 & 8 & 9 & 15 & 1 & 4 & $w_{379}$ & N & can. \\
4485 & 0 & 0 & 0 & 0 & 0 & 1 & 2 & 4 & 0 & 1 & 3 & 6 & 4 & 8 & 10 & 5 & 1 & 3 & $w_{378}$ & N & can. \\
4486 & 0 & 0 & 0 & 0 & 0 & 1 & 2 & 4 & 0 & 1 & 3 & 6 & 4 & 8 & 10 & 9 & 1 & 4 & $w_{427}$ & N & can. \\
4487 & 0 & 0 & 0 & 0 & 0 & 1 & 2 & 4 & 0 & 1 & 3 & 6 & 4 & 8 & 10 & 12 & 1 & 4 & $w_{379}$ & N & can. \\
4488 & 0 & 0 & 0 & 0 & 0 & 1 & 2 & 4 & 0 & 1 & 3 & 6 & 4 & 8 & 10 & 14 & 1 & 4 & $w_{427}$ & N & can. \\
4489 & 0 & 0 & 0 & 0 & 0 & 1 & 2 & 4 & 0 & 1 & 3 & 6 & 4 & 8 & 11 & 5 & 1 & 4 & $w_{423}$ & N & can. \\
4490 & 0 & 0 & 0 & 0 & 0 & 1 & 2 & 4 & 0 & 1 & 3 & 6 & 4 & 8 & 11 & 9 & 1 & 4 & $w_{428}$ & N & can. \\
4491 & 0 & 0 & 0 & 0 & 0 & 1 & 2 & 4 & 0 & 1 & 3 & 6 & 4 & 8 & 11 & 12 & 1 & 4 & $w_{428}$ & N & can. \\
4492 & 0 & 0 & 0 & 0 & 0 & 1 & 2 & 4 & 0 & 1 & 3 & 6 & 4 & 8 & 11 & 14 & 2 & 4 & $w_{423}$ & N & can. \\
4493 & 0 & 0 & 0 & 0 & 0 & 1 & 2 & 4 & 0 & 1 & 3 & 6 & 4 & 8 & 11 & 15 & 1 & 4 & $w_{379}$ & N & can. \\
4494 & 0 & 0 & 0 & 0 & 0 & 1 & 2 & 4 & 0 & 1 & 3 & 6 & 4 & 8 & 12 & 9 & 2 & 4 & $w_{379}$ & N & can. \\
4495 & 0 & 0 & 0 & 0 & 0 & 1 & 2 & 4 & 0 & 1 & 3 & 6 & 4 & 8 & 12 & 11 & 2 & 4 & $w_{379}$ & N & can. \\
4496 & 0 & 0 & 0 & 0 & 0 & 1 & 2 & 4 & 0 & 1 & 3 & 6 & 4 & 8 & 12 & 14 & 1 & 4 & $w_{428}$ & N & can. \\
4497 & 0 & 0 & 0 & 0 & 0 & 1 & 2 & 4 & 0 & 1 & 3 & 6 & 4 & 8 & 12 & 15 & 1 & 4 & $w_{379}$ & N & can. \\
4498 & 0 & 0 & 0 & 0 & 0 & 1 & 2 & 4 & 0 & 1 & 3 & 6 & 4 & 8 & 13 & 9 & 1 & 4 & $w_{429}$ & N & can. \\
4499 & 0 & 0 & 0 & 0 & 0 & 1 & 2 & 4 & 0 & 1 & 3 & 6 & 4 & 8 & 13 & 10 & 1 & 4 & $w_{427}$ & N & can. \\
4500 & 0 & 0 & 0 & 0 & 0 & 1 & 2 & 4 & 0 & 1 & 3 & 6 & 4 & 8 & 13 & 14 & 1 & 4 & $w_{467}$ & N & can. \\
4501 & 0 & 0 & 0 & 0 & 0 & 1 & 2 & 4 & 0 & 1 & 3 & 6 & 4 & 8 & 13 & 15 & 1 & 4 & $w_{428}$ & N & can. \\
4502 & 0 & 0 & 0 & 0 & 0 & 1 & 2 & 4 & 0 & 1 & 3 & 6 & 4 & 8 & 14 & 5 & 1 & 4 & $w_{419}$ & N & can. \\
4503 & 0 & 0 & 0 & 0 & 0 & 1 & 2 & 4 & 0 & 1 & 3 & 6 & 4 & 8 & 14 & 9 & 1 & 4 & $w_{427}$ & N & can. \\
4504 & 0 & 0 & 0 & 0 & 0 & 1 & 2 & 4 & 0 & 1 & 3 & 6 & 4 & 8 & 14 & 10 & 1 & 4 & $w_{355}$ & N & can. \\
4505 & 0 & 0 & 0 & 0 & 0 & 1 & 2 & 4 & 0 & 1 & 3 & 6 & 4 & 8 & 14 & 12 & 1 & 4 & $w_{428}$ & N & \#4463 \\
4506 & 0 & 0 & 0 & 0 & 0 & 1 & 2 & 4 & 0 & 1 & 3 & 6 & 4 & 8 & 15 & 5 & 1 & 4 & $w_{425}$ & N & can. \\
4507 & 0 & 0 & 0 & 0 & 0 & 1 & 2 & 4 & 0 & 1 & 3 & 6 & 4 & 8 & 15 & 9 & 1 & 4 & $w_{428}$ & N & can. \\
4508 & 0 & 0 & 0 & 0 & 0 & 1 & 2 & 4 & 0 & 1 & 3 & 6 & 4 & 8 & 15 & 10 & 2 & 4 & $w_{379}$ & N & \#4494 \\
4509 & 0 & 0 & 0 & 0 & 0 & 1 & 2 & 4 & 0 & 1 & 3 & 6 & 4 & 8 & 15 & 12 & 1 & 4 & $w_{430}$ & N & can. \\
4510 & 0 & 0 & 0 & 0 & 0 & 1 & 2 & 4 & 0 & 1 & 3 & 6 & 4 & 8 & 15 & 13 & 2 & 4 & $w_{428}$ & N & can. \\
4511 & 0 & 0 & 0 & 0 & 0 & 1 & 2 & 4 & 0 & 1 & 3 & 6 & 5 & 6 & 8 & 10 & 2 & 4 & $w_{442}$ & N & can. \\
4512 & 0 & 0 & 0 & 0 & 0 & 1 & 2 & 4 & 0 & 1 & 3 & 6 & 5 & 6 & 8 & 12 & 1 & 4 & $w_{467}$ & N & can. \\
4513 & 0 & 0 & 0 & 0 & 0 & 1 & 2 & 4 & 0 & 1 & 3 & 6 & 5 & 6 & 8 & 13 & 1 & 4 & $w_{425}$ & N & can. \\
4514 & 0 & 0 & 0 & 0 & 0 & 1 & 2 & 4 & 0 & 1 & 3 & 6 & 5 & 8 & 9 & 10 & 1 & 4 & $w_{429}$ & N & can. \\
4515 & 0 & 0 & 0 & 0 & 0 & 1 & 2 & 4 & 0 & 1 & 3 & 6 & 5 & 8 & 9 & 11 & 1 & 4 & $w_{430}$ & N & can. \\
4516 & 0 & 0 & 0 & 0 & 0 & 1 & 2 & 4 & 0 & 1 & 3 & 6 & 5 & 8 & 9 & 12 & 2 & 4 & $w_{430}$ & N & can. \\
4517 & 0 & 0 & 0 & 0 & 0 & 1 & 2 & 4 & 0 & 1 & 3 & 6 & 5 & 8 & 9 & 14 & 1 & 4 & $w_{470}$ & N & can. \\
4518 & 0 & 0 & 0 & 0 & 0 & 1 & 2 & 4 & 0 & 1 & 3 & 6 & 5 & 8 & 9 & 15 & 1 & 4 & $w_{428}$ & N & can. \\
4519 & 0 & 0 & 0 & 0 & 0 & 1 & 2 & 4 & 0 & 1 & 3 & 6 & 5 & 8 & 10 & 9 & 1 & 4 & $w_{429}$ & N & can. \\
4520 & 0 & 0 & 0 & 0 & 0 & 1 & 2 & 4 & 0 & 1 & 3 & 6 & 5 & 8 & 10 & 12 & 1 & 4 & $w_{428}$ & N & can. \\
4521 & 0 & 0 & 0 & 0 & 0 & 1 & 2 & 4 & 0 & 1 & 3 & 6 & 5 & 8 & 10 & 13 & 1 & 4 & $w_{427}$ & N & can. \\
4522 & 0 & 0 & 0 & 0 & 0 & 1 & 2 & 4 & 0 & 1 & 3 & 6 & 5 & 8 & 10 & 14 & 1 & 4 & $w_{429}$ & N & can. \\
4523 & 0 & 0 & 0 & 0 & 0 & 1 & 2 & 4 & 0 & 1 & 3 & 6 & 5 & 8 & 11 & 9 & 1 & 4 & $w_{430}$ & N & can. \\
4524 & 0 & 0 & 0 & 0 & 0 & 1 & 2 & 4 & 0 & 1 & 3 & 6 & 5 & 8 & 11 & 12 & 1 & 4 & $w_{430}$ & N & can. \\
4525 & 0 & 0 & 0 & 0 & 0 & 1 & 2 & 4 & 0 & 1 & 3 & 6 & 5 & 8 & 11 & 14 & 1 & 4 & $w_{428}$ & N & can. \\
4526 & 0 & 0 & 0 & 0 & 0 & 1 & 2 & 4 & 0 & 1 & 3 & 6 & 5 & 8 & 11 & 15 & 1 & 4 & $w_{428}$ & N & can. \\
4527 & 0 & 0 & 0 & 0 & 0 & 1 & 2 & 4 & 0 & 1 & 3 & 6 & 5 & 8 & 12 & 11 & 1 & 4 & $w_{428}$ & N & can. \\
4528 & 0 & 0 & 0 & 0 & 0 & 1 & 2 & 4 & 0 & 1 & 3 & 6 & 5 & 8 & 12 & 14 & 1 & 4 & $w_{430}$ & N & can. \\
4529 & 0 & 0 & 0 & 0 & 0 & 1 & 2 & 4 & 0 & 1 & 3 & 6 & 5 & 8 & 12 & 15 & 1 & 4 & $w_{428}$ & N & can. \\
4530 & 0 & 0 & 0 & 0 & 0 & 1 & 2 & 4 & 0 & 1 & 3 & 6 & 5 & 8 & 13 & 9 & 2 & 4 & $w_{470}$ & N & can. \\
4531 & 0 & 0 & 0 & 0 & 0 & 1 & 2 & 4 & 0 & 1 & 3 & 6 & 5 & 8 & 13 & 10 & 1 & 4 & $w_{429}$ & N & can. \\
4532 & 0 & 0 & 0 & 0 & 0 & 1 & 2 & 4 & 0 & 1 & 3 & 6 & 5 & 8 & 13 & 11 & 1 & 4 & $w_{428}$ & N & can. \\
4533 & 0 & 0 & 0 & 0 & 0 & 1 & 2 & 4 & 0 & 1 & 3 & 6 & 5 & 8 & 13 & 14 & 1 & 4 & $w_{470}$ & N & can. \\
4534 & 0 & 0 & 0 & 0 & 0 & 1 & 2 & 4 & 0 & 1 & 3 & 6 & 5 & 8 & 13 & 15 & 1 & 4 & $w_{430}$ & N & can. \\
4535 & 0 & 0 & 0 & 0 & 0 & 1 & 2 & 4 & 0 & 1 & 3 & 6 & 5 & 8 & 14 & 12 & 1 & 4 & $w_{430}$ & N & can. \\
4536 & 0 & 0 & 0 & 0 & 0 & 1 & 2 & 4 & 0 & 1 & 3 & 6 & 5 & 8 & 15 & 9 & 1 & 4 & $w_{430}$ & N & can. \\
4537 & 0 & 0 & 0 & 0 & 0 & 1 & 2 & 4 & 0 & 1 & 3 & 6 & 5 & 8 & 15 & 11 & 1 & 4 & $w_{430}$ & N & can. \\
4538 & 0 & 0 & 0 & 0 & 0 & 1 & 2 & 4 & 0 & 1 & 3 & 6 & 5 & 8 & 15 & 12 & 1 & 4 & $w_{441}$ & N & can. \\
4539 & 0 & 0 & 0 & 0 & 0 & 1 & 2 & 4 & 0 & 1 & 3 & 6 & 5 & 8 & 15 & 13 & 1 & 4 & $w_{430}$ & N & \#4528 \\
4540 & 0 & 0 & 0 & 0 & 0 & 1 & 2 & 4 & 0 & 1 & 3 & 6 & 8 & 10 & 9 & 13 & 2 & 4 & $w_{443}$ & N & can. \\
4541 & 0 & 0 & 0 & 0 & 0 & 1 & 2 & 4 & 0 & 1 & 3 & 6 & 8 & 10 & 9 & 14 & 1 & 4 & $w_{443}$ & N & can. \\
4542 & 0 & 0 & 0 & 0 & 0 & 1 & 2 & 4 & 0 & 1 & 3 & 6 & 8 & 10 & 14 & 13 & 6 & 4 & $w_{443}$ & N & can. \\
4543 & 0 & 0 & 0 & 0 & 0 & 1 & 2 & 4 & 0 & 1 & 3 & 6 & 8 & 11 & 9 & 13 & 2 & 4 & $w_{470}$ & N & can. \\
4544 & 0 & 0 & 0 & 0 & 0 & 1 & 2 & 4 & 0 & 1 & 3 & 6 & 8 & 12 & 9 & 14 & 3 & 3 & $w_{438}$ & N & can. \\
4545 & 0 & 0 & 0 & 0 & 0 & 1 & 2 & 4 & 0 & 1 & 3 & 8 & 2 & 6 & 8 & 4 & 1 & 4 & $w_{392}$ & N & can. \\
4546 & 0 & 0 & 0 & 0 & 0 & 1 & 2 & 4 & 0 & 1 & 3 & 8 & 2 & 6 & 8 & 13 & 1 & 4 & $w_{428}$ & N & can. \\
4547 & 0 & 0 & 0 & 0 & 0 & 1 & 2 & 4 & 0 & 1 & 3 & 8 & 2 & 6 & 8 & 14 & 1 & 4 & $w_{379}$ & N & can. \\
4548 & 0 & 0 & 0 & 0 & 0 & 1 & 2 & 4 & 0 & 1 & 3 & 8 & 2 & 6 & 9 & 12 & 1 & 4 & $w_{379}$ & N & can. \\
4549 & 0 & 0 & 0 & 0 & 0 & 1 & 2 & 4 & 0 & 1 & 3 & 8 & 2 & 6 & 9 & 14 & 2 & 4 & $w_{427}$ & N & can. \\
4550 & 0 & 0 & 0 & 0 & 0 & 1 & 2 & 4 & 0 & 1 & 3 & 8 & 2 & 6 & 9 & 15 & 1 & 4 & $w_{281}$ & N & can. \\
4551 & 0 & 0 & 0 & 0 & 0 & 1 & 2 & 4 & 0 & 1 & 3 & 8 & 2 & 6 & 10 & 4 & 2 & 4 & $w_{347}$ & N & can. \\
4552 & 0 & 0 & 0 & 0 & 0 & 1 & 2 & 4 & 0 & 1 & 3 & 8 & 2 & 6 & 10 & 15 & 1 & 4 & $w_{379}$ & N & can. \\
4553 & 0 & 0 & 0 & 0 & 0 & 1 & 2 & 4 & 0 & 1 & 3 & 8 & 2 & 6 & 11 & 14 & 1 & 4 & $w_{428}$ & N & can. \\
4554 & 0 & 0 & 0 & 0 & 0 & 1 & 2 & 4 & 0 & 1 & 3 & 8 & 2 & 6 & 12 & 14 & 1 & 4 & $w_{428}$ & N & can. \\
4555 & 0 & 0 & 0 & 0 & 0 & 1 & 2 & 4 & 0 & 1 & 3 & 8 & 2 & 6 & 12 & 15 & 1 & 4 & $w_{379}$ & N & can. \\
4556 & 0 & 0 & 0 & 0 & 0 & 1 & 2 & 4 & 0 & 1 & 3 & 8 & 2 & 6 & 13 & 4 & 2 & 3 & $w_{378}$ & N & can. \\
4557 & 0 & 0 & 0 & 0 & 0 & 1 & 2 & 4 & 0 & 1 & 3 & 8 & 2 & 6 & 13 & 8 & 1 & 4 & $w_{428}$ & N & can. \\
4558 & 0 & 0 & 0 & 0 & 0 & 1 & 2 & 4 & 0 & 1 & 3 & 8 & 2 & 6 & 13 & 14 & 1 & 4 & $w_{429}$ & N & can. \\
4559 & 0 & 0 & 0 & 0 & 0 & 1 & 2 & 4 & 0 & 1 & 3 & 8 & 2 & 6 & 14 & 4 & 2 & 4 & $w_{347}$ & N & can. \\
4560 & 0 & 0 & 0 & 0 & 0 & 1 & 2 & 4 & 0 & 1 & 3 & 8 & 2 & 6 & 14 & 11 & 1 & 4 & $w_{428}$ & N & can. \\
4561 & 0 & 0 & 0 & 0 & 0 & 1 & 2 & 4 & 0 & 1 & 3 & 8 & 2 & 6 & 14 & 12 & 1 & 4 & $w_{379}$ & N & can. \\
4562 & 0 & 0 & 0 & 0 & 0 & 1 & 2 & 4 & 0 & 1 & 3 & 8 & 2 & 6 & 14 & 13 & 1 & 4 & $w_{427}$ & N & can. \\
4563 & 0 & 0 & 0 & 0 & 0 & 1 & 2 & 4 & 0 & 1 & 3 & 8 & 2 & 6 & 15 & 8 & 1 & 4 & $w_{428}$ & N & can. \\
4564 & 0 & 0 & 0 & 0 & 0 & 1 & 2 & 4 & 0 & 1 & 3 & 8 & 2 & 6 & 15 & 12 & 1 & 4 & $w_{428}$ & N & can. \\
4565 & 0 & 0 & 0 & 0 & 0 & 1 & 2 & 4 & 0 & 1 & 3 & 8 & 2 & 6 & 15 & 13 & 1 & 4 & $w_{430}$ & N & can. \\
4566 & 0 & 0 & 0 & 0 & 0 & 1 & 2 & 4 & 0 & 1 & 3 & 8 & 2 & 7 & 8 & 5 & 1 & 4 & $w_{266}$ & N & can. \\
4567 & 0 & 0 & 0 & 0 & 0 & 1 & 2 & 4 & 0 & 1 & 3 & 8 & 2 & 7 & 8 & 14 & 2 & 4 & $w_{267}$ & N & can. \\
4568 & 0 & 0 & 0 & 0 & 0 & 1 & 2 & 4 & 0 & 1 & 3 & 8 & 2 & 7 & 9 & 13 & 1 & 4 & $w_{379}$ & N & can. \\
4569 & 0 & 0 & 0 & 0 & 0 & 1 & 2 & 4 & 0 & 1 & 3 & 8 & 2 & 7 & 9 & 14 & 1 & 4 & $w_{354}$ & N & can. \\
4570 & 0 & 0 & 0 & 0 & 0 & 1 & 2 & 4 & 0 & 1 & 3 & 8 & 2 & 7 & 10 & 14 & 1 & 4 & $w_{281}$ & N & can. \\
4571 & 0 & 0 & 0 & 0 & 0 & 1 & 2 & 4 & 0 & 1 & 3 & 8 & 2 & 7 & 12 & 8 & 1 & 4 & $w_{281}$ & N & can. \\
4572 & 0 & 0 & 0 & 0 & 0 & 1 & 2 & 4 & 0 & 1 & 3 & 8 & 2 & 7 & 12 & 14 & 2 & 4 & $w_{354}$ & N & can. \\
4573 & 0 & 0 & 0 & 0 & 0 & 1 & 2 & 4 & 0 & 1 & 3 & 8 & 2 & 7 & 13 & 5 & 7 & 3 & $w_{280}$ & Y & can. \\
4574 & 0 & 0 & 0 & 0 & 0 & 1 & 2 & 4 & 0 & 1 & 3 & 8 & 2 & 7 & 13 & 14 & 1 & 4 & $w_{379}$ & N & can. \\
4575 & 0 & 0 & 0 & 0 & 0 & 1 & 2 & 4 & 0 & 1 & 3 & 8 & 2 & 7 & 14 & 8 & 1 & 4 & $w_{281}$ & N & can. \\
4576 & 0 & 0 & 0 & 0 & 0 & 1 & 2 & 4 & 0 & 1 & 3 & 8 & 2 & 7 & 14 & 13 & 1 & 4 & $w_{428}$ & N & can. \\
4577 & 0 & 0 & 0 & 0 & 0 & 1 & 2 & 4 & 0 & 1 & 3 & 8 & 2 & 7 & 15 & 8 & 1 & 4 & $w_{281}$ & N & can. \\
4578 & 0 & 0 & 0 & 0 & 0 & 1 & 2 & 4 & 0 & 1 & 3 & 8 & 2 & 7 & 15 & 11 & 1 & 4 & $w_{379}$ & N & can. \\
4579 & 0 & 0 & 0 & 0 & 0 & 1 & 2 & 4 & 0 & 1 & 3 & 8 & 2 & 7 & 15 & 13 & 1 & 4 & $w_{379}$ & N & can. \\
4580 & 0 & 0 & 0 & 0 & 0 & 1 & 2 & 4 & 0 & 1 & 3 & 8 & 2 & 12 & 5 & 15 & 1 & 4 & $w_{281}$ & N & can. \\
4581 & 0 & 0 & 0 & 0 & 0 & 1 & 2 & 4 & 0 & 1 & 3 & 8 & 2 & 12 & 6 & 11 & 2 & 4 & $w_{428}$ & N & can. \\
4582 & 0 & 0 & 0 & 0 & 0 & 1 & 2 & 4 & 0 & 1 & 3 & 8 & 2 & 12 & 6 & 15 & 1 & 4 & $w_{379}$ & N & can. \\
4583 & 0 & 0 & 0 & 0 & 0 & 1 & 2 & 4 & 0 & 1 & 3 & 8 & 2 & 12 & 11 & 6 & 4 & 4 & $w_{281}$ & N & can. \\
4584 & 0 & 0 & 0 & 0 & 0 & 1 & 2 & 4 & 0 & 1 & 3 & 8 & 2 & 13 & 5 & 8 & 2 & 4 & $w_{267}$ & N & can. \\
4585 & 0 & 0 & 0 & 0 & 0 & 1 & 2 & 4 & 0 & 1 & 3 & 8 & 2 & 13 & 5 & 11 & 1 & 4 & $w_{379}$ & N & can. \\
4586 & 0 & 0 & 0 & 0 & 0 & 1 & 2 & 4 & 0 & 1 & 3 & 8 & 2 & 13 & 6 & 11 & 2 & 4 & $w_{429}$ & N & can. \\
4587 & 0 & 0 & 0 & 0 & 0 & 1 & 2 & 4 & 0 & 1 & 3 & 8 & 2 & 13 & 6 & 14 & 1 & 4 & $w_{427}$ & N & can. \\
4588 & 0 & 0 & 0 & 0 & 0 & 1 & 2 & 4 & 0 & 1 & 3 & 8 & 2 & 13 & 8 & 5 & 2 & 4 & $w_{379}$ & N & can. \\
4589 & 0 & 0 & 0 & 0 & 0 & 1 & 2 & 4 & 0 & 1 & 3 & 8 & 2 & 13 & 8 & 6 & 1 & 4 & $w_{379}$ & N & can. \\
4590 & 0 & 0 & 0 & 0 & 0 & 1 & 2 & 4 & 0 & 1 & 3 & 8 & 2 & 13 & 8 & 12 & 1 & 4 & $w_{379}$ & N & can. \\
4591 & 0 & 0 & 0 & 0 & 0 & 1 & 2 & 4 & 0 & 1 & 3 & 8 & 2 & 13 & 9 & 4 & 6 & 4 & $w_{373}$ & N & can. \\
4592 & 0 & 0 & 0 & 0 & 0 & 1 & 2 & 4 & 0 & 1 & 3 & 8 & 2 & 13 & 9 & 7 & 1 & 4 & $w_{281}$ & N & can. \\
4593 & 0 & 0 & 0 & 0 & 0 & 1 & 2 & 4 & 0 & 1 & 3 & 8 & 2 & 13 & 10 & 14 & 4 & 4 & $w_{379}$ & N & can. \\
4594 & 0 & 0 & 0 & 0 & 0 & 1 & 2 & 4 & 0 & 1 & 3 & 8 & 2 & 13 & 11 & 6 & 2 & 4 & $w_{355}$ & N & can. \\
4595 & 0 & 0 & 0 & 0 & 0 & 1 & 2 & 4 & 0 & 1 & 3 & 8 & 2 & 14 & 5 & 8 & 2 & 4 & $w_{432}$ & N & \#4300 \\
4596 & 0 & 0 & 0 & 0 & 0 & 1 & 2 & 4 & 0 & 1 & 3 & 8 & 2 & 14 & 5 & 11 & 1 & 4 & $w_{379}$ & N & \#4585 \\
4597 & 0 & 0 & 0 & 0 & 0 & 1 & 2 & 4 & 0 & 1 & 3 & 8 & 2 & 14 & 5 & 13 & 1 & 4 & $w_{379}$ & N & can. \\
4598 & 0 & 0 & 0 & 0 & 0 & 1 & 2 & 4 & 0 & 1 & 3 & 8 & 2 & 14 & 5 & 15 & 1 & 4 & $w_{434}$ & N & \#4306 \\
4599 & 0 & 0 & 0 & 0 & 0 & 1 & 2 & 4 & 0 & 1 & 3 & 8 & 2 & 14 & 6 & 11 & 2 & 4 & $w_{423}$ & N & can. \\
4600 & 0 & 0 & 0 & 0 & 0 & 1 & 2 & 4 & 0 & 1 & 3 & 8 & 2 & 14 & 6 & 15 & 1 & 4 & $w_{428}$ & N & can. \\
4601 & 0 & 0 & 0 & 0 & 0 & 1 & 2 & 4 & 0 & 1 & 3 & 8 & 2 & 14 & 8 & 5 & 2 & 4 & $w_{432}$ & N & \#4300 \\
4602 & 0 & 0 & 0 & 0 & 0 & 1 & 2 & 4 & 0 & 1 & 3 & 8 & 2 & 14 & 8 & 6 & 1 & 4 & $w_{379}$ & N & \#4589 \\
4603 & 0 & 0 & 0 & 0 & 0 & 1 & 2 & 4 & 0 & 1 & 3 & 8 & 2 & 14 & 8 & 15 & 1 & 4 & $w_{434}$ & N & \#4306 \\
4604 & 0 & 0 & 0 & 0 & 0 & 1 & 2 & 4 & 0 & 1 & 3 & 8 & 2 & 14 & 9 & 4 & 6 & 4 & $w_{432}$ & N & \#4312 \\
4605 & 0 & 0 & 0 & 0 & 0 & 1 & 2 & 4 & 0 & 1 & 3 & 8 & 2 & 14 & 9 & 15 & 1 & 4 & $w_{432}$ & N & \#4293 \\
4606 & 0 & 0 & 0 & 0 & 0 & 1 & 2 & 4 & 0 & 1 & 3 & 8 & 2 & 14 & 11 & 6 & 2 & 4 & $w_{423}$ & N & can. \\
4607 & 0 & 0 & 0 & 0 & 0 & 1 & 2 & 4 & 0 & 1 & 3 & 8 & 2 & 14 & 11 & 15 & 1 & 4 & $w_{428}$ & N & can. \\
4608 & 0 & 0 & 0 & 0 & 0 & 1 & 2 & 4 & 0 & 1 & 3 & 8 & 2 & 15 & 5 & 11 & 1 & 4 & $w_{428}$ & N & can. \\
4609 & 0 & 0 & 0 & 0 & 0 & 1 & 2 & 4 & 0 & 1 & 3 & 8 & 2 & 15 & 5 & 14 & 1 & 4 & $w_{432}$ & N & \#4304 \\
4610 & 0 & 0 & 0 & 0 & 0 & 1 & 2 & 4 & 0 & 1 & 3 & 8 & 2 & 15 & 6 & 14 & 1 & 4 & $w_{429}$ & N & can. \\
4611 & 0 & 0 & 0 & 0 & 0 & 1 & 2 & 4 & 0 & 1 & 3 & 8 & 2 & 15 & 8 & 14 & 1 & 4 & $w_{432}$ & N & \#4304 \\
4612 & 0 & 0 & 0 & 0 & 0 & 1 & 2 & 4 & 0 & 1 & 3 & 8 & 2 & 15 & 9 & 14 & 2 & 4 & $w_{427}$ & N & \#4330 \\
4613 & 0 & 0 & 0 & 0 & 0 & 1 & 2 & 4 & 0 & 1 & 3 & 8 & 4 & 6 & 8 & 11 & 2 & 4 & $w_{430}$ & N & can. \\
4614 & 0 & 0 & 0 & 0 & 0 & 1 & 2 & 4 & 0 & 1 & 3 & 8 & 4 & 6 & 8 & 13 & 1 & 4 & $w_{429}$ & N & can. \\
4615 & 0 & 0 & 0 & 0 & 0 & 1 & 2 & 4 & 0 & 1 & 3 & 8 & 4 & 6 & 8 & 14 & 1 & 4 & $w_{428}$ & N & can. \\
4616 & 0 & 0 & 0 & 0 & 0 & 1 & 2 & 4 & 0 & 1 & 3 & 8 & 4 & 6 & 9 & 13 & 1 & 4 & $w_{430}$ & N & can. \\
4617 & 0 & 0 & 0 & 0 & 0 & 1 & 2 & 4 & 0 & 1 & 3 & 8 & 4 & 6 & 9 & 14 & 1 & 4 & $w_{430}$ & N & can. \\
4618 & 0 & 0 & 0 & 0 & 0 & 1 & 2 & 4 & 0 & 1 & 3 & 8 & 4 & 6 & 10 & 9 & 1 & 4 & $w_{428}$ & N & can. \\
4619 & 0 & 0 & 0 & 0 & 0 & 1 & 2 & 4 & 0 & 1 & 3 & 8 & 4 & 6 & 10 & 14 & 1 & 4 & $w_{430}$ & N & can. \\
4620 & 0 & 0 & 0 & 0 & 0 & 1 & 2 & 4 & 0 & 1 & 3 & 8 & 4 & 6 & 10 & 15 & 2 & 4 & $w_{427}$ & N & can. \\
4621 & 0 & 0 & 0 & 0 & 0 & 1 & 2 & 4 & 0 & 1 & 3 & 8 & 4 & 6 & 11 & 14 & 1 & 4 & $w_{441}$ & N & can. \\
4622 & 0 & 0 & 0 & 0 & 0 & 1 & 2 & 4 & 0 & 1 & 3 & 8 & 4 & 6 & 14 & 9 & 1 & 4 & $w_{428}$ & N & can. \\
4623 & 0 & 0 & 0 & 0 & 0 & 1 & 2 & 4 & 0 & 1 & 3 & 8 & 4 & 6 & 14 & 10 & 1 & 4 & $w_{428}$ & N & can. \\
4624 & 0 & 0 & 0 & 0 & 0 & 1 & 2 & 4 & 0 & 1 & 3 & 8 & 4 & 6 & 14 & 11 & 1 & 4 & $w_{430}$ & N & can. \\
4625 & 0 & 0 & 0 & 0 & 0 & 1 & 2 & 4 & 0 & 1 & 3 & 8 & 4 & 6 & 15 & 11 & 1 & 4 & $w_{441}$ & N & can. \\
4626 & 0 & 0 & 0 & 0 & 0 & 1 & 2 & 4 & 0 & 1 & 3 & 8 & 4 & 7 & 6 & 11 & 4 & 4 & $w_{426}$ & N & can. \\
4627 & 0 & 0 & 0 & 0 & 0 & 1 & 2 & 4 & 0 & 1 & 3 & 8 & 4 & 7 & 8 & 13 & 1 & 4 & $w_{428}$ & N & can. \\
4628 & 0 & 0 & 0 & 0 & 0 & 1 & 2 & 4 & 0 & 1 & 3 & 8 & 4 & 7 & 8 & 15 & 2 & 4 & $w_{427}$ & N & can. \\
4629 & 0 & 0 & 0 & 0 & 0 & 1 & 2 & 4 & 0 & 1 & 3 & 8 & 4 & 7 & 9 & 13 & 1 & 4 & $w_{428}$ & N & can. \\
4630 & 0 & 0 & 0 & 0 & 0 & 1 & 2 & 4 & 0 & 1 & 3 & 8 & 4 & 7 & 9 & 15 & 2 & 4 & $w_{423}$ & N & can. \\
4631 & 0 & 0 & 0 & 0 & 0 & 1 & 2 & 4 & 0 & 1 & 3 & 8 & 4 & 7 & 10 & 14 & 1 & 4 & $w_{379}$ & N & can. \\
4632 & 0 & 0 & 0 & 0 & 0 & 1 & 2 & 4 & 0 & 1 & 3 & 8 & 4 & 7 & 11 & 14 & 1 & 4 & $w_{428}$ & N & can. \\
4633 & 0 & 0 & 0 & 0 & 0 & 1 & 2 & 4 & 0 & 1 & 3 & 8 & 4 & 7 & 11 & 15 & 1 & 4 & $w_{429}$ & N & can. \\
4634 & 0 & 0 & 0 & 0 & 0 & 1 & 2 & 4 & 0 & 1 & 3 & 8 & 4 & 7 & 14 & 11 & 1 & 4 & $w_{430}$ & N & can. \\
4635 & 0 & 0 & 0 & 0 & 0 & 1 & 2 & 4 & 0 & 1 & 3 & 8 & 4 & 9 & 13 & 14 & 2 & 4 & $w_{379}$ & N & can. \\
4636 & 0 & 0 & 0 & 0 & 0 & 1 & 2 & 4 & 0 & 1 & 3 & 8 & 4 & 9 & 13 & 15 & 6 & 4 & $w_{355}$ & N & can. \\
4637 & 0 & 0 & 0 & 0 & 0 & 1 & 2 & 4 & 0 & 1 & 3 & 8 & 4 & 9 & 14 & 13 & 2 & 4 & $w_{430}$ & N & can. \\
4638 & 0 & 0 & 0 & 0 & 0 & 1 & 2 & 4 & 0 & 1 & 3 & 8 & 4 & 9 & 15 & 13 & 2 & 4 & $w_{429}$ & N & can. \\
4639 & 0 & 0 & 0 & 0 & 0 & 1 & 2 & 4 & 0 & 1 & 3 & 8 & 4 & 10 & 13 & 14 & 1 & 3 & $w_{440}$ & N & can. \\
4640 & 0 & 0 & 0 & 0 & 0 & 1 & 2 & 4 & 0 & 1 & 3 & 8 & 4 & 10 & 14 & 13 & 1 & 3 & $w_{440}$ & N & \#4639 \\
4641 & 0 & 0 & 0 & 0 & 0 & 1 & 2 & 4 & 0 & 1 & 3 & 8 & 4 & 11 & 12 & 14 & 1 & 3 & $w_{440}$ & Y & can. \\
4642 & 0 & 0 & 0 & 0 & 0 & 1 & 2 & 4 & 0 & 1 & 3 & 8 & 4 & 11 & 12 & 15 & 1 & 4 & $w_{430}$ & N & can. \\
4643 & 0 & 0 & 0 & 0 & 0 & 1 & 2 & 4 & 0 & 1 & 3 & 8 & 4 & 11 & 13 & 15 & 1 & 3 & $w_{438}$ & N & can. \\
4644 & 0 & 0 & 0 & 0 & 0 & 1 & 2 & 4 & 0 & 1 & 3 & 8 & 4 & 11 & 14 & 13 & 1 & 4 & $w_{441}$ & N & can. \\
4645 & 0 & 0 & 0 & 0 & 0 & 1 & 2 & 4 & 0 & 1 & 3 & 8 & 4 & 13 & 10 & 14 & 4 & 3 & $w_{440}$ & Y & can. \\
4646 & 0 & 0 & 0 & 0 & 0 & 1 & 2 & 4 & 0 & 1 & 3 & 8 & 4 & 13 & 11 & 15 & 1 & 3 & $w_{438}$ & N & can. \\
4647 & 0 & 0 & 0 & 0 & 0 & 1 & 2 & 4 & 0 & 1 & 3 & 8 & 4 & 13 & 14 & 10 & 5 & 3 & $w_{440}$ & Y & can. \\
4648 & 0 & 0 & 0 & 0 & 0 & 1 & 2 & 4 & 0 & 1 & 3 & 8 & 4 & 13 & 15 & 9 & 4 & 4 & $w_{379}$ & N & can. \\
4649 & 0 & 0 & 0 & 0 & 0 & 1 & 2 & 4 & 0 & 1 & 3 & 8 & 4 & 14 & 8 & 15 & 8 & 3 & $w_{438}$ & N & \#4373 \\
4650 & 0 & 0 & 0 & 0 & 0 & 1 & 2 & 4 & 0 & 1 & 3 & 8 & 4 & 14 & 9 & 15 & 1 & 4 & $w_{434}$ & N & \#4376 \\
4651 & 0 & 0 & 0 & 0 & 0 & 1 & 2 & 4 & 0 & 1 & 3 & 8 & 4 & 15 & 8 & 14 & 8 & 3 & $w_{377}$ & N & \#4375 \\
4652 & 0 & 0 & 0 & 0 & 0 & 1 & 2 & 4 & 0 & 1 & 3 & 8 & 6 & 11 & 7 & 15 & 2 & 4 & $w_{470}$ & N & can. \\
4653 & 0 & 0 & 0 & 0 & 0 & 1 & 2 & 4 & 0 & 1 & 3 & 8 & 6 & 11 & 12 & 15 & 4 & 4 & $w_{441}$ & N & can. \\
4654 & 0 & 0 & 0 & 0 & 0 & 1 & 2 & 4 & 0 & 1 & 3 & 8 & 6 & 11 & 13 & 14 & 6 & 4 & $w_{429}$ & N & can. \\
4655 & 0 & 0 & 0 & 0 & 0 & 1 & 2 & 4 & 0 & 1 & 3 & 8 & 6 & 11 & 13 & 15 & 2 & 4 & $w_{475}$ & N & can. \\
4656 & 0 & 0 & 0 & 0 & 0 & 1 & 2 & 4 & 0 & 1 & 3 & 8 & 6 & 14 & 11 & 15 & 4 & 4 & $w_{476}$ & N & can. \\
4657 & 0 & 0 & 0 & 0 & 0 & 1 & 2 & 4 & 0 & 1 & 6 & 8 & 2 & 8 & 11 & 4 & 24 & 4 & $w_{356}$ & N & can. \\
4658 & 0 & 0 & 0 & 0 & 0 & 1 & 2 & 4 & 0 & 1 & 6 & 8 & 2 & 8 & 11 & 15 & 2 & 4 & $w_{281}$ & N & can. \\
4659 & 0 & 0 & 0 & 0 & 0 & 1 & 2 & 4 & 0 & 1 & 6 & 8 & 2 & 8 & 14 & 11 & 1 & 4 & $w_{427}$ & N & can. \\
4660 & 0 & 0 & 0 & 0 & 0 & 1 & 2 & 4 & 0 & 1 & 6 & 8 & 2 & 9 & 13 & 14 & 1 & 3 & $w_{377}$ & Y & can. \\
4661 & 0 & 0 & 0 & 0 & 0 & 1 & 2 & 4 & 0 & 1 & 6 & 8 & 2 & 11 & 9 & 14 & 1 & 4 & $w_{432}$ & N & can. \\
4662 & 0 & 0 & 0 & 0 & 0 & 1 & 2 & 4 & 0 & 1 & 6 & 8 & 2 & 11 & 13 & 8 & 1 & 4 & $w_{428}$ & N & can. \\
4663 & 0 & 0 & 0 & 0 & 0 & 1 & 2 & 4 & 0 & 1 & 6 & 8 & 2 & 13 & 8 & 15 & 2 & 3 & $w_{377}$ & Y & \#4398 \\
4664 & 0 & 0 & 0 & 0 & 0 & 1 & 2 & 4 & 0 & 1 & 6 & 8 & 2 & 13 & 9 & 14 & 6 & 3 & $w_{377}$ & N & \#4660 \\
4665 & 0 & 0 & 0 & 0 & 0 & 1 & 2 & 4 & 0 & 1 & 6 & 8 & 2 & 14 & 8 & 13 & 1 & 4 & $w_{428}$ & N & can. \\
4666 & 0 & 0 & 0 & 0 & 0 & 1 & 2 & 4 & 0 & 1 & 6 & 8 & 2 & 14 & 12 & 11 & 1 & 4 & $w_{379}$ & N & can. \\
4667 & 0 & 0 & 0 & 0 & 0 & 1 & 2 & 4 & 0 & 1 & 6 & 8 & 3 & 6 & 11 & 8 & 2 & 4 & $w_{428}$ & N & can. \\
4668 & 0 & 0 & 0 & 0 & 0 & 1 & 2 & 4 & 0 & 1 & 6 & 8 & 3 & 6 & 11 & 12 & 1 & 4 & $w_{428}$ & N & can. \\
4669 & 0 & 0 & 0 & 0 & 0 & 1 & 2 & 4 & 0 & 1 & 6 & 8 & 3 & 6 & 11 & 15 & 1 & 4 & $w_{428}$ & N & can. \\
4670 & 0 & 0 & 0 & 0 & 0 & 1 & 2 & 4 & 0 & 1 & 6 & 8 & 3 & 6 & 13 & 5 & 2 & 4 & $w_{426}$ & N & can. \\
4671 & 0 & 0 & 0 & 0 & 0 & 1 & 2 & 4 & 0 & 1 & 6 & 8 & 3 & 6 & 13 & 14 & 1 & 4 & $w_{430}$ & N & can. \\
4672 & 0 & 0 & 0 & 0 & 0 & 1 & 2 & 4 & 0 & 1 & 6 & 8 & 3 & 6 & 13 & 15 & 2 & 4 & $w_{426}$ & N & can. \\
4673 & 0 & 0 & 0 & 0 & 0 & 1 & 2 & 4 & 0 & 1 & 6 & 8 & 3 & 6 & 15 & 8 & 4 & 4 & $w_{427}$ & N & can. \\
4674 & 0 & 0 & 0 & 0 & 0 & 1 & 2 & 4 & 0 & 1 & 6 & 8 & 3 & 6 & 15 & 13 & 1 & 4 & $w_{429}$ & N & can. \\
4675 & 0 & 0 & 0 & 0 & 0 & 1 & 2 & 4 & 0 & 1 & 6 & 8 & 3 & 7 & 11 & 9 & 1 & 4 & $w_{428}$ & N & can. \\
4676 & 0 & 0 & 0 & 0 & 0 & 1 & 2 & 4 & 0 & 1 & 6 & 8 & 3 & 7 & 11 & 12 & 1 & 4 & $w_{434}$ & N & can. \\
4677 & 0 & 0 & 0 & 0 & 0 & 1 & 2 & 4 & 0 & 1 & 6 & 8 & 3 & 7 & 11 & 14 & 1 & 4 & $w_{428}$ & N & \#4665 \\
4678 & 0 & 0 & 0 & 0 & 0 & 1 & 2 & 4 & 0 & 1 & 6 & 8 & 3 & 7 & 12 & 15 & 1 & 4 & $w_{434}$ & N & \#4676 \\
4679 & 0 & 0 & 0 & 0 & 0 & 1 & 2 & 4 & 0 & 1 & 6 & 8 & 3 & 7 & 13 & 15 & 1 & 4 & $w_{430}$ & N & can. \\
4680 & 0 & 0 & 0 & 0 & 0 & 1 & 2 & 4 & 0 & 1 & 6 & 8 & 3 & 8 & 15 & 5 & 4 & 4 & $w_{379}$ & N & can. \\
4681 & 0 & 0 & 0 & 0 & 0 & 1 & 2 & 4 & 0 & 1 & 6 & 8 & 3 & 9 & 11 & 14 & 1 & 4 & $w_{430}$ & N & can. \\
4682 & 0 & 0 & 0 & 0 & 0 & 1 & 2 & 4 & 0 & 1 & 6 & 8 & 3 & 9 & 15 & 13 & 1 & 3 & $w_{438}$ & N & can. \\
4683 & 0 & 0 & 0 & 0 & 0 & 1 & 2 & 4 & 0 & 1 & 6 & 8 & 3 & 10 & 7 & 15 & 3 & 4 & $w_{432}$ & N & \#4661 \\
4684 & 0 & 0 & 0 & 0 & 0 & 1 & 2 & 4 & 0 & 1 & 6 & 8 & 3 & 10 & 15 & 13 & 2 & 4 & $w_{430}$ & N & can. \\
4685 & 0 & 0 & 0 & 0 & 0 & 1 & 2 & 4 & 0 & 1 & 6 & 8 & 3 & 11 & 7 & 12 & 1 & 4 & $w_{434}$ & N & \#4676 \\
4686 & 0 & 0 & 0 & 0 & 0 & 1 & 2 & 4 & 0 & 1 & 6 & 8 & 3 & 11 & 13 & 15 & 1 & 4 & $w_{470}$ & N & can. \\
4687 & 0 & 0 & 0 & 0 & 0 & 1 & 2 & 4 & 0 & 1 & 6 & 8 & 3 & 11 & 15 & 8 & 2 & 4 & $w_{434}$ & N & \#4676 \\
4688 & 0 & 0 & 0 & 0 & 0 & 1 & 2 & 4 & 0 & 2 & 3 & 6 & 0 & 8 & 10 & 5 & 4 & 4 & $w_{457}$ & N & can. \\
4689 & 0 & 0 & 0 & 0 & 0 & 1 & 2 & 4 & 0 & 2 & 3 & 6 & 0 & 8 & 13 & 9 & 2 & 4 & $w_{469}$ & N & can. \\
4690 & 0 & 0 & 0 & 0 & 0 & 1 & 2 & 4 & 0 & 2 & 3 & 6 & 0 & 8 & 15 & 5 & 4 & 4 & $w_{446}$ & N & can. \\
4691 & 0 & 0 & 0 & 0 & 0 & 1 & 2 & 4 & 0 & 2 & 3 & 6 & 2 & 5 & 8 & 11 & 4 & 4 & $w_{426}$ & N & can. \\
4692 & 0 & 0 & 0 & 0 & 0 & 1 & 2 & 4 & 0 & 2 & 3 & 6 & 2 & 8 & 10 & 9 & 6 & 4 & $w_{466}$ & N & can. \\
4693 & 0 & 0 & 0 & 0 & 0 & 1 & 2 & 4 & 0 & 2 & 3 & 6 & 5 & 8 & 10 & 9 & 6 & 4 & $w_{477}$ & N & can. \\
4694 & 0 & 0 & 0 & 0 & 0 & 1 & 2 & 4 & 0 & 2 & 3 & 6 & 8 & 11 & 9 & 13 & 72 & 4 & $w_{478}$ & N & can. \\
4695 & 0 & 0 & 0 & 0 & 0 & 1 & 2 & 4 & 0 & 2 & 3 & 8 & 0 & 5 & 14 & 13 & 1 & 4 & $w_{470}$ & N & can. \\
4696 & 0 & 0 & 0 & 0 & 0 & 1 & 2 & 4 & 0 & 2 & 3 & 8 & 0 & 10 & 13 & 5 & 1 & 4 & $w_{429}$ & N & can. \\
4697 & 0 & 0 & 0 & 0 & 0 & 1 & 2 & 4 & 0 & 2 & 3 & 8 & 0 & 10 & 15 & 11 & 6 & 3 & $w_{468}$ & N & can. \\
4698 & 0 & 0 & 0 & 0 & 0 & 1 & 2 & 4 & 0 & 2 & 3 & 8 & 2 & 5 & 11 & 14 & 2 & 4 & $w_{430}$ & N & can. \\
4699 & 0 & 0 & 0 & 0 & 0 & 1 & 2 & 4 & 0 & 2 & 3 & 8 & 2 & 5 & 14 & 11 & 1 & 4 & $w_{430}$ & N & can. \\
4700 & 0 & 0 & 0 & 0 & 0 & 1 & 2 & 4 & 0 & 2 & 3 & 8 & 4 & 9 & 5 & 13 & 2 & 4 & $w_{470}$ & N & can. \\
4701 & 0 & 0 & 0 & 0 & 0 & 1 & 2 & 4 & 0 & 2 & 3 & 8 & 4 & 9 & 5 & 15 & 6 & 4 & $w_{469}$ & N & can. \\
4702 & 0 & 0 & 0 & 0 & 0 & 1 & 2 & 4 & 0 & 2 & 3 & 8 & 4 & 11 & 5 & 13 & 1 & 4 & $w_{475}$ & N & can. \\
4703 & 0 & 0 & 0 & 0 & 0 & 1 & 2 & 4 & 0 & 2 & 3 & 8 & 4 & 12 & 5 & 11 & 2 & 4 & $w_{441}$ & N & can. \\
4704 & 0 & 0 & 0 & 0 & 0 & 1 & 2 & 4 & 0 & 2 & 3 & 8 & 4 & 14 & 11 & 15 & 6 & 3 & $w_{479}$ & N & can. \\
4705 & 0 & 0 & 0 & 0 & 0 & 1 & 2 & 4 & 0 & 2 & 3 & 8 & 5 & 11 & 12 & 15 & 4 & 4 & $w_{476}$ & N & can. \\
4706 & 0 & 0 & 0 & 0 & 0 & 1 & 2 & 4 & 0 & 2 & 7 & 8 & 0 & 13 & 11 & 2 & 12 & 4 & $w_{427}$ & N & can. \\
4707 & 0 & 0 & 0 & 0 & 0 & 1 & 2 & 4 & 0 & 2 & 7 & 8 & 1 & 5 & 8 & 11 & 3 & 4 & $w_{429}$ & N & can. \\
4708 & 0 & 0 & 0 & 0 & 0 & 1 & 2 & 4 & 0 & 2 & 7 & 8 & 1 & 5 & 14 & 13 & 4 & 4 & $w_{470}$ & N & can. \\
4709 & 0 & 0 & 0 & 0 & 0 & 1 & 2 & 4 & 0 & 2 & 8 & 13 & 1 & 5 & 11 & 8 & 36 & 4 & $w_{470}$ & N & \#4708 \\
4710 & 0 & 0 & 0 & 0 & 0 & 1 & 2 & 4 & 0 & 2 & 8 & 13 & 5 & 11 & 14 & 7 & 24 & 4 & $w_{480}$ & N & can. \\
4711 & 0 & 0 & 0 & 0 & 0 & 1 & 2 & 4 & 0 & 2 & 8 & 15 & 1 & 10 & 15 & 6 & 60 & 3 & $w_{440}$ & Y & can. \\
4712 & 0 & 0 & 0 & 1 & 0 & 2 & 4 & 7 & 0 & 4 & 6 & 3 & 8 & 14 & 10 & 13 & 5760 & 2 & $w_{481}$ & N & can. \\
4713 & 0 & 0 & 0 & 1 & 0 & 2 & 4 & 7 & 0 & 4 & 6 & 3 & 8 & 14 & 11 & 12 & 384 & 3 & $w_{481}$ & N & \#4712 \\

\end{longtable}
\end{footnotesize}


\end{document}